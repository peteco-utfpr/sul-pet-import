\addcontentsline{toc}{section}{A EDUCAÇÃO POLÍTICA NA FORMAÇÃO PETIANA: A EMANCIPAÇÃO POSSÍVEL FRENTE À BARBÁRIE}
\includepdf{pdfs/A-EDUCACAO-POLITICA-NA-FO}

\addcontentsline{toc}{section}{A História da Inserção Feminina nos cursos de Engenharia do Brasil}
\includepdf{pdfs/A-Historia-da-Insercao-Fe}

\addcontentsline{toc}{section}{A Humanização no Contexto dos Cuidados Paliativos por Meio de uma Ação Social: Relato de Experiência}
\includepdf{pdfs/A-Humanizacao-no-Contexto}

\addcontentsline{toc}{section}{A VERMICOMPOSTAGEM COMO ATIVIDADE DE EDUCAÇÃO AMBIENTAL}
\includepdf{pdfs/A-VERMICOMPOSTAGEM-COMO-A}

\addcontentsline{toc}{section}{A extensão rural de forma digital }
\includepdf{pdfs/A-extensao-rural-de-forma}

\addcontentsline{toc}{section}{ACADÊMICOS DE EDUCAÇÃO FÍSICA DA UFRGS E DA UFPEL NA PANDEMIA DE COVID19: ESTUDO SOBRE VARIÁVEIS ASSOCIADAS À SAÚDE, ACESSO À TECNOLOGIA E ENSINO REMOTO }
\includepdf{pdfs/ACADEMICOS-DE-EDUCACAO-FI}

\addcontentsline{toc}{section}{ARMADILHA FOTOGRÁFICA}
\includepdf{pdfs/ARMADILHA-FOTOGRAFICA}

\addcontentsline{toc}{section}{Alicerce}
\includepdf{pdfs/Alicerce}

\addcontentsline{toc}{section}{Aplicação do Framework Scrum no gerenciamento de projetos do grupo PET Engenharia de Produção UFSC.}
\includepdf{pdfs/Aplicacao-do-Framework-Sc}

\addcontentsline{toc}{section}{Artistas da Bio: Conhecendo nossa multiplicidade}
\includepdf{pdfs/Artistas-da-Bio--Conhecen}

\addcontentsline{toc}{section}{Atividade de Oratória do Treinamento com Tutor}
\includepdf{pdfs/Atividade-de-Oratoria-do-}

\addcontentsline{toc}{section}{Atividades realizadas pelo PET Engenharia Florestal nas mídias sociais em tempo de pandemia}
\includepdf{pdfs/Atividades-realizadas-pel}

\addcontentsline{toc}{section}{Atividades virtuais do PET Pedagogia UEM na quarentena de 2020}
\includepdf{pdfs/Atividades-virtuais-do-PE}

\addcontentsline{toc}{section}{Auditório Teixeirão}
\includepdf{pdfs/Auditorio-Teixeirao}

\addcontentsline{toc}{section}{Avaliação de métricas de aceitação do curso on-line “Boas Práticas de Fabricação e adaptações durante a pandemia”}
\includepdf{pdfs/Avaliacao-de-metricas-de-}

\addcontentsline{toc}{section}{BIOGÁS: A PRÓXIMA FONTE DE ENERGIA ELÉTRICA}
\includepdf{pdfs/BIOGAS--A-PROXIMA-FONTE-D}

\addcontentsline{toc}{section}{Brincando e Aprendendo: apresentando conceitos de eletrônica básica}
\includepdf{pdfs/Brincando-e-Aprendendo--a}

\addcontentsline{toc}{section}{CICLO DE PALESTRAS: PROPORCIONANDO UM NOVO CONTATO COM A FORMAÇÃO PROFISSIONAL}
\includepdf{pdfs/CICLO-DE-PALESTRAS--PROPO}

\addcontentsline{toc}{section}{COMPARTILHANDO SABERES NO I CICLO DE DEBATES SOCIOAMBIENTAIS PROMOVIDO PELO GRUPO PET CONEXÕES - GESTÃO AMBIENTAL}
\includepdf{pdfs/COMPARTILHANDO-SABERES-NO}

\addcontentsline{toc}{section}{COMPILADO DE DINÂMICAS NÃO-ODONTOLÓGICAS DO GRUPO PET: RODA DE CONVERSA, FHC E SETEMBRO AMARELO}
\includepdf{pdfs/COMPILADO-DE-DINAMICAS-NA}

\addcontentsline{toc}{section}{CURSO DE FORMAÇÃO PARA NOVOS PETIANOS: UMA AÇÃO DO PET LITORAL  SOCIAL}
\includepdf{pdfs/CURSO-DE-FORMACAO-PARA-NO}

\addcontentsline{toc}{section}{Caminhos formativos - Juventude, Políticas? ?Públicas? ?e? ?Educação?}
\includepdf{pdfs/Caminhos-formativos---Juv}

\addcontentsline{toc}{section}{Conexões de Saberes: Cine PET e Morte e Vida Severina como interfaces da Questão  Agrária no Brasil}
\includepdf{pdfs/Conexoes-de-Saberes--Cine}

\addcontentsline{toc}{section}{Conexões de Saberes: Cine PET e Morte e Vida Severina como interfaces da Questão Agrária no Brasil}
\includepdf{pdfs/Conexoes-de-Saberes--Cine}

\addcontentsline{toc}{section}{Contaminação do leite e a microbiologia preditiva}
\includepdf{pdfs/Contaminacao-do-leite-e-a}

\addcontentsline{toc}{section}{Criação de material didático: vídeos eletrônica.}
\includepdf{pdfs/Criacao-de-material-didat}

\addcontentsline{toc}{section}{Criação de um modelo de procedimento para cursos e palestras online do grupo PET Engenharia de Produção UFSC}
\includepdf{pdfs/Criacao-de-um-modelo-de-p}

\addcontentsline{toc}{section}{Curadoria de material de apoio para aprendizado da Linguagem C}
\includepdf{pdfs/Curadoria-de-material-de-}

\addcontentsline{toc}{section}{Cursos promovidos pelo PET ECV UFSC}
\includepdf{pdfs/Cursos-promovidos-pelo-PE}

\addcontentsline{toc}{section}{DESENVOLVIMENTO PROJETO SOCIAL - BÁRBARA MAIX}
\includepdf{pdfs/DESENVOLVIMENTO-PROJETO-S}

\addcontentsline{toc}{section}{Desenvolvimento de cursos em modalidade remota para a comunidade acadêmica da UTFPR }
\includepdf{pdfs/Desenvolvimento-de-cursos}

\addcontentsline{toc}{section}{Desenvolvimento de pesquisa coletiva no Programa de Educação Tutorial do curso de Direito da UFPR}
\includepdf{pdfs/Desenvolvimento-de-pesqui}

\addcontentsline{toc}{section}{Diagnóstico dos ingressantes do curso de agronomia da UTFPR Campus Pato Branco e efeitos da pandemia de COVID-19}
\includepdf{pdfs/Diagnostico-dos-ingressan}

\addcontentsline{toc}{section}{Diagnósticos PET: ferramentas e práticas estatísticas para qualificação da graduação}
\includepdf{pdfs/Diagnosticos-PET--ferrame}

\addcontentsline{toc}{section}{Dossiê Petiano: histórias, afetos e dinâmicas de grupo}
\includepdf{pdfs/Dossie-Petiano--historias}

\addcontentsline{toc}{section}{E-book de receitas: o aproveitamento dos alimentos além do convencional}
\includepdf{pdfs/E-book-de-receitas--o-apr}

\addcontentsline{toc}{section}{Encontros de Língua Estrangeira}
\includepdf{pdfs/Encontros-de-Lingua-Estra}

\addcontentsline{toc}{section}{Ensino Remoto - Monitoria Digital}
\includepdf{pdfs/Ensino-Remoto---Monitoria}

\addcontentsline{toc}{section}{Entrevista Coletiva X Individualizada na Seleção do PET Geografia da UEL: uma análise comparativa}
\includepdf{pdfs/Entrevista-Coletiva-X-Ind}

\addcontentsline{toc}{section}{Flores de corte como fonte de renda aos pequenos produtores rurais no Alto Vale do Itajaí, SC}
\includepdf{pdfs/Flores-de-corte-como-font}

\addcontentsline{toc}{section}{Futuro Profissional - Viabilizando o Intercâmbio de Saberes}
\includepdf{pdfs/Futuro-Profissional---Via}

\addcontentsline{toc}{section}{I Ciclo de Debates Socioambientais do Grupo PET Conexões - Gestão Ambiental: organização e resultados}
\includepdf{pdfs/I-Ciclo-de-Debates-Socioa}

\addcontentsline{toc}{section}{I Frutipet: Frutíferas para pequenas propriedades rurais}
\includepdf{pdfs/I-Frutipet--Frutiferas-pa}

\addcontentsline{toc}{section}{II TECNOLEITE: TECNOLOGIAS APLICADAS A PRODUÇÃO DE LEITE – O PET PRODUÇÃO LEITEIRA EM ATUAÇÃO NO SETOR AGROPECUÁRIO}
\includepdf{pdfs/II-TECNOLEITE--TECNOLOGIA}

\addcontentsline{toc}{section}{IMPACTO DA RENDA NA FORMAÇÃO DOS JOVENS PETIANOS}
\includepdf{pdfs/IMPACTO-DA-RENDA-NA-FORMA}

\addcontentsline{toc}{section}{Impacto da adaptação da atividade "Engenharia em Foco" para ensino remoto}
\includepdf{pdfs/Impacto-da-adaptacao-da-a}

\addcontentsline{toc}{section}{Implantação de ações afirmativas no Programa de Educação Tutorial do curso de Direito da UFPR}
\includepdf{pdfs/Implantacao-de-acoes-afir}

\addcontentsline{toc}{section}{Importância da Pesquisa Tecnológica no Desenvolvimento Acadêmico}
\includepdf{pdfs/Importancia-da-Pesquisa-T}

\addcontentsline{toc}{section}{Incentivo às criações alternativas na Agricultura Familiar}
\includepdf{pdfs/Incentivo-as-criacoes-alt}

\addcontentsline{toc}{section}{Instrumentalização de petianos: um espaço de desenvolvimento e aprendizado}
\includepdf{pdfs/Instrumentalizacao-de-pet}

\addcontentsline{toc}{section}{Jogo web de conscientização ao covid-19}
\includepdf{pdfs/Jogo-web-de-conscientizac}

\addcontentsline{toc}{section}{Jogos de Integração do Centro de Ciências Computacionais - JIC3}
\includepdf{pdfs/Jogos-de-Integracao-do-Ce}

\addcontentsline{toc}{section}{LEITURA LITERÁRIA DURANTE A PANDEMIA ATRAVÉS DE PLATAFORMAS DE STREAMING}
\includepdf{pdfs/LEITURA-LITERARIA-DURANTE}

\addcontentsline{toc}{section}{Literatura e Matemática: Dentre Horizontes Possíveis, Malba Tahan}
\includepdf{pdfs/Literatura-e-Matematica--}

\addcontentsline{toc}{section}{MINICURSOS: CONTRASTE DO MINICURSO PRESENCIAL X REMOTO}
\includepdf{pdfs/MINICURSOS--CONTRASTE-DO-}

\addcontentsline{toc}{section}{NO SEU PESCOÇO, CHIMAMANDA NGOZI ADICHIE: NAS PÁGINAS DO LIVRO, UM  MUNDO A DESCOBRIR}
\includepdf{pdfs/NO-SEU-PESCOCO--CHIMAMAND}

\addcontentsline{toc}{section}{NOSSA TRAJETÓRIA LITERÁRIA: O INTERPETS COMO MOMENTO PARAA FORMAÇÃO LITERÁRIA DE GRUPOS PETS.}
\includepdf{pdfs/NOSSA-TRAJETORIA-LITERARI}

\addcontentsline{toc}{section}{Notas técnicas Abril Branco  e Animal Topics- PET Produção Leiteira conexão do conhecimento}
\includepdf{pdfs/Notas-tecnicas-Abril-Bran}

\addcontentsline{toc}{section}{O impacto da pandemia da COVID-19 nos grupos PET da UFFS}
\includepdf{pdfs/O-impacto-da-pandemia-da-}

\addcontentsline{toc}{section}{Oficina de Pensamento Computacional}
\includepdf{pdfs/Oficina-de-Pensamento-Com}

\addcontentsline{toc}{section}{Oficina de Produção de Produtos de Limpeza como Fonte Alternativa de Renda para Mães em Situação de Vulnerabilidade Social}
\includepdf{pdfs/Oficina-de-Producao-de-Pr}

\addcontentsline{toc}{section}{Oficinas de introdução aos gêneros acadêmicos: (re)pensando a práxis empregada }
\includepdf{pdfs/Oficinas-de-introducao-ao}

\addcontentsline{toc}{section}{Os desafios e oportunidades das atividades on-line do grupo PETAMB durante a pandemia}
\includepdf{pdfs/Os-desafios-e-oportunidad}

\addcontentsline{toc}{section}{PERFIL DE EGRESSOS DO CURSO DE TECNOLOGIA EM ALIMENTOS DA UNIVERSIDADE TECNOLOGICA FEDERAL DO PARANÁ – CAMPUS FRANCISCO BELTRÃO}
\includepdf{pdfs/PERFIL-DE-EGRESSOS-DO-CUR}

\addcontentsline{toc}{section}{PERFIL, CONDIÇÕES E DESAFIOS DA FORMAÇÃO DO(A)S ESTUDANTES DE GRADUAÇÃO NOTURNA DA SAÚDE/UFRGS:  Serviço Social, Odontologia, Psicologia e Saúde Coletiva}
\includepdf{pdfs/PERFIL--CONDICOES-E-DESAF}

\addcontentsline{toc}{section}{PET Convida (IGTV)}
\includepdf{pdfs/PET-Convida-(IGTV)}

\addcontentsline{toc}{section}{PET Discute a Engenharia Civil}
\includepdf{pdfs/PET-Discute-a-Engenharia-}

\addcontentsline{toc}{section}{PET Explica: Conhecimento para Além da Universidade}
\includepdf{pdfs/PET-Explica--Conhecimento}

\addcontentsline{toc}{section}{PET NEWS, INTEGRANDO PROFISSIONAIS E SUAS ÁREAS DE ATUAÇÃO ESPECÍFICAS, COMPILADAS E DISTRIBUÍDAS PARA O ENTENDIMENTO E DESENVOLVIMENTO SOCIAL}
\includepdf{pdfs/PET-NEWS--INTEGRANDO-PROF}

\addcontentsline{toc}{section}{PET Zootecnia no Ensino Médio}
\includepdf{pdfs/PET-Zootecnia-no-Ensino-M}

\addcontentsline{toc}{section}{PET faz Arte}
\includepdf{pdfs/PET-faz-Arte}

\addcontentsline{toc}{section}{PET-Eventos e a pandemia}
\includepdf{pdfs/PET-Eventos-e-a-pandemia}

\addcontentsline{toc}{section}{PROJETO COVID-19}
\includepdf{pdfs/PROJETO-COVID-19}

\addcontentsline{toc}{section}{PROJETO FEQ/IEQ: aplicação de ferramentas computacionais nas disciplinas de Engenharia Química.}
\includepdf{pdfs/PROJETO-FEQ-IEQ--aplicaca}

\addcontentsline{toc}{section}{Para Além da Leitura: Cidadania em Ação}
\includepdf{pdfs/Para-Alem-da-Leitura--Cid}

\addcontentsline{toc}{section}{Pavimentação Utilizando Concreto Permeável}
\includepdf{pdfs/Pavimentacao-Utilizando-C}

\addcontentsline{toc}{section}{Percepção dos alunos do curso de Agronomia sobre o estímulo da família e da universidade no processo de sucessão familiar}
\includepdf{pdfs/Percepcao-dos-alunos-do-c}

\addcontentsline{toc}{section}{Podcast como Meio de Transformação Social}
\includepdf{pdfs/Podcast-como-Meio-de-Tran}

\addcontentsline{toc}{section}{Preparando para o Mundo Profissional (PMP)}
\includepdf{pdfs/Preparando-para-o-Mundo-P}

\addcontentsline{toc}{section}{Projeto Arboreto}
\includepdf{pdfs/Projeto-Arboreto}

\addcontentsline{toc}{section}{Projeto Brotar em Classe: Uma intervenção escolar sobre saneamento hídrico na Educação Remota   }
\includepdf{pdfs/Projeto-Brotar-em-Classe-}

\addcontentsline{toc}{section}{Projeto Vamos Entender}
\includepdf{pdfs/Projeto-Vamos-Entender}

\addcontentsline{toc}{section}{Projeto “Capacitação e qualificação PETiana”: promoção de práticas de ensino e formação  profissional ao futuro turismólogo. }
\includepdf{pdfs/Projeto-Capacitacao-e-qu}

\addcontentsline{toc}{section}{Projeto “Programa de Atendimento ao Calouro - PAC”}
\includepdf{pdfs/Projeto-Programa-de-Aten}

\addcontentsline{toc}{section}{REDES PEDAGÓGICAS: AS TECNOLOGIAS DIGITAIS COMO FERRAMENTAS DE AÇÕES DO PET PEDAGOGIA EM TEMPOS DE PANDEMIA}
\includepdf{pdfs/REDES-PEDAGOGICAS--AS-TEC}

\addcontentsline{toc}{section}{Relato de Experiência: o uso da rede social Instagram como ferramenta para disseminação de conteúdo sobre alimentação e nutrição.}
\includepdf{pdfs/Relato-de-Experiencia--o-}

\addcontentsline{toc}{section}{Relato de experiência: PET Talks contribuindo na escolha da área de atuação do estudante de nutrição.}
\includepdf{pdfs/Relato-de-experiencia--PE}

\addcontentsline{toc}{section}{Resíduo Eletrônico: Descarte, reciclagem e conscientização.}
\includepdf{pdfs/Residuo-Eletronico--Desca}

\addcontentsline{toc}{section}{Revista Informe Letras como veículo de divulgação do conhecimento sobre discursos de resistência}
\includepdf{pdfs/Revista-Informe-Letras-co}

\addcontentsline{toc}{section}{SABERES PEDAGÓGICOS: DIÁLOGOS COM JOVENS PESQUISADORES}
\includepdf{pdfs/SABERES-PEDAGOGICOS--DIAL}

\addcontentsline{toc}{section}{Saúde e resistência da população negra e indígena: um relato de experiência}
\includepdf{pdfs/Saude-e-resistencia-da-po}

\addcontentsline{toc}{section}{Seminários odontológicos: desenvolvendo o conhecimento, a pesquisa e o senso crítico}
\includepdf{pdfs/Seminarios-odontologicos-}

\addcontentsline{toc}{section}{Simpósio Online: Do desafio à oportunidade}
\includepdf{pdfs/Simposio-Online--Do-desaf}

\addcontentsline{toc}{section}{Série Documental: Controle Social nas comunidades periféricas  (Episódio 1 - Controle e Participação Social)}
\includepdf{pdfs/Serie-Documental--Control}

\addcontentsline{toc}{section}{TED-PET: MÉTODO DE APERFEIÇOAMENTO DA ORATÓRIA NO PET ODONTO}
\includepdf{pdfs/TED-PET--METODO-DE-APERFE}

\addcontentsline{toc}{section}{TransformAção - Convertendo gestos em objetos}
\includepdf{pdfs/TransformAcao---Converten}

\addcontentsline{toc}{section}{Utilização de redes sociais como ferramenta para a disseminação das geociências pelo PET Geologia UFPR}
\includepdf{pdfs/Utilizacao-de-redes-socia}

\addcontentsline{toc}{section}{VERSOS DO ÍNDICO: GRUPO CÊNICO-LITERÁRIO CONTAROLANDO (PET  PEDAGOGIA UFSC) NA PANDEMIA}
\includepdf{pdfs/VERSOS-DO-INDICO--GRUPO-C}

\addcontentsline{toc}{section}{VI Semana da Agricultura Familiar: Mulheres Rurais, Mulheres de Direitos, Mulheres de Respeito}
\includepdf{pdfs/VI-Semana-da-Agricultura-}

\addcontentsline{toc}{section}{Vivências do grupo PET Comunidades do Campo: a adaptação para o modelo remoto }
\includepdf{pdfs/Vivencias-do-grupo-PET-Co}

\addcontentsline{toc}{section}{Yellow Cow: reformulando o aprendizado de idiomas}
\includepdf{pdfs/Yellow-Cow--reformulando-}

\addcontentsline{toc}{section}{Zoopet - O futuro da zootecnia chegou: Uma possibilidade de aprimoramento por meio  remoto}
\includepdf{pdfs/Zoopet---O-futuro-da-zoot}

