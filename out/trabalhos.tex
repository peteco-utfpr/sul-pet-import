\addcontentsline{toc}{section}{A Educação Política Na Formação Petiana: A Emancipação Possível Frente À Barbárie}

\section*{A Educação Política Na Formação Petiana: A Emancipação Possível Frente À Barbárie}

Osmar Fabiano De Souza Filho, Thainara Santos de Campos; Joyce Morais de Lima; José Vinicius dos Santos Pires;  Gabriel Ferreira de Bovi; Rodrigo Batista da Silva; Thiago Bordin.  Bolsistas do PET Geografia - Universidade Estadual de Londrina/UEL

INTRODUÇÃO
Este relato de experiência apresenta atividades desenvolvidas no Programa de Educação 
Tutorial – PET da Geografia, da Universidade Estadual de Londrina – UEL, cujo objetivo se
orienta à formação humana crítica da realidade do/da petiano/a, como parte de um processo de 
ruptura com a alienação humana. Destacam-se algumas atividades desenvolvidas, como Roda de 
Literatura, Ciclo de Seminários e Roda Discussão Política e de Atualidades, as quais são balizadas 
pela criticidade e formação política emancipatória. Sendo um programa que preza, sobretudo, pela 
formação humana dos seus membros, o PET da Geografia da UEL, desenvolve essas atividades 
que vão no sentido contrário à lógica da alienação humana, ou seja, estas se fazem enquanto um 
conjunto de ações humanizadoras e emancipadoras, cuja análise qualitativa expõe a sua
importância na construção cidadã e emancipatória.
AÇÕES PETIANA DE EDUCAÇÃO POLÍTICA VISANDO A CIDADANIA E O AGIR 
PRÁTICO 
A alienação humana, discutida por Marx (2010), possuí efeitos práticos na vida cotidiana 
e nas relações sociais humanas. Este processo de negação da humanidade, promovida pelo capital, 
auxilia na explicação de fenômenos sociais que direcionam a sociedade como um todo, sobretudo 
a brasileira, no caminho da barbárie. Esta selvageria se apresenta na prática quando mesmo com a 
perda diária de quase um mil brasileiros em decorrência da Covid-19, as elites dirigentes, tentam 
impor à sociedade um “novo normal”. Qual normal pode existir com a morte de quase um mil 
brasileiros/as todos os dias? Outras questões que traduzem essa barbárie e alienação na/da vida 
prática, se dá quando o Brasil assiste 19 milhões de pessoas sofrerem novamente o flagelo da 
forme, e nada acontecer. Por fim, a barbárie contemporânea, e alienação humana, também, se 
expressa, quando cinco brasileiros detêm renda, e riqueza, equivalente à cem milhões de nacionais. 
Somente uma sociedade alienada, como a que se vive hoje, números como este são 
normalizados, pois, ao conjunto da sociedade, que sofre na vida cotidiana com esses fatores, não 
lhe é oferecida a possibilidade de pensar. Visando à formação cidadã e o agir prático frente a 
negação humana, o PET Geografia da UEL, desenvolveu três atividades que foram de muita valia 
para o processo formativo e cidadão. 
A roda de literatura, se dá com a apresentação de uma obra literária, escolhida pelo 
petiano/a, visando ressignificar as geograficidades contida nas narrativas que possam contribuir 
para pensar a atualidade e o agir prático. Esta atividade contribui para a formação política, uma 
vez que ela problematiza a realidade, por meio de outra linguagem, questões que estão presentes 
no cotidiano da sociedade moderna, desenvolvendo o senso crítico através da leitura e discussão
da obra.
O Ciclo de Seminários Científico, da mesma forma, tem sua importância formação política, 
pois, se dando a partir da discussão de obras científicas de outras áreas da ciência, proporciona a
interação dos petianos/as com linhas de pesquisa e de pensamentos diferentes, possibilitando aos 
membros do grupo conhecer outras realidades e perspectivas, enriquecendo, também, seu
conhecimento através dos debates.
Por fim, a Roda de Discussão Política de Atualidades, traz contribuições desde seu 
processo de formulação, em que ocorrem discussões para escolha de temas que envolvam a vida 
cotidiana e atualidade política. Com sua aplicação cria-se um ambiente de múltiplas opiniões que
faz com que os participantes entrem em contato com uma multiplicidade de visões acerca de pautas
que permeiam sua vida cotidiana, e que, assim, o faça refletir sobre a realidade vivenciada.
CONSIDERAÇÕES FINAIS
As ações práticas desenvolvidas pelo PET Geografia visam a formação humana e cidadã 
de seus membros, bem como daqueles externos ao grupo, mas que participam destas atividades. 
Ressalta-se que a tríade universitária se presentifica, pois cada atividade requer pesquisa 
bibliográfica e, em alguns casos, a empírica, além do ensino com apropriação da didática para o 
planejamento, apresentação, discussão e arguição das temáticas e, por fim, a extensão, pois são 
abertas à comunidade, promovendo reflexão crítica para além dos limites acadêmicos. Acreditase que o diálogo, o debate, e reflexão das ideias seja possível contribuir para a construção de seres 
ativos, emancipados, e, sobretudo, humanizados para a formação de uma outra sociedade possível
que esta da barbárie e da alienação.
PALAVRAS-CHAVE: Alienação; Emancipação; Agir Prático; Humanização; Política.
REFERÊNCIAS
MARX, Karl. Manuscritos econômico-filosóficos. São Paulo: Boitempo, 2010.

\includepdf{pdfs/A-Educacao-Politica-Na-Fo}

\addcontentsline{toc}{section}{A História Da Inserção Feminina Nos Cursos De Engenharia Do Brasil}

\section*{A História Da Inserção Feminina Nos Cursos De Engenharia Do Brasil}

Ana Carolina Rubio Klein, Gabriella Lucena,Ana Clara Prado Carvalho,Fernanda Gubert de Souza,Isac Gonçalves de Oliveira,Estevãn Martins de Oliveira

A sociedade brasileira possui uma desigualdade de gênero notável, devido a questões
culturais e sociais, predispondo mulheres à realização de tarefas do lar, enquanto o espaço
público de ensino e empresarial é destinado aos homens.
Ao longo dos tempos, as mulheres vêm tomando espaços jamais ocupados, possibilitando
a inserção da própria no mercado de trabalho. Isso se deve ao avanço e ao crescimento da
industrialização no Brasil, aos quais proporcionaram transformações na estrutura produtiva,
gerando um processo contínuo de urbanização e consequentemente, uma redução das taxas de
fecundidade. Sendo assim, o termo “representatividade” vem se tornando extremamente
necessário e debatido. Lugares que eram ocupados exclusivamente por homens, como as áreas de
Engenharias, no século XXI é constituído também pelo público feminino. Devido a este fato, é
perceptível que a inserção de alunas no âmbito acadêmico-científico das áreas de Engenharias
por sua vez ainda sofre grande preconceito.
Assim, o presente trabalho tem como objetivo apresentar a história da inserção da mulher
na Engenharia, expondo a luta travada pelas primeiras Engenheiras, demonstrando sua evolução
e analisando sua participação; e conscientizar a população acadêmica e social com relação aos
paradigmas que devemos combater.
O desenvolvimento deste trabalho se deu através de uma pesquisa descritiva com base
nos materiais bibliográficos, levando em virtude a inclusão das mulheres na Engenharia, a qual
obteve uma quebra de valores que as discriminam em carreiras consideradas majoritariamente
masculinas. Pelo fato dessas carreiras ainda serem ocupadas, ou ao menos serem observadas,
como cargos masculinos, o grande tabu, o qual é gerado pelo machismo estrutural em que
vivemos na sociedade brasileira, causa queda no interesse das meninas pela profissão.
Para fazer uma contextualização sobre a inserção feminina no âmbito da Engenharia, é
interessante apresentar alguns fatos históricos encontrados sobre o Brasil. Segundo CASTRO,
2010 o início dos cursos de Engenharia no Brasil foi em 1792, no Rio de Janeiro, tendo como
precursor a Real Academia de Artilharia, Fortificação e Desenho. Da documentação legal que
contém a inserção de mulheres no ensino superior do Brasil, encontra-se o Decreto de número
7.247 de 1879, quase 100 anos depois.
Das primeiras turmas de Engenharia, apenas em 1917, na Escola Politécnica da UFRJ,
Edwiges Maria Becker Hom’meil se tornou a primeira mulher brasileira formada em Engenharia
no país, seguida de Enedina Alves Marques, que se formou em 1945 pela UFPR como a primeira
mulher negra engenheira no país. Somente em 1997, o ITA (Instituto Tecnológico de
Aeronáutica) abriu vagas para mulheres nos cursos de Engenharias mais concorridos do país,
formando-se assim, em 2000, as suas primeiras Engenheiras de infraestrutura e aeronáutica.
No estado do Paraná, de 2013 a 2018 o número de engenheiras registradas aumentou
24%, sendo um total de 12.546 mulheres, a partir de uma pesquisa realizada pelo Conselho
Regional de Engenharia e Agronomia do Paraná (CREA - PR). De acordo com o Conselho
Federal de Engenharia e Agronomia (CONFEA), entre os anos de 2016 e 2018, o número anual
de mulheres engenheiras registradas no Brasil cresceu 42%, sendo que o número total de
mulheres com registros no sistema, é de 196.372, segundo uma matéria publicada no Conselho
Regional de Engenharia e Agronomia de Alagoas (CREA - AL) em março de 2019. Esse valor
corresponde a 15% do total de 1.109.628 de engenheiros registrados no Brasil.
Conclui-se que a inclusão da mulher na Engenharia ainda é um assunto recente cujo qual
precisa ser debatido, devido ao fato de ainda haver grandes tabus e pensamentos retrógrados
nesse meio. Assim, existe um dever histórico de debater e desenvolver a representatividade
feminina nas áreas de Engenharia do Brasil.
Referências
CASTRO, R. N. A. Teorias do currículo e suas repercussões nas diretrizes curriculares dos
cursos de Engenharia. Educativa, Goiânia, v. 13, n. 2, p. 307-322, jul./dez. 2010.
Dispara o número de mulheres engenheiras registradas no Brasil. CREA-AL. Disponível em:
.
Acesso em AGO. 2021.
10 engenheiras que marcaram história no mundo. Associação de Engenheiros do Brasil.
Disponível em:
. Acesso em:
AGO. 2021.
MENDONÇA, L. K. et al. Mulheres na Engenharia: desafios encontrados desde a
Universidade até o chão de fábrica na Engenharia de Produção na Paraíba. 18° REDOR,
2014. UFRP.

\includepdf{pdfs/A-Historia-Da-Insercao-Fe}

\addcontentsline{toc}{section}{A Humanização No Contexto Dos Cuidados Paliativos Por Meio De Uma Ação Social: Relato De Experiência}

\section*{A Humanização No Contexto Dos Cuidados Paliativos Por Meio De Uma Ação Social: Relato De Experiência}

Alana Flavia Rezende, Larissa Padoin Lopes,Vitória Goularte de Oliveira,Laís Moreira Martins,Lais Kaori Sato Murrugarra,Vanessa Denardi Antoniassi Baldissera,Universidade Estadual de Maringá,Jhenicy Rubira Dias

Introdução: A Organização Mundial de Saúde (WHO, 2016) define Cuidados Paliativos (CP’s)
como “uma abordagem que promove a qualidade de vida de pacientes e seus familiares, que
enfrentam doenças que ameacem a continuidade da vida, por meio da prevenção e alívio do
sofrimento” (OMS, 2016). Contemplam estratégias focadas nos aspectos bio-psico-sociais e
espirituais para atenção integral e holística com vistas à qualidade de vida (BANDEIRA et al.,
2020). Nessa direção, instrumentaliza o atendimento humanizado por quem cuida e a percepção
acolhedora por quem é cuidado. Objetivo: Relatar a experiência de uma ação social aplicada ao
contexto das pessoas que estão em cuidados paliativos. Metodologia: Trata-se de um estudo
descritivo do tipo relato de experiência acerca da atividade de ação social “Cartinhas de Amor”
realizada pelo Programa de Educação Tutorial de Enfermagem (PET Enfermagem) da
Universidade Estadual de Maringá em parceria com a Rede Feminina de Combate ao Câncer de
Maringá, no ano de 2021, voltado para o público-alvo de 109 pacientes em cuidados paliativos.
As cartinhas foram escritas pelos estudantes do curso de enfermagem da instituição, sob
coordenação dos alunos participantes do PET Enfermagem. Para a escrita das cartinhas o
primeiro passo foi receber inscrição voluntária de alunos do curso de enfermagem para esse fim.
Após as inscrições, os participantes receberam orientações prévias a respeito de CP’s,
humanização da atenção interdisciplinar e as formas de linguagem ao adulto e à criança, por
meio de palestras ministradas por profissionais convidados e um curso disponível na plataforma
UNA-SUS. A escrita das cartas foram anônimas, com sigilo do destinatário, para não expor os
envolvidos. Resultados e Discussão: Durante a elaboração, fez-se necessário que os alunos
refletissem sobre a condição de saúde e vulnerabilidade biopsicossocial-espiritual, de forma a
desenvolver as habilidades de empatia, linguagem escrita e humanização da relação, ainda que
de forma indireta. Portanto, foram escritas mensagens de carinho, apoio e esperança para motivar
o enfrentamento desse momento ao suscitar sentimentos de acolhimento, compaixão e resiliência
daqueles que as receberam. As cartinhas escritas pelos alunos de enfermagem foram
encaminhadas via endereço eletrônico para o grupo PET Enfermagem, que as imprimiram e as
levaram até a sede da Rede Feminina de Combate ao Câncer de Maringá, para que chegassem ao
destinatário no momento da retirada de benefícios sociais oferecidos pela rede. Conclusão: A
presente ação social permitiu aproximação profissional e pessoal junto a esse público,
enaltecendo a formação cidadã. Além disso, proporcionou contato com esta temática ainda na
formação básica na graduação. Já para as pessoas em CP’s, o trabalho promoveu acolhimento e
supostamente conforto como motivadores ao enfrentamento da condição que vivem.

\includepdf{pdfs/A-Humanizacao-No-Contexto}

\addcontentsline{toc}{section}{A Vermicompostagem Como Atividade De Educação Ambiental}

\section*{A Vermicompostagem Como Atividade De Educação Ambiental}

Maria Julia Pavesi Marcal, Evelyn Fernanda Latarulo de Moraes,Helen Silva dos Santos,Igor de Souza Batista,Leiddi Laura Maria Leal,Mylena Kellyn de Paula Rosetti,Laercio Mantovani Frare

Devido a alta geração de resíduos sólidos, sendo que grande parte corresponde a resíduos orgânicos e que parte de sua destinação é inadequada, surge a necessidade de uma educação ambiental e o investimento na gestão de resíduos sólidos. Com isso, tem-se a ideia da disseminação da técnica da vermicompostagem de forma educacional, por meio de sua construção para tratamento de  resíduos domésticos. Portanto, foram descrito os materiais e o passo-a-passo para a construção e manuseio da vermicomposteira. Com a utilização da técnica, conseguiu-se tratar os resíduos aplicados adequadamente por meio da decomposição da matéria orgânica, não sendo nocivo para as minhocas. Além disso, obteve-se um  composto de cor escura, densidade homogênea e sem odor, sendo possível a sua aplicação como húmus para adubação e o chorume como biofertilizante. Conclui-se então, que essa é uma técnica ecologicamente correta, economicamente viável e de fácil manuseio. 

\includepdf{pdfs/A-Vermicompostagem-Como-A}

\addcontentsline{toc}{section}{A Extensão Rural De Forma Digital }

\section*{A Extensão Rural De Forma Digital }

Gilmar Franklin Machado, BORSOI, A. L.,BAZZANELLA, C. M.,CASTANHA, A. A.,GIONGO, F.,POSSENTI, J. C.

A ciência mundial sempre foi tida como complexa, dotada de grandiosidade e dificuldades
na sua elaboração e desenvolvimento. Não é à toa que as grandes mentes humanas dos últimos 
séculos pertenceram a brilhantes cientistas que marcaram época. É consenso que a ciência pode 
melhorar, intensificar, prolongar e ainda, tornar o modo de vida da humanidade mais confortável 
e longeva, independendo do modo e forma que isso venha a acontecer. 
O fato é que a ciência contribui na qualidade de vida dos seres que habitam a terra. O que 
limita a intensidade com que as informações produzidas pela ciência sejam distribuídas e aplicadas 
por todos, são os diferentes interesses de quem as produziu. Por vezes a complexidade com que as 
informações saem das academias e centros de pesquisa, colabora para os resultados não sejam tão 
popularizados como deveriam ser. O entendimento daquilo que é produzido pelos cientistas tem 
uma importância impar no processo de uso, pois basicamente, a ciência divide a pesquisa em básica 
e aplicada. Esta por sua vez, trata de dar aplicação ao que foi produzido cientificamente.
O papel de traduzir para a população, a informação produzida pela ciência é da extensão.
A extensão rural por sua vez, trata de levar ao campo, de forma traduzida, as informações 
científicas produzidas pelas universidades e instituições de pesquisa da área. O presente trabalho 
foi desenvolvido pelo Grupo PET Conexões de Saberes – Agricultura Familiar, Saberes e Fazeres 
da Vida no Campo (PET AF), pertencente à Universidade Tecnológica Federal do Paraná, Campus 
Dois Vizinhos. Professores pesquisadores das Ciências Agrárias, foram convidados para produção 
de lives técnicas e material digital como vídeos de curta duração, traduzindo pesquisa cientifica 
para extensão rural aplicada. Este material foi veiculado por meio de plataformas de mídia 
disponíveis na internet. O material gerado teve abordagem simples e entendível pelos agricultores, 
os quais são os verdadeiros aplicadores práticos da informação produzida pelos pesquisadores. 
A escolha da plataforma que seria responsável por acomodar o material produzido, foi o 
primeiro passo dado, sendo então selecionado o YouTube® como plataforma principal. A criação 
do canal PET Agricultura Familiar foi feita e em seguida, realizou-se a procura por 
pesquisadores do corpo docente da UTFPR-DV com interesse em participar da iniciativa. Após a 
seleção inicial dos professores, deu-se início a produção do material. 
Cada professor que participava, escolhia um trabalho dentro do seu espectro de atuação e 
um pequeno roteiro era criado e discutido. Produziram-se lives no formato de palestras, de mesas 
redondas e também de Webinar, bem como vídeos gravados. No caso de gravações, estas eram
realizadas na Estação Experimental do campus, onde as pesquisas vinham sendo conduzidas. Não 
foram usados equipamentos sofisticados, apenas celular smartphone, com auxílio de um tripé.
Após o material gerado, o mesmo era devidamente editado, faziam-se os ajustes 
necessários, para que a divulgação pudesse ser feita.
Com pouco mais de um ano desde a sua criação, o canal do Grupo PET AF, conta com 
mais de 310 inscritos e mais de 6.500 visualizações. A repercussão regional do trabalho tem
apresentado comentários positivos, não só oriundos dos produtores que acompanham os vídeos 
produzidos, mas também dos profissionais que atuam no campo. Desta maneira, o Grupo PET AF, 
tem colaborado com a extensão universitária, partilhando com a comunidade externa pagadora de 
impostos, resultados das pesquisas internamente realizadas.
Pode-se concluir, que a produção e veiculação de conteúdo digital por meio das redes 
sociais, é uma importante ferramenta para facilitar a comunicação entre a universidade e o público 
externo. A iniciativa deste projeto, desencadeou uma série de projetos semelhantes dentro do 
campus, o que fortalece a ideia inicial e agrega valor para a sociedade.

\includepdf{pdfs/A-Extensao-Rural-De-Forma}

\addcontentsline{toc}{section}{Acadêmicos De Educação Física Da Ufrgs E Da Ufpel Na Pandemia De Covid19: Estudo Sobre Variáveis Associadas À Saúde, Acesso À Tecnologia E Ensino Remoto }

\section*{Acadêmicos De Educação Física Da Ufrgs E Da Ufpel Na Pandemia De Covid19: Estudo Sobre Variáveis Associadas À Saúde, Acesso À Tecnologia E Ensino Remoto }

Arthur Damacena Elias, Camila Corletto (UFRGS),Igor Monteiro (UFRGS),Laura Martins (UFRGS),Andréa Gonçalves (UFRGS),Alisson Gularte (UFRGS),Michael Alves (UFRGS),Paulo Guedes (UFRGS),Lucas Richter (UFRGS),Maria Vitória Bernadotte (UFRGS),Gabriel Coscia (UFRGS),Matheus Goulart (UFRGS),Georgia Rodrigues (UFRGS)

O trabalho apresenta um estudo  cross-sectional com delineamento através do survey, com o intuito de analisar impactos da pandemia de COVID-19 nos estudantes dos cursos de Educação Física da Universidade Federal do Rio Grande do Sul (UFRGS) e Universidade Federal de Pelotas (UFPel). Os dados para o estudo foram coletados pela plataforma Google Forms e contou com um total de 265 alunos do curso de Educação Física, licenciatura ou bacharelado das duas universidades, para uma melhor análise as respostas foram divididas em 3 grupos baseado na autopercepção de saúde durante a pandemia em comparação ao período antes do distanciamento social: Igual, Pior e Melhor.

A comparação entre os 3 grupos foi com o método qui-quadrado, e os resultados obtidos pelas analises mostraram uma diferença significante na pratica de atividade física durante a pandemia, destaca-se que o grupo que se autopercebeu Pior mostrou maior percentual de não realização de atividade física, assim como também avalia de modo diferenciado que o ensino remoto não deveria ser implantado pela dificuldade de acesso das pessoas. 
 


\includepdf{pdfs/Academicos-De-Educacao-Fi}

\addcontentsline{toc}{section}{Armadilha Fotográfica}

\section*{Armadilha Fotográfica}

Victor Hugo Igino Bezerra, Italo Ribeiro Fabiani

Devido a dificuldade de mapeamento de populações de animais em seus habitats naturais, como
florestas e até mesmo a grande mata atlântica, tornou-se necessário o estudo de melhorias para o
biomonitoramento, uma vez que o que existe atualmente é caro, de difícil aquisição por falta de
modelos nacionais e pouco eficiente. Para aumentar a eficiência, deve-se investir em melhorar o
tempo de vida útil da bateria (que possibilita grandes períodos sem a interferência presencial
humana), a qualidade das fotos e das filmagens, e principalmente o preço da tecnologia.
Para tanto, foi usado uma plataforma baseada no processador ARM-v8, o Raspberry Pi 4, que
pode ser usado como um computador e é encontrado nas lojas tradicionais de venda de
componentes eletrônicos. Um módulo de câmera OV5647, de 5 Mpxs para uso com o Raspberry,
um sensor de presença DYP-ME003 e uma bateria (powerbank) de 10400mAh TP-LINK
TL-PB10400. Um programa foi executado na linguagem Python, inicialmente usando o
Raspbian (atualmente chamado de Raspberry OS) para a programação direta no dispositivo.
O algoritmo desenvolvido para a detecção do movimento na imagem foi baseado no algoritmo
de subtração de background, onde imagens são capturadas em sequência e em seguida é
“retirado” o fundo imóvel da imagem, mantendo apenas o objeto que se movimentou. Feito isso,
foi comparada a imagem atual com a anterior, realizando a subtração matricial das mesmas e
encontrando o módulo dessa matriz de diferença. Se o módulo for muito pequeno, da ordem de
500 a 1000, significa que não houve um movimento considerável no ambiente ou que o
movimento observado foi ruído de imagem. Se ele for superior a isso, a foto onde o movimento
foi detectado é armazenada. Para evitar erros em ambientes com pouca luminosidade, foi
adicionado ao algoritmo uma detecção de movimentos por sensor de presença. Se o sensor
detectar algum movimento que o algoritmo não detectou, é capturada a foto.
Os testes foram feitos em bancada com objetos sendo movimentados na frente da objetiva e o
dispositivo desenvolvido funcionou apropriadamente. Sempre que houve um movimento
significativo o sistema registrou a foto. Os testes com detector de presença mostraram que o
sensor é muito lento para ser utilizado sozinho. Testes de duração da bateria e de gasto
energético ainda não foram feitos. Uma melhoria, utilizando uma câmera de infravermelho, que
tornará o sistema bem mais eficiente em fotos no escuro, está em andamento. Após a consecução
destes testes, estão programados ensaios de campo para fotografia de animais silvestres em
ambientes de floresta preservada na região de Curitiba e da Ilha do Mel.
Após sua conclusão final, o sistema tem grande possibilidade de ser usado nos trabalhos de
pesquisa de Engenharia Biomédica, especialmente naqueles de Biomonitoramento animal.
GRANDE, KARIN CRISTINE; Schneider, Nicole Becker ; Sato, Gilson Yukio ; Schneider, Bertoldo . Passive Acoustic Localization
Based on Time of Arrival Trilateration. IFMBE Proceedings. -ed.: Springer Singapore, 2019, v. , p. 519-524.
GOMES, F. H. ; GRANDE, K. C. ; SCHNEIDER JR, B. . Desenvolvimento de tecnologia de um dispositivo de longo alcance e baixo
consumo para rastreabilidade de animais ameaçados. In: XII SIMPÓSIO DE ENGENHARIA BIOMÉDICA, 2019, Uberlândia. IX
SIMPÓSIO DE INSTRUMENTAÇÃO E IMAGENS MÉDICAS. Uberlândia, 2019.
GRANDE, K. C.; GOMES, F. H. ; SANTIAGO, E. ; GEWEHR, P. M. ; BERGOSSI, V. H. D. ; SCHNEIDER JR, B. . LORA based
biotelemetry system for large land mammals. In: European Test and Telemetry Conference, 2018, Nürnberg. European Test and
Telemetry Conference- Chapter 5. Time-space position technologies. Berlin: Ama Science, 2018. p. 101-105.
GRANDE, K. C.; Schneider, N B ; Sato, G Y ; SCHNEIDER JR, B. . Passive acoustic localization based on time of arrival trilateration.
In: XXVI Congresso Brasileiro de Engenharia Biomédica, 2018, Buzios. Anais do XXVI Congresso Brasileiro de Engenharia
Biomédica, 2018.
GRANDE, K. C.; SCHNEIDER JUNIOR, B. . USO DE TECNOLOGIAS GLOBAIS PARAA DETERMINAÇÃO DE HABITATS DE
ESPÉCIES BIOINDICADORAS E TRANSFORMAÇÕES DE TERRITÓRIOS.. In: XXV Congresso Brasileiro de Engenharia
Biomédica (XXV CBEB), 2016, Foz do Iguaçu. Anais do XXV Congresso Brasileiro de Engenharia Biomédica (XXV CBEB). Curitiba:
UTFPR, 2016. v. 1. p. 2161-2164.
A DETECÇÃO DE MOVIMENTO COM O OPENCV. [S. l.], 13 jul. 2021. Disponível em:
https://cadernodelaboratorio.com.br/a-deteccao-de-movimento-com-o-opencv/. Acesso em: 11 set. 2021.
OPENCV UTILIZANDO PROGRAMAÇÃO EM PYTHON: Detecção de movimento com Python e OpenCV. Disponível em:
http://www.galirows.com.br/meublog/opencv-python/opencv2-python27/capitulo2-deteccao/deteccao-movimento/. Acesso em: 11 set.
2021

\includepdf{pdfs/Armadilha-Fotografica}

\addcontentsline{toc}{section}{Alicerce}

\section*{Alicerce}

Laysa Samara Da Silva, Milena Mayumi Costa Makimori,Vinício Tiossi  Schincaglia,Tutor: Jorge Luís Nunes de Góes

É previsto pelo Art. 4o da Resolução CNE/CES 11/2002 (a qual institui Diretrizes 
Curriculares Nacionais do Curso de Graduação em Engenharia) que o engenheiro, em sua 
formação, possua conhecimentos requeridos para desenvolver e utilizar ferramentas para o 
exercício da profissão. Entretanto, alguns aspectos fundamentais para o trabalho do engenheiro, 
como o domínio de alguns softwares de modelagem e dimensionamento de edifícios e alguns 
esclarecimentos sobre trâmites burocráticos para a aprovação de projetos, não são contemplados 
em algumas disciplinas por falta de tempo hábil durante o período letivo.
Diante disso, a mesma resolução enfatiza que devem ser estimuladas atividades para a 
formação complementar, a fim de se reduzir o tempo em sala de aula e favorecer o trabalho 
individual e em grupo. Assim, o aprendizado de ferramentas complementares à formação do aluno 
é diretamente relacionado a quantidade e qualidade de cursos extracurriculares ofertados.
Uma das atividades comuns presentes nos grupos PET de engenharia, é a realização de 
cursos capacitando os graduandos nas chamadas hard e soft skills. O grupo PET Civil da UTFPR 
costumeiramente assume essa função, por meio do Projeto Alicerce, onde os próprios membros do 
grupo ou discentes e docentes convidados ofertam capacitações para a comunidade acadêmica da 
própria universidade, principalmente. Tal Projeto foi criado no ano de 2013 e tem como objetivo 
promover a transformação do acadêmico por meio de cursos de formação complementar, de tal 
forma a sanar as lacunas comuns existentes na grade curricular.
Entretanto, devido a pandemia do Covid-19 e a suspensão das atividades letivas presenciais 
por parte da UTFPR desde o dia 16 de março de 2020, o grupo PET Civil – UTFPR ficou 
impossibilitado de prosseguir com algumas das atividades planejadas para o ano de 2020 e 2021
por falta, principalmente, do ambiente apropriado e da comunidade acadêmica que retornou para 
suas cidades de origem.
Dessa forma, para que o projeto Alicerce fosse mantido, o grupo precisou inovar a maneira 
de ofertar tais cursos, uma vez que, mesmo com a paralisação das aulas, o desenvolvimento da 
sociedade acadêmica não deve cessar. Para isso, por meio de capacitações semanais o grupo vem 
ofertando cursos onde são ensinados softwares das mais variadas áreas, não só com enfoque na 
área de Engenharia Civil, mas também em conhecimentos da plataforma Excel, Illustrator e 
Photoshop.
Considerando as dificuldades de reuniões presenciais causadas pela pandemia o grupo
decidiu realizar as capacitações internas virtualmente utilizando a plataforma Jitsi Meet a qual 
mais tarde foi substituída pelo Google Meet, que permite um número de até cem pessoas online, 
suficiente para atender a demanda dos cursos. No ano de 2021 também se passou a utilizar a 
plataforma YouTube, que possibilitava disponibilizar os cursos por meio de lives e vídeos que 
ficam acessíveis por determinado período de tempo.
Para estrear o novo método de ensino, o primeiro curso ofertado foi o Ftool e, além dos 
elogios que os participantes fizeram tanto à plataforma, que nesse caso foi o Jitsi Meet, quanto ao 
grupo PET, foi útil para averiguar que a plataforma estava adequada para a continuação do uso. 
Além do Ftool, foram ofertados pelo grupo, cursos de Excel básico, Revit, Robot, e o Workshop 
de mídias digitais, no qual engloba o Illustrator e o Photoshop. Em relação ao novo meio de 
disponibilização dos cursos em 2021, houve uma ótima avaliação dos alunos pois, os mesmos 
viram vantagem em poder ver as aulas no horário que melhor julgassem.
Para a realização de cada um dos cursos, houve um planejamento prévio, desde o
formulário de inscrição até o formulário final de satisfação, no qual os participantes avaliavam o 
desempenho do grupo e do curso. Foi estipulado um cronograma contendo o período de inscrição, 
a duração, tanto em dias ou horas ministradas, e, por fim, contendo um período para que fosse 
entregue o certificado.
Na maioria dos cursos, a meta no número de inscrições foi superada, visto que,
presencialmente os mesmos cursos não tinham tanta adesão quanto ofertados virtualmente como 
na situação atual. Em relação aos cursos ofertados até o momento da elaboração deste trabalho, 
todos obtiveram respostas satisfatórias e os ministrantes e a organização foram elogiadas, tanto 
para com o evento, mas também para com a iniciativa de ofertar cursos visando melhorar o 
conhecimento sobre determinadas ferramentas digitais.
Comparando o número de participantes no Projeto Alicerce desde 2013, é possível observar 
um grande incremento do número de participantes nos cursos ofertados pelo Projeto Alicerce no 
ano de 2020 e 2021. Ressalta-se que nos anos de 2013 até 2019 todos os cursos foram ofertados 
de forma presencial, já em relação aos anos de 2020 e 2021, os cursos foram ofertados 
virtualmente.
A alteração do Projeto Alicerce costumeiramente formada por atividades didáticas
presenciais, para atividades não presenciais síncronas e assíncronas, possibilitou um incremento 
da participação de acadêmicos, inclusive de outras universidades. Tal afirmação pode ser 
confirmada pelo fato de que durante o período em que os cursos foram ministrados de maneira 
remota, foi notório o aumento de alunos participantes quando comparados com os anos anteriores, 
revelando um ponto positivo do cenário atual.
REFERÊNCIAS: 
BRASIL. Resolução CNE/CES 11, de 11 de março de 2002. Diário Oficial da União, 
Brasília, DF, 9 de abril de 2002. Seção 1, p. 32.

\includepdf{pdfs/Alicerce}

\addcontentsline{toc}{section}{Aplicação Do Framework Scrum No Gerenciamento De Projetos Do Grupo Pet Engenharia De Produção Ufsc.}

\section*{Aplicação Do Framework Scrum No Gerenciamento De Projetos Do Grupo Pet Engenharia De Produção Ufsc.}

Marcos Miyahara Hirano, João Vitor Goedert,Eduardo Daniel,Antonio Cezar Bornia,Ana Luiza da Costa Garcia

Dentre as formas de capacitação disponíveis para os membros do PET Engenharia de 
Produção da UFSC, uma das mais impactantes é a realização de projetos, práticos e teóricos, cujos 
temas variam entre diversas áreas dos cursos de Engenharias de Produção, Civil, Elétrica e 
Mecânica. Eles são realizados em parceria com organizações internas e externas à universidade e 
possibilitam aos PETianos oportunidades para aplicar os conhecimentos acadêmicos na prática, 
inserindo-os no ambiente profissional.
Até 2019, o gerenciamento de projetos do grupo era baseado no método cascata (PMBOK), 
uma metodologia tradicional caracterizada pela definição de objetivos fixos, cronogramas rígidos 
e inflexíveis, falta de momentos específicos para feedbacks com os stakeholders envolvidos em 
cada projeto, entre outras características, que geravam constantes atrasos nas entregas pretendidas. 
Visando solucionar essa problemática, buscou-se por uma alternativa ao modelo tradicional que 
apresentasse resultados mais ágeis e que estivesse melhor adaptada aos interesses do grupo: o
Framework Scrum.
O Framework Scrum é uma metodologia ágil que se adapta ao contexto da equipe, trazendo 
maior dinamismo na realização dos projetos e maior valor agregado ao produto final. Este preza 
pela constante comunicação entre a equipe de projetos e os clientes; dessa forma, os feedbacks 
tornam-se frequentes e as oportunidades de melhoria ficam mais evidentes. Além disso, as 
constantes entregas - outra característica do Scrum - acarretam em prazos mais adaptáveis à 
realidade da equipe de projeto e dos stakeholders.
A ideia da implementação do Framework Scrum surgiu a partir da formulação de 
uma ação estratégica, que segue o modelo de Gestão Estratégica já existente no grupo. A primeira 
etapa dessa ação foi um estudo em literaturas que abordam o Scrum como uma metodologia ágil 
para gerenciamento de projetos. O principal livro utilizado para este estudo foi o “Scrum - A arte 
de fazer o dobro na metade do tempo” de J.J. Sutherland e do cocriador do framework, Jeff 
Sutherland. Após o estudo, a próxima etapa foi a realização de benchmarkings com três 
organizações que possuem estrutura semelhante ao PET Engenharia de Produção e que já 
utilizavam o framework, a fim de identificar que tipo de adaptações foram realizadas para tornar a 
metodologia viável a uma entidade estudantil.
Com o conhecimento adquirido e consolidado, partiu-se para a criação de um treinamento 
focado na teoria e prática do Scrum, a fim de repassar o conhecimento para o restante do grupo e 
perpetuar o funcionamento desta nova forma de gerenciamento de projetos. Por fim, como o grupo 
PET Engenharia de Produção possui um Sistema de Gestão da Qualidade implementado segundo 
a norma ISO 9001:2015, foi necessário reformular toda a documentação relacionada ao 
procedimento de realização de projetos técnicos dentro do grupo, o qual era fundamentado em 
uma metodologia mais tradicional de gerenciamento de projetos.
O novo processo de gerenciamento do grupo se inicia com a definição dos papéis de
Product Owner, Scrum Master e Time de Desenvolvimento entre todas as partes envolvidas 
(clientes, professor orientador, grupo de projetos, coordenador e trainee de projetos). Em seguida, 
o progresso das atividades do projeto até a sua finalização ocorre através das Sprints, que podem 
ser definidas como períodos limitados destinados à realização de atividades pré-determinadas que 
fazem parte do objetivo geral que se pretende atingir com o projeto, e ela é composta por quatro 
principais reuniões, denominadas como as “cerimônias” do Scrum.
A primeira é a Reunião de Planejamento, onde são discutidos e elencados todos os 
objetivos que se pretende alcançar ao final da Sprint, juntamente com os prazos associados. 
Posteriormente, são realizadas, semanalmente, as Reuniões de Weekly, cujo objetivo é alinhar, 
com a equipe do projeto, o andamento das atividades e se estão ocorrendo de acordo com o 
planejado, provendo auxílio sempre que necessário. Chegando ao prazo final para as atividades, é 
realizada a Reunião de Revisão junto ao cliente do projeto, para validar os resultados alcançados 
na Sprint e coletar feedbacks que possam ser implementados visando sempre a satisfação do 
mesmo. Por fim, é realizada a Reunião de Retrospectiva entre os membros do projeto, onde são 
discutidos os pontos que deram certo durante os trabalhos e os que podem ser melhorados para as 
próximas etapas/sprints, sempre com a ideia de melhoria contínua. Assim, todas as atividades 
seguem este ciclo, até que todos os objetivos e escopo do projeto sejam alcançados.
Após a implementação e adequação dos princípios do Scrum para a realidade do grupo, o 
gerenciamento dos projetos tornou-se muito mais dinâmico, através da adesão das cerimônias e da 
definição dos papéis para cada projeto, entre outros princípios adotados do framework. Observouse que, apesar da recente implementação, a aplicação do Scrum trouxe benefícios evidentes ao 
grupo, como a redução dos atrasos e maior satisfação dos clientes a respeito dos resultados 
desenvolvidos nos projetos.

\includepdf{pdfs/Aplicacao-Do-Framework-Sc}

\addcontentsline{toc}{section}{Artistas Da Bio: Conhecendo Nossa Multiplicidade}

\section*{Artistas Da Bio: Conhecendo Nossa Multiplicidade}

Jaqueline Goldani Becker, Alessandra Maria Couto Figueira,Fernanda Zanini dos Santos Bentancur,Lucca Bragança Castagnino Viana,Roberta Delgado Bauer,Ana Júlia Vicari

O projeto Artistas da Bio surgiu da visão de que a arte se apresenta como uma aliada no período 
de quarentena - que inquieta, limita nossas trocas e experiências e nos faz conviver intensamente 
com nós mesmos. Além de sustento econômico, ela serve como uma facilitadora no entendimento 
e externalização de sentimentos. Para demonstrar nossa multiplicidade como indivíduos, não nos 
privando de explorar outras áreas de interesse e de procurar meios de expressão e catarse, 
realizamos um chamamento de colegas (e ex-colegas) do curso de Ciências Biológicas para 
compartilhar suas artes. O artista compartilhava conosco suas criações artísticas e escrevia um 
pequeno texto contando um pouco sobre si e sua relação com a arte. A cada semana publicamos 
em nossas redes sociais o material de algum participante. Finalizamos essa parte do projeto em 
dezembro, tendo a participação de 11 alunas em 14 postagens, nas quais divulgamos pinturas e 
desenhos digitais e manuais, crochês, poemas, lettering, pinturas em parede e cerâmica e vídeos 
de declamação de texto. Na busca por outras abordagens de incentivo à arte, este ano iniciamos a 
elaboração de fanzines com a proposta de cada edição ter um tema base abrangente para que os 
participantes exponham suas interpretações sobre o tema, na intenção de aproximar sensibilidades 
e o viver de perspectivas individuais em uma construção coletiva. Para a primeira edição, 
convidamos nossos colegas a compartilharem conosco suas variadas manifestações artísticas 
produzidas durante o período de quarentena. Sua elaboração funcionou como uma curadoria 
somada a uma produção artística. Para que houvesse fluidez e diálogo entre as artes, além do 
material enviado pelos participantes, novos textos e imagens foram criados pelos organizadores 
do projeto. A PETZINE foi disponibilizada digitalmente em nossas redes sociais e cópias físicas 
serão disponibilizadas na Biblioteca do IB, no DAIB e na sala do PET Biologia.

\includepdf{pdfs/Artistas-Da-Bio--Conhecen}

\addcontentsline{toc}{section}{Atividade De Oratória Do Treinamento Com Tutor}

\section*{Atividade De Oratória Do Treinamento Com Tutor}

Giovanna Maria Pires De Moura, Julio Akira Tanabe (Membro),Anamaria Malachini Miotto Farah (Tutora),Matheus Afonso Pinto de Mello (Membro),Matheus Augusto Basso (Membro),Luann Felippe Lima Martins (Membro),Saulo Bonetti Buogo (Membro)

A fala é resultado da evolução humana e hoje é uma ação intrínseca e fundamental aos 
homens. Nesse mesmo contexto, observa-se que, por conta de muitas das funções exercidas pelo 
engenheiro civil terem caráter impessoal — como análise e desenvolvimento de projetos e 
processos — e requererem numerosas horas de estudo e pesquisa, o ato da fala torna-se 
frequentemente alvo de pouca atenção e, consequentemente, ínfimo desenvolvimento. Tendo em 
vista que o presente cenário pandêmico prejudicou ainda mais esse processo — mantendo, para a 
segurança de todos, a população isolada e distante fisicamente — e que o PET objetiva a melhoria 
da graduação e o desenvolvimento dos PETianos, uma das maneiras encontradas para enfrentar 
essa problemática foi a promoção de um ciclo de treinamento em oratória como atividade do 
Treinamento com Tutor — atividade interna do PET Engenharia Civil da UEM. 
A atividade realizada teve por finalidade o desenvolvimento e a aprimoração de uma série 
de habilidades cognitivas, tendo como foco a fala e acompanhada de processos como organização 
e planejamento do tema e do enredo da apresentação, material de apoio como slides e gestão de 
tempo. Ao promover esse ciclo de oratória, pretendia-se principalmente tornar mais natural ao 
aluno o processo de transmitir de maneira clara e objetiva suas ideias, além de prepará-lo para o 
mercado de trabalho dando especial ênfase a um aspecto pouco incentivado na universidade.
A fim de alcançar o objetivo proposto e solucionar a problemática a respeito de oratória 
dentro do grupo PET de Engenharia Civil da UEM, foi desenvolvido um cronograma semanal. O 
cronograma foi estabelecido por meio de votação dos PETianos, decidindo assim a frequência de 
2 apresentações semanais de 10 minutos cada, para que fosse possível avaliar também a gestão de 
tempo de cada PETiano. A ordem das apresentações foi decidida internamente.
O conteúdo das apresentações não era estritamente relacionado a Engenharia Civil, 
possibilitando um leque maior de opções para os mais diversos assuntos. Essa escolha objetivava 
o não sobrecarregamento dos PETianos — com a realização de mais pesquisas relacionadas ao 
curso —, além de proporcionar conforto ao discursar sobre um tema do qual tem domínio.
Após a realização da apresentação, foi dedicado determinado tempo para feedback dessa, 
elencando pontos como gestão de tempo, eloquência, vícios de linguagem, dicção, além de fatores 
indiretos como a elaboração da apresentação e planejamento do enredo. Ao fim do ciclo de 
oratória, também foi enviado aos PETianos um formulário, a fim de promover um feedback geral 
sobre a atividade e seus resultados.
Após a realização do feedback final, em que os dezesseis PETianos que responderam o 
formulário apresentaram sua opinião em relação à atividade como um todo, foi possível constatar 
que 62,5% dos PETianos alegaram que sua maior dificuldade foi controlar a ansiedade durante a 
apresentação — o que se deve, possivelmente, à modalidade online e à falta de experiência com 
apresentações. Sob esse mesmo viés, 18,8% acreditam que se preparar foi a maior problemática e 
os 18,8% restantes informaram outras dificuldades, como: falta de experiência e aplicar os 
conhecimentos recebidos no feedback. 
Outrossim, em relação ao conteúdo do feedback feito logo após as apresentações — em 
escala de 1 a 5 —, 68,8% dos PETianos apontaram nota 5, que as críticas e observações realizadas 
foram muito construtivas e de valiosa ajuda para o desenvolvimento pessoal. Do mesmo modo, 
6,3% responderam nota 3. Por sua vez, 18,8% expressaram nota 2, ou seja, não acreditam que as 
críticas de fato agregaram. Por unanimidade, todos os PETianos afirmaram que a atividade da 
oratória auxiliou no desenvolvimento da eloquência.
Ainda segundo as respostas do formulário de feedback, as apresentações também foram de 
grande valia ao passo que, por tratarem dos mais variados temas — como minimalismo, 
vegetarianismo, feminismo, vantagens da atividade física, higiene do sono, educação financeira, 
entre outros —, resultaram em aprendizado em áreas antes desconhecidas.
Diante dessa conjuntura, foi possível observar não só desafios — como adequar a atividade 
à modalidade online, a elaboração e planejamento das apresentações, o nervosismo, entre outros 
— com também êxitos conquistados — como a melhoria na elaboração das apresentações, 
desenvolvimento da fala e a aprendizagem em função dos muitos temas discorridos durante a 
atividade. Portanto, ainda que alguns fatores necessitem aprimoramento, conclui-se que a atividade 
de oratória do Treinamento com Tutor é uma alternativa interessante para o desenvolvimento da 
eloquência, dicção e todos os fatores direta e indiretamente relacionados a oratória.

\includepdf{pdfs/Atividade-De-Oratoria-Do-}

\addcontentsline{toc}{section}{Atividades Realizadas Pelo Pet Engenharia Florestal Nas Mídias Sociais Em Tempo De Pandemia}

\section*{Atividades Realizadas Pelo Pet Engenharia Florestal Nas Mídias Sociais Em Tempo De Pandemia}

Maiara Masiero Fianco, Camila Kreczkiuski - Aluna do curso de Engenharia Florestal,Carla Marins Santos Santana Viana - Aluna do curso de Engenharia Florestal,Vitoria Regina Pereira Betim - Aluna do curso de Engenharia Florestal,Patricia Fernandes - Professora doutora do curso de Engenharia Florestal,Dineia Tessaro - Professora doutora do curso de Engenharia Florestal

A pandemia provocada pelo Sars-Cov-2, fez com que as atividades presenciais programadas tivessem que ser adaptadas ao modelo remoto, desse modo, para manter a comunicação com demais discentes e com a comunidade, o Grupo PET iniciou em 2020 dois projetos nomeados como PET Indica e PET Compartilha - ambos através da rede social Instagram (© 2021 Instagram do Facebook); no primeiro são publicadas dicas, informações culturais e de entretenimento, já no segundo, são compartilhadas informações sobre o curso, oportunidades de emprego, etc. Outro projeto no qual o grupo PET auxilia na realização das atividades é o Podcast Caminhos da Floresta, em que são roteirizados episódios e elaboradas artes para divulgação do projeto nas redes sociais. A partir destes projetos iniciados após o início do distanciamento social, foi possível manter e ampliar a comunicação com o público, sendo, portanto, as ferramentas digitais importantes para o diálogo e acesso às informações, tanto por parte dos estudantes quanto pela comunidade em geral.

\includepdf{pdfs/Atividades-Realizadas-Pel}

\addcontentsline{toc}{section}{Atividades Virtuais Do Pet Pedagogia Uem Na Quarentena De 2020}

\section*{Atividades Virtuais Do Pet Pedagogia Uem Na Quarentena De 2020}

Emilly Fernanda Dorigan, Jenifer Fernanda Lopes da Silva,Leonardo Carbonera Girotto

No oriente, mais especificamente em Wuhan, na China, começou a se alastrar o vírus Sars-CoV2, vulgarizado como Coronavírus pelo fato de se assemelhar a uma coroa (corona em italiano). O 
vírus, registrado pela Organização Mundial da Saúde (OMS) em 31 de dezembro de 2019, 
começou a ser detectado no Brasil em março de 2020. As notícias correram o Brasil afora e a 
Universidade Estadual de Maringá (UEM) suspendeu o primeiro semestre de aula do calendário 
acadêmico de 2020. Devido a esse contexto, o Programa de Educação Tutorial (PET) da UEM 
continuou com suas atividades adaptadas ao modo home office. Visto que havia a impossibilidade 
de realização das atividades previstas a serem executadas presencialmente no primeiro semestre 
de 2020, em uma das reuniões administrativas que estavam acontecendo remotamente foi 
levantada a ideia de propor atividades voltadas para as famílias que estava em casa com seus/suas 
filhos/as, a fim de contribuir para a convivência familiar durante o isolamento social. Além disso, 
verificou-se a necessidade de manutenção das relações entre os/as acadêmicos/as e a comunidade 
externa neste momento de isolamento social. Decorrente desses fatores foi elaborada uma ação 
que teve como objetivo articular o tripé ensino, pesquisa e extensão, próprio da universidade e do 
Programa. Desse modo, o grupo elegeu temas relacionados à área educacional e dividiu-se em 
pequenas comissões, acordando que os conteúdos seriam publicados e socializados nas redes 
sociais (Instagram e Facebook) do grupo duas vezes por semana, além de serem divulgados nas 
redes sociais individuais dos próprios membros do Programa para maior divulgação, contendo 
sempre suas devidas explicações pedagógicas na descrição, sendo interativo, auxiliando e 
fornecendo aprendizados àqueles que fossem ter acesso. Para efetivação das publicações, essas 
comissões fizeram buscas acadêmicas em sites educativos e outras fontes confiáveis para serem 
desenvolvidas as atividades, com caráter cultural - como filmes e séries, brincadeiras de diferentes 
nacionalidades e épocas, músicas com gêneros diversos e outros temas considerados importantes 
pelo grupo. A atividade proporcionou enriquecimento cultural aos Petianos/as e à comunidade 
externa, além de auxiliar pais/responsáveis na nova dinâmica familiar e escolar durante o período 
inicial da quarentena. A avaliação dessa ação foi feita mediante o número de visualizações dos 
posts na rede social, os comentários feitos nas publicações e, semanalmente, em reunião com 
parecer de todos os Petianos/as, buscando relatar os pontos positivos e negativos, bem como 
sugestões de aprimoramento das publicações de acordo com as necessidades apresentadas durante 
a semana. Como resultado dessa ação, notamos o aumento da integração entre o grupo PET 
Pedagogia UEM e a comunidade acadêmica e externa à universidade, de modo a estimular o apreço 
pela cultura e auxiliar pais/responsáveis em atividades de ensino remotas durante o isolamento 
social. Além disso, a atividade propiciou uma reportagem na TV aberta regional e o aumento no 
número de seguidores nas redes sociais do PET Pedagogia. O grupo teve êxito em realizar a 
articulação do tripé ensino, pesquisa e extensão e fez com que a comunidade externa tenha contato 
com conteúdos relevantes para o desenvolvimento humano, de modo a oportunizar o lazer em 
tempos restritivos.

\includepdf{pdfs/Atividades-Virtuais-Do-Pe}

\addcontentsline{toc}{section}{Auditório Teixeirão}

\section*{Auditório Teixeirão}

Tamires Dos Santos, Daniel dos Anjos Tavares-PET/ECV,Giorgia Luccheta Pezzi-PET/ECV,Giulia Pimentel Cia Koike-PET/ECV,Juliana Ribas Nantes-PET/ECV,Samara Tiemi Nakashima Kobori-PET/ECV,Cláudio Cesar Zimmermann(Orientador)-PET/ECV

O auditório Luiz Antônio Teixeira (Teixeirão), alocado atualmente no prédio da
Engenharia Elétrica da UFSC, tem por finalidade servir de apoio para atividades acadêmicas dos
diversos cursos do Centro Tecnológico (CTC) em que há necessidade de um ambiente amplo e
confortável, com recursos de mídia digital como datashow e áudio.
A partir de uma demanda que contemplava a reforma do ambiente, vinda por parte da
direção do CTC em conjunto com o Departamento de Engenharia Elétrica (EEL) para oferecer
fácil acesso e ensino de qualidade, dentro das normas, o grupo PET/ECV/UFSC foi convidado
pela Engª Fernanda Scheidt, responsável por elaborar o projeto de reforma do auditório, e está
neste processo desde março de 2021.
O projeto de reestruturação contempla a reformulação arquitetônica e estética do
Auditório, consistindo em uma readequação de materiais e de acabamento que visam a melhoria
da parte acústica, da iluminação, da sonorização, bem como da proteção contra incêndio e
também da acessibilidade. Sendo que apenas a parte composta pelo projeto arquitetônico não foi
realizada pelos petianos, mas sim pela arquiteta Aline Monique Bortolini.
Primeiramente foram analisadas as demandas do projeto, sendo feito um estudo acerca do
caso, com buscas por referências de projetos, discussões e reuniões em ambientes de
comunicação virtual, para entender melhor a situação e, juntos, resolver as demandas do projeto.
A partir disso, o grupo concentrou, primeiramente, as atividades nos projetos luminotécnico e
PPCI (Projeto Preventivo contra Incêndio), pois este era independente, e aquele era quase um
pré-requisito para o projeto elétrico, vide seus dados, muito necessários para dimensionar
eletrodutos e potência de pontos de luz. Sempre com o auxílio de engenheiros experientes nas
áreas estudadas para então efetuar os projetos e constantemente verificando se está de acordo
com as normas NBR de cada projeto. Por fim, para a realização dos projetos foram utilizados
softwares como o AutoCAD®, Discord® e o Google Drive®.
Além do conhecimento constituído devido a toda busca de informações e à concepção
dos projetos, é possível associar alguns resultados potencialmente obtidos, como a melhoria para
os graduandos de ter uma sala totalmente voltada para o ensino profissionalizante, sem
distrações devido ao planejamento inadequado; conforto e confiança no uso dos instrumentos
disponibilizados no auditório, tendo em vista o correto dimensionamento do projeto elétrico, que
proporcionará tranquilidade para demandar uma larga quantidade de energia de forma simultânea
enquanto se compartilha o conhecimento; conforto, também, na estadia de quem vai às aulas, às
palestras e às apresentações em geral, em virtude da iluminância do espaço e das poltronas;
acessibilidade para todos, dando direito a todas as pessoas estudarem de maneira equitativa e
sem adversidades de acomodação e deslocamento; segurança, dado que, em caso de emergência,
há rotas de fuga sinalizadas e equipamentos para tentar gerir eventual situação de pânico, de
acordo com as normas técnicas vigentes.
Dessa forma, com um ambiente bem projetado, a Universidade Federal possa
proporcionar um ensino de qualidade e para todos visto que é um ambiente seguro em caso de
incêndios, bem planejado para não ocorrer choques elétricos e perda de energia com recorrência,
além de proporcionar uma experiência luminosa para melhor entendimento. Neste sentido, um
auditório de uma universidade federal precisa ser renovado de acordo com as mudanças de
normas, com o surgimento de novas tecnologias e demandas. Entretanto, vale ressaltar que
apenas o espaço físico adequado para todos não consegue contemplar de forma abrangente as
desigualdades de cunho histórico e que ainda há meios a serem revistos para o acesso
universalizado aos recursos. Por fim, no âmbito de capacitação dos integrantes do
PET/ECV/UFSC, este projeto traz a possibilidade de atuação dos mesmos em um caso de
demanda real, com particularidades e presença de interdisciplinaridade. Gerando assim, um
importante aprendizado na formação profissional dos integrantes.
ABNT. NBR 5410: Instalações elétricas de baixa tensão. Rio de Janeiro, 2004.
ABNT. NBR 5382: Verificação de iluminância de interiores. Rio de Janeiro, 1985.
ABNT. NBR ISO/CIE 8995-1: Iluminação de ambientes de trabalho Parte 1: Interior. Rio de
Janeiro, 2013.
ABNT. NBR 17240: Sistemas de detecção e alarme de incêndio – Projeto, instalação,
comissionamento e manutenção de sistemas de detecção e alarme de incêndio – Requisitos.
Rio de Janeiro, 2010.
Instrução Normativa - IN 1 - Parte 1: Processos Gerais de Segurança Contra Incêndio e
Pânico. Santa Catarina, nota técnica 62/2021 alterada.
Instrução Normativa - IN 1 - Parte 2: Sistemas e Medidas de Segurança Contra Incêndio e
Pânico, respectivamente. Santa Catarina, nota técnica 62/2021 alterada.
Instrução Normativa - IN 3: Carga de Incêndio. Santa Catarina, publicada em 17/12/2019.
Vigente a partir de 17/02/2020.
Instrução Normativa - IN 6: Sistema Preventivo por Extintores. Santa Catarina, nota técnica
50/2020 alterada.
Instrução Normativa - IN 9: Sistema de Alarme e Detecção de Incêndio. Santa Catarina, nota
técnica 60/2020 alterada.
Instrução Normativa - IN 11: Sistema de Iluminação de Emergência. Santa Catarina, nota
técnica 34/2018 alterada.
Instrução Normativa - IN 12: Sistema de Alarme e Detecção de Incêndio. Santa Catarina, nota
técnica 61/2021 alterada.
Instrução Normativa - IN 18: Controle de Materiais de Revestimento e Acabamento. Santa
Catarina, nota circular 03/DSCI/2019 alterada.
Instrução Normativa - IN 28: Brigada de Incêndio. Santa Catarina, nota técnica 17/2016
alterada.
Instrução Normativa - IN 31: Plano de Emergência. Santa Catarina, editada em 28/03/2014

\includepdf{pdfs/Auditorio-Teixeirao}

\addcontentsline{toc}{section}{Avaliação De Métricas De Aceitação Do Curso On-Line “Boas Práticas De Fabricação E Adaptações Durante A Pandemia”}

\section*{Avaliação De Métricas De Aceitação Do Curso On-Line “Boas Práticas De Fabricação E Adaptações Durante A Pandemia”}

Alessandro Bruno Machado Da Silva, FELIPE DA COSTA ROLIM,RAFAEL GUIMARÃES GARCIA,TAIRINE DA ROSA RIBEIRO,GUILHERME SANTOS MARTINS,PROGRAMA DE EDUCAÇÃO TUTORIAL DA ENGENHARIA DE ALIMENTOS DA UNIVERSIDADE FEDERAL DO RIO GRANDE (PET-EA/FURG)

A síndrome respiratória aguda grave coronavírus 2 (SARS-CoV-2) surgiu como um vírus 
zoonótico no final de 2019 e é o agente causador do COVID-19. Bloqueios de emergência foram 
iniciados em países em todo o mundo. Sendo o efeito sobre a saúde, o bem-estar, os negócios e 
outros aspectos da vida diária sentidos em todas as sociedades e nos indivíduos. Sem intervenções 
farmacológicas eficazes ou vacinas disponíveis no futuro iminente, reduzir a taxa de infecção se
tornou uma prioridade, e a prevenção é a melhor forma de atingir esse objetivo. (CHU, D. K. et 
al). Em razão do cenário de incertezas, e com o amparo dos órgãos regulatórios, os serviços de 
alimentação deram continuidade as suas atividades enfrentando o desafio para garantir a segurança 
e a saúde do consumidor.
As Boas Práticas de Fabricação (BPFs) são caracterizadas como um conjunto de medidas e práticas 
que devem ser adotadas pelas indústrias de alimentos e pelos prestadores de serviços de 
alimentação, de forma a garantir a qualidade sanitária e a conformidade dos alimentos com os 
regulamentos técnicos. Sendo que deve ser aplicada durante a produção, o manuseio e 
armazenamento dos alimentos e produtos. A lavagem das mãos é fundamental para evitar doenças, 
tais como gripe, diarreia, infecção estomacal, conjuntivite e dor de garganta. Apesar de ser um ato 
extremamente simples muitas pessoas, independente do grau de escolaridade ou classe social, não 
lavam as mãos habitualmente. (BRASIL, 2004)
Essas práticas diminuem o risco das doenças transmitidas pelos alimentos, pois focam na higiene 
e na qualidade em toda a cadeia produtiva. Assim, o fortalecimento das boas práticas pode auxiliar 
para reduzir a transmissão direta do COVID-19 no ambiente de produção (BRASIL, 2020).
Neste contexto, o Grupo PET Engenharia de Alimentos FURG desenvolveu um curso online, que 
foi hospedado na plataforma Coursify.me. e intitulado “Boas Práticas de Fabricação e adaptações 
durante a pandemia”. Esta prática foi adotada como alternativa de adequação de uma de suas 
atividades de extensão. O curso foi elaborado tendo por base as normativas da RDC Nº 216 
(BRASIL, 2004) e a RDC Nº 275 (BRASIL, 2002), que regulamentam as BPFs. estando de acordo 
também com a NT 48/2020 (BRASIL, 2020).
Tendo em vista a adequação adotada, o objetivo do trabalho foi avaliar a contribuição e o impacto 
do curso através de questionários de avaliação que foram aplicados a todos os participantes que 
finalizaram o curso, condição necessária para emissão do certificado.
O questionário foi elaborado contendo seis questões, e as cinco primeiras foram constituídas por 
alternativas que representam uma escala (5 = Ótimo, 4 = Bom, 3 = Regular, 2 = Ruim e 1 = 
Péssimo). As perguntas foram: 1. “Na sua opinião, o curso atendeu suas expectativas?”; 2. “Como 
você avalia seu aprendizado?”; 3. “Como você avalia a didática das ministrantes?”; 4. “Como você 
avalia a plataforma utilizada?”; 5. “Você indicaria esse curso para seus amigos?”; 6. “Atribua uma 
nota geral (0 a 10) ao curso”.
O curso teve 305 matrículas, das quais 170 iniciaram, porém não concluíram o curso, 102 pessoas
concluíram e responderam ao questionário de avaliação para obter o certificado. Os demais 
inscritos (33), não iniciaram curso. Os resultados mostram que 72,6% dos concluintes 
responderam que o curso atendeu as suas expectativas e que seu aprendizado foi satisfatório 
(72,5%). Já a didática dos ministrantes, obteve 56,9% de aprovação. A plataforma onde foi 
hospedado o curso obteve 100% de aprovação. Com relação a indicação do curso para seus amigos, 
93,1% dos respondentes assinalaram de maneira positiva a indicação, e a aprovação geral foi de 
92,2%. A troca de modalidade do curso foi uma experiência nova, e uma alternativa bem-sucedida
em relação ao momento pandêmico, porém alguns pontos merecem atenção para a sua melhoria.
Sendo assim, a readequação do projeto para a forma remota tem contribuído para a conscientização 
dos manipuladores e colaboradores dos serviços de alimentação no fortalecimento das BPFs e 
principalmente na redução da transmissão ao novo coronavírus (SARS-CoV-2).
Referências
BRASIL. Agência Nacional de Vigilância Sanitária.Nota Técnica nº48/2020/SEI/GIALI/GGFIS/
DIRE4/ANVISA. Documento orientativo para produção segura de alimentos durante a pandemia 
de Covid-19. Disponível em: 
. Acesso em 
15 de setembro de 2021.
BRASIL. Agência Nacional de Vigilância Sanitária (Anvisa). Resolução RDC n° 216, de 15
de setembro de 2004. Dispõe sobre Regulamento Técnico de Boas Práticas para Serviços de
Alimentação. Diário Oficial da União, Brasília, DF, 15 de setembro de 2004.
BRASIL. Ministério da Saúde. Agência Nacional de Vigilância Sanitária. Resolução –
RDC n° 275, de 21 de outubro de 2002. Dispõe sobre o Regulamento Técnico de
Procedimentos Padronizados Aplicados aos Estabelecimentos
Produtores/Industrializadores de Alimentos. Diário Oficial da União, Brasília, DF, 22 de
outubro de 2002. Republicada no D.O.U de 06/11/2002.
CHU, D. K. et al. Physical distancing, face masks, and eye protection to prevent person-to-person 
transmission of SARS-CoV-2 and COVID-19: a systematic review and meta-analysis. The Lancet, 
v. 395, p. 1973-1987, 2020. 

\includepdf{pdfs/Avaliacao-De-Metricas-De-}

\addcontentsline{toc}{section}{Biogás: A Próxima Fonte De Energia Elétrica}

\section*{Biogás: A Próxima Fonte De Energia Elétrica}

Thaina Delgado Bezerra, Paula Schneid Alves,Jamile Alves Leal,Thalia Delgado Bezerra

O Programa de Educação Tutorial - PET, do Ministério da Educação do Brasil – MEC, visa a melhoria do curso de graduação através de projetos de ensino, pesquisa e extensão. Dentre os projetos desenvolvidos pelo PET do curso de Engenharia Química (PET/EQ) da FURG, a comissão de maquetes busca explicativar, através de recursos visuais e educativos, o funcionamento dos equipamentos ou processos ligados a Engenharia Química. A comissão estudou sobre o processo de geração de energia elétrica, optando pela biodigestão anaeróbia como o processo de conversão da biomassa. A motivação deste artigo é inspirar e elaborar desenhos de um biodigestor, com o auxílio de “softwares” de projeções, utilizando materiais acessíveis e baratos, e visando analisar as vantagens de utilizar essa técnica para a geração de energia elétrica e de biogás produzido a partir da biodigestão anaeróbia.


\includepdf{pdfs/Biogas--A-Proxima-Fonte-D}

\addcontentsline{toc}{section}{Brincando E Aprendendo: Apresentando Conceitos De Eletrônica Básica}

\section*{Brincando E Aprendendo: Apresentando Conceitos De Eletrônica Básica}

Giovana Viegas Barros, Marília Abrahão Amaral

Este trabalho propõe um material para dar suporte à atividades de extensão para apresentação de conceitos introdutórios de eletrônica e robótica para crianças, por meio da construção de um artefato que se baseia em conceitos de atividade lúdica e imaginação na educação. É apresentado o material com o passo a passo da criação do artefato, para ser utilizado pelas crianças, bem como os materiais necessários e fundamentação teórica do trabalho.

\includepdf{pdfs/Brincando-E-Aprendendo--A}

\addcontentsline{toc}{section}{Ciclo De Palestras: Proporcionando Um Novo Contato Com A Formação Profissional}

\section*{Ciclo De Palestras: Proporcionando Um Novo Contato Com A Formação Profissional}

Paulo Vitor De Lima Carvalho, Enzo Sennhauser,Ezequias David,Ludmylla Weber Kienen Muller Simon,Luís Felipe Bavati Medri,Naiury da Silva Marcondes

O vasto leque de possibilidades de atuação que a Engenharia Química proporciona, somado 
com a falta de perspectiva da aplicação dos conhecimentos básicos das disciplinas iniciais do curso 
são os principais motivos que levam aos altos índices de reprovação e evasão escolar (PEREIRA 
et al., 2006).
De acordo com Mendes, Rodrigues e Duarte (2014) citam a interpretação errada da 
verdadeira área de atuação da Engenharia Química, uma vez que “muitos esperam pelo nome do 
curso uma grande habilidade na química, o que de fato não procede”.
Tendo em vista a problemática descrita acima, o grupo PET Engenharia Química da 
Universidade Federal do Paraná (UFPR) realiza anualmente desde 2009 um evento denominado 
Ciclo de Palestras. Apesar de originalmente apresentar como cerne a promoção de palestras para 
aproximar o corpo discente da realidade da atuação da/o engenheira/o química/o, o Ciclo no 
decorrer das 11 edições incorporou novos elementos que ampliaram o escopo da engenharia ou da 
química, de modo que as/os graduandas/os tenham contato com temas diversificados e não apenas 
voltados ao mercado de trabalho.
Assim, busca-se promover a integração do grupo com a comunidade acadêmica e externa 
ao disponibilizar às/aos alunas/os de graduação conhecimentos sobre assuntos extracurriculares, 
valorizando a formação. Pretende-se complementar a Proposta Pedagógica do curso, com a 
apresentação de palestrantes convidados, que sejam docentes ou profissionais atuantes com o 
intuito de abordar temas complementares da graduação, tais como gestão de negócios, 
oportunidades de trabalho e atuação profissional, e outros assuntos de interesse do curso.
Para a realização do evento, inicialmente é definida a semana em que ocorrerá a aplicação 
e são selecionados temas de acordo com o interesse do público discente, que é levantado a partir 
das sugestões e comentários realizados nas edições anteriores do projeto. Com base nisso, são 
contatadas/os possíveis palestrantes com domínio sobre as temáticas a serem abordadas.
Após a seleção das/os convidadas/os em função do seu interesse e disponibilidade, 
usualmente o grupo comunica a necessidade de reservar o auditório à coordenação do curso. No 
entanto, dada a impossibilidade de realizar a aplicação presencial do projeto em 2020, o grupo 
realizou uma pesquisa sobre as plataformas de videoconferência que melhor se encaixariam às 
necessidades do evento. Para tanto, foram selecionados o Microsoft Teams e o Google Meet.
Dessa maneira, foram criados links para o acesso às reuniões, que foram disponibilizados 
com antecedência às/aos palestrantes e à comunidade acadêmica por meio das redes sociais do 
grupo. Paralelamente, foi realizada a divulgação dos temas que seriam abordados e das/os 
palestrantes convidadas/os, tendo como base a identidade visual elaborada para a edição.
Após o segundo dia do evento, atendendo ao que foi requisitado pelas/os discentes, o grupo 
solicitou o consenso de cada convidada/o para que as reuniões passassem a ser gravadas. Essas 
filmagens foram posteriormente disponibilizadas no canal do YouTube do PET EQ UFPR.
Em 2020, em sua décima primeira edição, trouxe um total de 6 palestras, todas sendo 
ofertadas por funcionárias/os das empresas: Ambev, Ajinomoto, Bayer, Braskem, Evonik e por 
um discente que estagiou na Polícia Científica. Os temas abordados por cada palestra foram:
Possibilidades de Carreiras e vivências; Indústria 4.0; Diversidade e Inclusão na prática; 
Atribuições do Engenheiro Químico; Tecnologia do processo de produção de Rações e Ciências 
Forenses e a Polícia Científica do Paraná, respectivamente.
Somadas todas as palestras tiveram em média 46 participantes, já a palestra com maior 
número de pessoas teve 61 no total; de acordo com o feedback enviado pela maioria dessas 
pessoas, a organização do PET EQ UFPR ficou avaliada, majoritariamente, entre excelente ou boa. 
Os conteúdos abordados foram caracterizados com relevância percentual entre 68% e 96% 
para Muito Grande, ainda nenhum deles foi avaliado com pouca ou nenhuma relevância.
Na avaliação da atividade foi feito o formulário de feedback no qual continha, também, o 
link para outro formulário onde era contabilizada a presença, o que garantia que a maior parte 
das/dos presentes na palestra tivessem que responder para garantir seus certificados. 
Nas perguntas foi feito um levantamento sobre, além de qualidade das palestras, 
ministrantes e organização, comentários sobre o evento, assim como sugestões de temas para 
eventos futuros. 
A análise dos dados obtidos após cada evento mostra a importância do projeto para com a 
graduação. Isso é demonstrado pela alta adesão de discentes nas palestras, que possuem interesse 
pelos temas propostos, em muitos casos, não abordados em sala de aula, sendo notável o interesse 
em temas de cunho profissional e demonstrando a eficiência da divulgação realizada pelo grupo 
PET.
A presença de discentes em repetidas palestras de um mesmo evento indica uma boa 
organização do grupo. Além disso, o bom desempenho das/dos palestrantes e conteúdo abordado 
satisfatório também contribuem para o evento, garantindo mais presenças e demonstrando que há 
interesse na realização de novas edições do Ciclo de Palestras.
Dessa forma, o Ciclo de Palestras é de grande impacto a quem participa, de modo a 
conhecer mais o próprio curso em aspectos não abordados em geral.
Referências:
MENDES, D.; RODRIGUES, S. A.; DUARTE, E. R. Projeto Forma Engenharia: vivenciando 
engenharia química. Conexão UEPG, Ponta Grossa, v. 10, n. 1, p. 150-161, 2014.
PEREIRA, M. C.; FERREIRA, W. M.; BATISTA, E. A.; SCAMPINI JR., E.; FALCO, J. R. 
Evitando evasão em cursos de engenharia: um estudo de caso. In: CONGRESSO BRASILEIRO 
DE ENSINO DE ENGENHARIA, 34. Resumos... Passo Fundo: Ed. Universidade de Passo 
Fundo, 2006. p. 1726-1732.

\includepdf{pdfs/Ciclo-De-Palestras--Propo}

\addcontentsline{toc}{section}{Compartilhando Saberes No I Ciclo De Debates Socioambientais Promovido Pelo Grupo Pet Conexões - Gestão Ambiental}

\section*{Compartilhando Saberes No I Ciclo De Debates Socioambientais Promovido Pelo Grupo Pet Conexões - Gestão Ambiental}

Barbara Pereira Vidal, Vicente Behnck Lucena Soares - PET Conexões Gestão Ambiental IFRS,Dyowanne Hiulei Schmitt - PET Conexões Gestão Ambiental - IFRS,Patrícia Silveira de Barros PET Conexões Gestão Ambiental - IFRS,Lucas Alexandre Ferrari PET Conexões Gestão Ambiental - IFRS,Moacir Vargas Gaspar PET Conexões Gestão Ambiental - IFRS,Ana Maria de Jesus Cardozo PET Conexões Gestão Ambiental - IFRS,Prof. Dr. Celson Roberto Canto Silva - Orientador

O I Ciclo de Debates Socioambientais foi um evento de extensão promovido pelo
grupo do PET Conexões - Gestão Ambiental do Instituto Federal de Educação, Ciência e
Tecnologia do Rio Grande do Sul - Campus Porto Alegre. Buscando alternativas para promover
ações de forma remota devido ao contexto da pandemia, o grupo realizou o evento on-line de
forma segura e de fácil acesso trazendo três encontros com temáticas diferentes, mas que se
convergem no tema principal, que foi “Cidade e Meio Ambiente”. Para a realização do evento de
forma remota, a metodologia utilizada contou com a elaboração de um cronograma, contato com
palestrantes especialistas em cada temática proposta, que apresentaram as suas contribuições
durante a transmissão pelo Youtube, o que permitiu também a participação dos ouvintes através
do chat. Também foi organizada uma divulgação massiva nas mídias sociais do grupo.
Projetos Ambientais, Perspectivas e Desafios foi o primeiro tema que deu início ao ciclo de
debates. Trazendo como convidados Dilton de Castro, que falou sobre “Projetos socioambientais,
a importância das parcerias interinstitucionais” e Júlia Ilha, que discutiu sobre o “Programa
Macacos Urbanos e os desafios da conservação nas cidades” estes compartilharam com o público
do evento suas experiências no desenvolvimento de projetos socioambientais. A participação do
público com questionamentos e reflexões revelou a necessidade de espaços como estes na
divulgação e conscientização da importância dos projetos ambientais. Além disso, também
mostrou como se dá o desenvolvimento e aplicação destes projetos, bem como a perseverança e
criatividade dos realizadores destes projetos na superação de desafios.
Já no tema Pandemia e Resíduos Sólidos a discussão, segundo a ser abordado no Ciclo,
este se concentrou na conscientização sobre a natureza de cada tipo de resíduo gerado, como é
realizada esta classificação, o destino correto e as consequências de quando não é realizado o
tratamento correto destes, bem como as consequências da pandemia de COVID-19 na gestão e
gerenciamento dos resíduos hospitalares e domésticos. Este segundo dia de evento contou com a
presença da enfermeira Valquiria Martins, responsável pela gestão de riscos hospitalares no
Hospital Nossa Senhora da Conceição, e da Ana Paula Medeiros, representante da cooperativa de
catadores de resíduos do bairro Bom Jesus, em Porto Alegre. As convidadas enfatizaram a
importância de haver uma conscientização sobre a questão, descrevendo de forma muito didática
toda problemática envolvida no tema e demonstrando a importância da responsabilidade
compartilhada para gestão dos resíduos, onde os geradores, o poder público e a população são
corresponsáveis pelas diferentes etapas de gestão.
O último debate do Ciclo contou com a temática Cultura e Meio Ambiente, que de uma
maneira geral buscou abordar as interações existentes entre os dois temas. Numa perspectiva
sobre os monumentos culturais e em qual ambiente eles se encontram, tivemos a apresentação do
Museu de Percurso do Negro em Porto Alegre:”A arte no combate ao racismo ambiental”, em
que a convidada Dra. Prof. Aline Ferraz da Silva apresentou o roteiro do museu no Centro
Histórico de Porto Alegre e em outras partes da cidade. Este percurso resgata a memória do
protagonismo cultural e social dos africanos e seus descendentes, bem como a escravidão ao qual
foram submetidos. As presenças dos convidados Adriano Peixoto (Adriano Dplay) e Cássio de
Abreu propiciaram a socialização do Projeto Geloteca, que integra a arte do grafite à literatura,
cultura e sustentabilidade, utilizando-se de geladeiras que seriam descartadas, transformando-as
em bibliotecas, visando promover a consciência ecológica juntamente com a literatura,
permitindo que a comunidade tenha acesso a cultura.
A partir das discussões realizadas nos debates, concluiu-se que o evento foi um
aprendizado para todo o grupo e para a comunidade participante. Diante dos desafios que foram
impostos devido a natureza de um evento on-line, a receptividade do público evidenciou que a
troca de saberes e experiências de todos os envolvidos foi muito importante nesses tempos de
pandemia .

\includepdf{pdfs/Compartilhando-Saberes-No}

\addcontentsline{toc}{section}{Compilado De Dinâmicas Não-Odontológicas Do Grupo Pet: Roda De Conversa, Fhc E Setembro Amarelo}

\section*{Compilado De Dinâmicas Não-Odontológicas Do Grupo Pet: Roda De Conversa, Fhc E Setembro Amarelo}

Anna Julia Santiago Campanelli, Nicole Catherine Goltz Fokkema (UEM),Kemilly Soares de Castro (UEM),Maria Eduarda Fernandes (UEM),Flávia Akemi Nakayama Henschel (UEM),Gabriela Steckel Neme (UEM),Carlos Alberto Herrero de Morais (UEM)

A interdisciplinaridade é um fator de extrema importância quando se trata da formação acadêmica. A 
busca pela reorganização curricular dos graduandos se torna cada vez mais recorrente, a fim de levar 
à educação os parâmetros da integração interdisciplinar. Tendo esse enfoque, nota-se que a construção 
da estrutura universitária, que é fornecida aos discentes, precisa ser completa, ou seja, com uma 
concretização da totalidade curricular, e não uma apreciação isolada de cada conteúdo. Esse ponto de 
vista também é ressaltado quando se trata da necessidade de trazer pautas sociais para a rotina 
acadêmica, indo além daquilo que é preconizado por cada curso e expandindo a visão de seus 
integrantes, com o intuito de torná-los sujeitos críticos, reflexivos, autônomos, criativos, capazes de 
tomarem decisões e atuarem na sociedade. Desse modo, o objetivo deste trabalho é apresentar três 
diferentes dinâmicas realizadas pelo grupo PET-Odontologia da Universidade Estadual de Maringá 
(UEM) que têm a finalidade de integrar seus participantes a uma realidade excepcional à Odontologia, 
sendo elas: “Roda de conversa com os professores e/ou servidores técnicos do Departamento de 
Odontologia da UEM”, “FHC - Formação Humanística Cultural” e “Setembro Amarelo”.
A atividade “Roda de conversa com os professores e/ou servidores técnicos do Departamento de 
Odontologia da UEM” ocorre uma vez a cada semestre, sendo, normalmente, às quartas-feiras, e tem 
como objetivo compartilhar experiências de um determinado docente e/ou servidor técnico da 
Universidade para com os petianos e tutor. Anteriormente à dinâmica, os petianos responsáveis por 
ela entram em contato com o convidado a fim de realizar um convite para uma conversa e 
compartilhamento de experiências com o grupo. O exercício se dá pela forma como o(a) convidado(a) 
demandar, seja apenas através do bate-papo ou com a realização de alguma prática específica. Ao 
final, o grupo se reúne para discussão e conclui com agradecimentos e considerações.
A atividade “FHC – Formação Humanística Cultural” tem caráter livre e busca estimular a 
criatividade do grupo, somando, assim, conhecimentos diversos a respeito de fatores que não se 
restringem somente à graduação. Por ser uma dinâmica livre, proporciona aos petianos conduzi-la da 
forma como desejarem, podendo fazer uso de slides, fotos, objetos, conversas ou outras ideias, além 
de contribuir significativamente na formação holística e cultural do grupo. O \"FHC\'\' deve ser realizado 
em quatro datas diferentes ao longo do ano, normalmente às quartas-feiras, e, a cada vez, é organizado 
por uma turma diferente, sendo responsabilidade do 3º, 4º e 5º ano e também dos “aspiras” (candidatos 
do processo seletivo do PET Odontologia/UEM). Na reunião seguinte à da atividade, o grupo realiza 
uma avaliação a respeito do tema tratado e da decorrência do exercício.
A atividade “Setembro Amarelo” foi adquirida recentemente e é organizada anualmente pelo grupo 
PET Odontologia-UEM em conjunto com o Centro Acadêmico de Odontologia da Universidade 
Estadual de Maringá. A dinâmica surgiu no ano de 2020 com base na campanha “Setembro Amarelo 
– Prevenção ao Suicídio” e objetivando apoio aos discentes e docentes da Universidade. Trata-se de 
um exercício leve e empático, em forma de bate-papo, onde os participantes dividem experiências 
compartilhadas e individuais de suas rotinas, colocando em prática o emblema que a atividade leva: 
“falar é a melhor opção”. Dessa forma, objetiva-se trazer visibilidade e conscientização para a 
prevenção de suicídios e fomentar o vínculo entre discentes e docentes da Universidade. A atividade 
pode ser conduzida por um profissional da área da psicologia ou psiquiatria, a fim de ser, 
simultaneamente, produtiva e ponderada. Ademais, por se tratar de um momento tranquilizante, os 
organizadores optaram por não realizarem a emissão de certificados aos participantes.
Posto isto, a efetividade das três dinâmicas citadas acima se torna significativa ao proporcionar aos 
participantes do grupo PET - Odontologia UEM uma maior integridade com os docentes/servidores 
técnicos da Universidade e demais discentes, através das \"Rodas de Conversas\'\'. Além disso, auxiliam 
em um desenvolvimento considerável de criatividade nas organizações e participações do \"FHC\", 
agregando ricamente aos conhecimentos culturais do grupo. Por fim, também acarretam em uma 
personalidade empática e solidária aos seus participantes por meio do \"Setembro Amarelo\", em adição 
à consequente união frente à campanha de prevenção ao suicídio.
Desse modo, conclui-se que o grupo PET - Odontologia UEM prioriza levar aos seus integrantes a 
conscientização a respeito da realidade como um todo, englobando mais do que assuntos relacionados 
à Odontologia, mas, também, problemas sociais reais, através de atividades que envolvem discussão, 
aprendizado, acolhimento e, principalmente, união.


\includepdf{pdfs/Compilado-De-Dinamicas-Na}

\addcontentsline{toc}{section}{Curso De Formação Para Novos Petianos: Uma Ação Do Pet Litoral  Social}

\section*{Curso De Formação Para Novos Petianos: Uma Ação Do Pet Litoral  Social}

Barbara Abila Napoleao, Layliene Kawane de Souza,Wellyngton Fernando Leonel de  Souza,Caroline dos Santos Mesquita,Mayra Taiza Sulzbach

Diante da pandemia de Covid-19, em 2020 e 2021, o Programa de Educação Tutorial Litoral 
Social (PET LS) da Universidade Federal do Paraná (UFPR) vivenciou a dificuldade no 
acolhimento e preparação dos bolsistas e voluntários aprovados nos processos seletivos. Assim, 
foi pensado e planejado uma nova ação para auxiliar os novos ingressantes do Programa a se 
adaptarem com as atividades e rotina do Grupo. A ação se justifica pela rotatividade de 
integrantes no Programa, já que todos que adentram vão atuar no planejamento realizado no ano 
anterior. Neste ínterim, o presente resumo visa comunicar a atividade denominada “Curso de 
Formação para os Novos Petianos”, que tem o intuito de familiarizar os novos integrantes ao 
Programa nacional, bem como às ações que compõem o planejamento do Grupo e sua rotina. 
Para tanto, na fase final do processo seletivo os integrantes do Grupo se reúnem em subgrupos
focais para definição dos conteúdos do Curso, assim como para a elaboração de materiais de 
apoio. As etapas consistem em: i) elaboração do cronograma com as datas/horários das 
atividades que farão parte do curso; ii) divisão dos subgrupos responsáveis para apresentação de 
cada atividade; iii) elaboração de materiais: apresentação de imagens, revisão da cartilha de 
normativas e endereços eletrônicos para auxílio dos integrantes; e iv) escolha de textos,
documentários chaves e ferramentas utilizadas pelo Grupo a serem socializadas com os novos 
integrantes. Ao longo do Curso, para além da apresentação das atividades, aberta ao debate a 
partir dos materiais de base, são solicitadas algumas tarefas para os novos integrantes, como 
elaboração de infográfico, mapa mental, formulário e apresentação de seminário, a fim de que 
possam ter contato com ferramentas utilizadas com frequência pelo Grupo, como é o caso do 
Canva e Google Forms. Um momento especial do Curso são os ciclos de debates, quando
egressos do PET LS são convidados para apresentar o quanto o Programa agregou durante sua 
trajetória acadêmica, e vem contribuindo para a vida profissional e pessoal. O Curso também 
conta com um momento de avaliação, que acontece em dois períodos distintos, ao final: i) 
avaliação dos pontos positivos, negativos e ameaças, utilizando a ferramenta Strengths 
Weaknesses Opportunities Threats (SWOT) pelos subgrupos que o organizaram; e ii) avaliação 
pelos ingressantes no Grupo, por meio de um questionário através do Google Forms, dividido em 
três seções, sendo que na primeira os ingressantes elegem/apresentam seu perfil. Na segunda, a
avaliação do Curso nos quesitos: cronograma, materiais base e atividades solicitadas; e na
terceira, a avaliação é relativa à integração no Grupo. Em 2020 o PET LS realizou um processo 
seletivo virtual, com o ingresso de seis integrantes (quatro bolsistas e dois voluntários), dos quais 
todos realizaram o Curso. No processo avaliativo cinco responderam o Formulário de Avaliação 
uma vez que o mesmo não era obrigatório. Em relação à avaliação, 100% consideraram que o 
Curso foi realizado de forma dialogada, destacando o contanto com petianos de outras áreas de 
formação superior, dado que o PET LS é interdisciplinar. “A formação cidadã” foi eleita pelos 
cinco ingressantes como positiva por contribuir na melhoria da qualidade de vida da comunidade 
local. A respeito do desenvolvimento pessoal, 100% das respostas apontam que o curso auxiliou 
no trabalho em grupo e no desenvolvimento da responsabilidade coletiva, tópicos importantes 
para o funcionamento do PET LS. Na avaliação da integração com os já petianos, elencaram que 
esses não mediram esforços em apresentar as atividades planejadas e desenvolvidas pelo/e das 
ferramentas/plataformas digitais (Canva, Google Forms). Com o resultado da avaliação dos 
subgrupos, o Curso atingiu o seu objetivo, que era o de auxiliar no processo de adaptação dos
integrantes do Grupo. Diante da experiência da nova forma de realizar as atividades, essa foi 
avaliada como apta a novas ofertas, especialmente porque por meio dessa atividade ocorre a 
troca e interação de conhecimentos entre os integrantes do Grupo, além de ser um instrumento 
incentivo e motivação a realização do trabalho em grupo, considerando que a participação de 
todos nas diferentes atividades é fundamental.

\includepdf{pdfs/Curso-De-Formacao-Para-No}

\addcontentsline{toc}{section}{Caminhos Formativos - Juventude, Políticas? ?Públicas? ?E? ?Educação?}

\section*{Caminhos Formativos - Juventude, Políticas? ?Públicas? ?E? ?Educação?}

Brenda Barros Dias, Mariana Freitas Pinto - UFRGS,Richer Rodrigues - UFRGS,Frederico Machado - UFRGS

O acesso e permanência no ensino superior tornou-se uma pauta importante, em que várias
estratégias foram implementadas para favorecer esses indicadores. No entanto, o contato com
estudantes de ensino fundamental e médio aponta que muitas pessoas ainda não conhecem as
possibilidades existentes para a concretização deste direito. Alguns relatos que chegaram até
membros do nosso grupo mostram que alguns estudantes de ensino fundamental e médio não
sabiam sequer que as universidades públicas eram gratuitas. Isto nos alertou para a carência de
informação dos jovens sobre os caminhos e as formas de acessar o direito à educação, portanto,
temos como objetivo diminuir esta problemática. A distância entre o ensino secundário e o de
nível superior é um desafio a ser superado e a criação de pontos de contato e intercomunicação
que funcionem longitudinalmente podem contribuir tanto para a disseminação de informações
sobre a universidade, como para a construção de um ambiente universitário mais adequado para
receber novos alunos. A elaboração da cartilha, visa a divulgação de informações a respeito de
direito à educação como as políticas de acesso e permanência estudantil, além disso, informar
sobre: gratuidade do transporte interestadual para jovens (ID Jovem); direito a saúde pública de
qualidade; direito a moradia; Projovem, entre outros, fomentar reflexões sobre o direito à
educação de qualidade e seu acesso e permanência principalmente nas Universidade Públicas e
Institutos Federais. Muitas vezes a falta de informação impede de tomar melhores decisões a
respeito do nosso futuro. Por isso, consideramos importante para o desenvolvimento do projeto a
criação da cartilha contendo o mapeamento em lista de instituições com informações referentes a
cursinhos populares, EJA, cursos profissionalizantes, técnicos e superiores gratuitos em Porto
Alegre. Assim, espera-se promover ações informativas referentes a oportunidades de ensino para
alunos de baixa renda do ensino médio em escolas públicas de alunos das periferias de Porto
Alegre e Região Metropolitana que estejam abertas ao projeto. A proposta é distribuir este
material em escolas públicas e, por meio de rodas de conversa, oportunizar a informação a
jovens que necessitam de algum estímulo nesse sentido. Portanto, o projeto tem como objetivo
instigar e informar jovens dos anos finais do ensino fundamental e do ensino médio de escolas
públicas sobre políticas de acesso e permanência à educação, serviços e ações afirmativas que os
contemplem. Espera-se que, além de aproximar o grupo de petianos dos estudantes de escolas
públicas para discutir direitos civis, políticos e sociais e as possibilidades de participação nas
diversas instâncias de tomada de decisão para a definição de políticas públicas. A elaboração da
cartilha do projeto, durante o mapeamento das políticas, vem propiciando ao grupo PET
interação e a troca de conhecimentos entre estudantes de graduação de diferentes cursos e
estudantes de escolas públicas. No debate investiga-se como desenvolver o espírito investigativo
e a curiosidade dos estudantes com relação ao seu contexto sócio cultural; Problematizar a
realidade de vida dos estudantes de maneira crítica e estimular a busca de estratégias criativas de
construção de novas perspectivas de futuro fornecendo essas informações como subsídio, de
forma a estimular a curiosidade dos estudantes com relação ao seu contexto sociocultural no
desenvolvimento de estratégias criativas de construção de novas perspectivas de futuro
promovendo ações informativas referentes a oportunidades de ensino para alunos de baixa renda
do ensino médio em escolas públicas. Contudo, o projeto caminhos formativos acredita que o
jovem bem informado sobre seus direitos, sobre os serviços de utilidade pública e as políticas
para a juventude, estarão mais bem preparados para o mercado de trabalho e, sobretudo, para
exercerem sua cidadania de forma consciente.

\includepdf{pdfs/Caminhos-Formativos---Juv}

\addcontentsline{toc}{section}{Conexões De Saberes: Cine Pet E Morte E Vida Severina Como Interfaces Da Questão  Agrária No Brasil}

\section*{Conexões De Saberes: Cine Pet E Morte E Vida Severina Como Interfaces Da Questão  Agrária No Brasil}

Ediane Hirle, Jandir Rodrigues,,Dalila Prado  Rodrigues Gonçalves,Ahmed Hammaud Chansedine,Angela Vieira Rodrigues,Evens Pierre,Wellignton de Souza Lima

O PET Conexões de Saberes da UNILA tem como característica principal a 
Interdisciplinaridade, e é composto por estudantes de diversos cursos de graduação: Antropologia, 
Ciência Política e Sociologia, Ciências Econômicas, Cinema e Audiovisual, Desenvolvimento 
Rural e Segurança Alimentar, Letras, Matemática, Relações Internacionais e Integração e Serviço 
Social. O grupo conta também com diversas nacionalidades: brasileira, colombiana, haitiana, 
libanesa e venezuelana. Nosso programa é fundamentado na construção da Integração da 
Universidade no projeto Latino-americano por meio da Cultura e Literatura, internamente e com 
a comunidade da região de Foz do Iguaçu e da fronteira. Levando em conta essa rica experiência, 
apresentamos o relato de uma atividade dentre as desenvolvidas no decorrer dos anos de 2020 e 
2021 através de encontros virtuais, inseridas nos três principais eixos: Pesquisa, Conexões de 
Saberes e o Cine PET.
A partir da pesquisa e estudo da obra literária “Morte e Vida Severina”, de João Cabral de 
Melo Neto, realizamos debates sobre a Questão Agrária no Brasil inserido na atividade Conexões 
de Saberes, na qual tivemos a oportunidade compartilhar da experiência de vida de Sandra Marli 
da Rocha Rodrigues, mulher camponesa militante do Movimento de Mulheres Camponesas 
(MMC). Além disso, esta atividade esteve integrada com a realização do Cine PET, já que 
abordamos o tema a partir da exibição de dois curta-metragem: “Vida Maria” e “Sozinhas –
Violência contra mulheres que vivem no campo”, ambos que tratam da questão de gênero.
É importante situar que o MMC é um movimento social popular brasileiro, fundado em 
2004, composto por mulheres camponesas em sua diversidade no país: elas são agricultoras, 
arrendatárias, meeiras, ribeirinhas, posseiras, boias-frias, diaristas, parceiras, extrativistas, 
quebradeiras de coco, pescadoras artesanais, sem terra, assentadas; são mulheres indígenas, negras, 
descendentes de europeus. O movimento defende a Reforma Agrária e a construção de um projeto 
popular de agricultura agroecológica para alcançar a soberania alimentar, produzir alimentos 
saudáveis e garantir comida de verdade no prato do povo brasileiro, contribuindo para melhorar a 
qualidade de vida no campo e na cidade, preservando a agrobiodiversidade.
Nas discussões, foi possível traçar um paralelo entre “Morte e Vida Severina” e a Questão 
Agrária atual no Brasil; podemos afirmar que a estrutura agrária brasileira permanece inalterada e 
a concentração de terras acelerou consideravelmente na atualidade, ampliando as fronteiras 
agrícolas para a produção de commodities, com um elemento a mais, que é a estrangeirização de 
terras em território brasileiro. Já os elementos abordados nos curta-metragem trazem um recorte 
na questão de gênero, expondo a divisão sexual do trabalho, por exemplo, retratado no curta “Vida 
Maria”, no qual o “lugar” da mulher está limitado à casa e ao seu entorno, espaço da reprodução 
social da força de trabalho, o que vai sendo naturalizado e incorporado por gerações de “Marias”.
Se em “Morte e Vida Severina” encontramos a representação de um corpo racializado e 
afetado pela fome e violação de direitos, “Se somos severinos / iguais em tudo na vida, / morremos 
de morte igual / mesma morte severina: / que a morte que se morre / de velhice antes dos trinta / 
de emboscada antes dos vinte / de fome um pouco por dia”, de igual modo nos dias atuais podemos 
fazer uma análise interseccional entre gênero, classe e raça, sobre como a Questão Agrária no 
Brasil tem raízes coloniais e poucas medidas de justiça, reparação, demarcação e redistribuição de 
terras para afrodescendentes e indígenas são verificadas na história do País.
Como resultados, o contato com as obras literária e audiovisuais, somados à experiência 
vivenciada e compartilhada pela convidada, tivemos um debate intenso entre os petianos e 
petianas, buscando compreender mais sobre a Questão Agrária, a violência no campo, de gênero, 
de geração e racial (indígenas, quilombolas, povos e comunidades tradicionais), agregando 
conhecimentos diferenciados no grupo por sua diversidade. 

\includepdf{pdfs/Conexoes-De-Saberes--Cine}

\addcontentsline{toc}{section}{Conexões De Saberes: Cine Pet E Morte E Vida Severina Como Interfaces Da Questão Agrária No Brasil}

\section*{Conexões De Saberes: Cine Pet E Morte E Vida Severina Como Interfaces Da Questão Agrária No Brasil}

Ediane Hirle, Jandir Rodrigues,Ahmed Hammaud Chansedine,Angela Vieira Rodrigues,Evens Pierre,Wellington de Souza Lima,Dalila Prado Rodrigues Gonçalves

O PET Conexões de Saberes da UNILA tem como característica principal a 
Interdisciplinaridade, e é composto por estudantes de diversos cursos de graduação: Antropologia, 
Ciência Política e Sociologia, Ciências Econômicas, Cinema e Audiovisual, Desenvolvimento 
Rural e Segurança Alimentar, Letras, Matemática, Relações Internacionais e Integração e Serviço 
Social. O grupo conta também com diversas nacionalidades: brasileira, colombiana, haitiana, 
libanesa e venezuelana. Nosso programa é fundamentado na construção da Integração da 
Universidade no projeto Latino-americano por meio da Cultura e Literatura, internamente e com 
a comunidade da região de Foz do Iguaçu e da fronteira. Levando em conta essa rica experiência, 
apresentamos o relato de uma atividade dentre as desenvolvidas no decorrer dos anos de 2020 e 
2021 através de encontros virtuais, inseridas nos três principais eixos: Pesquisa, Conexões de 
Saberes e o Cine PET.
A partir da pesquisa e estudo da obra literária “Morte e Vida Severina”, de João Cabral de 
Melo Neto, realizamos debates sobre a Questão Agrária no Brasil inserido na atividade Conexões 
de Saberes, na qual tivemos a oportunidade compartilhar da experiência de vida de Sandra Marli 
da Rocha Rodrigues, mulher camponesa militante do Movimento de Mulheres Camponesas 
(MMC). Além disso, esta atividade esteve integrada com a realização do Cine PET, já que 
abordamos o tema a partir da exibição de dois curta-metragem: “Vida Maria” e “Sozinhas –
Violência contra mulheres que vivem no campo”, ambos que tratam da questão de gênero.
É importante situar que o MMC é um movimento social popular brasileiro, fundado em 
2004, composto por mulheres camponesas em sua diversidade no país: elas são agricultoras, 
arrendatárias, meeiras, ribeirinhas, posseiras, boias-frias, diaristas, parceiras, extrativistas, 
quebradeiras de coco, pescadoras artesanais, sem terra, assentadas; são mulheres indígenas, negras, 
descendentes de europeus. O movimento defende a Reforma Agrária e a construção de um projeto 
popular de agricultura agroecológica para alcançar a soberania alimentar, produzir alimentos 
saudáveis e garantir comida de verdade no prato do povo brasileiro, contribuindo para melhorar a 
qualidade de vida no campo e na cidade, preservando a agrobiodiversidade.
Nas discussões, foi possível traçar um paralelo entre “Morte e Vida Severina” e a Questão 
Agrária atual no Brasil; podemos afirmar que a estrutura agrária brasileira permanece inalterada e 
a concentração de terras acelerou consideravelmente na atualidade, ampliando as fronteiras 
agrícolas para a produção de commodities, com um elemento a mais, que é a estrangeirização de 
terras em território brasileiro. Já os elementos abordados nos curta-metragem trazem um recorte 
na questão de gênero, expondo a divisão sexual do trabalho, por exemplo, retratado no curta “Vida 
Maria”, no qual o “lugar” da mulher está limitado à casa e ao seu entorno, espaço da reprodução 
social da força de trabalho, o que vai sendo naturalizado e incorporado por gerações de “Marias”.
Se em “Morte e Vida Severina” encontramos a representação de um corpo racializado e 
afetado pela fome e violação de direitos, “Se somos severinos / iguais em tudo na vida, / morremos 
de morte igual / mesma morte severina: / que a morte que se morre / de velhice antes dos trinta / 
de emboscada antes dos vinte / de fome um pouco por dia”, de igual modo nos dias atuais podemos 
fazer uma análise interseccional entre gênero, classe e raça, sobre como a Questão Agrária no 
Brasil tem raízes coloniais e poucas medidas de justiça, reparação, demarcação e redistribuição de 
terras para afrodescendentes e indígenas são verificadas na história do País.
Como resultados, o contato com as obras literária e audiovisuais, somados à experiência 
vivenciada e compartilhada pela convidada, tivemos um debate intenso entre os petianos e 
petianas, buscando compreender mais sobre a Questão Agrária, a violência no campo, de gênero, 
de geração e racial (indígenas, quilombolas, povos e comunidades tradicionais), agregando 
conhecimentos diferenciados no grupo por sua diversidade. 
Referências:
SILVA, Isabela Costa da. Movimento de Mulheres Camponesas na Trajetória Feminista 
Brasileira: Uma Experiência de Luta por Direitos e Liberdade. Dissertação de Mestrado em 
Serviço Social. Universidade Federal de Juiz de Fora, 2013. P. 99.
MELO NETO, João Cabral de. Morte e vida severina e outros poemas. Rio de Janeiro: Alfaguara, 
2007.
VIDA MARIA. Direção Marcio Ramos. 3º Prêmio Ceará de Cinema e Vídeo, Trio Filmes, 
VIACG. DVD (8 min. e 34 seg).
SOZINHAS-Violência contra mulheres que vivem no campo. Disponível em: 
https://www.youtube.com/watch?v=XEuJ9XT2yX8. Acesso em 15 de Setembro de 2021.

\includepdf{pdfs/Conexoes-De-Saberes--Cine}

\addcontentsline{toc}{section}{Contaminação Do Leite E A Microbiologia Preditiva}

\section*{Contaminação Do Leite E A Microbiologia Preditiva}

Ana Carolina Rubio Klein, Pietro Serraglio Figueiredo,Isac Gonçalves de Oliveira,Fernanda Gubert de Souza,Estevãn Martins de Oliveira

A pecuária leiteira do Brasil é responsável por 7% da produção global, segundo estudo divulgado
pela Companhia Nacional de Abastecimento (Conab). Em 2016, o país produziu 32,1 milhões de
toneladas, de acordo com dados do Instituto Brasileiro de Geografia e Estatística (IBGE) e do
Departamento de Agricultura dos EUA (USDA). Devido à sua composição o leite é de grande
importância na nutrição humana e com alto valor nutritivo, por esse motivo cada tipo de
processamento tem que ocorrer corretamente, para não afetar a qualidade do produto. Podemos
dividir o processamento do leite em três tipos, leite cru, leite pasteurizado e leite esterilizado. O
leite cru nada mais é que o leite que foi retirado das tetas do animal e não passou por nem um
processo, proveniente de uma vaca, de uma cabra, ou mesmo de uma ovelha. Deve apresentar
características sensoriais: líquido branco opalescente e homogêneo e odor característico. O leite
pasteurizado é aquele que foi submetido a um tratamento térmico nas temperaturas que podem
variar entre 72 ºC a 75 ºC por cerca de 15 a 20 segundos e a refrigeração de 2 ºC a 5 ºC. Podendo
ser integral, semidesnatado ou desnatado. A pasteurização é um processo cuja finalidade é
eliminar micro-organismos patogênicos do leite, porém, nem todos são eliminados, por isso
necessita da refrigeração, com o intuito de evitar a proliferação e crescimento microbiológico no
leite. Já o leite esterilizado, cujo tratamento térmico denomina-se UHT (Ultra High temperature),
é aquecido em 70ºC em fluxo contínuo e esterilizado na própria embalagem, à temperatura de
109 ºC a 120 ºC, durante 20 a 40 minutos, sofrendo resfriamento de 20 ºC a 35 ºC, o leite pode
ser integral ou desnatado. Microbiologia preditiva é a área que utiliza modelos matemáticos
baseados em dados experimentais, como atividade de água, pH e temperatura, para modelar as
curvas de crescimento ou diminuição de concentração de microrganismos em determinado
alimento. Existem três tipos de modelos preditivos, sendo eles: modelo primário, que gera curvas
em um gráfico de concentração de microrganismos (UFC/ml) por tempo; o modelo secundário,
que explica como a curva do gráfico primário varia conforme a alteração de diferentes
ambientes, como ph e temperatura; e o modelo terciário, que pode combinar os dois primeiros,
utilizando softwares para calcular como o microrganismo se comportará. O consumo de leite e
derivados está em constante em crescimento no Brasil e no mundo, e por esta razão, a
microbiologia preditiva deve ser utilizada sempre para o auxílio nas pesquisas de segurança
destes produtos. A microbiologia preditiva constitui-se uma das ferramentas mais seguras para se
calcular o tempo de prateleira de um alimento, como também para se calcular as melhores
formas de inibir o crescimento de patógenos, garantindo assim a segurança do alimento
(SCHLEI et al., 2018). Um dos modelos mais utilizados é o modelo de Baranyi e Roberts, pois o
mesmo leva em consideração o mecanismo biológico de crescimento de microrganismos, onde
sua curva tem o formato semelhante a um sigmóide (formato de um “S”) porém apresenta um
comportamento linear em sua fase intermediária (ROBERTS \& BARANYI, 1994). Todavia, a
microbiologia preditiva apresenta algumas limitações, devido aos modelos propostos não
poderem exceder os intervalos para quais foram descritos, como faixas de temperaturas e
atividade de água, pois simulações fora dos intervalos dos experimentos são susceptíveis a erros
que podem comprometer a qualidade do produto final (FAKRUDDIN et al., 2011). Portanto, o
responsável pela qualidade do produto deve atentar-se ao modelo que será utilizado, respeitando
sempre suas limitações e para quais microorganismos o modelo é destinado, aproximando-se o
máximo possível da realidade para assim obter bons resultados.
Brasil é responsável por 7% da produção de leite. Companhia Nacional de Abastecimento
(Conab). Disponível em:
https://www.conab.gov.br/ultimas-noticias/2634-brasil-e-responsavel-por-cerca-de-7-do-leite-pro
duzido-no-mundo
FAKRUDDIN, M.; MAZUMDER, R.M.; MANNAN, K.S.B. Predictive microbiology:
Modeling microbial responses in food. Ceylon Journal of Science [online], v. 40, n. 2,
p.121-131, 2011. Disponível em: http://www.sljol.info/index.php/CJSBS/article/view/3928
Acesso em: 07 set. 2021.
BARANYI, J.; ROBERTS, T.A. A dynamic approach to predicting bacterial growth in food.
International journal of food microbiology, 23(3-4), 277-294, 1994
SCHLEI K. P.; REITER M. G. R.; BERTOLI S. L.; LICODIEDOFF S.; CARVALHO L. F. \&
SOUZA C. K. Microbiologia Preditiva: aspectos gerais e tendências. Perspectivas da
Ciência e Tecnologia, v.10, p. 52-68, 2018.

\includepdf{pdfs/Contaminacao-Do-Leite-E-A}

\addcontentsline{toc}{section}{Criação De Material Didático: Vídeos Eletrônica.}

\section*{Criação De Material Didático: Vídeos Eletrônica.}

Leonardo Sales Galvao, None

A mecatrônica é uma área de conhecimento amplo, desenvolvendo a integração das áreas 
de mecânica, eletrônica, programação e elétrica. Segundo ALCIATORE, HISTAND, 2014, p.2. 
ela é utilizada como “desenvolvimento de produtos e cuja função depende da integração dos 
componentes eletrônicos e mecânicos” 
Com isso, o estudante de mecatrônica necessita desenvolver conhecimentos na área de 
eletrônica digital para poder integrar os conhecimentos de eletrônica com as outras áreas de 
estudo da mecatrônica.
Analisando o escasso material existente, verificou-se que a metodologia aplicada nesse material, 
era descritiva ou explicativa, o que gera lacunas no aprendizado para o estudante de mecatrônica 
na área de eletrônica. 
Para propor uma alternativa a este problema, foram utilizadas as duas metodologias 
associadas a fim de se construir um conteúdo que apresente o tema de forma simples, facilitando 
a aprendizagem por meio de imagens demonstrativas e ilustrativas e explicações sobre os 
processos e métodos usados.
Possuindo uma abordagem qualitativa e utilizando as metodologias descritiva e 
explicativa e tendo em vista o objetivo de introduzir as pessoas no tema, a primeira etapa desse 
trabalho foi a seleção dos tópicos principais: componentes eletrônicos, portas lógicas e software. 
Para dar sequencia de uma maneira didática os vídeos foram produzidos explicando 
componentes simples de eletrônica para somente depois seguir para lógica digital onde são 
utilizados esses componentes. Por último, foi ensinado como abordado o software logisim onde 
se utiliza a lógica digital. 
Com base na análise dos planos de ensinos das disciplinas de eletrônica digital e 
eletrônica analógica do curso de Engenharia Mecatrônica foram abordados conceitos básicos de: 
capacitor, resistores, portas lógicas, logisim, circuito integrado, somador de bits entre outros.
As aulas se iniciam com uma apresentação do bolsista do grupo PET mecatrônica, do 
Instituto Federal de Santa Catarina campus Florianópolis. Na sequência é apresentado o objeto 
de estudo da aula e de acordo com suas características são explorados recursos metodológicos 
diferentes. Quando o objeto de estudo não permite uma visualização, por exemplo: a 
eletricidade; são utilizadas analogias com processos do cotidiano para facilitar a compreensão. 
Outros objetos de estudo, por exemplo: softwares e circuitos integrados, são apresentados ao 
estudante por partes para que ele consiga compreender o todo. Seguindo o desenvolvimento do 
tema são discutidos os conceitos do objeto de estudo, suas características, usos, formas de 
integração, cuidados e etc.

\includepdf{pdfs/Criacao-De-Material-Didat}

\addcontentsline{toc}{section}{Criação De Um Modelo De Procedimento Para Cursos E Palestras Online Do Grupo Pet Engenharia De Produção Ufsc}

\section*{Criação De Um Modelo De Procedimento Para Cursos E Palestras Online Do Grupo Pet Engenharia De Produção Ufsc}

Bruna Schiavini Hoepers, Gustavo Borba (PETEPS/UFSC),Beatriz Locatelli (PETEPS/UFSC),Gabriel Henrique Silva Cidade (PETEPS/UFSC),João Paulo Maximiano Almeida (PETEPS/UFSC)

No ano de 2020, o mundo vivenciou um novo período, provocado pela pandemia da COVID-19, 
doença causada pelo vírus SARS-CoV-2. Nesse cenário, governos promoveram ações para 
redução da transmissão da doença, adotando, entre outras, medidas de distanciamento social, que 
ocasionaram o fechamento de escolas e universidades, comércios e áreas de lazer, entre outros. 
Em virtude da situação, as atividades da Universidade Federal de Santa Catarina foram adaptadas 
para um modelo de ensino remoto e, com isso, os cursos, antes promovidos de forma presencial 
pelo PET Engenharia de Produção, necessitaram ser adaptados para o modo online.
Estes cursos são tradicionalmente ministrados pela entidade e seguem procedimentos mapeados e 
descritos seguindo um padrão de sistema de gestão da qualidade, que apresenta, através de 
fluxogramas e registros de qualidade, uma metodologia que tenta garantir uma melhoria contínua 
e o bom funcionamento de todos os processos da entidade. Para os cursos presenciais, existe um 
procedimento bem estruturado e padronizado, onde está descrito o que os membros devem fazer 
antes, durante e depois de cada evento, para prevenir qualquer imprevisto e assegurar que nada 
seja esquecido. Todo esse mapeamento foi montado baseado na experiência adquirida ao longo 
dos anos de história do grupo e, apesar de ser revisto periodicamente, não estava adequado para 
essa nova realidade de cursos totalmente desenvolvidos de forma online, onde são encontradas 
particularidades, como o número maior de participantes, inclusive de fora da universidade, e a 
necessidade de meios de gravação e disponibilização dos materiais de forma digital. Sentiu-se, 
então, necessidade de criar um procedimento que se ajustasse às novas necessidades dos alunos e 
dos membros, modificando tarefas que não eram mais necessárias, como alocar espaço físico, e 
adicionando novas, como a criação de salas virtuais em plataformas que comportam a nova 
demanda de participantes.
O projeto foi gerido com auxílio do framework Scrum, metodologia de gerenciamento de projetos 
ágil, que visa tornar trabalhos mais dinâmicos e adaptáveis dividindo o tempo de entrega em 
períodos menores de trabalho, conhecidos como sprints, onde resultados parciais são apresentados 
aos clientes no final de cada um desses períodos, para garantir o alinhamento entre as partes. Para 
o trabalho em questão, dividiu-se o período de entrega em três sprints. No primeiro momento, 
todas as demandas do cliente do projeto - o Coordenador de Marketing e Eventos do PET 
Engenharia de Produção - foram analisadas para estruturar o Product Backlog, que consiste em 
uma lista das atividades que devem ser realizadas ao longo do período de trabalho e é usado como 
referência para definir o que será feito em cada sprint. Em seguida, iniciou-se uma etapa de 
pesquisa, na qual foram levantados: qual seria a plataforma que melhor atenderia às necessidades 
de transmissão do/da curso/palestra; termos de uso e imagem que possibilitassem a gravação dos 
eventos, bem como a disponibilização dessa gravação e, por fim, possíveis plataformas para 
disponibilizar os materiais do evento.
Após a pesquisa, definiram-se as melhores opções para o projeto, com a validação do cliente. Com 
essas definições, reestruturou-se o mapeamento do processo de curso/palestra presencial, 
adequando-o à modalidade virtual. Para esta tarefa, foi remodelado o fluxograma do processo que 
constava nos arquivos do PET, de maneira que, para esse procedimento, constata-se o passo a 
passo para realizá-lo, tanto de forma online quanto presencial, para que ambas sejam maneiras 
viáveis no futuro.
Durante a fase de desenvolvimento do projeto, além de adequar o procedimento, se fez necessário 
adaptar a documentação interna relacionada ao mesmo, presente no sistema de gestão da qualidade 
do PET Engenharia de Produção. Em relação às adaptações, pode-se mencionar a lista de 
verificação do procedimento, que consiste no registro de realização das etapas deste novo modelo; 
o formulário de inscrição, a lista de presença e, por fim, o questionário de avaliação do evento, 
que foram transformados em formulários online. Junto às alterações citadas, foi definida a criação 
de dois novos documentos que, posteriormente, foram integrados ao sistema de gestão da 
qualidade do grupo: um registro automatizado, no formato de planilha, e os Termos de Uso de 
Curso/Palestra. Esse registro tem como finalidade auxiliar na verificação da frequência mínima 
necessária para obtenção de certificado de participação no evento — tal documentação recebe 
como “entrada” os dados obtidos com o relatório emitido pela própria plataforma escolhida para a 
realização do evento. Por sua vez, o documento com os Termos de Uso de Curso/Palestra foi 
criado, com o auxílio de profissionais da área, e consiste em um contrato eletrônico — sendo este 
documento exclusivamente ligado ao modelo para a realização de eventos online — estabelecido 
entre as partes envolvidas, objetivando limitar a responsabilidade do PET no que diz respeito à 
opinião de terceiros, direcionar a forma de utilização e esclarecer possíveis dúvidas acerca dos 
cursos/palestras realizados pela entidade.
Portanto, ao realizar esse projeto, o grupo PET Engenharia de Produção da UFSC vem cumprindo 
seu objetivo com o Programa de Educação Tutorial, possibilitando impactar todos os pilares da 
tríade universitária. O impacto no âmbito da pesquisa ocorreu porque, para remodelar o 
procedimento, foi necessário buscar conhecimento em diversas fontes, para estruturá-lo passo a 
passo e adequá-lo ao sistema de gestão da qualidade adotado na instituição. Já os pilares de ensino 
e extensão foram abrangidos, pois a criação do procedimento permitiu ao grupo realizar eventos 
de qualidade, seja na modalidade online ou na presencial. Além disso, o fato de também expandir 
o número de vagas disponibilizadas para trezentos participantes corrobora com a disseminação de 
conhecimentos para a comunidade sobre temas relacionados à Engenharia de Produção ou a 
habilidades comportamentais importantes para a vida profissional.

\includepdf{pdfs/Criacao-De-Um-Modelo-De-P}

\addcontentsline{toc}{section}{Curadoria De Material De Apoio Para Aprendizado Da Linguagem C}

\section*{Curadoria De Material De Apoio Para Aprendizado Da Linguagem C}

Daniel Eduardo Vieira, Victor Hugo Garret,Saulo Jafet Gusmão,Cesar Augusto Tacla

Neste trabalho organizamos um conjunto de materiais com intuito de apoiar o desenvolvimento
inicial de novos membros do nosso grupo que desejam aprender a linguagem de programação C,
levando em consideração os problemas provenientes do método da educação a distância
meramente expositivo. Para isso, buscamos e selecionamos produções já existentes no contexto
da Universidade e conteúdos disponíveis na internet, seguindo critérios próprios que
estabelecemos em testes internos, para estabelecer a coesão dos materiais reutilizados.
Encontradas e selecionadas as produções, construímos índices para permitir ao grupo de
estudantes fazer um percurso de aprendizado e, então, disponibilizamos o índice e realizamos a
divulgação para os estudantes.
No início do ano de 2021, decidimos criar uma base de suporte para integrantes de períodos
iniciais que permitisse o contato desses integrantes com projetos mais avançados do grupo
PET-ECo da UTFPR, um dos elementos dessa base é o conjunto de grupos de estudo, pensados
para gerar coesão entre os membros e ao mesmo tempo integrar os novatos aos projetos dos
veteranos. Esses grupos apoiam-se no sistema de apadrinhamento (sistema que estabelece uma
relação de um veterano do grupo para cada novato, para que o veterano desempenhe o papel de
orientador) já existente para criar um ambiente de interação e construção de conhecimento,
sendo que a cada processo seletivo realizado os grupos de estudo são formados pelos novatos,
que serão orientados pelos veteranos nos temas de interesse.
O grupo de estudo sobre a linguagem C é a primeira versão instanciada pelo PET-ECo e
apresentou os primeiros desafios de implementação que enfrentamos. A começar pelo problema
de definir o método de construção de conhecimento a ser utilizado, uma vez que queríamos
evitar uma abordagem baseada na Educação a Distância em que a informação flui em único
sentido, cenário no qual é frequente pensarmos em um aluno estudando os conteúdos sozinho e o
computador sendo utilizado como uma “máquina de ensinar”. Nessa concepção, os
computadores representam uma evolução das mídias e não modificam o modelo de comunicação
de massa, predominantemente unidirecional, que tipicamente caracteriza a abordagem
instrucionista-massiva que ainda hoje é muito praticada na modalidade a distância (PIMENTEL,
2020).
Com o intuito de definir uma estrutura para o processo de integração de novos membros ao
PET-ECo, a curadoria tem a função de contextualizar um novo membro no primeiro ambiente de
introdução ao RobôFun (KLEIN et allia, 2020), que é um projeto de longo prazo criado para
desenvolver robótica com uma abordagem interdisciplinar.
O método de construção da atividade foi baseado no Scrum (SCHWABER, 2020), através de
reuniões semanais dos membros participantes. Nas reuniões de validação, avaliamos
informalmente o material e suas limitações, através da resolução de problemas usados nas
disciplinas de Fundamentos de Programação, cedidos gentilmente pela professora Leyza Baldo
Dorini, esses problemas deveriam ser solucionados com as ferramentas disponíveis nos materiais
selecionados.
O material foi usado como base para as discussões das reuniões, onde os membros mais
experientes guiaram o membro estudante na resolução. Quando o material apresentou alguma
deficiência, buscamos em outras fontes complementos para chegar a uma solução do problema e,
consequentemente, validar informalmente o conteúdo selecionado.
Para gestão do conhecimento e nova aplicação da atividade, publicamos o material validado, os
problemas e soluções. A forma empregada foi a criação de um curso no Moodle para
armazenamento e exposição do material. Organizado em tópicos por hierarquia de seguimento de
aprendizado e adicionados tópicos para feedbacks dos estudantes. Dada a nossa preocupação
com a forma de construção de conhecimento, é sempre necessário que os estudantes sejam
guiados por membros mais experientes, de forma síncrona, como um grupo de estudos.
Referências:
PIMENTEL, Mariano; CARVALHO, Felipe da Silva Ponte. Princípios da Educação Online: para
sua aula não ficar massiva nem maçante! SBC Horizontes, maio de 2020. ISSN 2175-9235.
Disponível em: .
Acesso em: 23 Agosto. 2021.
SCHWABER, Ken; SUTHERLAND, Jeff. O Guia do Scrum. O Guia definitivo para o Scrum:
As regras do jogo. Creative Commons, novembro de 2020. Disponível em:
.
Acesso em: 31 Agosto 2021.
KLEIN, Luan C.; CORDEIRO, Julia Z.; BARBOZA, Guilherme G.; TACLA, Cesar A. (2021).
Robô: Lições Aprendidas em um Projeto de Atração de Novos Estudantes e de Redução da
Evasão. Communications And Innovations Gazette, v.5, n. 2, p. 52-63.
https://doi.org/10.5902/2448190462034.

\includepdf{pdfs/Curadoria-De-Material-De-}

\addcontentsline{toc}{section}{Cursos Promovidos Pelo Pet Ecv Ufsc}

\section*{Cursos Promovidos Pelo Pet Ecv Ufsc}

Tamires Dos Santos, Samara Tiemi Nakashima Kobori-PET/ECV,Daniel Tavares dos Anjos-PET/ECV,Giorgia Luccheta Pezzi-PET/ECV,Giulia Pimentel Cia Koike-PET/ECV,Juliana Ribas Nantes-PET/ECV,Cláudio Cesar Zimmermann(orientador)-PET/ECV

Um engenheiro civil deve sempre estar atualizado com as novas tecnologias de seu ramo,
já que frequentemente novas técnicas são criadas para reduzir eventuais erros humanos que, na
Engenharia Civil, podem custar diversas vidas. Assim, esta profissão exige maior discernimento
das novidades que surgem ao longo do tempo e o conhecimento sobre as ferramentas
computacionais da área torna-se cada vez mais importante entre os alunos da graduação - hoje,
indispensáveis à sua capacitação.
Pensando nisso e nas dificuldades trazidas pela pandemia do COVID-19, o
PET/ECV/UFSC busca promover a qualificação e a integração de estudantes e profissionais da
Universidade por meio de cursos online de softwares de fácil acesso à comunidade acadêmica, já
que este conhecimento técnico tem-se mostrado tão fundamental no mercado de trabalho, agora
majoritariamente de maneira remota.
Desta forma, a atividade contribui para o desenvolvimento técnico tanto dos participantes
quanto dos petianos, que o adquirem lecionando cada aula, e também para a formação de
profissionais socialmente conscientes, propiciando uma melhor integração do PET com a
sociedade e a transformação da comunidade através do trabalho em grupo e da elevação do
conhecimento na Universidade.
Os petianos e o professor tutor ministram semestralmente cursos de softwares muito
utilizados na Engenharia Civil, com alternância especial entre o AutoCAD® e o Revit®, ambos
da AutoDesk. Durante a pandemia, eles são gratuitos, têm emissão de certificado e as vagas são
preenchidas por ordem de inscrição, sem nenhuma restrição ou preferência.
Para o desenvolvimento e a aplicação dos cursos foram elaboradas duas apostilas de
apoio, uma para o curso de Revit® e outra para o de AutoCAD®. Nelas estão contidas
informações detalhadas acerca dos usos e funções das principais ferramentas dos softwares, bem
como a organização de cada aula, juntamente com o passo a passo de toda competência a ser
ensinada, facilitando o ensino e o acompanhamento dos alunos durante a aplicação dos cursos.
Ao início de cada semestre, os petianos voluntariam-se para serem professores ou
monitores do curso. Os professores são os responsáveis pela elaboração e pela aplicação das
aulas, que são realizadas de maneira síncrona, por meio da plataforma de encontros virtuais
Google Meet. Já os monitores também estão presentes em todas as aulas, cumprindo a função de
sanar dúvidas e de garantir que os encontros ocorram de maneira propícia. Além disso, cada um
deles disponibiliza um horário semanal que é destinado para o atendimento e para a orientação
dos alunos inscritos no curso, tirando dúvidas subjacentes às aulas.
Os membros do programa também são responsáveis pelas funções administrativas, isto é,
elaboram formulários de inscrição e todo o material de divulgação, o que inclui a confecção de
artes e de textos utilizados nas mídias sociais e no fórum da graduação.
Após o período de inscrição, as aulas são oferecidas 2 vezes na semana. Ao final do
curso, um trabalho de avaliação é solicitado aos alunos, sendo os petianos os responsáveis pela
sua correção e pela aprovação dos participantes.
Após finalizada a atividade, o conteúdo do curso é divulgado no site do PET no canal do
YouTube, e é enviado um formulário de avaliação aos participantes para averiguar sua satisfação
sobre o curso, para que em ocasiões futuras possam haver eventuais melhorias.
Durante o período de organização do curso é possível verificar o forte desenvolvimento
de habilidades de comunicação e de pensamento crítico entre os petianos e orientadores para
garantir que todo o material passado aos participantes do curso seja utilizado de maneira clara e
coerente, além de buscarem fazê-lo de maneira acessível ao público e incentivando a integração
entre a graduação e o grupo PET. Além disso, o próprio planejamento dos professores sobre as
aulas a serem lecionadas já aprimora seu domínio sobre o software, melhorando inclusive sua
oratória e sua segurança durante a aplicação do curso.
Ademais, os inscritos do curso ganham aprendizados que podem ajudar a sociedade
futuramente, visto que o conhecimento adquirido pode ser melhorado e aplicado durante sua
jornada profissional diretamente sobre a sociedade e sobre o avanço da Educação,
principalmente considerando que a transmissão do conhecimento de Revit® e de AutoCAD® de
maneira remota é muito viável e segura dada a situação da pandemia.
Assim, percebe-se que o desenvolvimento da atividade referida contribui imensamente
para o enriquecimento da experiência universitária e do conhecimento técnico de todos os
envolvidos. Por ser uma atividade oferecida ao público geral, toda a comunidade acadêmica pode
ser beneficiada, por meio da aquisição de novos conhecimentos a partir de um material de
qualidade, promovendo a integração e a troca de conhecimento entre bolsistas e estudantes.
E evidencia-se o elevado grau de conhecimento adquirido pelos membros do
PET/ECV/UFSC, já que são responsáveis por todas as etapas da atividade, que incluíram desde a
elaboração do material de apoio e das aulas até à divulgação e à estruturação dos diversos cursos
ofertados, além da capacitação de cidadãos mais qualificados para o mercado de trabalho mesmo
em momentos de incerteza tais como o da pandemia do COVID-19.
VÍDEO AULA Revit 01: Introdução ao Programa, Utilidades e Configurações Básicas por PET
Engenharia Civil UFSC. YouTube (15:26 min). Disponível em:
https://www.youtube.com/watch?v=brTNRRNuOfo. Acesso em: 20 setembro 2021.
TUTORIAL AutoCAD #01 - Cotas por PET Engenharia Civil UFSC. YouTube (04:40 min),
Disponível em: https://www.youtube.com/watch?v=H4df6E-Jzv8. Acesso em: 20 setembro
2021.
ENSINO. PET Engenharia Civil UFSC. Disponível em: < https://petecv.ufsc.br/ensino/>.
Acesso em: 20/09/2021, 16:30h.

\includepdf{pdfs/Cursos-Promovidos-Pelo-Pe}

\addcontentsline{toc}{section}{Desenvolvimento Projeto Social - Bárbara Maix}

\section*{Desenvolvimento Projeto Social - Bárbara Maix}

Filipe Belchor Barcelos, ALFREDO HENRIQUE SUPTIZ¹,GABRIEL ALENCAR PASINATTO,GABRIEL RAMBO,MARCOS BONINI,CLAUDIR JOSÉ BASSO

O Programa de Educação Tutorial (PET) Ciências Agrárias do curso de agronomia da
Universidade Federal de Santa Maria, campus Frederico Westphalen, tem como objetivo realizar
e desenvolver ações junto à comunidade acadêmica e comunidade em geral, tendo como seus
principais vértices o ensino, a pesquisa e a extensão. Visando gerar experiências e estimular o
espírito altruísta dos alunos e demais envolvidos, o PET Ciências Agrárias desenvolveu na área
da extensão ações na Comunidade Terapêutica Feminina Bárbara Maix (Cotebma), localizada na
linha Encruzilhada no município de Frederico Westphalen RS/Brasil, a instituição Bárbara Maix
tem como principal objetivo auxiliar na recuperação de mulheres de vários municípios da região,
vítimas da dependência de drogas, álcool e de medicamentos, proporcionando um ambiente de
apoio e acompanhamento necessários para as residentes, como tratamentos psicológico e social
individualizado. Nesse sentido, desenvolve na Comunidade Terapêutica, atividades educativas,
esportivas e profissionalizantes que ajudem na reintegração da mulher a sociedade com a
dignidade e respeito devido a vida humana.
As ações realizadas pelo grupo PET Ciências Agrárias, em conjunto com a Comunidade
Terapêutica Feminina Bárbara Maix, possibilitaram a revitalização do espaço, sendo adotadas
melhorias na horta da casa, através da limpeza da área, revolvimento e formação de canteiros,
para posterior implantação de mudas e sementes de diferentes hortaliças e plantas medicinais.
Além da revitalização da área, foram realizados tratos culturais nas plantas já existentes no local,
através da realização do manejo e poda de videiras e a realização de limpeza, tratos culturais e
implantação de flores em canteiros já existentes na localidade e canteiros os quais foram
implantados a partir das atividades realizadas pelo grupo, com objetivo de tornar o ambiente
mais aconchegante e harmonioso.
O trabalho oferecido pelo grupo é todo realizado de maneira voluntária, trabalho este,
como realização de oficinas, o estímulo aos trabalhos em grupo e a manutenção de estoques de
alimentos e utensílios de higiene para a casa, sendo estes, provenientes de doações de membros
da comunidade acadêmica, da comunidade em geral e de empresas da região. A grande maioria
dos materiais utilizados para as ações de revitalização da horta, videiras e jardim da casa, foram
doados por empresas e agropecuárias da região, arrecadados através de atitudes solidárias da
sociedade.
Pode-se perceber, nas visitas de acompanhamento realizados pelos integrantes do grupo
PET Ciências Agrárias até o local, que posteriormente a realização das ações de ensino e
extensão, que as residentes da Comunidade Terapêutica Feminina Bárbara Maix, prosseguiram
com os trabalhos, mostrando-se interessadas, determinadas e felizes com os conhecimentos
adquiridos e com os trabalhos realizados em conjunto no local. Demonstrando assim, a
importância de trabalhos como este, o qual busca estimular o trabalho em grupo e principalmente
passar conhecimento para as residentes dos locais. As moradoras da casa, se demonstraram
muito contentes com as ações realizadas, sempre tentando se inteirar acerca dos tratos culturais e
espécies cultivadas, como, por exemplo, discutindo as melhores formas de cultivo e as diferentes
finalidades das plantas medicinais.

\includepdf{pdfs/Desenvolvimento-Projeto-S}

\addcontentsline{toc}{section}{Desenvolvimento De Cursos Em Modalidade Remota Para A Comunidade Acadêmica Da Utfpr }

\section*{Desenvolvimento De Cursos Em Modalidade Remota Para A Comunidade Acadêmica Da Utfpr }

Cecilia Eduarda Gnoatto Tomazini, Ana Carolina Ferreira- UTFPR,Vanessa Santin Guerra-UTFPR,Marcelo Izidro-UTFPR,Julia Casagrande-UTFPR,Regis Luis Missio-UTFPR

As atividades realizadas para a formação complementar de acadêmicos e para a sociedade contribuem para o desenvolvimento e enriquecimento curricular do ser humano moderno, preocupado com as questões ambientais, econômicas e sociais do país. Pensando nisso, o Grupo PET juntamente com a universidade desenvolve diversos cursos e atividades complementares para promover o crescimento pessoal do acadêmico e estimular o conhecimento crítico. Essas atividades são realizadas dentro dos três pilares básicos das instituições de ensino superior (IES), ensino, pesquisa e extensão. Além disto, os cursos proporcionam enriquecimento extracurricular para quem organiza bem como os colaboradores e ministrantes Os cursos, são frequentemente requisitados pela comunidade acadêmica que busca conhecimento diversificado nas mais diferentes áreas da ciência. Levando-se em conta o momento de pandemia ocasionado pelo vírus da COVID-19, as atividades tiveram que ser reformuladas e reprogramadas para serem oferecidas de forma segura e que possibilitasse aos participantes a agregação de conhecimento. Entretanto com a possibilidade de transmissão online, os conhecimentos repassados tornaram-se mais acessível à sociedade e conectando diferentes instituições de ensino pelo território nacional, permitindo a troca de relatos e realidades.

\includepdf{pdfs/Desenvolvimento-De-Cursos}

\addcontentsline{toc}{section}{Desenvolvimento De Pesquisa Coletiva No Programa De Educação Tutorial Do Curso De Direito Da Ufpr}

\section*{Desenvolvimento De Pesquisa Coletiva No Programa De Educação Tutorial Do Curso De Direito Da Ufpr}

Victoria Brasil, Profa Dra Heloísa Camara (Orientadora),Julia Favaretto Deschamps,Heloisa Nerone

As atividades do Programa de Educação Tutorial do curso de Direito da Universidade
Federal do Paraná ao longo de um ano são guiadas por um tema geral, a partir do qual o PET
elabora o edital do processo seletivo e desenvolve as atividades de pesquisa coletiva. Estas visam
à elaboração de artigos de forma conjunta, por meio da contribuição de todas do grupo sobre
cada artigo desenvolvido em dupla, para posterior elaboração de obra para publicação. O
processo de escolha do tema central se dá através de reuniões com debates sobre as ideias
levantadas pelas integrantes e pela professora tutora.
O tema de estudo escolhido em 2021 foi “As encruzilhadas da subjetividade jurídica
brasileira a partir do Sul”, pois buscou-se investigar, neste ano, a subjetividade jurídica no Brasil,
enquanto país periférico, do sul global e com grupos de vulnerabilizados que foram excluídos do
processo de construção do Estado, em um processo forjado em projeto colonial.
O fenômeno jurídico, imbricado ao campo social, político, cultural, econômico etc.,
figura como altamente complexo; complexidade esta que ganhou tons ainda mais desafiadores
com a ruína, nas últimas décadas, de diversas categorias jurídicas, tidas até então como eternas e
universais. O ponto de partida da pesquisa de 2021 foi a concepção de que a construção da figura
do sujeito de direito enquanto universal é excludente. A pretexto de construir “o” sujeito, em
geral, excluem-se grupos inteiros, como mulheres, negros, povos indígenas etc. Assim, ao
colocar o debate sobre subjetividade jurídica no Brasil, pretende-se pensar justamente nas
pessoas que foram excluídas da proteção e reconhecimento do Direito, além de possibilitar
pensar o próprio Direito na relação com essas pessoas. O tema do ano, portanto, buscou
incentivar pesquisas que revelem o que se esconde na normatividade abstrata e técnica, a partir
das lentes da subjetividade jurídica não-universal e do comprometimento com os grupos
vulnerabilizados.
Assim, com base no tema central, o processo seletivo requisitou a leitura de autores e
autoras que tratam da subjetividade jurídica nos países periféricos e de abordagens críticas à
colonialidade. As obras selecionadas foram as seguintes: “A categoria político-cultural
amefricanidade” de Lélia González; “Colonialidade do poder, eurocentrismo e América Latina”
de Aníbal Quijano; “Direito e Relações Raciais - Uma introdução crítica ao racismo” de Dora
Lúcia de Lima Bertúlio; além do desenho “América Invertida”, de Joaquín Torres García. Na
segunda fase do processo seletivo, as candidatas apresentaram seus projetos de pesquisa com
base no tema de 2021.
Assim, cada dupla desenvolve, ao longo do ano, um artigo científico cujo tema específico
está abarcado pelo tema geral. Desde a escolha do recorte teórico, os trabalhos são
periodicamente compartilhados com o grupo, como forma de colocar em discussão ideias e
dificuldades de pesquisa, além de garantir o alinhamento de todas as pesquisas com o tema do
ano. Ao final, os trabalhos produzidos são reunidos em uma obra para publicação.
Na primeira reunião do PET sobre a pesquisa coletiva de 2021, as duplas apresentaram
seu recorte temático e objetivos. Em grupo, decidiu-se que a data de entrega do primeiro capítulo
seria no dia 18/06 e que uma dupla ficaria responsável pela leitura e sugestões sobre o capítulo
de outra dupla. Os outros integrantes do PET também fizeram comentários adicionais, assim
como a professora tutora orientou o encaminhamento do artigo.
Como prazo para o 2° capítulo, estabeleceu-se a data de 13/08. Após o segundo capítulo,
não houve discussão em reunião, mas somente por comentários da dupla responsável pela
revisão. A data final para entrega da introdução, terceiro capítulo e conclusão foi estabelecida
para o dia 10/09. O grupo decidiu que as reuniões para discussão do artigo completo fossem nos
dias 14/09 e 21/09. A data final para devolução pela professora dos nove artigos foi estabelecida
para o dia 01/10, com a data final do artigo pronto pelos discentes no dia 15/10.
Além da pesquisa, as atividades desenvolvidas pelo PET durante o ano também versam
sobre o tema geral, de forma a fornecer discussões e embasamentos teóricos para a pesquisa
final. Um dos eventos realizados nesse sentido foi o no dia 06/04/2021, intitulado “O que é
subjetividade jurídica?”. Durante as bancas do Processo Seletivo 2021, percebeu-se que os
convidados tinham diferentes concepções do que é subjetividade jurídica. Assim, tendo em vista
a relevância da discussão para o tema do ano, convidamos o Professor Ricardo Prestes Pazello e
o Lugan Thierry para aprofundar o tema, respondendo perguntas como: O que é a subjetividade
jurídica? Como pesquisá-la? Ela é a mesma coisa que o sujeito de direitos? O que significa se
reivindicar enquanto sujeito de direitos hoje? Ou melhor, o que pode um sujeito de direitos?
Em outro formato, mas também com o objetivo de melhor atravessar as encruzilhadas da
subjetividade jurídica brasileira a partir do Sul, com todas suas complexidades e perspectivas de
raça, gênero, classe e descoloniais, o PET Direito realizou um grupo de estudos chamado “Grupo
de estudos do PET: (re)imaginando a subjetividade jurídica”. Os textos escolhidos foram:
\"Discurso sobre o colonialismo\", de Aimé Césaire; \"Pele negra, máscaras brancas\", de Frantz
Fanon; \"Pelo espaço: uma nova política da espacialidade\", de Doreen Massey; e \"Saberes
localizados: a questão da ciência para o feminismo e o privilégio da perspectiva parcial\", de
Donna Haraway. O grupo de estudos contou com convidados como Daniel Fauth, Luis Fernando
Lopes Pereira, Maria Helena Lenzi. Assim, com base nas discussões promovidas sob o tema
anual, os membros do PET podem aprofundar suas pesquisas e concretizar os debates feitos no
grupo, com autonomia para definição do recorte específico do artigo, mas também contando com
tutoria e construção coletiva.

\includepdf{pdfs/Desenvolvimento-De-Pesqui}

\addcontentsline{toc}{section}{Diagnóstico Dos Ingressantes Do Curso De Agronomia Da Utfpr Campus Pato Branco E Efeitos Da Pandemia De Covid-19}

\section*{Diagnóstico Dos Ingressantes Do Curso De Agronomia Da Utfpr Campus Pato Branco E Efeitos Da Pandemia De Covid-19}

João Paulo Gonzatto, Cecília Eduarda Gnoatto Tomazini,Gabriela Pilatti,Bárbara de Farias,Ana Beatriz de Souza Serafim,Vitor Augusto Librelato,Regis Luis Missio,Universidade Tecnológica Federal do Paraná

Objetivou-se identificar o perfil dos ingressantes no curso de Agronomia da Universidade
Tecnológica Federal do Paraná (UTFPR)/Campus Pato Branco, bem como, identificar os efeitos
da pandemia de coronavírus no aspecto assistencial aos ingressantes. A metodologia foi
desenvolvida e aplicada com base em um questionário online, com perguntas de caráter
socioeconômico e relacionadas ao curso (interesses, perspectivas, dificuldades, etc.), enviados com
auxílio da coordenação do curso. O público alvo da pesquisa foram os ingressantes desde o ano de
2015 até o primeiro semestre de 2021. Contudo, para o segundo semestre de 2020 e o primeiro
semestre de 2021 algumas perguntas específicas sobre a pandemia de coronavírus foram
acrescentadas ao questionário dos novos ingressantes.
Foram 454 questionários respondidos desde o ano de 2015. Verificou-se que 61,7% dos
ingressantes do curso de Agronomia são do sexo masculino. Além disso, 68,5% dos alunos
afirmam terem cursado o ensino fundamental e médio em escola pública, 18,5% em escolas
particulares e 13% em ambas. A maioria dos ingressantes do curso de agronomia da UTFPRCampus Pato Branco é oriunda do meio urbano (68,5%), o que demonstra que os aspectos práticos
do curso de graduação assumem relevante importância para a formação profissional. Por outro
lado, 79,5% dos ingressantes no curso de agronomia possuem familiares ligados ao meio rural, o
que pode influenciar na decisão pela escolha do curso de agronomia. Dentre os ingressantes no
curso de agronomia, 80,8% admitem serem dependentes economicamente dos pais ou
responsáveis, uma vez que optaram por terminarem o ensino médio e ingressarem diretamente na
universidade.
Dentre os ingressantes do referido curso de Agronomia, verificou-se que que uma parcela
relevante (43,2%) é oriunda da região (até 100 km de distância de Pato Branco), dos quais 27,8% 
residem nesta cidade. Por outro lado, 15,8% dos entrevistados são oriundos de cidades com 100 a 
500 km de distância de Pato Branco e 13,2% de cidades com mais de 500 km de distância da cidade
em que se localiza o Campus de Pato Branco da UTFPR. A respeito de como conhecerama 
universidade, 42,8% afirmaram terem o primeiro contato com a universidade através de indicações 
do colégio, parentes ou amigos; 32,5% conheciam o campus através de visitas e eventos 
organizados pela UTFPR; 9,3% dos ingressantes residem próximo ao campus e 15,4% declararam 
ter conhecimento por via digital, o que demonstra a importância das mídias sociais para a
divulgação da universidade e atrair novos alunos.
Parte dos ingressantes (34,9%) do curso de Agronomia da UTFPR/Campus Pato Branco 
responderam dominar o idioma inglês, 13,5% dominavam o espanhol, 24,8% afirmaram ter
domínio parcial de outras línguas e 26,8% não compreendiam nenhum outro idioma além do
português. Dentre os ingressantes do curso de agronomia, 59,9% afirmaram morar com
os pais, 15,2% residiam sozinhos, 22,2% com amigos e 2,2% eram casados. Quanto às motivações 
para a escolha do curso de agronomia, 59,9% consideraram a escolha do curso de agronomia ser a 
melhor opção, 17,6% afirmaram possuir um futuro financeiro garantido, 19,4% esperavam 
conhecer melhor o curso e 3,1% alegaram não conhecer satisfatoriamente a profissãoescolhida. 
Estes resultados inferem que uma parcela dos ingressantes (20,7%) pode ter optado pelo curso de 
agronomia como segunda opção no processo de entrada. Em relação a fonte de renda, 90,5% 
informaram serem dependentes financeiramente dos pais ou responsáveis sem haver a necessidade
de auxílios disponibilizados pela instituição, 5,7% são independentes e 3,8% dependem dos
auxílios. Quando comparado aos ingressantes do período da pandemia (63 entrevistados), 28,9%
dos mesmos afirmaram depender de algum tipo de auxílio, o que demonstraos impactos nefastos da
pandemia sobre a economia e a renda das famílias brasileiras.
Os resultados das perguntas específicas sobre os reflexos da pandemia feitas para os
ingressantes do segundo semestre de 2020 e primeiro semestre de 2021, mostraram aspectos
interessantes. Uma das questões perguntava se o aluno possuía os meios necessários e suficientes
para acesso as aulas online, do total de 63 respostas, 92,1% afirmaram possuir os meios
necessários para acessar as aulas remotas, 3,2% não possuíam e 4,8% dependiam de horário ou
local para um acesso satisfatório. Outra pergunta estava relacionada ao acesso à internet de
qualidade, sendo que 84,1% possuíam acesso razoável, 6,3% não detinham, além de 9,5% que
dependiam de horário e local para acesso suficiente do sinal de internet.
Outras questões abordaram os impactos da pandemia sobre os ingressantes e suas famílias
(questão com múltiplas escolhas). Verificou-se que 33,3% dos entrevistados afirmaram não terem
sido afetados pela pandemia de nenhuma forma, 12,7% responderamque a saúde foi afetada, 25,4% 
tiveram a renda afetada, 33,3% tiveram o bem-estar afetado e 7,9% apresentaram outros aspectos 
afetados. Por fim, foi questionado aos 63 ingressantes do segundo semestre de 2020 e primeiro 
semestre de 2021 se eles continuariam a graduação e a posterior conclusão do curso de forma 
presencial na UTFPR Campus Pato Branco. Dentre o total de respostas, 100% delas afirmaram 
que iriam concluir o curso presencialmente após o término das medidas restritivas da COVID-19. 
Verifica-se que os efeitos negativos da pandemia afetaram a maioria dos estudantes quando
somadas as problemáticas de bem-estar, renda e saúde. Todos os entrevistados da pesquisadesejam
voltar às atividades na UTFPR de forma presencial assim que possível. Grande parte dos
ingressantes no curso de Agronomia não tiveram problemas com internet e meios necessários para
acessar as aulas no início de suas atividades acadêmicas em forma remota. Isso, de certa forma,
demonstra um bom nível socioeconômico dos ingressantes do curso de Agronomia da
UTFPR/Campus Pato Branco.
Os ingressantes do curso de Agronomia da UTFPR/Campus Pato Branco ainda são em sua
maioria do sexo masculino. Ações a fim de que estereótipos da profissão possam ser superados
são necessárias de forma que se possa aumentar a participação de mulheres em cursos das ciências
agrárias. Por fim, em função dos impactos econômicos da pandemia, pode-se destacar um aumento
do interesse aos auxílios estudantis, sendo necessário também a divulgação dos programas de
auxílios ofertados pela universidade. Da mesma maneira, assuntos relacionados à melhoria do
acesso às atividades remotas paraalunos com acesso restrito aos meios de comunicação. Apesar 
disso se manifestar numa porçãoreduzida dos ingressantes do curso, é um fato essencial para o 
desenvolvimento das disciplinas na forma de Atividades Pedagógicas Não Presenciais (APNP).

\includepdf{pdfs/Diagnostico-Dos-Ingressan}

\addcontentsline{toc}{section}{Diagnósticos Pet: Ferramentas E Práticas Estatísticas Para Qualificação Da Graduação}

\section*{Diagnósticos Pet: Ferramentas E Práticas Estatísticas Para Qualificação Da Graduação}

Leticia Barreto Assad Bruel, Alex de Lima Ferreira (UFPR),Ana Camille Kroin (UFPR),Felipe Adrian de Assis Vaz (UFPR),Heloisa Motelewski Trippia (UFPR),Rafaela Zimkovicz (UFPR)

Propondo a adesão das instâncias administrativas e pedagógicas ao modelo de oferta de
disciplinas à distância, em decorrência do enfrentamento da pandemia de Covid-19, a
Universidade Federal do Paraná, assim como outras instituições do país, acabou por suscitar a
constituição de uma conjuntura inédita, em meio à qual se tornou perceptível a extensão de
novas problemáticas à comunidade acadêmica. Tendo em vista suas repercussões primordiais no
entremeio organizacional das atividades de ensino e aprendizagem, através das propostas do
Ensino Remoto Emergencial (ERE), ocorrido entre os meses de julho de 2020 a março de 2021,
e de retomada do calendário acadêmico para o segundo semestre de 2020, realizada entre maio e
agosto de 2021, pautaram-se novos debates sobre a aplicação de ferramentas e ensino e avaliação
por meios integralmente remotos. Contudo, estes recém formados espaços de discussão
assinalaram uma participação discente limitada, engendrando, em um primeiro momento, um
distanciamento entre as esferas docentes e administrativas universitárias e a comunidade
estudantil. Desse modo, tal situação passou a mostrar-se como uma questão central para um
efetivo desenvolvimento de espaços virtuais de ensino conformados às necessidades e aos
anseios de suas/seus alunas/os/es. Nesse sentido, o grupo PET História UFPR, visando ao
estabelecimento de diálogos mais profícuos entre os grupos de docentes e discentes dos cursos
de História - Licenciatura e/ou Bacharelado e de História - Memória e Imagem, esboçou
mecanismos de integração da demanda estudantil aos espaços de planejamento dos novos
períodos letivos, congregando-os na aplicação dos Diagnósticos PET. Questionários
disponibilizados às/aos estudantes de ambos os cursos no intervalo dos referidos períodos, os
Diagnósticos pretenderam, por essa forma, a coleta de informações mais apuradas sobre o perfil
da comunidade discente, e, mais destacadamente, suas expectativas, necessidades e requisições
para a retomada das atividades acadêmicas nos novos espaços virtuais. Em termos
metodológicos, essas edições dos Diagnósticos PET corresponderam à estruturação de
formulários avaliativos virtuais por meio da plataforma gratuita Google Forms e de suas
possibilidades de combinação de perguntas de caráter quali-quantitativo. Para a elaboração deles,
amparamo-nos no intuito de formação de um survey descritivo, isto é, de um estudo panorâmico
que, por meio de uma mixagem entre questões objetivas e abertas, fosse capaz de indicar: I) os
principais aspectos socioeconômicos do corpo discente de cada curso, de modo a chamar atenção
para as distribuições de bolsas e auxílios, desigualdades internas e para eventuais dificuldades
acadêmicas decorrentes do cenário social do país; II) os percentuais de participação da totalidade
estudantil no sistema de ensino remoto e as razões mais pronunciadas de não participação; III) a
qualidade das ferramentas digitais e métodos de ensino e avaliação empregados no âmbito geral
das graduações, bem como os empecilhos de ordem psicoemocional enfrentados no cenário
pandêmico; e IV) a classificação dos níveis de funcionalidade, aproveitamento e utilidade das
disciplinas ministradas, visando a uma reunião de ponderações de práticas a serem alteradas nos
períodos posteriores, através da mobilização de dez critérios (autoavaliação; pontualidade,
aproveitamento, assiduidade; domínio de conteúdo; clareza e objetividade na exposição;
organização da disciplina; disponibilidade; pesquisa/extensão; avaliações; normativas do ERE).
Na primeira edição dos Diagnósticos, obteve-se 131 respostas, número que decresceu nas
próximas, 78 na segunda e 81 na terceira. Ao longo dos períodos, a taxa de adesão ao ensino
remoto aumentou: no primeiro, 16% não participaram, percentual que diminuiu para 12,8% no
segundo, sendo 9% que não aderiram a ambos os períodos e 3,8 que, após a experiência com o
primeiro, preferiram não continuar com as aulas virtuais. Apenas 4% não aderiu ao Ensino
Remoto no último período, que não foi mais facultativo - quem não realizou matrícula em pelo
menos uma disciplina não pôde concorrer a auxílios estudantis e sofreu trancamento
administrativo, ainda que esse tempo não tenha contado para o jubilamento. Comparando os
resultados das três edições, observou-se algumas melhorias de um período para o outro
motivadas pelo feedback proporcionado pelos Diagnósticos, como a troca de plataformas para
realização das aulas síncronas. Na primeira edição, a plataforma Jitsi foi a mais utilizada -
adotada, inclusive, pelo DEHIS para realização de reuniões internas -, porém obteve uma
avaliação muito baixa (2,64 em uma escala até 5). Os/as discentes apontaram como problemas a
instabilidade do software, assim como a dificuldade de gravação das aulas. No período seguinte,
boa parte das/os professores trocaram para outras alternativas, como o Google Meet ou o
Microsoft Teams, vinculado aos emails institucionais da universidade. Além disso, algumas
disciplinas foram ofertadas mais de uma vez desde que iniciou o Ensino Remoto, e o resultado
dos diagnósticos permitiu mudanças em seu planejamento e execução. Uma delas obteve na
primeira edição uma nota de 3,11 - média dos nove critérios -, e alguns comentários negativos
que remetiam a uma carga excessiva de trabalhos tendo em vista a duração mais curta do período
e a conjuntura sanitária e social. O parâmetro referente a isso obteve 2,08 de média. No semestre
seguinte, a docente mudou seu método, de modo que sua disciplina obteve média geral 4,33 e 4,5
no critério avaliações. Os Diagnósticos serviram ainda como apoio às coordenações para mapear
as necessidades discentes acerca de próximas disciplinas a serem ofertadas, auxiliando, assim,
não só o corpo docente, mas também proporcionando que os/as estudantes tivessem suas
demandas atendidas. A última pergunta do formulário é destinada para comentários e sugestões
acerca do formato dos Diagnósticos e, conforme as sugestões recebidas, eles foram modificados:
acrescentou-se uma seção para as disciplinas de estágio e orientação de monografia, além de
adicionadas questões sobre tipos de avaliação (prova, fichamento, ensaio etc). Com base nos
elementos acima dispostos, nota-se explícita concatenação do trabalho em análise com os
princípios de aprimoramento e qualificação progressiva dos cursos de graduação, inseridos como
entes basilares da premissa de promoção de educação tutorial que rege os grupos PET. Ao se
voltarem para o todo das estruturas de funcionamento dos períodos letivos da IES em questão, os
diagnósticos produzem um mapeamento das necessidades estudantis, o que se dá tanto por um
rankeamento numérico das condutas e metodologias docentes, quanto pelo levantamento de
considerações individuais, de manifestação subjetiva via escrita. Desse modo, obtém-se, ainda
que por meio de interações indiretas, um processo dialógico de incentivo à transformatividade
das dinâmicas que constituem o Ensino Superior, em consonância com os pressupostos petianos
de responsabilidade e capacidade de ação coletivas para agenciamento de mudanças na esfera
educacional

\includepdf{pdfs/Diagnosticos-Pet--Ferrame}

\addcontentsline{toc}{section}{Dossiê Petiano: Histórias, Afetos E Dinâmicas De Grupo}

\section*{Dossiê Petiano: Histórias, Afetos E Dinâmicas De Grupo}

Lucas Matozo Milan, Mariana Obino,Eloisa Akemi Nakao,Hugo de Alencar Ipólito,Larissa Michelle Lara

Em função dos acontecimentos decorrentes da pandemia do SARS-CoV-2, muito se 
discute sobre o isolamento social e a forma como esse contexto afeta nossas relações afetivas com 
o próximo. Tendo em vista as adaptações sociais realizadas virtualmente como forma de reconectar 
as pessoas ao meio social, o grupo PET Educação Física decidiu elaborar um projeto de 
aproximação entre os integrantes. Iniciado no ano de 2020, o projeto “Dossiê Petiano” teve como 
base a aproximação entre os membros do grupo.
A finalidade desse projeto é aproximar os/as petianos/as no intuito de se conhecerem 
melhor. Caracteriza-se como uma forma rápida, fácil e divertida de conhecimento do outro, de
seus anseios e histórias pessoais. Segundo Ortega (1999), o objetivo mais vantajoso para a 
educação social é alcançar a integração e a participação dos indivíduos e grupos numa sociedade 
conjunta. Observado que a atividade poderá trazer benefícios sociais ao grupo, o “Dossiê Petiano” 
proporciona ao grupo uma apresentação narrada pelos/as petianos/as, que contam, por exemplo, 
algumas experiências, hobbies, talentos, expectativas com o curso, entre outras informações 
pessoais que possam contribuir para a qualificação do trabalho do grupo.
A atividade “Dossiê Petiano” surge com o objetivo de aproximar os participantes e 
melhorar o vínculo entre acadêmicos/as do Programa PET Educação Física, e consequentemente, 
aprimorar a comunicação e o funcionamento do grupo. A ideia da criação do projeto inspira-se 
em um projeto do PET Pedagogia, intitulado Dia do Conto, que teoricamente, tinha o mesmo 
objetivo, porém se organizava de uma forma diferente, considerando a distribuição do tempo de 
apresentação, temáticas, entre outros. Os integrantes do grupo PET Educação Física participaram 
da apresentação do projeto “Dia do Conto” junto ao grupo PET Pedagogia e, então, levaram a ideia 
ao grupo. Com a aprovação de todos/as, o projeto foi implementado com o nome “Dossiê Petiano”.
Para as primeiras apresentações do projeto, o grupo selecionou algumas datas, distribuindoas entre dois a três petianos/as em cada encontro. As apresentações aconteceram durante as 
reuniões semanais do grupo e, conforme a demanda de pautas a serem discutidas, adaptava-se a 
quantidade de apresentações. Dessa forma, a primeira apresentação aconteceu no dia 27 de agosto 
de 2020 e a última no dia 08 de março de 2021. Conforme o grupo recebe novos integrantes, outras 
datas são organizadas para que calouros/as façam as suas apresentações. No total, o grupo teve 16 
apresentações voltadas para a vida pessoal de petianos/as, como, por exemplo, família, hobbies, 
viagens, entre outros. Assim, cada petiano/a escolhe o que quer apresentar ao grupo, ficando à 
vontade para incluir ou excluir alguns assuntos que não queira expor.
Considerando o Ensino Remoto Emergencial (ERE) e o formato online adotado para as 
reuniões do grupo PET Educação Física, as apresentações aconteceram pela plataforma Google 
Meet, com tempo pré-estabelecido de aproximadamente 10 minutos de fala. A forma de 
apresentação era livre, com possibilidade de mostrar slides, fotografias ou apenas uma fala sobre 
si. Após as apresentações, o grupo realizava perguntas e trocas de experiências com relação ao que 
fora apresentado por meio do chat da plataforma ou por microfone diretamente com o/a petiano/a.
A abordagem e a comunicação feita pelos/as petianos/as na atividade tornou o projeto 
atrativo, sobretudo pela importância da troca de afetos e também pela satisfação de conhecer as 
individualidades e os diferenciais de cada petiano/a no curso e em suas vivências pessoais. O 
projeto funcionou também como forma de expressão social em relação ao que o/a petiano/a
considerava de si mesmo, das pessoas ao seu redor, e também do curso que escolheu seguir, tendo 
sua relevância também durante a troca de experiências. Trata-se de uma forma de construção da 
história pessoal, sintetizada em minutos e, dependendo da quantidade de histórias e experiências
acumuladas, a seleção dos fatos que integrariam (ou não) a atividade ficava mais complexa.
Em relação aos dossiês apresentados, alguns fatos chamaram a atenção e são aqui 
destacados: a) a tutora do PET Educação Física contou que foi também petiana na UEM, no 
período de 1992 a 1995, e que a experiência no programa fez toda a diferença em sua formação; 
b) uma petiana caloura no curso de Educação Física, já formada em Engenharia Civil, escolheu o 
curso por motivação pessoal, notadamente pelo interesse na dança; c) vários petianos/as 
demonstraram ter animais de estimação e mostraram como eles qualificam seu cotidiano; d) 
vários/as petianos/as apresentaram suas famílias e demonstraram como o afeto familiar é 
importante em suas vidas. Por fim, a estruturação do projeto também foi um fator positivo ao 
grupo, considerando a simplicidade de execução e adaptação das diferentes vivências, propiciando 
mais intimidade entre os/as participantes, além da aquisição de novas habilidades, experiências 
profissionais, entre outros, tornando a convivência e a comunicação facilitadas, bem como a 
distribuição de funções entre seus membros.
A experiência “Dossiê Petiano” possibilita que os/as integrantes do PET Educação Física 
possam realizar uma apresentação, partilhando suas vivências, rotina e particularidades, tendo 
como objetivo a troca de experiências e a melhora do vínculo do grupo, ao desfrutar de um 
momento de integração. Devido ao momento de pandemia atual ocasionado pelo SARS-CoV-19, 
as reuniões do grupo PET Educação Física, no ano de 2021, acontecem de maneira remota e muitos 
integrantes ainda não se conhecem de modo presencial. Assim, o projeto é uma oportunidade de 
quebrar, em alguns aspectos, esse distanciamento, de modo que afetos possam ser criados a partir 
de momentos de aproximação entre seus membros a partir da história pessoal de cada um.
Dessa maneira, o “Dossiê Petiano” é classificado pelo grupo como uma experiência que 
contribui para o enriquecimento do coletivo e para o processo formativo. Seu desenvolvimento é 
dinâmico, haja vista que a ideia é relacionar temas diversos a serem trabalhados, como, por 
exemplo, a experiência profissional/acadêmica, a trajetória na universidade, entre outros, para que, 
assim, petianos/as possam se conhecer cada vez mais e refinar a capacidade de comunicação e 
diálogo entre si, sempre com conteúdo novo, ainda não apresentado.
REFERÊNCIA
ORTEGA, J. Educación social especializada. Barcelona: Ariel, 1999.

\includepdf{pdfs/Dossie-Petiano--Historias}

\addcontentsline{toc}{section}{E-Book De Receitas: O Aproveitamento Dos Alimentos Além Do Convencional}

\section*{E-Book De Receitas: O Aproveitamento Dos Alimentos Além Do Convencional}

Fernanda Dias Cardoso, Amanda Thais Heylmann,Thayse de Oliveira Schmalfuss,Alessandro de Oliveira Rios

O relatório intitulado “Índice de Desperdício de Alimentos 2021”, estudo liderado pela
ONU, revela que no ano de 2019, 17% da produção total de alimentos do mundo foram
descartados. De acordo com o estudo, a maior parte, cerca de 61%, é proveniente das residências
familiares e o restante resultante dos serviços de alimentação e do comércio em geral. Esses
dados são preocupantes visto a quantidade de pessoas em situação de fome e desnutrição no
Brasil e no mundo. A partir desses dados, o Programa de Educação Tutorial (PET) da Engenharia
de Alimentos da Universidade Federal do Rio Grande do Sul (UFRGS), desenvolveu o projeto
intitulado “E-book: receitas com subprodutos” com o objetivo de difundir informações e receitas
relacionadas ao aproveitamento integral dos alimentos, além de informar sobre as propriedades
nutritivas dos subprodutos utilizados, enriquecendo a alimentação humana. O e-book conta com
10 receitas, cada qual desenvolvida por um integrante do grupo. A publicação é separada em 4
capítulos que classificam as receitas de acordo com o principal subproduto utilizado. Cada
receita compreende 4 tópicos: ingredientes, com as quantidades em medidas caseiras e em peso;
descrição do modo de preparo, com cada etapa detalhada; item “PET Informa” com a descrição
dos benefícios do subproduto utilizado a partir de pesquisas científicas; e uma imagem do
alimento pronto registrada pelo próprio discente. O e-book representa um livro eletrônico
gratuito registrado com o ISBN 978-65-00-25430-3, sendo divulgado nas redes sociais do PET
Engenharia de Alimentos, no site da Universidade e em outros meios de comunicação como
forma a atingir o maior número de pessoas, interna e externamente à Universidade. Após 5 dias
da publicação do e-book, na página do Instagram do grupo, 428 impressões foram registradas e
53 ações executadas a partir da publicação, sendo elas 45 curtidas e 17 compartilhamentos. O
projeto, além de contribuir para a formação dos discentes através das pesquisas relacionados aos
subprodutos, contribui diretamente com a problemática ambiental relacionada à produção e
desperdício dos alimentos, e social referente à situação de fome e desnutrição. Em contribuição
com os pontos mencionados, o projeto colabora diretamente com a meta 3 do 12º Objetivo de
Desenvolvimento Sustentável (ODS) da ONU que pretende reduzir pela metade o desperdício de
alimentos per capita mundial até 2030.

\includepdf{pdfs/E-Book-De-Receitas--O-Apr}

\addcontentsline{toc}{section}{Encontros De Língua Estrangeira}

\section*{Encontros De Língua Estrangeira}

Gabriela Di Diego, Amanda de Campos Cerioli,Amanda Fernandes Alves,Felipe Pergher,Gabriele Pergher,João Vicente Cardoso Kohem,Natalia Fernanda Silveira da Pureza

Com o Ensino Remoto Emergencial (ERE) na UFRGS, acompanhado de aulas virtuais e
distanciamento social, percebemos uma lacuna na parte de conversação das disciplinas de
línguas do curso de Letras. As aulas assíncronas não permitem a prática da comunicação oral nas
línguas que estão sendo aprendidas pelos alunos e, mesmo nas aulas síncronas, muitos não se
sentem confortáveis para usar sua segunda língua oralmente na frente do(a) professor(a) e/ou
dos(as) colegas, já que as turmas geralmente são compostas por mais de 25 alunos Além disso,
vários discentes ao longo do curso vocalizaram sua dificuldade em permanecer em contato com
a(s) língua(s) que estudam durante o período de férias da Universidade, algo que foi parte da
motivação do presente projeto.
Pensando nisso, o PET Letras decidiu organizar os Encontros de Língua Estrangeira. Estes
são grupos de conversação ministrados por alguns de nossos bolsistas nos seguintes idiomas:
alemão, inglês e italiano. Foram disponibilizados grupos somente desses três idiomas porque são
eles os estudados pelos atuais bolsistas do PET que tinham disponibilidade para ministrar os
encontros. No entanto, o curso de Letras da UFRGS conta com mais quatro ênfases em línguas
modernas: Espanhol, Francês, Japonês e LIBRAS.
Para pensar o funcionamento dos Encontros, foi elaborado um formulário para sondagem dos
alunos do Instituto de Letras que acompanhava a inscrição no projeto, feito pelo Google Forms.
Nele, os interessados e interessadas foram indagados quanto ao seu nível na língua escolhida,
suas preferências de temas de estudo, disponibilidade de horários, etc. O formulário foi dividido
em três seções, uma para cada idioma que seria trabalhado. A partir dele, foram definidos os dias
e horários dos encontros, conforme a disponibilidade dos petianos e dos inscritos, resultando em
quatro grupos: um de alemão, dois de inglês (devido ao considerável número de inscritos) e um
de italiano. Tomamos cuidado para que as inscrições fossem encerradas no momento em que o
número de inscritos considerado ideal — cerca de 15 pessoas — fosse atingido. Foi definido que
os encontros ocorreriam semanalmente e teriam a duração de uma hora e trinta minutos. A
atividade teve início na primeira semana de julho de 2021, durante as férias de inverno, e cada
grupo teve em média cinco encontros. Os tópicos discutidos em cada semana foram decididos
em conjunto com os participantes de acordo com seus interesses e necessidades, assim como a
escolha de materiais, mediada pelos petianos responsáveis.
No geral, os tópicos mais pedidos foram aqueles relacionados a arte e cultura. Houve debates
e discussões sobre literatura, música, séries e filmes produzidos nos respectivos idiomas. Os
encontros foram sempre conduzidos de maneira informal. Então, apesar de os participantes
serem encorajados a usar somente a língua estrangeira em questão, poderiam falar em português
se tivessem alguma dificuldade. Dessa forma, tanto os ministrantes quanto os outros
participantes puderam ajudar uns aos outros conforme as dúvidas surgiam.
Como avaliação da atividade, em primeiro lugar, levamos em conta o número de inscritos
inicialmente. Ao todo, recebemos 66 inscrições: 15 para o italiano, 21 para o alemão e 30 para o
inglês. Naturalmente, nem todos que se inscreveram no primeiro momento tiveram
disponibilidade para participar de fato. Então, a presença também foi considerada: tivemos uma
média de 6 participantes por grupo e por dia. Para a emissão de um certificado de participação,
75% de presença era necessária. Assim, 20 pessoas estavam aptas para recebê-lo. O número
reduzido de participantes foi encarado de uma forma positiva por nós, pois assim todos têm mais
espaço para se expressar e se sentem mais confortáveis para tal.
As quatro ministrantes dos encontros de língua inglesa ainda solicitaram que os participantes
respondessem anonimamente um formulário de avaliação sobre o funcionamento do projeto. De
acordo com as respostas, os Encontros de Língua Estrangeira foram positivos para a prática da
língua estudada, os materiais de apoio foram adequados e há interesse em continuar participando
do projeto em uma edição futura. Dentre as nove pessoas que responderam o formulário, apenas
uma disse “não tenho certeza” para essas questões, e nenhuma respondeu “não”. Uma segunda
edição para os encontros de língua inglesa já está sendo planejada.
Os encontros de italiano e de alemão não tiveram avaliação formal, somente o relato dos
ministrantes desses grupos. No de italiano, os participantes gostaram muito das atividades, tanto
que logo que esta edição chegou ao fim, já se organizaram para continuar os encontros
independentemente do PET Letras, já que o ministrante do idioma não tinha disponibilidade. Isso
foi encarado de forma muito positiva por nós, pois significa que os alunos se apropriaram desse
espaço e deram continuidade para as nossas iniciativas. Por outro lado, no grupo de alemão, foi
descrito que não havia uma participação muito engajada e que não houve pedidos para
continuidade. Quando questionamos o porquê disso, a ministrante nos informou que utilizou uma
abordagem de revisão de conteúdos vistos nas aulas do idioma, diferentemente dos outros
grupos, que optaram por conversar sobre arte e cultura. Logo, essa abordagem não teve sucesso e
para que haja mais edições dos encontros de alemão, ela teria que ser modificada de modo que
se assemelhasse à dos outros grupos.
A criação de ambientes para treinar a comunicação oral em línguas estrangeiras foi positiva
tanto para os alunos do curso, quanto para os bolsistas do PET que ministraram os encontros. O
caráter informal possibilitou que os participantes se sentissem mais confortáveis para de fato
conversarem com outras pessoas no idioma em que estudam no curso de Letras. Acreditamos
que a prática da fala em línguas adicionais é imprescindível para seu aprendizado, e a troca de
experiências e indicações de obras que falam sobre determinada cultura é uma das melhores
maneiras de realizá-la.

\includepdf{pdfs/Encontros-De-Lingua-Estra}

\addcontentsline{toc}{section}{Ensino Remoto - Monitoria Digital}

\section*{Ensino Remoto - Monitoria Digital}

Rebeca Cristina Araujo De Almeida, Ana Flávia Spolti Ferreira

Introdução
Atualmente uma forma de estudo que vem crescendo cada vez mais e é utilizada por 
muitos estudantes, são as vídeo-aulas, um modo prático de alcançar discentes de qualquer parte
do Brasil e do mundo. Dessa forma, os alunos têm uma vasta quantidade de vídeo-aulas para 
escolher, mas que muitas vezes não tem uma continuidade de conteúdo ou são vídeos muito 
longos, abordando muitos tópicos de uma vez, o que pode acabar atrapalhando na concentração
do estudante.
Além disso, as vantagens da utilização dessa forma de aprendizado estão nos fatos de que 
um conteúdo já visto pode ser revisado quantas vezes for necessário para o entendimento, se um 
aluno precisa faltar na aula não será prejudicado se tiver o mesmo conteúdo disponível na 
internet para ver a qualquer momento, ademais, não há atraso do conteúdo em sala de aula por 
causa da falta de compressão dos alunos. [1] É importante ressaltar que nenhum desses fatores tira 
a importância das aulas em sala de aula, as vídeo-aulas funcionam apenas como um 
complemento daquilo que já foi visto.
Sentindo então uma falta de conteúdos específicos da Engenharia Química na internet de 
forma visual e visando a disseminação do conhecimento científico, o Programa de Educação 
Tutorial de Engenharia Química da Universidade Estadual de Maringá (UEM), promove a 
elaboração de vídeo-aulas que são postados no canal do YouTube, abordando matérias da grade 
da Engenharia Química, como por exemplo, Introdução à Engenharia Química (durante o ano de 
2020). Atualmente estão sendo produzidos os conteúdos de Fundamentos de Engenharia 
Química, Fenômenos de Transporte, Materiais e Utilidades e Termodinâmica, e o objetivo desse
projeto de ensino é aumentar cada vez mais a quantidade de conteúdos abordados.
Metodologia
O PET Engenharia Química da UEM é estruturado em várias comissões. A comissão 
responsável pelas vídeo-aulas é a Monitoria Digital, composta por 5 membros, sendo um deles o 
coordenador, responsável por estruturar a comissão e dar andamento na atividade. Essa comissão 
fica responsável por escrever os roteiros da parte teórica e por escolher e resolver os exercícios 
das matérias propostas. Todos os roteiros teóricos e de exercícios passam por revisão de um 
professor qualificado. 
Essa comissão também fica encarregada da parte da edição das vídeo-aulas, mas a 
gravação das mesmas é disponibilizada para todos os PETianos. O PETiano interessado em 
ministrar uma aula fica encarregado de fazer os slides do conteúdo com base em um arquivo 
padrão no PowerPoint, formulado pela comissão. O ministrante também tem acesso a um POP 
(Procedimento Operacional Padrão) [2]
formulado pela comissão, no qual constam instruções de
gravação, escrita de roteiros e edição do PowerPoint. A aula é gravada utilizando um software de 
gravação da tela do computador, sem mostrar o PETiano ministrante. Utilizam-se fones de 
ouvido para gravar o áudio das aulas e um áudio externo (celular), para garantir a qualidade do 
som das vídeo-aulas. 
Depois de editadas, as aulas são postadas no canal do YouTube e é realizada a divulgação 
das vídeo-aulas no Instagram do PET Engenharia Química da UEM. Também é elaborado 
mapas mentais de todas as aulas teóricas, com o objetivo de ajudar os graduandos na fixação da 
matéria. Esses mapas mentais são disponibilizados nas redes sociais do PET Engenharia Química 
da UEM. Foi efetuada uma pesquisa com a graduação sobre o impacto das vídeo-aulas 
disponibilizadas no YouTube. Essa pesquisa foi realizada por meio do formulário do Google 
Drive e teve 19 respostas.
[3] Além disso, também foi exportada uma planilha do YouTube Studio, 
na qual obteve-se vários dados pertinentes aos vídeos postados. [4]
Resultados e discussão
Com base na pesquisa feita com a graduação, foi possível analisar que a vídeo-aula é um 
dos três melhores métodos de estudo, pois 47,2% dos graduandos responderam que esse é o 
método que mais funciona para eles. Ademais, analisando os dados do YouTube foi possível
aferir a média de visualizações de todos os vídeos postados, dando uma média de 
aproximadamente 212 visualizações por vídeo. 
Outro ponto importante da pesquisa, foi que 78,9% responderam que preferem vídeoaulas de resolução de exercícios. A partir desse resultado, foi implementado mais vídeo-aulas de 
exercícios com o quadro “PET Resolve”, voltado para a resolução de exercícios das principais 
matérias do terceiro ano da graduação. Esse quadro em específico, somente os PETianos do 
quarto ano podem gravar as aulas, pois são os únicos que já realizaram essas matérias. 
Por fim, foi averiguada qual a frequência que os graduandos recorriam às vídeo-aulas do 
PET Engenharia Química. Em uma escala de 1 a 5, sendo 1 muito pouco e 5 muitas vezes, 
26,6% responderam 2; 42,1% responderam 3; 26,3% responderam 4 e 5,3% responderam 5. 
Analisando os dados obtidos é possível constatar que as vídeo-aulas disponibilizadas no 
YouTube do PET Engenharia Química da UEM impactam positivamente a graduação, auxiliando 
no aprendizado dos alunos. 
Conclusão
Em conclusão, o projeto realizado tem uma grande importância e afeta diretamente os 
estudantes de Engenharia Química da Universidade Estadual de Maringá, como foi possível 
perceber pela pesquisa feita. Além disso, o projeto tem atuado como um agente auxiliador de 
ensino disseminando o conhecimento científico para lugares além da UEM, como foi mostrado
pelos dados obtidos do canal do YouTube.
Referências 
[1] CANDEIAS, C. N. B., CARVALHO, L. H. P. O uso de videoaulas como ferramenta no 
processo de ensino e aprendizagem em química. 7º Simpósio Internacional de Educação e 
Comunicação (SIMEDUC), Aracaju (SE), ISSN: 2179-4901, Setembro de 2016.
[2] SPOLTI, A. F., PETTENUCI, B., VINÍCIUS, J., FAVARETTO, L. Procedimento 
Operacional Padrão Monitoria Digital. 2020. Disponível em . 
Acesso em 14 de setembro de 2021.
[3] MEIRA, G. J., IMAMURA, N. M. Pesquisa com os calouros. 2021. Disponível em
. Acesso em 14 de setembro de 2021. 
[4] Canal do YouTube PET Engenharia Química UEM. 2021. Disponível em <
https://docs.google.com/spreadsheets/d/19xoAUtLRqIBoEbdOYbBYvjm5STAuHPbcgb07TkeG
H08/edit?usp=sharing>. Acesso em 23 de setembro de 2021.

\includepdf{pdfs/Ensino-Remoto---Monitoria}

\addcontentsline{toc}{section}{Entrevista Coletiva X Individualizada Na Seleção Do Pet Geografia Da Uel: Uma Análise Comparativa}

\section*{Entrevista Coletiva X Individualizada Na Seleção Do Pet Geografia Da Uel: Uma Análise Comparativa}

Mariana Mantovani De Quadros Rapacci, Caio Akyama da Silva,Júlia Casagrande Luiz,Lucas de Freitas Botega,Marcelo Correa Porto,Nathalya Glenda Mayer Chagas,Tainá Araujo,Jeani Delgado Paschoal Moura

Em 2020, devido ao cenário pandêmico causado pela disseminação do vírus SARS-Cov-2 e as significativas alterações cotidianas promovidas visando implementar medidas sanitárias para o combate da Covid-19, tornou-se necessário adaptar-se à realidade experienciada e os diversos desafios que surgiram ao longo do enfrentamento da pandemia. Inserido nesse contexto, o grupo PET de Geografia da Universidade Estadual de Londrina (UEL) se viu diante da necessidade de replanejar grande parte de suas atividades, entre elas o processo de seleção para novos bolsistas, levando em consideração a adoção do sistema de ensino remoto e, consequentemente, as novas formas de interações coletivas exclusivas em ambientes virtuais.

Desse modo, este trabalho teve por objetivo realizar uma análise comparativa, com base nos relatos de experiência dos petianos (as), entre o modelo de entrevista individualizada, empregado nos processos seletivos anteriores à pandemia, e o modelo de entrevista coletiva, realizado em junho de 2021, a fim de suprir as demandas impostas pela pandemia. Haguette (1997) conceitua entrevista como um processo de interação social, com o objetivo de se obter informações por parte do outro, o entrevistado. Tanto as entrevistas individualizadas, quanto as coletivas, aplicadas no âmbito das seleções do PET, buscaram uma imersão nesse processo de interação social, para obter, da melhor forma possível, informações objetivas e subjetivas dos candidatos, visando avaliar o potencial para o perfil que se espera formar em um futuro petiano.

Com o intuito de se fazer uma análise comparativa entre os dois modelos de entrevistas aplicados nos processos de seleção, foi elaborado um questionário através da plataforma on-line Google Forms e, posteriormente, aplicado ao grupo PET de Geografia, incluindo seus bolsistas e colaboradores.

Através das respostas obtidas pelo questionário foi constatado que 86,7% dos petianos (as) participaram do modelo de entrevista individual em seu processo seletivo, enquanto somente 13,3% participaram do modelo de entrevista coletivo. Percebe-se que, com base nos relatos obtidos, a maioria dos participantes que foram entrevistados individualmente se sentiram intimidados devido a presença dos demais petianos (as) durante a entrevista, porém, um fator

tranquilizador deste modelo em relação ao modelo coletivo é a ausência dos demais concorrentes para a vaga pretendida. Já dentre os petianos (as) que foram entrevistados no modelo coletivo, poucos relatam terem se sentido desconfortáveis em algum momento do processo, sendo o maior fator de intimidação a presença dos demais concorrentes e a realização de dinâmicas em grupo.

Quando questionados a respeito de qual modelo de entrevista possibilitou aos entrevistadores analisarem melhor os perfis individuais de cada um dos candidatos para a vaga, 26,7% dos petianos (as) afirmaram ser o modelo de entrevista individualizado, enquanto 33,3% acreditam que o modelo coletivo é o mais adequado para tal análise e 40% dos participantes consideram que ambos os modelos possibilitam igualmente conhecer os participantes e seus interesses. Em contrapartida, quando questionados qual modelo de entrevista permite melhor ao candidato expressar seus interesses e demonstrar as suas aptidões, 40% dos participantes responderam ser o modelo individualizado, 40% optaram pelo modelo coletivo e 20% afirmaram que ambos são adequados para tal.

Por fim, a maioria dos entrevistados apontam que o modelo de entrevista coletiva, adotado em virtude da pandemia, é considerado mais enriquecedor para a seleção do grupo PET de Geografia devido ao seu caráter dinâmico e que fomenta maior interação entre entrevistadores e candidatos. Além disso, o processo de elaboração deste modelo de entrevista também se sobressai em relação a entrevista individualizada, exercitando a criatividade individual, a colaboração em grupo e o fortalecimento das relações interpessoais entre o grupo PET. A intenção deste trabalho não foi a de discorrer de forma exaustiva sobre a técnica de entrevista, mas a de confrontar as duas formas aplicadas, demonstrando que ambas apresentam limitações, sobre as quais se deve desenvolver um olhar cauteloso para tornar o processo avaliativo o mais adequado possível para os fins que se almejam.

Nesse sentido, este trabalho conclui que, mesmo diante dos inúmeros desafios impostos pela pandemia, o grupo PET de Geografia da UEL foi capaz de se reinventar e apresentar resultados positivos que poderão ser utilizados para além deste momento e gerar impactos significativos na elaboração e aplicação dos próximos processos seletivos.

REFERÊNCIAS:

HAGUETTE, Teresa Maria Frota. Metodologias qualitativas na Sociologia. 5ed. Petrópolis: Vozes, 1997.

\includepdf{pdfs/Entrevista-Coletiva-X-Ind}

\addcontentsline{toc}{section}{Flores De Corte Como Fonte De Renda Aos Pequenos Produtores Rurais No Alto Vale Do Itajaí, Sc}

\section*{Flores De Corte Como Fonte De Renda Aos Pequenos Produtores Rurais No Alto Vale Do Itajaí, Sc}

Vinicius Petermann Benedicto, Alexandra Goede de Souza,Daniela Munch; Eduardo Affonso Jung;  Gabrieli Wasilkosky; Maria Luiza Rodrigues Soriano de Aquino

A floricultura brasileira vem se desenvolvendo nos últimos anos, apresentando crescimento anual de 8% a 10% no volume de plantas produzidas, e um dos ramos de produção na floricultura são as  
 flores de corte que podem representar uma ótima alternativa de diversificação e geração de renda para pequenos produtores rurais, especialmente na região do Alto Vale do Itajaí, SC, onde predominam pequenas propriedades rurais. Desta forma objetivou realizar a produção e e acompanhamento com 3 tipos de flores de corte em propriedades da região do Alto Vale do Itajai-SC, e avaliar a produtividade e a viabilidade da comercialização para rentabilizar pequenas propriedades. Como resultado obteve-se bom desempenho das diferentes espécies de plantas, apresentaram baixo custo de produção e ótima viabilidade para comercialização local das hastes.

\includepdf{pdfs/Flores-De-Corte-Como-Font}

\addcontentsline{toc}{section}{Futuro Profissional - Viabilizando O Intercâmbio De Saberes}

\section*{Futuro Profissional - Viabilizando O Intercâmbio De Saberes}

Lucas Jardim Da Silva, Laura Lourenço Morel, Lara Krusser Feltraco, Jéssica Ellen Gomes, Júlia Rodrigues Burket, Laura Barreto Moreno e Josué Martos.

1. INTRODUÇÃO
Nota-se grande diferença do papel do Cirurgião-Dentista de hoje com o do passado, o leque de  opções de trabalho está ainda maior bem como as possibilidades de especialização, no entanto,  bem como a ampliação nas áreas a serem trabalhadas o Cirurgião Dentista (CD) também conta  com a maior oferta de profissionais a procura de trabalho e com experiência de carreira semelhantes. Tendo em vista as dificuldades encontradas dos recém graduados na busca por um futuro  profissional satisfatório e que atenda às suas expectativas, o presente Projeto de Ensino traz aos alunos participantes do Programa de Educação Tutorial em Odontologia (PET – Odontologia) o  projeto de ensino intitulado “Futuro Profissional”, onde os ex-integrantes do grupo PET que já  tiveram experiência no mercado de trabalho ou estão inseridos, retornam para palestras e discussões com o grupo, quanto às suas vivências, dificuldades, escolhas de formação complementar e experiências pessoais após a graduação, até o momento em que se encontram. Possibilitam com esse link entre os acadêmicos e profissionais, as vivências e possibilidades futuras de carreira após a formação, bem como serve como guia para a identificação ou não dos mesmos à essas possibilidades.

2. METODOLOGIA
A atividade é desenvolvida com a participação de egressos do PET, professores e outros profissionais da área da saúde que são convidados a relatarem sua trajetória acadêmica adquirida durante a graduação e a pós-graduação. A escolha dessa pessoa é realizada pelo grupo durante nossas reuniões semanais e a presença dos nossos egressos do PET é maioria quando analisamos os últimos eventos. A atividade é realizada na plataforma Google Meet ou WebConf da UFPel, com a presença do tutor e todos os componentes do grupo. Os convidados apresentam sua trajetória profissional e acadêmica e após isso tiram as dúvidas dos integrantes do grupo. Ao final, nosso convidado é presenteado com um certificado de participação no nosso evento. Antes do período pandêmico essa atividade ocorria de forma presencial na sala do PET Odontologia na Universidade Federal de Pelotas. 

3. RESULTADOS E DISCUSSÃO
Assim como já documentado por PEREIRA et al. (2007), na questão da escolha do futuro profissional, os dados sugerem que a influência social, particularmente para o estabelecimento de uma carreira como meta, é percebida principalmente como originária dos membros com experiência profissional, que participam da rede social dos acadêmicos. No caso da escolha profissional, a participação principal destes, parece estar baseada na experiência que estes membros trazem ao grupo quanto a suas próprias escolhas profissionais. O atual Projeto do Futuro Profissional, traz à tona os temores, angustias, anseios e expectativas dos acadêmicos, no que se refere ao término de sua graduação, o que por sua vez é acompanhado e segue sob orientação do tutor do grupo, as dúvidas nos diferentes aspectos abordados nas palestras oferecidas pelos egressos do PET, são sanadas bem como novos questionamentos sobre o tema surgem, a fim de elucidar e muitas vezes trazer à tona possibilidades antes esquecidas pelos próprios acadêmicos quanto ao seu futuro como CD. O trabalho ainda encontra-se em andamento, uma vez que como o grupo alvo trata-se dos acadêmicos do grupo PET Odontologia, este está sempre em constante mudança, assim que ocorre a colação de grau, os membros deixam de fazer parte do grupo e partem ao mercado de trabalho, colocando em prática aquilo que foi possível extrair de conhecimento das atividades passadas com o Futuro Profissional, e ainda assim com a partida dos membros mais antigos o grupo volta a se renovar com uma nova seleção onde acadêmicos dos semestres iniciais ingressam, dando 
continuidade ao trabalho que está sempre se renovando em busca de sanar e criar possibilidades de carreira para os acadêmicos, atendendo a demanda que eles mesmo criam e entendem como necessárias para si, ou ainda aquela vistas como essenciais pelo tutor do grupo. Esse retorno dos egressos do PET ainda é percebido como fundamental pelo grupo, uma vez que estes retornam ao local de origem de sua formação acadêmica, possibilitando a transmissão de seu conhecimento, experiências e ponderações sobre como se deu a sua conclusão da graduação. Tudo isso ajuda a sanar dúvidas dos acadêmicos do grupo PET- Odontologia e os mesmo manifestam-se favoráveis ao projeto de ensino e ainda extremamente satisfeitos com a possibilidade de se conseguir essa ligação entre a graduação que os cerca e o que o mundo do mercado de trabalho lhes reserva.

4. CONCLUSÕES
A partir da explanação acerca das possibilidades de futuro profissional, os petianos poderão refletir sobre o seu papel dentro da Instituição e como podem contribuir positivamente no seu curso, instituição e sociedade e qual o impacto destas ações sobre o teu próprio futuro profissional. Concluímos também que a atividade promove a formação ampla e de qualidade acadêmica dos petianos estimulando a formação de valores que reforcem a cidadania e a consciência social de todos os participantes.

5. REFERÊNCIAS BIBLIOGRÁFICAS
PEREIRA, F.N; GARCIA, A. Amizade e escolha profissional: influência ou cooperação? Revista Brasileira de Orientação Profissional. 2007;8(1):71-86

\includepdf{pdfs/Futuro-Profissional---Via}

\addcontentsline{toc}{section}{I Ciclo De Debates Socioambientais Do Grupo Pet Conexões - Gestão Ambiental: Organização E Resultados}

\section*{I Ciclo De Debates Socioambientais Do Grupo Pet Conexões - Gestão Ambiental: Organização E Resultados}

Willian Axl Espindola, Julia Detzel Kipper - IFRS POA,Evelyn Dias dos Santos - IFRS POA,Leandro Maciel de Abreu - IFRS POA,Valesca Martins Thumé - IFRS POA,Celson Roberto Canto Silva - IFRS POA

O I Ciclo de Debates Socioambientais do grupo PET Conexões - Gestão Ambiental do Instituto
Federal do Rio Grande do Sul - Campus Porto Alegre (IFRS) tratou-se do primeiro evento de
extensão virtual remoto promovido pelo grupo PET. Teve como objetivo organizar um espaço
para a discussão de tópicos atrelados às questões socioambientais a partir da participação de
convidados que atuam em projetos, empresas, instituições e também profissionais de diferentes
áreas, no intuito de socializar suas experiências com a comunidade em geral. Delineou-se que o
tema principal do evento seria Cidade e Meio Ambiente, uma alusão ao tema do Dia Mundial do
Meio Ambiente de 2021 - Restauração de Ecossistemas.
O evento foi constituído por três encontros que ocorreram ao vivo, sendo transmitidos pelo canal
do grupo PET na plataforma de compartilhamento de vídeos YouTube, utilizando a Conferência
WEB - RNP como serviço de comunicação para a organização das salas virtuais. Após a
definição do cronograma do evento, programou-se três temáticas vinculadas ao tema principal
em cada dia de encontro, assim determinadas: projetos ambientais: perspectivas e desafios;
pandemia e resíduos sólidos; e cultura e meio ambiente.
Para a organização do evento, o grupo PET dividiu-se em três comissões, com tarefas definidas,
como: contato com os palestrantes e intérpretes de Libras, elaboração de planilhas (de inscrição,
presença e avaliação), elaboração de roteiros de mediação, atuação como mediador, testes de
transmissões, elaboração de publicações nas redes sociais e divulgação do evento, dentre outras
funções. Os encontros tiveram duração, em média, de 90 minutos, sendo constituídos por um
mediador, que era um dos bolsistas do grupo, dois painelistas convidados para a Mesa Redonda,
um bolsista responsável pela transmissão, duas intérpretes de Libras e os demais bolsistas
responsáveis pela interação com o público durante a transmissão.
Quanto ao público, o evento teve um total de 206 inscritos e 108 ouvintes efetivos. Como
resultado das três transmissões gravadas, observamos que estas tiveram, até o momento, 820
visualizações. Houve um decréscimo de audiência entre os três encontros, o que de certa forma é
bem comum nesse tipo de evento. Entretanto, é possível que o espaçamento de duas semanas
entre cada encontro possa ter colaborado para isso.
Com base na avaliação da satisfação dos ouvintes, realizada através do preenchimento de
questionário de avaliação enviado aos participantes após o fim do evento, constatou-se que: o
principal motivo para a não participação de todos os encontros foi o esquecimento (57,2%); a
maioria do público do evento foi de estudantes, professores, ex-estudantes do IFRS (82%) e
pessoas externas ao IFRS (18%). Quanto à informação sobre a ocorrência do evento, cerca de
48,4% ficou sabendo através de amigos, cerca de 33,3% através das redes sociais e cerca de
18,2% através da página do IFRS. Observou-se que a quase totalidade (97%) respondeu que
recomendariam o evento para conhecidos. No questionário de avaliação também foi
disponibilizada a opção de adicionar sugestões, críticas e apontamentos: 45,45% das
manifestações contiveram sugestões diversas (temáticas novas, sugestão de projetos a serem
abordados em novas edições e questões organizacionais), 30,3% foram de respostas neutras (sem
sugestões, críticas ou apontamentos) e 42,42% contiveram elogios diversos quanto à organização
do evento, ao evento em si e as temáticas abordadas, aos convidados e à equipe organizacional.
A partir desta avaliação e da avaliação realizada pela equipe executora, notou-se que o evento
atingiu seus objetivos, apesar de não ter alcançado plenamente o público estimado. Cabe
destacar, entretanto, que esta estimativa foi feita baseada apenas na quantidade de ouvintes em
eventos virtuais remotos similares ocorridos durante a pandemia. Observamos também que
algumas questões organizacionais podem ser melhoradas em futuras edições ou eventos
semelhantes a este, tais como a utilização otimizada das plataformas de transmissão e um menor
espaçamento entre os encontros. Por fim, o desenvolvimento de atividades de extensão em
grupos oferece maior dinamicidade na realização de demandas, desenvolvimento de trabalho em
equipe e individual, o que auxilia na formação pessoal e profissional do estudante bolsista.

\includepdf{pdfs/I-Ciclo-De-Debates-Socioa}

\addcontentsline{toc}{section}{I Frutipet: Frutíferas Para Pequenas Propriedades Rurais}

\section*{I Frutipet: Frutíferas Para Pequenas Propriedades Rurais}

Edilson Malikoski, Jonas Linzmeyer,Lucas Odorizzi,Marino Jubanski,Victor Lacerda,Yuri Back Salvador,Alexandra Goede de Souza

A fruticultura é um dos setores do agronegócio de grande destaque no cenário nacional, 
especialmente devido à grande variedade de frutíferas produzidas em todo o país, proporcionado 
pela diversidade edafoclimática. A região do Alto Vale do Itajaí, SC possui clima subtropical 
úmido com verão quente, Cfa, segundo a classificação de Köppen, permitindo o cultivo de 
frutíferas de clima temperado. O cultivo de nogueira pecan (Carya illinoinensis), goiabeira-serrana 
(Acca sellowiana) e o consórcio de frutíferas com outras espécies vegetais em sistema agroflorestal 
(SAF) são projetos implantados pelo programa PET Agroecologia Rural Sustentável do Instituto 
Federal Catarinense – Campus Rio do Sul. O objetivo dos projetos é o estudo da adaptação das 
culturas à região, empregando o manejo agroecológico na produção, além de servir como 
laboratório didático e de pesquisa aos estudantes do curso de Agronomia da Instituição e ofertar a 
extensão direta ao produtor rural por meio da divulgação dos resultados dos trabalhos, na tentativa 
de trazer para discussão temas importantes como a produção agroecológica de alimentos.
Com a atual crise sanitária vigente no mundo em decorrência de Sars-CoV-2 (Covid-19), 
fomos submetidos a situação de pandemia e isolamento social. Com isso, o contato humano e as 
relações interpessoais modificaram e se tornaram mais distantes. A extensão rural, responsável 
pela transmissão do conhecimento técnico-científico, gerado nos trabalhos de pesquisa, aos 
produtores rurais também sofreu grande impacto. Tradicionalmente, as ações de extensão nesta 
área sempre foram realizadas por meio de dias de campo, oficinas, visitas técnicas, entre outros. 
Todos envolvendo o contato direto entre as pessoas. Para mitigar os prejuízos, foi necessário 
procurar caminhos alternativos para que as informações chegassem aos produtores rurais e 
comunidade acadêmica e geral. A utilização das mídias sociais e plataformas digitais foi a 
alternativa encontrada pelo PET Agroecologia Rural Sustentável para divulgar os resultados dos 
trabalhos de pesquisa desenvolvidos no IFC – Campus Rio do Sul. 
Desta forma, foi organizado o “I FRUTIPET - Frutíferas para pequenas propriedades 
rurais” (Figura 1A), para apresentar à comunidade os trabalhos da área de fruticultura realizados 
pelo grupo e os principais resultados alcançados, mostrando que é possível a produção de frutíferas 
adotando o sistema agroecológico de cultivo. O evento foi totalmente online na forma de 
Webinário (Figura 1D), e organizado, transmitido e apresentados por alunos bolsistas do PET.
Foram apresentados os trabalhos desenvolvidos com as culturas da nogueira pecan, 
goiabeira-serrana (Figura 1B) e frutíferas diversas cultivados no SAF (Figura 1C). Após a abertura 
as seguintes palestras foram abordadas: Goiaba-serrana no Alto Vale do Itajaí; Cultivo 
Agroecológico de nogueira pecan; Fruticultura em Sistemas Agroflorestais; e Extrato de alho 
como regulador de Crescimento. Também foi aberto espaço, ao final das apresentações, para 
perguntas do público, as quais foram respondidas pelos apresentadores.
O evento ocorreu no dia 15 de julho de 2021, e foi transmitido pelo canal do PET 
Agroecologia no YouTube, apresentando resultados obtidos através dos trabalhos técnicocientíficos realizados na área da fruticultura (Figura 1E, F, G e H). 
Houve a participação de diversos alunos dos cursos de Agronomia e Técnico em 
Agropecuária e Agroecologia da instituição, além de professores, produtores rurais, extensionistas 
e pesquisadores da área de fruticultura. Além de ser uma forma de abordar com segurança os temas 
propostos, o evento online permitiu a interação com o público, por meio das respostas aos 
questionamentos realizados, e maior alcance, uma vez que houve o total de 419 pessoas 
visualizando o evento.
Este trabalho confirmou a importância da utilização das plataformas digitais como veículos 
de divulgação de conteúdos técnico-científico para projetos extensionistas na esfera da produção 
de frutíferas em sistema agroecológico, aumentando a popularização deste tema tão caro às 
questões produtivas e ambientais, compartilhando conhecimento e despertando a importância do 
cultivo agroecológico para produção sustentável de alimentos.
Por fim, vale ressaltar que o uso de plataforma digital escolhida, representou uma 
ferramenta importante para a divulgação dos conteúdos técnico-científicos produzidos pelo PET, 
e que exigiu por parte dos envolvidos não só a capacidade crítica e conhecimento acadêmico, mas 
também é capacidade de traduzir conteúdos complexos para que sejam apresentados de forma 
fluida e de fácil entendimento para diversos atores da sociedade, fazendo-se cumprir os principais 
pilares da extensão universitária; Além da experiência que representa a organização de eventos.

\includepdf{pdfs/I-Frutipet--Frutiferas-Pa}

\addcontentsline{toc}{section}{Ii Tecnoleite: Tecnologias Aplicadas A Produção De Leite – O Pet Produção Leiteira Em Atuação No Setor Agropecuário}

\section*{Ii Tecnoleite: Tecnologias Aplicadas A Produção De Leite – O Pet Produção Leiteira Em Atuação No Setor Agropecuário}

Renan Quisini, Ilana Niqueli Talino dos Santos,Joeliton Campani dos Santos,Luiz Fernando Klein,Mateus Gomes,Bruna Meirelles Fragata,Fernando Kuss

A atividade do leite se destaca na Região Sudoeste do Paraná, neste estado e no Brasil, envolvendo uma longa cadeia de produção, gerando renda a milhares de famílias do campo e da cidade, bem como participando significativamente do produto interno bruto (PIB) do nosso país. Eventos que promovam a disseminação de tecnologias é fundamental para o aprimoramento tecnológico dessa cadeia. Neste sentido, embora com a pandemia do COVID-19, o PET Produção Leiteira não paralisou suas atividades de extensão e executou o II TECNOLEITE - Tecnologias Aplicadas a Produção de Leite. Este projeto buscou a transferência de tecnologias para produtores, técnicos extencionistas, docentes, pesquisadores e alunos de graduação e pós-graduação, e deu continuidade as ações do PET Produção Leiteira em âmbito mundial por meio da transmissão de palestras online pelo YouTube. Contribuindo com o desenvolvimento da pecuária leiteira.

\includepdf{pdfs/Ii-Tecnoleite--Tecnologia}

\addcontentsline{toc}{section}{Impacto Da Renda Na Formação Dos Jovens Petianos}

\section*{Impacto Da Renda Na Formação Dos Jovens Petianos}

Natacha Rosane Antonio Coelho, Bruna Beltrame dos Santos

1. INTRODUÇÃO
Este artigo pretende analisar o impacto da renda, em específico, na formação dos jovens
petianos, expondo e desenvolvendo os desafios enfrentados por eles. Com isso, a partir de uma
pesquisa realizada pelo grupo PET Engenharia Civil da UFRGS, no ano de 2021, que será
melhor descrita ao longo do texto, busca-se aprofundar quais são os aspectos mais afetados pela
renda na vida estudantil e como os petianos amenizam-nos com o auxílio financeiro.
Esta pesquisa teve como objetivo principal analisar a importância do auxílio financeiro
proveniente do Programa de Educação Tutorial (PET), do FNDE/MEC, na vida dos estudantes
petianos da UFRGS. O tema ganhou visibilidade no ano de 2021 devido aos constantes
problemas e dificuldades vividos pelos petianos em relação ao custeio, despertando assim o
interesse em analisar quais são os principais impactos da renda na formação destes jovens, já que
apesar de toda discussão, normalmente o impacto real na vida dos estudantes não têm recebido a
mesma visibilidade, mesmo sendo algo decisivo para a permanência dos discentes no programa
de educação tutorial e na universidade.
2. METODOLOGIA
Realização de coleta de dados, através do preenchimento de um formulário on-line
promovido pelo grupo PET Engenharia Civil da UFRGS intitulado “Formação dos petianos:
impacto do auxílio financeiro na economia familiar”; entre os dias 8 e 20 de maio de 2021. Este
questionário foi enviado pelas plataformas WhatsApp e Gmail para os 16 Grupos PET
pertencentes à UFRGS. Participaram da pesquisa 35 alunos Petianos.
3. RESULTADOS E DISCUSSÃO
Através desta pesquisa, pode-se constatar que quase 80% dos petianos afirmaram que
precisam com certeza ou talvez de outra fonte de renda caso não recebessem auxílio financeiro
do PET. Aliado a isso, 66,7% dos participantes indicaram que a importância da bolsa PET na sua
vida é alta ou muito alta. Também, para 21,2% dos participantes a bolsa não é de uso
especialmente pessoal, indicando que muitas famílias dependem do incentivo financeiro recebido
pelos petianos. Quando perguntado aos 35 participantes quais são os aspectos da sua graduação
que são afetados em função da renda entre lazer, alimentação, transporte, atendimento
psicológico, moradia, material escolar, higiene pessoal, tempo para estudos, eletrônicos,
internet, vestuário e atendimento médico, as respostas foram: 16 alunos lazer; 15 alimentação,
transporte e atendimento psicológico; 13 moradia, 12 material escolar, higiene pessoal, tempo
para estudos e eletrônicos, 8 Internet, 7 Vestuário e 5 Atendimento médico. A pergunta foi feita
no formato “caixa de seleção” através de um formulário on-line, e os participantes poderiam
marcar todas as opções com as quais se identificassem dentre as citadas acima.
Sendo assim, foi relatado que, sem a renda muitos alunos não teriam as condições
necessárias para se manter na academia; teriam dificuldades em aproveitar os períodos de
divertimento, descanso ou desenvolvimento emocional, que são tão importantes quanto a saúde
física, de alimentação, psicológicos, de locomoção e até de moradia, entre outros, todos aspectos
que prejudicam a saúde psicológica dos discentes, cujo tema é abordado por Ponde e Cardoso
(2003).
Estes dados reforçam a importância do custeio para manutenção de urgências básicas que
vão muito além da universidade. Além disso, obteve-se que quase 70% dos participantes
acreditam que sua graduação sofrerá atraso. Apesar de, para alguns estudantes, o tempo
empregado na graduação não ser decisivo em outros aspectos de sua vida, há alunos que
dependem deste avanço “pré-programado” para conseguir estágios e empregos que pagam
maiores salários aos seus funcionários, só assim essas pessoas têm condições de ajudar suas
famílias.
4. CONCLUSÃO DO TRABALHO
Levando-se em consideração os aspectos relatados, percebeu-se que, apesar de o recurso
da bolsa ser baixo (se comparado ao valor do salário mínimo) ele é de grande importância para a
permanência dos estudantes na universidade e para o desenvolvimento pessoal e profissional dos
petianos, pois mais de 60% dos participantes da pesquisa sinalizaram que a importância da bolsa
PET na sua vida é alta ou muito alta e, além disso, a bolsa de 400 reais oferecida pelo
FNDE/MEC garante a manutenção da pluralidade nos grupos, pois permite a presença de
estudantes de diferentes classes sociais no programa, incluindo estudantes em situação de
vulnerabilidade socioeconômica, que provavelmente não poderiam participar sem o auxílio
financeiro anteriormente citado.
5. REFERÊNCIAS BIBLIOGRÁFICAS
PONDE, M. P.; CARDOSO, C. Lazer como Fator de Proteção da Saúde Mental. Revista de
Ciências Médicas. Campinas/SP, 12(2), 163-172, 2003.

\includepdf{pdfs/Impacto-Da-Renda-Na-Forma}

\addcontentsline{toc}{section}{Impacto Da Adaptação Da Atividade "Engenharia Em Foco" Para Ensino Remoto}

\section*{Impacto Da Adaptação Da Atividade "Engenharia Em Foco" Para Ensino Remoto}

Fabiano Piechontcoski, Emely Luane Pscheidt,Geovana Fernanda Lara Barboza,Isis Fernandes do Carmo,Juliana de Paula Martins

O texto traz uma análise comparativa de dados entre o êxito na divulgação e presença do público no cenário pré-pandemia e pós-pandemia, tendo como parâmetro relatórios do PET-EQ e informações obtidas através das plataformas utilizadas. Com a utilização dessas informações, chegou-se a conclusão de que a média de participação nas atividades aumentou no pós-pandemia, notando-se também que algumas atividades tiveram participação maior que as outras.

\includepdf{pdfs/Impacto-Da-Adaptacao-Da-A}

\addcontentsline{toc}{section}{Implantação De Ações Afirmativas No Programa De Educação Tutorial Do Curso De Direito Da Ufpr}

\section*{Implantação De Ações Afirmativas No Programa De Educação Tutorial Do Curso De Direito Da Ufpr}

Victoria Brasil, Profa Dra Heloisa Camara (Orientadora),Julia Favaretto Deschamps,Heloisa Nerone

O Programa de Educação Tutorial de Direito da Universidade Federal do Paraná, no
segundo semestre de 2020, efetivou a implementação de ações afirmativas para o ingresso de
discentes, tendo por fundamento a necessidade de adequar o grupo à política afirmativa que é
adotada pela UFPR desde 2004 e promover maior diversidade. A experiência visou atender a
demanda por inclusividade conjugando os pilares de ensino, pesquisa e extensão com maior
comprometimento acadêmico e social.
Atualmente, nota-se uma relevante diversidade de pessoas que ocupam os bancos da
Faculdade de Direito, que se torna ainda mais relevante considerando a composição desta mesma
Faculdade menos de 10 anos atrás. Lima (2016) aponta, porém, que há um desafio e uma
necessidade de se pensar na permanência dos estudantes negros (as) na universidade e em seus
espaços, em um constante cuidado e adequação para que estes não se tornem espaços de
exclusão ou de silenciamento destes alunos. Isto inclui o grupo PET Direito UFPR.
Apesar da adoção de cotas pela UFPR em 2004, até 2020 a questão não havia sido
implementada no PET Direito UFPR. Esse processo somente aconteceu no final de 2020, com o
objetivo de integração de grupos historicamente excluídos de determinados espaços, como o
ambiente universitário e com o objetivo de que o grupo PET passasse a representar de fato a os
estudantes, cujo perfil havia mudado nos últimos anos em decorrência do sucesso das políticas
de cotas. Após deliberação, aprovou-se a Resolução n.º 01/2020, que estabeleceu uma política
de cotas para o ingresso no grupo.
Em linhas gerais, a resolução prevê a reserva de 50% do total das vagas disponíveis no
processo seletivo para estudantes que se autodeclararem pertencentes às minorias políticas -
negras (pretas e pardas), indígenas, quilombolas, trans (transexuais e transgêneros), travestis,
migrantes humanitárias e refugiadas, bem como pessoas com deficiência -, a forma das
inscrições (documentos de autoidentificação) e maneira como se dará a distribuição das vagas e
bolsas ao final do processo seletivo.
A partir dos debates, o grupo decidiu não vincular a autodeclaração da candidata à forma
de ingresso no UFPR, isto é, ao fato de ter sido aprovada no vestibular pela modalidade de cotas
ou ampla concorrência. Por isso, a resolução prevê a possibilidade de realização de banca de
identificação, apenas para aquelas pessoas que não ingressaram na UFPR como cotistas (vez que
estas pessoas já passaram pela banca de heteroidentificação da universidade), nos moldes
estabelecidos pela SIPAD, e a depender de previsão no edital de seleção.
Ainda, de modo a preencher de significado a implementação de cotas do processo
seletivo, o documento estabelece que a banca do processo seletivo deverá obrigatoriamente ser
composta por pelo menos uma representante das minorias políticas citadas. Além disso, as
avaliações de ingresso deverão incluir referências produzidas por estas minorias que versem
preferencialmente sobre: Teoria Crítica da Raça, racismo estrutural e institucional, estudos de
gênero e sexualidade, pensamento decolonial e estudos críticos ao capacitismo.
As mudanças no edital do processo seletivo de 2021 para ingresso do PET Direito se
basearam nas atribuições garantidas à tutora do grupo PET. A adequação do edital à Resolução
nº 01/2020 do PET Direito ocorreu com as definições: I) O tema do ensaio e de estudos do PET
em 2021 segue aqueles apontados na resolução, a saber: “As encruzilhadas da subjetividade
jurídica a partir do Sul”,; II) A banca teve em sua composição uma representante das minorias
políticas citadas na resolução;, III) Seguindo a Resolução nº 01/2020, foram escolhidas
referências produzidas por pessoas das minorias e sobre os temas indicados na resolução. Foram
abordados autores e autoras que tratam da subjetividade jurídica nos países periféricos e de
abordagens críticas à colonialidade. As obras selecionadas foram as seguintes: “A categoria
político-cultural amefricanidade” de Lélia González; “Colonialidade do poder, eurocentrismo e
América Latina” de Aníbal Quijano; “Direito e Relações Raciais - Uma introdução crítica ao
racismo” de Dora Lúcia de Lima Bertúlio; além do desenho “América Invertida”, de Joaquín
Torres García.
No processo seletivo para o ingresso em 2021 houve, ao todo, 22 inscrições, cinco com
declaração de cotista. Dessas, quatro pleiteantes declararam-se negras e uma parda. Em relação
aos resultados, ingressaram 7 alunas, sendo três autodeclaradas negras e uma parda. A partir de
pesquisa junto aos membros que ingressaram no PET nos últimos 3 anos (2018-2020),
verificou-se que apenas 3 (15,8%) membros, entre as 19 respostas, declararam que teriam
ingressado na categoria de cotista: duas pardas e uma negra.
Com o processo seletivo de 2021, há diferença gritante com os resultados dos anos
anteriores, que apontaram que apenas uma pessoa se autodeclarou negra das dentre as 19 (5,5%)
respostas recebidas na pesquisa. É nítido que a política de ação afirmativa já demonstrou um
expressivo impacto na composição racial do grupo, e ainda se aguarda seus efeitos a longo prazo,
ligados com outras políticas de inclusão para ingresso na universidade. Também se espera que
essa ação influencie outros grupos da universidade, não somente PETs, a adotarem tais políticas.
Referências
LIMA, S. M. A. A permanência de estudantes negros(as) na universidade federal do
paraná: aspectos material e simbólico. 2016. Dissertação (Mestrado em Educação) –
Universidade Federal do Paraná. Orientador: Paulo Vinicius Baptista da Silva.

\includepdf{pdfs/Implantacao-De-Acoes-Afir}

\addcontentsline{toc}{section}{Importância Da Pesquisa Tecnológica No Desenvolvimento Acadêmico}

\section*{Importância Da Pesquisa Tecnológica No Desenvolvimento Acadêmico}

Walter Luiz Candido Closs, Eloá Vitória Corrêa Santos,Gabrielli Barros Silva,Mariana Barra Ferreira,Regina Cavalcanti Rodrigues,Victor Henrique Pelincer,Neusa Fátima Seibel.

Em decorrência da crise provocada pela pandemia do novo Coronavírus, no ano de 2020 houve a 
paralisação das atividades presenciais, em todo o Brasil. Inicialmente as Universidades 
suspenderam as atividades in loco por um determinado tempo, em seguida, transformaram todas 
as possíveis para a modalidade remota, assim, as aulas práticas e o desenvolvimento de pesquisas 
foram afetados. Essa situação também refletiu no planejamento do Grupo PET Tecnologia em 
Alimentos da UTFPR do Câmpus Londrina, pois o desenvolvimento da pesquisa tecnológica 
prevista para o ano de 2020 foi totalmente inviabilizado. O objetivo da atividade “Pesquisa 
Tecnológica com Alimentos” prevista e descrita no planejamento anual deste grupo é 
‘desenvolver pesquisas tecnológicas nos grãos e produtos derivados da soja, utilizando análises 
físico-químicas padronizadas ou adaptando-as, quando necessário’. Este alimento foi escolhido 
devido a sua grande importância econômica, pois é a principal cultura agrícola da região de 
localização deste grupo PET e também do Brasil, além de ser uma das linhas de pesquisa da 
tutora, assim, todos os integrantes adquirem conhecimentos atualizados do assunto em questão. 
Para a execução, inicialmente os alunos fazem uma pesquisa bibliográfica sobre as análises e as 
metodologias a serem aplicadas, em seguida, ocorrem os testes e adaptações dos métodos, visto 
que há diferenças nos equipamentos disponíveis, logo após, as análises definitivas são realizadas, 
avaliadas e os seus resultados são comparados e discutidos com os da literatura. Assim, durante o 
ano de 2020 os integrantes do grupo PET trabalharam somente na fase teórica, não obtendo a 
experiência prática da execução das atividades laboratoriais. No primeiro semestre de 2021, a 
partir da flexibilização dos decretos, municipais e estaduais, e do progresso da vacinação, foi 
possível retornar aos laboratórios com as devidas medidas de segurança estabelecidas pela 
Organização Mundial da Saúde (OMS, 2021) e pelas Instruções Normativas da Universidade 
(UTFPR, 2021). Primeiramente a tutora organizou os integrantes em sub-grupos para a 
realização das diferentes análises, objetivando o aprendizado total da metodologia aplicada, 
assim como, instruiu-os sobre a execução dos trabalhos laboratoriais de forma teórica e os 
ensinou na prática todas as determinações e técnicas efetuadas. Os grãos de soja de quatro 
diferentes safras foram avaliados pela quantificação de fibras alimentares, proteínas solúveis, 
compostos fenólicos, atividade antioxidante, pH e propriedades emulsificantes. Após as 
atividades laboratoriais, a tutora ensinou aos discentes como é realizado o cálculo de cada análise 
e em seguida foram avaliados quanto a sua repetibilidade. Com o exercício desta atividade 
pretende-se que os discentes aprimorem os conhecimentos relacionados às práticas laboratoriais 
já recebidos nas disciplinas do curso e desenvolvam uma análise crítica ao comparar e discutir os 
resultados obtidos com dados já publicados, e posteriormente publiquem os dados em revistas 
e/ou eventos específicos da área, para a socialização dos mesmos. Visa-se também trabalhar em 
equipe dentro do laboratório, aprender a lidar com situações inesperadas e adquirir conhecimento 
técnico, contribuindo para a elevação da qualidade da formação acadêmica dos alunos de 
graduação e estimular a formação de profissionais e docentes de elevada qualificação técnica, 
científica, tecnológica e acadêmica. Mas durante a execução desta atividade foi observada uma 
grande dificuldade dos discentes, pois a maioria não tinha tido aulas práticas, somente na forma 
remota, consequentemente não conheciam as metodologias, alguns materiais e seu manuseio e os 
equipamentos laboratoriais, necessitando de um maior auxílio. Notando-se assim, os múltiplos 
efeitos negativos no aprendizado dos estudantes com a queda no rendimento acadêmico, tendo 
em vista a ausência das práticas laboratoriais. Neste caso, os petianos foram privilegiados, pois 
com a execução da atividade “Pesquisa Tecnológica com Alimentos” estes puderam retornar às 
atividades práticas no laboratório, onde aprenderam fazendo todas as etapas: organização e 
separação dos materiais necessários, preparo das soluções, manuseio dos materiais e 
equipamentos, limpeza e organização, cálculos e avaliação dos resultados. Os discentes também 
destacaram que com essa atuação é possível expandir o conhecimento, pois há análises que não 
são realizadas no decorrer das disciplinas ou executadas somente de forma parcial devido ao 
tempo da aula, além do desenvolvimento do espírito crítico e analítico na avaliação dos dados. 
Conclui-se assim, que a pesquisa tecnológica tem extrema importância no desenvolvimento 
acadêmico, desempenho pessoal e profissional dos alunos, pois é nesse momento que estes 
adquirem a experiência prática que será necessária no decorrer da sua vida, assim como 
melhoram as características de responsabilidade, organização e dedicação. 
Referências 
OMS (2021). Organização Mundial da Saúde. Disponível em: https://www.paho.org/pt/brasil
UTFPR (2021). Universidade Tecnológica Federal do Paraná. Disponível em: 
http://portal.utfpr.edu.br

\includepdf{pdfs/Importancia-Da-Pesquisa-T}

\addcontentsline{toc}{section}{Incentivo Às Criações Alternativas Na Agricultura Familiar}

\section*{Incentivo Às Criações Alternativas Na Agricultura Familiar}

Janaina Hillesheim, Naiara Vitoria Ferreira Cortes Koprovski,Eloize de Souza,Fabiana Rankrape,Gabriela Salete Vasconcelos,Mayara Cristina Stumm,Simone Wagner Menegotto,Karina Ramirez Starikoff (tutora)

A diversificação das propriedades rurais com atividades alternativas é uma prática que
auxilia na atenuação dos riscos de um único tipo de produção, contribuindo com o
desenvolvimento rural e com a garantia de geração de renda às famílias de agricultura familiar.
Com o atual crescimento populacional, diminuição de residentes do campo e o uso errôneo dos
recursos naturais, há uma urgente preocupação com o incentivo de atividades que sejam
sustentáveis econômica e ambientalmente (SORDI, 2016).
Em vista disso, o grupo PET Medicina Veterinária Agricultura Familiar, da Universidade
Federal da Fronteira Sul (UFFS) campus Realeza – PR, desenvolveu a atividade denominada
“Criações alternativas na Agricultura Familiar”. Essa evento teve como objetivo incentivar a
implantação de criações econômicas alternativas nas propriedades de agricultura familiar, para
que espaços em ócio sejam utilizados para geração de renda das famílias, auxiliando na
diminuição do êxodo rural. Além disso, teve como intuito difundir informações sobre diferentes
criações, incentivando o interesse, por parte dos acadêmicos, em diferentes áreas de atuação na
Medicina Veterinária.
A atividade propôs a realização de palestras sobre culturas alternativas, que sejam de fácil
implantação e manejo, com a participação de palestrantes que trabalham na área, com capacidade
de oferecer informações técnicas e práticas acerca de cada assunto.
A criação de coelhos é uma cultura alternativa que pode ser implantada nas propriedades
rurais com grande potencial de geração de renda, possibilitando a comercialização de carne, pele,
pelos e outros produtos derivados da espécie. A criação de coelhos é de fácil manejo, pois são
pequenos, se alimentam de uma grande variedade de alimentos e adaptam-se a alojamentos
simples, mas, apesar de ser atrativa, ainda é pouco explorada no Brasil, o qual se encontra na 36°
posição no ranking mundial da população de coelhos (SORDI, 2016).
Dessa forma, no dia 23 de junho de 2021 foi realizada a palestra intitulada “Cunicultura
de Corte na Agricultura Familiar”, com a zootecnista Ana Carolina Kohlrausch Klinger, que é
membro da Associação Científica Brasileira de Cunicultura (ACBC) e colunista do Boletim de
Cunicultura. O evento foi transmitido online via plataforma Cisco Webex Meetings, teve duração
de 1h15min e contou com a participação de 43 pessoas, entre professores, produtores rurais e
alunos da graduação de diversos cursos e universidades. Foram tratados assuntos como as
aptidões dos coelhos, manejo geral da criação, instalação de alojamentos e ninhos, alimentação e
retornos econômicos. A atividade possibilitou intensa interação entre os espectadores e a
palestrante, a qual transmitiu muitas informações técnicas aos mesmos, fato que pôde ser
evidenciado a partir do formulário de satisfação, respondido pelos participantes ao final do
evento.
A agricultura familiar é a fonte da produção de alimentos básicos consumidos
internamente no Brasil, tal como leite e hortaliças. Apesar de ocupar pequena porcentagem das
terras agricultáveis do país, promove geração de empregos, redistribuição de renda e apresenta
potencial desenvolvimento sustentável. Todavia, a diversificação dessas propriedades rurais
surge como alternativa para minimizar os riscos econômicos, como preços e comercialização,
riscos ambientais, incertezas climáticas, pragas e doenças, que podem afetar a produção (ESAU;
DEPONTI, 2020).
Garantir a segurança da renda passa a ter uma importância fundamental, principalmente
para os pequenos, que têm menos condição de resistir a grandes impactos no orçamento familiar.
Além disso, existe uma correlação positiva entre diversificação, renda familiar e a diversidade da
dieta, mostrando que essa adaptação também contribui para a sobrevivência e segurança
alimentar das famílias rurais (SAMBUICHI et al, 2014).
As lideranças institucionais, assistência técnica e atores da extensão rural devem estar
atualizados e conscientes para promover orientação técnica que dê suporte para que os
agricultores familiares possam diversificar sua produção e desenvolver subsistemas de produção,
aproveitando os nichos específicos e demandas de mercado por produtos alimentícios, a fim de
superar as limitações impostas pela cultura local (ESAU; DEPONTI, 2020).
A atividade desenvolvida pelo grupo reconhece a importância de dar luz ao processo de
diversificação das pequenas propriedades familiares com o cunho de incentivar o
desenvolvimento e permanência da população na zona rural levando informação e conhecimento
para produtores e acadêmicos. Pois, quanto mais diversificada for a unidade produtiva, maiores
serão as possibilidades de escolha e mais amplas as estratégias que poderão ser estabelecidas
para o combate da vulnerabilidade.
Referências
SAMBUICHI, Regina Helena Rosa et al. A diversificação produtiva como forma de viabilizar o
desenvolvimento sustentável da agricultura familiar no Brasil. 2014.
SORDI, Victor Fraile et al. Estratégia de diversificação em propriedades rurais: o caso da
cunicultura. Revista Brasileira de Produtos Agroindustriais, v. 18, n. 3, p. 325-333, 2016.
ESAU, Carlos; DEPONTI, Cidonea Machado. Tomada de decisão pela diversificação: uma
alternativa para agricultura familiar na microrregião de Santa Cruz do Sul/RS.
DRd-Desenvolvimento Regional em debate, v. 10, p. 439-460, 2020.

\includepdf{pdfs/Incentivo-As-Criacoes-Alt}

\addcontentsline{toc}{section}{Instrumentalização De Petianos: Um Espaço De Desenvolvimento E Aprendizado}

\section*{Instrumentalização De Petianos: Um Espaço De Desenvolvimento E Aprendizado}

Camila Segatto Hartmann, Arthur Danzmann Chaves,Giovanna Leal Klein,Jessica Carvalho de Oliveira,Luiz Fillipi Fleck,Rafaela Fernandes Borin,Luísa Helena do Nascimento Tôrres

Considerando que o PET consiste em um grupo de discentes sob tutoria docente que realiza 
atividades extracurriculares como complemento da formação acadêmica, o Grupo PET 
Odontologia da Universidade Federal de Santa Maria elaborou um projeto de ensino que objetiva 
instrumentalizar seus próprios acadêmicos integrantes do Grupo PET através de metodologias 
diversificadas sobre temas em áreas de interesse dos petianos e/ou relacionadas às atividades 
realizadas pelo grupo para as quais há necessidade de aperfeiçoamento. Dessa forma, a elaboração 
de minicursos, palestras, capacitações e oficinas para os alunos em áreas distintas do curso, como 
Língua Brasileira de Sinais, Gestão, Metodologia Científica, Comunicação, entre outras, atua na 
promoção de uma formação global. Para começar a busca por temas de interesse, o petiano 
coordenador, responsável por gerenciar as atividades desse projeto, disponibilizou aos demais 
integrantes um formulário Google com sugestões de temas a serem trabalhados pelo grupo. Após, 
definiu-se o assunto da primeira palestra como oratória e comunicação, e um integrante se 
voluntariou para convidar um profissional que pudesse abordar o tema. Dessa forma, as atividades 
do projeto tiveram início no dia 31 março de 2021 com a realização da palestra “Comunicação e 
oratória: como ter uma comunicação assertiva”, ministrada por uma fonoaudióloga, formada e 
mestranda da UFSM, com duração de duas horas. Em reunião administrativa posterior, os 
integrantes do grupo realizaram o relato das percepções em relação a palestra e o retorno foi muito 
positivo, tendo a aprovação de todos os participantes. A escolha do tema de oratória e comunicação 
veio para suprir o anseio do grupo referente a apresentação de trabalhos e realização das atividades 
do PET que envolvem principalmente falar em público. Para o segundo encontro, considerou-se 
as exigências de escrita e uso de excel nos projetos do Grupo. Sendo assim, o tema escolhido foi 
aprofundar conhecimento em Word e Excel. Dessa forma, nos dias 06 e 13 de abril de 2021, uma 
professora doutora do curso de Odontologia da UFSM ministrou o “Minicurso: formatação no 
Word e Excel básico”, com duração de quatro horas. A segunda e terceira atividades foram bem 
aceitas pelos participantes do Grupo, que relataram em reunião administrativa terem aprendido 
novas ferramentas para uso nos programas. Por fim, a última atividade realizada até o momento 
surgiu frente a necessidade de aprendizagem da língua brasileira de sinais por parte dos integrantes 
do Grupo devido a elaboração de um projeto de extensão com pacientes surdos. Portanto, sob 
organização de um integrante, uma professora da disciplina de Libras da UFSM ministrou a 
palestra “Introdução à língua brasileira de sinais”, no dia 21 de junho de 2021 com duração de 
duas horas. Embora o projeto ainda seja recente, já foi possível observar, por parte dos acadêmicos 
integrantes do Grupo PET Odontologia UFSM, a implementação dos conhecimentos adquiridos 
nos projetos e atividades do Grupo, bem como a otimização do processo de trabalho, uma vez que 
foi possível colocar em ação os aprendizados no dia-a-dia das atividades. Ainda, todos 
participaram da escolha dos temas abordados, das atividades elaboradas e demonstraram 
entusiasmo com as próximas atividades. Portanto, o presente projeto possibilita ao Grupo 
aprofundar o conhecimento em temas de interesse e relevância, além de exigir a participação ativa 
dos integrantes no planejamento e organização das atividades. Dessa forma, proporciona aos 
participantes uma qualificação interdisciplinar, contribuindo tanto para a formação profissional 
quanto pessoal dos integrantes do grupo PET Odontologia UFSM.

\includepdf{pdfs/Instrumentalizacao-De-Pet}

\addcontentsline{toc}{section}{Jogo Web De Conscientização Ao Covid-19}

\section*{Jogo Web De Conscientização Ao Covid-19}

Gustavo De Souza Santos, Caio Machado dos Santos,Marcos Felipe Friske dos Santos,Caroline Francisca,Kleber Ersching Tutor

Em dezembro de 2019 na China foi registrado o primeiro alerta sobre o surgimento da
covid-19, e essa notícia rapidamente se espalhou pelo mundo. No Brasil o coronavírus teve seu 
primeiro caso confirmado pelo Ministério da Saúde em 26 de fevereiro de 2020. Esse vírus levou 
a população mundial a enfrentar uma pandemia muito grave que levou a óbito mais de 4 milhões 
de pessoas no mundo (G1 Globo, 2021).
A Covid-19 é uma doença respiratória e a sua transmissão se dá de pessoa para pessoa 
por meio de gotículas do nariz ou da boca que se espalham quando alguém doente tosse ou espirra 
(BVSMS, 2021). Segundo a OMS algumas ações de prevenção ao vírus que as pessoas devem 
fazer é: 
• lavar as mãos frequentemente com água e sabonete;
• evitar tocar nos olhos, nariz e boca com as mãos não lavadas;
• evitar contato próximo de pessoas doentes (a recomendação é mais de um metro de 
distância);
• ficar em casa quando estiver doente;
• cobrir boca e nariz com um lenço de papel, ao tossir ou espirrar. 
• evitar o compartilhamento de copos, pratos ou outros objetos de uso pessoal;
• limpar e desinfetar objetos e superfícies que sejam tocadas com freqüência por várias 
pessoas;
• pessoas que estiveram em áreas onde o vírus circula, que tiveram contato físico com 
alguém diagnosticado ou que apresentem febre, tosse ou dificuldade para respirar, devem 
procurar atendimento médico de imediato.
Mesmo com as ações de prevenção sendo amplamente divulgadas e discutidas nas mídias 
os casos de covid aumentaram muito ao redor do mundo, fato que se deu por vários motivos, e um 
deles foi a problemática em relação à população mais jovem que tiveram dificuldades em entender 
a real dimensão e gravidade da pandemia, e isso os levaram a não praticarem de forma correta as 
medidas de prevenção proposta pela Organização Mundial de Saúde. 
Tendo em vista todo esse contexto da pandemia do covid-19, o PET em 2020 em parceria 
com o grupo de Defesa Civil do Estado de Santa Catarina, desenvolveu uma aplicação web (site) 
em formato de jogo quiz, com o intuito de conscientizar a população, em especial a infantil, sobre 
a prevenção ao coronavírus. O site possui uma atmosfera lúdica e nele é contada uma história que 
por meio de perguntas e respostas leva o jogador a se entreter e se conscientizar acerca da 
prevenção correta ao covid-19.
O jogo Recebeu o nome de “Jogo de prevenção ao Covid-19”, e foi desenvolvido utilizando 
as linguagens de marcação HTML, CSS e a linguagem de programação Javascript. O ambiente de 
desenvolvimento integrado (IDE - Integrated Development Environment) utilizado para o 
desenvolvimento foi o Visual Studio Code.
Após o término do desenvolvimento do site, o grupo PET e os professores do curso de 
defesa civil do Instituto Federal Catarinense - Campus Camboriú (IFC-Cam), fizeram uma ampla 
divulgação do site em grupos do WhatsApp e em redes sociais buscando assim atrair uma grande 
quantidade de jovens para que eles pudessem compreender de forma interativa como se prevenir 
da contaminação pela covid-19. 
E por fim o resultado final foi um site responsivo em formato quiz, que tem como objetivo 
atrair o público infantil de uma forma menos formal para que eles venham entender as medidas de 
prevenção que eles devem adotar no dia a dia. O jogo está disponível no endereço 
(http://www.defesacivil.ifc-camboriu.edu.br/jogoCovid/).
Referências:
G1 GLOBO. G1, Publicado em: 07/07/2021. O clube: Mundo passa de 4 milhões de mortes por Covid. 
Disponível em: . Acesso em: 14 de ago. de 2021.
Biblioteca Virtual em Saúde. BVSMS, c2021. Novo Coronavírus (Covid-19): informações básicas. 
Disponível em: . Acesso 
em: 14 de ago. de 2021.

\includepdf{pdfs/Jogo-Web-De-Conscientizac}

\addcontentsline{toc}{section}{Jogos De Integração Do Centro De Ciências Computacionais - Jic3}

\section*{Jogos De Integração Do Centro De Ciências Computacionais - Jic3}

Paulo Madson Da Silva, Breno G. Rodrigues Soares,Cristofer Herreira Santos,Rodrigo Kochenborger,Diana Adamatti

Os Jogos de Integração do Centro de Ciências Computacionais (JIC3) são um conjunto
de competições realizadas no início de cada ano letivo, com o objetivo de integrar os estudantes,
técnicos, professores e egressos do centro, bem como diminuir a evasão dos novos alunos. A
atividade tornou-se assídua desde sua criação, adequando-se ao contexto e às possibilidades de
realização. Em tempos comuns, as competições trazem esportes regulares, como futebol e
voleibol, em um espaço organizado e preparado para o acontecimento das atividades presenciais.
Isto demanda uma grande organização, tanto para locação de um local apropriado, do material
utilizado e profissionais para gerenciamento dos jogos, quanto para o regulamento, inscrição e a
divulgação. Por esta razão, o projeto sempre busca voluntários para auxiliar durante todo o
evento, sempre organizado para que os voluntários também possam participar das competições.
Qualquer pessoa ligada ao centro pode participar dos jogos, e para ajudar em causas sociais,
sendo a única exigência de inscrição a doação de um alimento não perecível. Durante o
distanciamento social e ensino remoto, não existiu a possibilidade de realização do JIC3
presencial, assim, o grupo se estruturou para a realização dos jogos de maneira remota. As
competições foram em campeonatos de esportes eletrônicos, pegando os jogos mais famosos do
meio, como League of Legends, Counter Strike e Valorant. Cada regulamento foi desenvolvido
respeitando as regras de cada jogo, adaptado de campeonatos oficiais, e foi divulgado dias antes
das inscrições. Durante uma semana, os jogos foram divididos em etapas como
pré-classificatórias, fase semifinal e fase final, e foram transmitidos e comentados ao vivo no
canal do YouTube do grupo, para qualquer pessoa assistir. Divulgado durante semanas, as
competições contaram com 17 times inscritos, envolvendo aproximadamente 100 integrantes dos
três cursos do centro como jogadores, e uma média 100 telespectadores dos jogos por dia. Nessa
edição com esportes eletrônicos e remotos, a inscrição foi a doação de alimentos de forma
virtual, via Banco de Alimentos do Rio Grande do Sul. Além dos jogadores, todos poderiam
doar, sendo atingida a meta de mais de R\$1.005,51 em doações monetárias. O evento também
contou com alguns voluntários para comentar e narrar os jogos, e ao fim, resultou-se de maneira
positiva as opiniões colhidas através de um formulário. Após o fechamento, foi observado que os
jogos eletrônicos são opções bem populares, fáceis de gerenciar e organizar como campeonatos.
Assim, em futuros eventos, existe a possibilidade de integrar os esportes tradicionais com os
e-esports, conseguindo abranger um público ainda maior

\includepdf{pdfs/Jogos-De-Integracao-Do-Ce}

\addcontentsline{toc}{section}{Leitura Literária Durante A Pandemia Através De Plataformas De Streaming}

\section*{Leitura Literária Durante A Pandemia Através De Plataformas De Streaming}

Alisson Castro Batista, Paola Cassuriaga Sandim,; Luiz Ariel Miranda Ribeiro da  Silva,Paloma Evelise Wiegand,Cinara Tonello Postringer,Letícia Oliveira Vilela,Angelica dos Santos Karsburg

PROBLEMÁTICA
Uma das principais atividades que vêm sendo desenvolvidas nos últimos anos pelo PET 
Educação são as práticas de leitura literária, realizadas presencialmente em escolas, bibliotecas, 
museus e outros ambientes culturais. De acordo com PAULINO (2004), a leitura literária ocorre 
quando “a ação do leitor constitui predominantemente uma prática cultural de natureza artística, 
estabelecendo com o texto lido uma interação prazerosa”. No atual contexto de isolamento social, 
pensamos maneiras de adaptar nossas atividades que até então eram realizadas presencialmente.
JUSTIFICATIVA
A fim de proporcionar práticas de leitura literária ao público prioritário de nossas ações, 
criamos dois programas de áudio que foram disponibilizados em plataformas de Streaming: o 
“Primeiras Páginas” e o “Minutos Literários”. De acordo com DE JESUS ADÃO (2006), 
“Streaming” trata-se uma tecnologia que “permite ao cliente visualizar os ficheiros de áudio e 
vídeo sem que estes tenham sido completamente descarregados do servidor.” Ou seja, o conteúdo 
é exibido de forma online, sem a necessidade de que seja feita uma cópia dos arquivos no 
dispositivo pessoal de quem acessa. Proporcionando economia de dados e rápido acesso.
METODOLOGIA
O primeiro passo ao realizar uma prática de leitura literária é a seleção da obra. Neste 
sentido, são realizados, pelo grupo PET Educação da UFPel, constantes estudos acerca de critérios 
que fundamentam uma seleção de obras coerente com o contexto e público para quem será 
realizada a leitura. Na criação dos programas “Minutos Literários” e “Primeiras Páginas”, 
inicialmente, realizamos uma seleção de obras assentadas em nosso acervo pessoal. Escolhidos e 
aprovados pela orientadora, passamos ao ensaio das leituras e, posteriormente, a gravação em 
áudios, em nossos smartphones. O objetivo era publicar esses áudios e disponibilizá-los na 
plataforma WhatsApp, junto com dados de autores e ilustradores, além de imagens das capas dos 
livros. Esse material foi salvo em ficheiros virtuais e organizados de acordo com o programa 
correspondente. Então, foi realizado um trabalho de produção fonográfica nos arquivos de áudio, 
com a intencionalidade de melhorar a qualidade da faixa, remover ruídos e barulhos indesejados 
e, também, corrigir os níveis de volume e de ganho. Além disso, foi mixada uma trilha sonora 
adequada ao contexto da obra, a fim de proporcionar uma experiência mais envolvente aos 
ouvintes. A plataforma de streaming escolhida foi a SoundCloud, que é gratuita. Porém, após 
publicarmos uma grande quantidade de arquivos, percebemos que a plataforma dispunha de um 
limite na versão gratuita e já havíamos atingido este limite. Por esta razão e pelo considerável 
alcance que os programas obtiveram, decidimos migrar para a plataforma de streaming de vídeos 
YouTube e para a plataforma de áudios Spotify, nas quais não existem limites de publicações.
Atualmente, estamos no processo de troca de plataforma, publicando todo o material já 
disponibilizado na plataforma anterior. A principal diferença metodológica observada foi: 
enquanto no SoundCloud são publicados áudios (geralmente em formato mp3), no YouTube são 
publicados vídeos. Isso demanda que transformemos o áudio que tínhamos até então, em vídeo. 
Para tal, utilizamos a foto da capa do livro lido como imagem fixa no vídeo, enquanto é 
reproduzido o áudio da leitura.
RESULTADOS E DISCUSSÕES
Iremos considerar as estatísticas disponibilizadas diretamente pela plataforma 
SoundCloud, pois, no YouTube e no Spotify, ainda estamos nos estágios iniciais da publicação do 
conteúdo já produzido.
Como critérios de confiabilidade destacamos: 1. Áudios postados; 2. Quantidade de 
reproduções; 3. Artefato de conectividade; 4. Onde mora quem ouviu; 5. Mais ouvido. De acordo 
com a plataforma, o programa voltado ao público infantil, intitulado Minutos Literários, teve até 
o momento,1928 reproduções, nos seus 58 arquivos de áudios publicados. Aproximadamente 80% 
das reproduções foram realizadas a partir de smartphones e 20% foram realizadas a partir de 
computadores. Destas, 1885 foram realizadas no Brasil e 955 foram realizadas na cidade de 
Pelotas. Os textos mais reproduzidos foram: Viva Voz!, escrito por Léo Cunha e lido por Paloma 
Wiegand; e Coisa de Menina, escrito por Pri Ferrari e lido por Cristina Maria Rosa. No programa 
voltado ao público adulto, intitulado Primeiras Páginas tivemos, até o momento, um total de 701 
reproduções, nos seus 55 áudios publicados. Aproximadamente 60% das reproduções foram 
realizadas a partir de smartphones e 40% foram realizadas a partir de computadores. Destas, 682 
foram realizadas no Brasil e 309 foram realizadas na cidade de Pelotas. Os textos mais 
reproduzidos foram: A Touca de Bolinha, escrito por Sergio Faraco e lido por Cristina Maria Rosa; 
e Rosa de Hiroshima, escrito por Vinicius de Moraes e lido por Cinara Tonello Postringer. Ao 
todo, os dois programas tiveram um alcance de aproximadamente 2600 reproduções.
REFERÊNCIAS
FRADE, Isabel Cristina Alves da Silva. VAL, Maria da Graça Costa. BREGUNCI, Maria das 
Graças de Castro. (orgs). Glossário Ceale: termos de alfabetização, leitura e escrita para 
educadores. Belo Horizonte: UFMG/Faculdade de Educação, 2014. Link de acesso: 
http://ceale.fae.ufmg.br/app/webroot/glossarioceale/referencia/paulino-g-cosson-r-org s-leituraliter-ria-a-media-o-escolar-belo-horizonte-fale-ufmg-2004- Acesso em 24 de setembro de 2020. 
DE JESUS ADÃO, C.M.C. Tecnologias de Streaming em Contextos de Aprendizagem. 2006. 
Dissertação (mestrado em Sistemas de Informação) - Departamento de Sistemas de Informação, 
Escola de Engenharia, Universidade do Minho.

\includepdf{pdfs/Leitura-Literaria-Durante}

\addcontentsline{toc}{section}{Literatura E Matemática: Dentre Horizontes Possíveis, Malba Tahan}

\section*{Literatura E Matemática: Dentre Horizontes Possíveis, Malba Tahan}

Midia Barbosa, César Augusto Espitia Pedreros,Karen Dayanna Salinas Peña,Einer Jesus Castro Cabarcas,Jose Carlos Martinez Oñate,Heloisa Marques Gimenez

É comum associar a matemática escolar como um dos conhecimentos que desperta nos
estudantes uma grande dificuldade e às vezes aversão. Em contrapartida, a literatura tem uma
melhor aceitação entre o alunado. Com o propósito de aproximar essas áreas do saber, uma vez
que o eixo de pesquisa do grupo perpassa a literatura, foi proposto a leitura do livro O Homem
que Calculava de Malba Tahan.
Como parte do processo formativo estudantil na Unila e do PET Conexões de Saberes Literatura
e Cultura especificamente, tem-se como eixo orientador a tarefa de encontrar, ou devolver, uma
correlação entre diferentes saberes e cursos. Além disso, a interdisciplinaridade que há no grupo
PET Unila nos permite criar um espaço de discussão que aponte os saberes da Literatura e
Matemática com a intenção de aproximar e ampliar o público interessado na temática proposta.
Portanto, após a leitura e discussão entre os membros Petianos do livro “O Homem que
Calculava” foi estabelecido que enquanto atividade do eixo Conectando Saberes e Práticas seria
desenvolvido um minicurso online no dia 28 de agosto de 2021, das 14:00 as 16:00. Com o
objetivo de propiciar um ambiente interativo entre os petianos e os ouvintes, optamos por utilizar
o jamboard Google, vídeos do YouTube, assim como textos ilustrativos.
Podemos descrever o evento em três grandes momentos: No primeiro, buscamos refletir sobre o
que vem a ser o conhecimento matemático e literário, com isso foi possível discutir tais
concepções desde a perspectiva daqueles presentes, isto é, sem defini-las de maneira rígida. As
seguintes perguntas orientaram nossa discussão: Qual a relação entre esses saberes? Quais os
diálogos possíveis na sala de aula?
Em um segundo momento procuramos fundamentar as possibilidades de vinculação a partir de
exemplos trazidos no livro Literatura e Matemática de Jacques Fux, exclusivamente trazendo o
conto O Livro de Areia do escritor argentino Jorge Luis Borges e o romance Planolândia de
Edwin A. Abbott, além do poema Oda a Los Números do escritor chileno Pablo Neruda. Dessa
forma, o que se almeja a partir dos exemplos é compreender a matemática como plano de leitura
das obras literárias. Já para o último ponto de discussão foi apresentado “O problema dos quatro
quatros” presente no livro O Homem que Calculava, desde nossa concepção a obra se enquadra
no conceito do gênero literário “literatura de viagem”, devido o relato descritivo do livro se
enquadrar enquanto característica do gênero literário mencionado.
Diante do que foi mencionado surgiram novas aprendizagens e saberes tanto no campo
conceitual como no âmbito prático de estratégias pedagógicas. Dentre os participantes não
petianos do minicurso se encontravam estudantes de licenciatura em matemática, letras,
pedagogia, engenharia, os mesmos explicitaram que além dessa correlação entre conceitos
matemáticos e gêneros literários também poderíamos estabelecer uma correspondência entre
Literatura e as demais áreas do saber, a exemplo a Química. Por esse motivo, sugestiona-se o uso
do material trabalhado no minicurso, especificamente o livro O Homem que Calculava, nas aulas
de matemáticas, bem como, nas de literatura. Tendo em consideração a postura adotada pelos
participantes entendemos que a proposta do grupo PET Conexões de Saberes Literatura e Cultura
contribuiu com a discussão e no processo de formação discente, seja dos membros petianos ou
dos discentes participantes, pois a atividade poderia ser transposta para o âmbito escolar
colocando em prática a proposta pedagógica e ampliando assim as possibilidades no processo de
ensino e aprendizagem.
Ao final das nossas discussões sobre matemática e literatura conseguimos visualizar novas
estratégias para o ensino e aprendizagem no âmbito escolar e fora dele. As provocações feitas em
cada um dos momentos geraram respostas positivas por parte dos participantes. A reflexão a
partir dos conhecimentos e memórias escolares trazidos pelos mesmos possibilitou construir uma
intervenção propositiva que pensasse nas conexões entre os diversos conceitos dos saberes por
eles aprendidos ao longo de suas vidas escolares e acadêmicas. Esse primeiro exercício
interativo, permitiu que os participantes, especialmente os discentes dos cursos de licenciatura
pensassem desde sua concepção de aluno, bem como, de futuro professor.
Além disso, tivemos falas com referência a outros autores latino-americanos, a exemplo o autor
peruano Carlos Augusto Salaverry enfatizando sua produção literária, mas ainda o
compartilhamento de experiência com a matemática no âmbito da engenharia, corroborando com
a proposta interdisciplinar do minicurso.
De igual relevância, durante o processo nos questionamos acerca da necessidade e possibilidade
de trazer/conhecer outros temas através da literatura e matemática, a exemplo citamos o racismo,
o sexismo. As propostas apresentadas permeiam a matemática escolar, como é o caso do livro O
Homem que Calculava, mas também extrapolam tal espaço, pois no conto O Livro de Areia
podemos trabalhar conceitos presentes na teoria dos números, especificamente a noção de
conjuntos numéricos não enumeráveis. Dito isso, vale ressaltar que se une aos objetivos do
minicurso a intenção de proporcionar novos planos de leitura para as obras literárias, isto é, não
conhecer os objetos matemáticos não impede a compreensão da mesma, porém ao revelar as
estruturas matemáticas almejamos potencializar a interpretação dos livros.
Por ser uma primeira experiência, não buscamos concluir o que vem a ser a literatura e a
matemática, tão pouco engessar suas possibilidades de inserção, o que consideramos por
concluído foi o aprendizado e o compartilhamento de experiências durante toda a trajetória de
execução.

\includepdf{pdfs/Literatura-E-Matematica--}

\addcontentsline{toc}{section}{Minicursos: Contraste Do Minicurso Presencial X Remoto}

\section*{Minicursos: Contraste Do Minicurso Presencial X Remoto}

Enzo Sennhauser, Ezequias David,Ludmylla Weber Kiene Muller Simon,Luís Felipe Bavati Medri,Naiury da Silva Marcondes,Paulo Vitor de Lima Carvalho

É notável que há alguns anos o grupo PET Engenharia Química da Universidade Federal
do Paraná oferece minicursos para os discentes da graduação com o intuito de enriquecer e
auxiliá-los por meio de noções básicas de diversos programas utilizados nos ramos de tecnologia
e engenharia, como, por exemplo, Planilhas Eletrônicas, Aspen Plus® e Scilab.
As aulas dos minicursos além de revisarem conceitos de matérias já abordadas também
realizam a introdução aos softwares com a resolução de exercícios propostos. Portanto
promovendo o uso de novas ferramentas aplicadas à solução de problemas, complementando a
graduação e preparando a/o discente para o mercado de trabalho.
Antes da pandemia, todos os semestres eram oferecidos ao menos dois minicursos. No
entanto, com o período de aulas remotas, foi necessária uma adaptação por parte do
planejamento do grupo PET, onde foi aprovado o desenvolvimento de uma aplicação piloto de
forma online, devido a um fator crucial, à demanda das/os discentes.
Sendo assim, todo o material didático foi atualizado pelas/os PETianas/os focando nas
adversidades da aplicação remota e das plataformas disponíveis para isso. Vale lembrar que os
minicursos ofertados pelo grupo têm uma grande procura, visto que há um rápido preenchimento
das vagas em poucas horas após a abertura das inscrições. Em 2020, devido a adaptação, foi
realizado 1 minicurso piloto: Planilhas Eletrônicas Módulo I, carga horária de 8 horas. Já em
2021, foram planejados para aplicação 3 minicursos no primeiro semestre: Planilhas Eletrônicas
Módulo I, Planilhas Eletrônicas Módulo II e Scilab, os dois últimos com carga horária de 10
horas. Todos os minicursos foram aplicados no formato síncrono, enquanto que o Scilab foi
aplicado no formato híbrido (síncrono e assíncrono). Quanto às plataformas utilizadas, foram
analisadas o Microsoft Teams, Zoom e o Google Meet, sendo escolhido o Teams devido às suas
vantagens e maior profissionalidade.
Em todos esses minicursos o grupo PET Engenharia Química alcançou satisfação
superior a 95% e 100% de recomendação, informações coletadas através de formulários online,
que também questionam o desempenho da/o ministrante e monitoras/es, relevância do conteúdo,
velocidade da aula e comentários. Os minicursos contemplaram, na pandemia, 103 discentes,
com 253 inscrições. Esses resultados e a alta concorrência por vagas, fazem dessa prática uma
tradição do PET Engenharia Química. Um ponto importante a ser mencionado é que nesse
período remoto, o grupo está se reinventando e tentando levar os minicursos para outras
universidades, abordando também o pilar da extensão (algo facilitado por eventos online). Neste
próximo semestre, teremos a aplicação do minicurso de Excel Extensão para a Semana
Acadêmica de Engenharia Química da Universidade Federal Rural do Rio de Janeiro. Além de
estarmos planejando uma aplicação extra do minicurso de Planilhas Eletrônicas Módulo I para a
Universidade Federal do Rio Grande (FURG).
Por fim, levando em conta as avaliações positivas por parte dos participantes, a alta
concorrência por vagas e efetiva participação da comunidade acadêmica, podemos concluir que
essa prática é uma tradição do PET Engenharia Química da UFPR. Vale ressaltar, que continuar
produzindo e aplicando minicursos continua sendo um desafio de ousadia e inovação por parte
desta entidade em meio a um período que pede adaptações, tornando-se algo reconhecido tanto
pelos discentes quanto pelos docentes, que auxiliam na divulgação dos minicursos e opinam
sobre conteúdos abordados nos mesmos. A partir disso, entende-se que a aplicação dos
minicursos desenvolvida pelo grupo atinge seus objetivos iniciais e quando não, os superam.

\includepdf{pdfs/Minicursos--Contraste-Do-}

\addcontentsline{toc}{section}{No Seu Pescoço, Chimamanda Ngozi Adichie: Nas Páginas Do Livro, Um  Mundo A Descobrir}

\section*{No Seu Pescoço, Chimamanda Ngozi Adichie: Nas Páginas Do Livro, Um  Mundo A Descobrir}

Juliana Breuer Pires, Jayziela Jessica Fuck,Elizabeth de  Souza Neckel,Lucas Rodrigues Menezes,Rafael da Silva,Carlos Henrique de Moraes Barbosa,Eliane Santana Dias Debus

O PET Pedagogia da Universidade Federal de Santa Catarina (UFSC) organiza suas ações 
a partir de três eixos temáticos: Infância e Literatura, Educação das Relações Étnico-Raciais 
(ERER) e Educação de Jovens e Adultos (EJA) e busca quando possível trabalhar de forma 
integralizada. Neste evento socializamos o projeto “Para além das páginas: vivenciando Leituras” 
que tem como objetivo a leitura do livro No seu pescoço (2009), da autora africana Chimamanda 
Ngozi Adichie. O projeto nasce da sugestão da petiana Juliana Breuer quando em novembro de 
2020 planejávamos as ações para o ano de 2021 e constatou-se a necessidade de ampliar a 
compreensão sobre o Continente Africano e suas culturas. Desse modo, o objetivo geral foi o de 
organizar uma roda de leitura que possibilitasse o contato dos estudantes com a literatura africana 
e as demais temáticas abordadas no âmago do texto escolhido, desdobrando-se nos seguintes 
objetivos específicos: ler e debater sobre o título lido e desenvolver análise escrita sobre o título 
lido.
A importância do projeto consiste na relevância de se ler e estudar as literaturas africanas, 
e desta forma, compreendendo o continente africano em sua pluralidade, rompendo com visões 
estereotipadas. Percepções errôneas que, se não trabalhadas na nossa formação acadêmica podem 
perpassar o futuro espaço de docência, um risco principalmente ao se tratar da pedagogia. Por 
outro lado, a Lei 10.639, sancionada em 9 de janeiro de 2003 e com ela se altera os artigos da Lei 
de Diretrizes e Bases da Educação (LDB) de 20 de dezembro de 1996 determina o ensino da 
história e da cultura africana e afro-brasileira, apoiada pelos documentos de implementação, entre 
eles as Diretrizes Curriculares Nacionais para a Educação das Relações Étnico-Raciais (BRASIL, 
2004) fortalecendo a proposta de trazer ao grupo a Educação das Relações Étnico-Raciais (ERER), 
por meio da palavra literária.
O livro No meu pescoço é estruturado em doze contos que abordam diversas temáticas e 
vivências culturais de diferentes personagens. Assim, metodologicamente o projeto foi 
desenvolvido no período de fevereiro a agosto de 2021, e foram organizados 13 encontros. No 
primeiro encontro, a bolsista responsável pelo projeto apresentou a autora do livro Chimamanda 
Ngozi Adichie, a estrutura do livro e o primeiro conto. Nos demais encontros, cada estudante se 
responsabilizou em socializar um dos contos e após cada apresentação, ocorreu o debate sobre a 
experiência da leitura e os pontos que se considerou mais relevantes da narrativa. 
Ao longo dos encontros e das discussões a respeito dos contos apresentados foi possível 
debater sobre algumas temáticas abordadas na narrativa pela a autora, tais como: relações 
familiares, questões de gênero, imigração, preconceito racial, dentre outros. O levantamento pelos 
petianos e petianas destas temáticas potencializou nossas discussões, atravessando a temática 
inicial proposta. Para além de enxergar a pluralidade cultural do Continente Africano e das 
temáticas que atravessaram as discussões, e, sem sombra de dúvidas, o projeto nos permitiu criar 
laços com os novos bolsistas, o que é um desafio no momento da COVID 19, em que o isolamento 
social nos colocou a prova nas nossas relações de sociabilidade. 
Com a leitura do livro e os debates acerca deste, os estudantes se apropriaram dasliteraturas
africanas, assim como conheceram um pouco das culturas apresentadas nas narrativas. O projeto 
de extensão Para além das páginas: vivenciando Leituras, permitiu nos formarmos não só como 
futuros pedagogos(as), mas como sujeitos, e nos mostrou a importância de buscar o conhecimento. 
Referências
ADICHIE,Chimamanda Ngozi. No seu pescoço. São Paulo: Companhia das Letras, 2017. 
BRASIL. Lei no
9.394, de 20 de dezembro de 1996. Estabelece as diretrizes e bases da educação 
nacional. Diário Oficial da União, Brasília, DF, 23 dez. 1996.
BRASIL. Lei no
10.639, de 9 de janeiro de 2003. Altera a Lei no 9.394, de 20 de dezembro de 
1996, que estabelece as diretrizes e bases da educação nacional, para incluir no currículo oficial 
da Rede de Ensino a obrigatoriedade da temática \"História e Cultura Afro-Brasileira\", e á outras 
providências. Diário Oficial da União, Brasília, DF, 10 jan. 2003.
BRASIL. Diretrizes Curriculares Nacionais para a Educação das Relações Étnico-Raciais e 
para o Ensino de História e Cultura Afro-Brasileira e Africana. Brasília: Conselho Nacional 
de Educação, 2004.

\includepdf{pdfs/No-Seu-Pescoco--Chimamand}

\addcontentsline{toc}{section}{Nossa Trajetória Literária: O Interpets Como Momento Paraa Formação Literária De Grupos Pets.}

\section*{Nossa Trajetória Literária: O Interpets Como Momento Paraa Formação Literária De Grupos Pets.}

Matthieu Octaveus, Daniele Drabeski, Matheus dos Santos Machado, Wellington dos Santos Machado, Luana Antonowicz de Souza, Josimeire Aparecida Leandrini ,

INTRODUÇÃO
De acordo com a SILVA et al. (2004), o âmbito acadêmico é um dos meios
incentivadores e levando os estudantes em busca de conhecimentos, enfatizando a leitura crítica
como se fosse uma forma de recuperar todas as informações acumuladas na história e usá-las de
maneira eficiente para o desenvolvimento do cotidiano. Assim, o hábito da leitura necessita de
um esforço pessoal de cada leitor, mas é um processo de construção ao longo do tempo, onde o
cérebro se desenvolverá, ainda na ausência desse exercício ele ficará em um estado de
desconforto, porque a leitura já faz parte da sua essência (SARTI et al., 2007).
Segundo Failla (2021), o Brasil perdeu, nos últimos quatro anos, mais de 4,6 milhões de
leitores, sendo que a porcentagem de leitores no Brasil caiu de 56% para 52%, de 2015 a 2019,
sendo observadas as maiores quedas no percentual de leitores entre as pessoas com ensino
superior (82% em 2015 para 68% em 2019). Deste modo, criar momentos que incentivem a
leitura acadêmica, pessoal e profissional entre os integrantes dos grupos PETs (sendo estes
também acadêmicos), é essencial para mudar e melhorar as dificuldades que ocorrem na vida
acadêmica dos estudantes.
METODOLOGIA
O InterPETs tem o intuito de aproximar os cinco grupos de Programas de Educação
Tutorial (PET), da Universidade Federal da Fronteira Sul, que está localizada em todos os
estados do sul do Brasil, onde cada encontro tem o objetivo de promover a interação entre os
grupos. Assim o PET Conexões de Saberes Políticas Públicas e Agroecologia do campus
Laranjeiras do Sul, foi o responsável pela organização do último encontro, com o tema “Nossa
Trajetória Literária\'\'.
Foi disponibilizado um questionário para os participantes responderem com informações
sobre a interação de cada um com a leitura. Foram coletadas 51 respostas, que foram
apresentadas e expostas durante a apresentação do evento e farão parte dos resultados .
RESULTADOS E DISCUSSÃO
De acordo com os dados coletados através do questionário, foi possível analisar que mais
de 23% dos respondentes, buscam ler todos os dias, 39% leem quando sobra algum tempo e 33%
com alguma frequência semanal. Isso mostra que os acadêmicos dos grupos PETs buscam ler
sempre que possível e que consideram a leitura muito importante tanto para a sua vida
acadêmica, quanto pessoal (98% das respostas). Porém, quando perguntamos sobre a quantidade
de livros lidos, mais de 50% destes, responderam ler entre 1 e 4 livros no ano, ou seja, um
número extremamente baixo e que reflete a realidade brasileira em sua quantidade. Sobre os
gêneros de obras preferidas para ler, as mais indicadas foram: Romance (60%); Ficção científica
(51%); Classicos (49%); contos (29%); e gibis e quadrinhos (27%).
Quando perguntados quais foram as pessoas que incentivaram a leitura na sua vida, nesta
questão poderiam ser indicados mais de uma pessoa, 76% afirmaram que os professores foram os
principais incentivadores, 43% pais e familiares e 35% foram incentivados por amigos. Nesse
sentido, os professores são os principais responsáveis pela leitura dos acadêmicos, pois são
considerados dependendo da forma que fala e trata a leitura desperta nos acadêmicos a
curiosidade e para novas leituras.
CONCLUSÃO
O debate realizado durante o encontro, sobre as obras que mais marcaram a vida de cada
um, se tornou um momento de descontração e aprendizado. Cada livro indicado pelo colega,
mostrou um pouco da experiência de cada um com os livros, com o conhecimento e com o
prazer de ler.
Além dos livros indicados, foi realizado um momento cultural, com apresentações de
danças, músicas, canções e declamações de poesia autoral ou não, de acadêmicos de diferentes
regiões, com culturas e valores diversificados. Descobrimos que alguns grupos têm escritores e
poetas que já publicaram obras. Contudo, descobrimos também que alguns acadêmicos são
incentivados somente à leitura científica e de uma única área, isto nos alertou da importância da
formação do indivíduo cidadão, da importância da leitura para entender a realidade.
Com relação às respostas obtidas, estas são importantes para compreendermos melhor
nossas limitações quanto leitores e estudantes, mas também, como podemos melhorar e criar
cada dia mais o hábito de ler, seja por necessidade ou por vontade, e que com o tempo,
consigamos criar o prazer de ler por nós mesmos.
REFERÊNCIAS
FAILLA, Z. (Org.). Retratos da Leitura no Brasil. 5. ed. São Paulo: Instituto pró-livro, 2021.
Disponível em:
. Acesso em: 09 set. 2021.
SARTI et al. Leituras Profissionais Docentes e Apropriação de Saberes
Acadêmico-educacionais. Cadernos de Pesquisa, v. 37, n. 131, maio/ago. 2007. Disponível em:
. Acesso
em: 09 set. 2021.
SILVA et al. A Avaliação da Compreensão em Leitura e o Desempenho Acadêmico de
Universitários. Psicologia em Estudo, Maringá, v. 9, n. 3, p. 459-467, set./dez. 2004. Disponível
em: .
Acesso em:09 set. 2021

\includepdf{pdfs/Nossa-Trajetoria-Literari}

\addcontentsline{toc}{section}{Notas Técnicas Abril Branco  E Animal Topics- Pet Produção Leiteira Conexão Do Conhecimento}

\section*{Notas Técnicas Abril Branco  E Animal Topics- Pet Produção Leiteira Conexão Do Conhecimento}

Lanna Cristyne De Oliveira Santos, Jéssica Bruna Verardo (Discente),Luana Pagliarini Castagnetti (Discente),Debora Kreczkiuski (Discente),Natasha Gabrielly Porrua (Discente),Ilana Niqueli Talino dos Santos (Discente),Fernando Kuss (Orientador)

Diante da dificuldade de transmitir conhecimento ou informações sobre assuntos que 
envolvem a cadeia leiteira para o público em geral de forma presencial, uma alternativa para 
resolver esse impasse foi a criação de notas técnicas por meio das redes sociais Facebook e 
Instagram do grupo PET- Produção Leiteira. Sendo assim, uma forma de repassar os conteúdos 
relevantes aprendidos pelos alunos integrantes do grupo no ensino acadêmico para o público em 
geral. Neste sentido, foram utilizados temas como Abril Branco sobre a campanha de prevenção 
das doenças metabólicas dos animais e também tópicos que envolvem tecnologia do leite. 
Utilizando o Facebook e Instagram como ferramentas importantes de contato entre o 
público que já segue a página do grupo e também com outros novos seguidores, essas plataformas 
foram escolhidas pelos integrantes pela sua notoriedade principalmente no contexto em que as 
informações são transmitidas de forma rápida e remota. Primeiramente, os integrantes do grupo 
PET- Produção Leiteira foram os responsáveis pela divisão dos assuntos abordados sobre o abril 
branco, que foram: hipocalcemia, deslocamento de abomaso, retenção de placenta e cetose. Dessa 
forma, totalizando 4 notas técnicas que foram escritas e publicadas como posts pelos petianos 
abordando essa temática sobre doenças bem como suas caracterizações, sinais clínicos, medidas 
preventivas e tratamentos.
A outra temática divulgada foi o Animal Topics, no qual os petianos escolheram 
conteúdos sobre freemartinismo, colostragem, silagem de sorgo, bem estar animal, enzimas e 
tecnologia do leite, uso de dejetos bovinos nas pastagens, vantagens no uso do sistema silvipastoril, 
importância da alimentação para bovinos leiteiros, importância da água no manejo alimentar 
bovinos leiteiros, fenação, e por fim, a bezerra de hoje será a vaca de amanhã. Desse modo, foram 
escritas 11 novas notas técnicas e publicadas em forma de posts no Instagram e Facebook.
Aproveitando as mídias sociais comumente já usadas pelo grupo para divulgação de 
fotos e convites de eventos, publicações das atividades desenvolvidas, entre outros, o perfil do 
Facebook e Instagram também se fizeram instrumentos essenciais de interação e de aprendizagem 
com o público pelo lançamento das notas técnicas. Ao abordar questões relacionadas às 
enfermidades que afetam a bovinocultura do leite é uma maneira de alertar a comunidade a 
relevância que a prevenção dessas doenças possui, que é de diminuir os danos que elas ocasionam 
(ECO, 2020). 
Além disso, essa experiência de compartilhamentos de conteúdos técnicos foi um 
desafio atingido, pois esses materiais foram redigidos com uma linguagem acessível para o público 
em geral sem perder seu caráter científico, que era um dos objetivos dessa atividade. Tal como 
foram abordados também os temas das publicações do Animal Topics, no qual teve como 
propósito de explicar e esclarecer pontos fundamentais da pecuária do leite.
Logo, com o advento da pandemia de COVID-19 e suas restrições de isolamento 
social, o modo de ensino e a realização das atividades do grupo PET- Produção Leiteira também 
tiveram que se adequar a essa nova realidade. Nesse viés, com a intenção de se manter conectado 
com a população externa e ainda compartilhar com a sociedade tópicos científicos pertinentes 
sobre a cadeia do leite, as redes sociais se mostraram um meio comunicativo eficaz, tanto pela 
quantidade de pessoas que já seguiam o perfil do grupo, ou seja, mais de mil pessoas, como pela 
rapidez que as informações postadas foram divulgadas. Somado a isso, todos esses informativos 
estão disponíveis livremente para quem tiver interesse de ler, tirar dúvidas e fazer comentários. 
Referências:
Lançamento: Abril branco [prevenção das doenças metabólicas]. ECO Diagnóstico Veterinário, 
2020. Disponível em:< https://ecodiagnosticavet.com.br/lancamento-abril-branco-prevencao-dasdoencas-metabolicas/>. Acesso em: 09 de set. 2021.

\includepdf{pdfs/Notas-Tecnicas-Abril-Bran}

\addcontentsline{toc}{section}{O Impacto Da Pandemia Da Covid-19 Nos Grupos Pet Da Uffs}

\section*{O Impacto Da Pandemia Da Covid-19 Nos Grupos Pet Da Uffs}

Guilherme Henrique Malinowski, João Vitor Pchirmer - petiano,Amanda Knorst Bellon - petiana,Maria Eduarda Artuso Schnorr - petiana,Mariana Valentini Casagrande - petiana,Adriana Kielek - petiana,Laura Dalcin Lorenzi - petiana,Karina Ramirez Starikoff - tutora

Em 2019 surgiu a Covid-19, doença que causaria um grande impacto político, econômico, 
social e de saúde mundial (OPAS/OMS, 2020). Para controlar o avanço da pandemia, medidas 
restritivas foram implantadas: o fechamento de creches, escolas e universidades, o uso de máscaras, 
distanciamento e isolamento social, em alguns momentos o lockdown.
O isolamento social pode causar um impacto negativo na saúde mental das pessoas. Além 
disso, o uso excessivo da internet tanto para amenizar a falta de sociabilidade, quanto para uso 
recreativo e profissional, também pode gerar problemas (CHANG; YUAN; WANG. 2020).
Com o objetivo de uma maior interação entre os grupos do Programa de Educação Tutorial -
PET da Universidade Federal da Fronteira Sul - UFFS, o PET Medicina Veterinária/Agricultura 
Familiar propôs um encontro virtual. Anteriormente ao encontro foi enviado um formulário online 
(Google Forms) à todos os petianos e tutores dos 5 grupos PETs da UFFS: PET Medicina 
Veterinária/Agricultura Familiar, PET Práxis - Conexões de saberes, PET Ciências, PET Assessoria 
Linguística e Literária, PET Conexão de Saberes – Políticas Públicas e Agroecologia.
O formulário foi dividido em 4 seções: na primeira as perguntas foram: Escreva em uma 
palavra ou expressão do sentimento vivido; Quais atividades participou?; Qual foi o maior impacto 
que sentiu/sofreu?; O que considera que foi bom? As respostas foram para cada trimestre da pandemia, 
de março de 2020 até fevereiro de 2021; A segunda seção foi sobre a Covid-19: Se contraiu?; Em que 
mês?; Se algum familiar contraiu?; A terceira seção continha perguntas sobre saúde mental como: Se 
está morando com quem?; Qual hábitos/hobbies desenvolveu durante a pandemia?; Se teve algum 
problema psicológico ou se já apresentava?; Se procurou ajuda?; E a quarta seção tratava sobre o PET: 
Quais atividades sofreram maior alteração?; E se teve dificuldades para conciliar as atividades do PET 
com o curso de forma online?.
No dia 15 de abril de 2021 ocorreu o encontro nomeado de INTERPET, em formato online 
pela plataforma Cisco Webex, que contou com a participação de 4 tutores e 55 petianos.
Durante o encontro foram apresentados os resultados obtidos no formulário, além de 
momentos para discussões, dinâmicas e apresentações realizadas por cada grupo, apresentando às 
soluções encontradas para atividades que foram impactadas pela pandemia.
Ao longo do ano de 2020, os acadêmicos dos 5 grupos PET da UFFS sentiram, principalmente, 
incerteza, ansiedade, esperança e medo. Essas palavras expressaram as dúvidas do início da pandemia 
e a volta de muitos para a casa de familiares, a ansiedade pela volta às aulas em formato online, além 
da esperança pela vacina e o medo pelo número crescente de casos e mortes pela Covid-19.
Além disso, atividades físicas, leituras, entretenimento por filmes ou séries foram as principais 
atividades desenvolvidas. Em um momento difícil da pandemia, os petianos elencaram o 
conhecimento pelas atividades e aulas, a presença da família e a vacina como os principais pontos 
positivos do ano de 2020. Porém, a Covid-19 também atingiu os estudantes, 7 petianos foram 
infectados e em todos os grupos tiveram casos de familiares afetados.
Os grupos relataram uma grande preocupação com seus familiares, com a volta às aulas e com 
saúde mental no início do evento, além da incerteza com os rumos da pandemia no Brasil e uma 
grande angústia pela vacinação. Ao fim, os participantes relataram a importância do evento como um 
momento de conversas e interação entre os grupos.
A pandemia da Covid-19 levou a sociedade a apresentar mudanças no dia a dia e nas formas 
de convívio, essas mudanças somadas ao ensino remoto desencadearam no agravamento dos 
problemas relacionados à saúde mental dos estudantes universitários (RODRIGUES et al., 2020). 
Questões de insegurança aumentaram durante a pandemia, estudantes universitários chineses tiveram 
um aumento significativo em casos de depressão e ansiedade a partir do confinamento social 
(RIBEIRO et al., 2020). Todavia, a rotina de muitos estudantes passou pela busca por atividades 
físicas, entretenimento e uma comunicação digital (MAIA e DIAS, 2020).
A incerteza e a dúvida que vieram junto com a pandemia da Covid-19, somado ao medo e a 
volta às aulas remotas evidenciaram problemas de saúde mental na vida dos petianos, todavia as 
atividades realizadas como leituras, atividades físicas e a interação entre os grupos promoveram
interações pessoais que contribuíram para melhora da saúde mental.
Referências:
CHANG, Jinghui; YUAN, Yuxin; WANG, Dong. Mental health status and its influencing factors 
among college students during the epidemic of COVID-19. Nan fang yi ke da xue xue bao= Journal
of Southern Medical University, v. 40, n. 2, p. 171-176, 2020.
MAIA, Berta Rodrigues; DIAS, Paulo César. Ansiedade, depressão e estresse em estudantes 
universitários: o impacto da COVID-19. Estudos de Psicologia (Campinas), v. 37, 2020.
ORGANIZAÇÃO PAN-AMERICANA DE SAÚDE. Histórico da pandemia de Covid-19.
OPAS/OMS, 2020.
RIBEIRO, Eugénia et al. Impacto psicológico da pandemia em estudantes universitários e a Linha de 
Apoio Psicológico SOS COVID-19 (APsi-UMinho e EPsi). 2020.
RODRIGUES, Bráulio Brandão et al. Aprendendo com o Imprevisível: Saúde mental dos 
universitários e Educação Médica na pandemia de Covid-19. Revista Brasileira de Educação 
Médica, v. 44, 2020.

\includepdf{pdfs/O-Impacto-Da-Pandemia-Da-}

\addcontentsline{toc}{section}{Oficina De Pensamento Computacional}

\section*{Oficina De Pensamento Computacional}

Victor Hugo Garrett, Cesar Tacla(PETECO-UTFPR),Diogo Da Silva Gouveia(PETECO-UTFPR),Felipe Augusto Lee(PETECO-UTFPR),Gabriela Bogomolof Taquegami(PETECO-UTFPR)

O que é Pensamento Computacional?
Para (Aho, 2011), Pensamento Computacional é entendido como a abordagem de
problemas que busca torná-los resolvíveis por uma sequência finita de passos bem definidos, isto
é, que possa ser processada por um computador. Algumas das características principais dessa
forma de pensar são: decomposição, abstração, reconhecimento de padrões e algoritmização.
Decompor um problema significa dividi-lo em partes menores que, embora coesas, tenham o
máximo de independência, o que permite tratar algo muito complexo como uma união de coisas
mais simples que, resolvidas uma a uma, levam à solução do todo. Abstrair é olhar através das
particularidades de um certo problema e ver sua forma fundamental, aquilo que o caracteriza. O
reconhecimento de padrões diz respeito a como um problema está ligado a outros, e por isso
depende fortemente da abstração, que permite encontrar equivalências apesar das diferenças
superficiais. Por fim, a algoritmização tem a ver com expressar a solução encontrada em passos
sequenciais genéricos que podem ser usados para resolver todas as instâncias do problema.
Por que ensinar Pensamento Computacional?
\"A educação não mudou para atender às necessidades do mundo a sua volta. O ambiente
do trabalho de hoje demanda que se trabalhe em pequenos grupos para solucionar problemas,
precisa de ferramentas digitais e que as pessoas estejam preparadas para desempenhar
multitarefas sem a supervisão de outros.” A partir desta contestação em um evento sobre gestão
educacional de 2013, Jim Langel, escritor e professor universitário, revela a importância de
adaptar suas habilidades às situações e problemas atuais. O Pensamento Computacional surge da
necessidade de ensinar competências contemporâneas diversas, como: acadêmicas, éticas,
políticas e tecnológicas. Assim, o ganho cognitivo dos estudantes reflete diretamente na forma de
análise e resolução de problemas e na descrição e explicação de situações complexas. Ademais,
jovens munidos de um pensamento computacional forte têm maior capacidade de analisar dados
logicamente, compreender temas abstratos e particionar problemas complexos por meio da
discussão e mapeamento de ideias.
Pensamento crítico, criatividade, colaboração e comunicação, são ingredientes essenciais
no currículo de uma escola, além de trazer benefícios para a autonomia dos estudantes, as
atividades ligadas à ciência da computação tem papel no desenvolvimento dessas habilidades,
com diversas aplicações nas mais diversas áreas. Atividades escolares podem ser aprofundadas
com o uso das habilidades cognitivas adquiridas, desde a criação de um algoritmo para
multiplicação de matrizes, modelagem de fenômenos físicos, até a criação de modelos sociais
através da abstração da realidade.
Um jogo para ensinar Pensamento Computacional
Com objetivo de introduzir o pensamento computacional a estudantes do ensino
fundamental e médio, foi desenvolvido um jogo simples em linguagem C++ com a biblioteca
gráfica SFML. O jogo é baseado no problema dos Missionários e Canibais. Nesse problema, o
objetivo é passar todos os personagens, missionários e canibais, de uma margem de um rio até a
outra utilizando um barco com dois espaços, com a condição de nunca ter, em nenhuma das
margens, mais canibais do que missionários. O jogo permite instanciar o problema com um ou
até quatro personagem de cada tipo, para dar ao usuário a noção do aumento da complexidade
devido à quantidade de elementos, uma vez que a solução com um personagem de cada tipo tem
apenas um movimento enquanto a solução com quatro personagens de cada tipo é impossível
com um barco de tamanho dois. Os estudantes resolvem todas as instâncias do problema, da mais
fácil até a mais difícil, e seus resultados são salvos para discussão, comparação e análise. Além
do aumento de complexidade pela observação do tamanho da solução, objetiva-se que os
participantes da atividade observem o que há de comum entre as soluções de crescente
dificuldade, exercitando a abstração e reconhecimento de padrões.
Resultados
O jogo foi utilizado em oficinas de cerca de 40 minutos com 150 estudantes do 8o ano de
três escolas do ensino público municipal de Curitiba nos anos de 2018 e 2019. As petianas/os
organizaram as turmas e conduziram a oficina. Também trocaram experiências com as
professoras/es das escolas durante a realização da atividade e puderam compreender de forma
mais detalhada as dificuldades de ensino e de motivação para estudo de matemática, disciplina
básica da computação. Percebeu-se grande interesse dos/das estudantes em solucionar o
problema e as/os professores se mostraram entusiastas da ideia de juntar matemática e
computação. A partir da experiência da realização das oficinas, as/os petianas/os vislumbraram o
desenvolvimento de um instrumento de coleta de dados mais apropriado, ou seja, que capture o
percurso de cada equipe na construção da solução ao problema. O grupo PET demonstrou
satisfação pelo conhecimento que levaram aos estudantes e pelo que aprenderam, vivenciando o
processo dialógico de ensino-aprendizagem.


\includepdf{pdfs/Oficina-De-Pensamento-Com}

\addcontentsline{toc}{section}{Oficina De Produção De Produtos De Limpeza Como Fonte Alternativa De Renda Para Mães Em Situação De Vulnerabilidade Social}

\section*{Oficina De Produção De Produtos De Limpeza Como Fonte Alternativa De Renda Para Mães Em Situação De Vulnerabilidade Social}

Maria Eduarda Goncalves, Andrieli Parolin PET-EQ UTFPR,Calina Razani PET-EQ UTFPR,Matheus dos Santos Macedo PET-EQ UTFPR

Levando em conta as dificuldades econômicas das mães envolvidas nesse projeto, o PET Engenharia Química da UTFPR desenvolveu uma oficina de produtos de limpeza para que os participantes pudessem aprender a produzir esses itens e utilizá-los para aumentar a renda familiar, assim ajudando no sustento deles e das crianças envolvidas.

\includepdf{pdfs/Oficina-De-Producao-De-Pr}

\addcontentsline{toc}{section}{Oficinas De Introdução Aos Gêneros Acadêmicos: (Re)Pensando A Práxis Empregada }

\section*{Oficinas De Introdução Aos Gêneros Acadêmicos: (Re)Pensando A Práxis Empregada }

Icaro Cesar Cainan Da Cunha Claro Olanda, Rafael Ramos Martins (Universidade Federal do Pampa - UNIPAMPA)

Distanciando-nos da Linguística Textual e de seu raciocínio analítico da língua, neste
relato de experiência, nos acercamos das reflexões do filósofo da linguagem Mikhail Bakhtin,
sobretudo em suas proposições a respeito das esferas da atividade humana e a sua intrínseca
relação com a linguagem, que nos possibilita o (re)pensar da práxis empregada em três oficinas
de “Introdução aos gêneros acadêmicos”, ministradas pelos bolsistas do Programa de Educação
Tutorial (PET Letras), vinculados aos cursos de Letras, da Universidade Federal do Pampa,
campus Jaguarão. Ministradas no primeiro semestre de 2021, estas oficinas, cabe justificar,
surgiram em parceria com o Projeto de Apoio Social e Pedagógico (PASP), um projeto
institucional ligado à mesma instituição de ensino mencionada, à convite da professora
responsável por sua execução, no contexto de ensino remoto. As oficinas foram delineadas para
apresentar, em especial, aos acadêmicos ingressantes os gêneros fichamento, resumo e resenha a
partir de uma roda de conversa (e mais três encontros) sobre gêneros discursivos que circulam na
esfera acadêmica. Neste sentido, nossas palavras aqui são/serão palavras de outros que “trazem
consigo a sua expressão, o seu tom valorativo que assimilamos, reelaboramos, e reacentuamos”
(BAKHTIN, 2016, p. 54), isto é, nossas contrapalavras são, em certa medida, espelhamentos
reacentuados das contribuições de linguistas como Geraldi (2010), dentre outros, pautadas no
viés bakhtiniano.
Ancorados no locus de enunciação das ciências humanas, as oficinas de “Introdução aos
gêneros discursivos acadêmicos” foram elaboradas sob o viés de uma metodologia qualitativa,
isto é, estamos interessados em um diálogo, não com o critério da ciência da exatidão da língua
análitica, mas com os sujeitos e o objeto de aprendizagem ou nas palavras de Bakhtin (2011, p.
394): “o critério não é a exatidão do conhecimento, mas a profundidade da penetração [...]”,
sendo este conhecimento localizado no humano heterogêneo, isto é, na individualidade de cada
sujeito.
Posto isso, as oficinas foram organizadas para que os seus participantes pudessem ter
uma experienciação com os gêneros fichamento, resumo e resenha, precedidos por uma roda de
conversa em consonância com a temática, dentro da esfera acadêmica. Elas foram aplicadas em
datas com um intervalo de 15 dias entre uma oficina e outra, sugeridas pelo PASP em acordo
com o grupo. Para tanto, os bolsistas do PET Letras já divididos, previamente, em grupos de
trabalho, escolheram por afinidade qual gênero discursivo seria apresentado nas oficinas. Então,
estes grupos começaram, juntamente com a tutora do PET, a se reunir para a preparação das
oficinas, uma vez que a relação Teoria/Prática precisa ser o reflexo consciente de ambas e da
prática docente (FREIRE, 2002). Neste sentido, os bolsistas fizeram reuniões para a discussão de
textos teóricos sugeridos por eles, bem como pela tutora, para a elaboração das oficinas que
contaram com a exposição do gênero escolhido. Também houve a seleção de textos (exemplos de
gêneros) e de uma apresentação destes em powerpoint que nortearam as discussões no dia de
cada atividade. Cabe ressaltar que esses movimentos foram essenciais no processo, entendido
por Freire, em seu livro Pedagogia da Autonomia, como um “não há docência sem discência”.
Se a natureza do enunciado, que por sua vez, se funde à condição da natureza humana, é
particular, a sua materialização está atrelada às distintas instituições sociais que, assim, elaboram
“seus tipos relativamente estáveis de enunciados, os quais denominamos gênero do discurso”
(BAKHTIN, 2016, p. 12). Contudo, em especial, na última década, no Brasil, o pensamento de
Bakhtin foi tomado, erroneamente, como um passe-livre para, o que denominou Geraldi (2010),
uma “gramaticalização dos gêneros [discursivos]”. Segundo este linguista, o trabalho com os
gêneros se configurou numa ruptura com a tradição de ensino das regras gramaticais normativas,
o que levou à perda de sua real função sociocomunicativa e passou a ser banalizado e reduzido a
uma repetição de estrutura fixa. Posto isso, nós, os bolsistas do grupo PET Letras Jaguarão, ao
elaborarmos as oficinas de \"Introdução ao gêneros discursivos acadêmicos”, nos propusemos a
não gramaticalizar os gêneros que foram apresentados, a saber: fichamento, resumo e resenha;
uma vez que, não poderíamos -nem acreditamos-, como muitos estudantes, neste caso, almejam
um método perfeito para seguir e, consequentemente, obter resultados positivos no que concerne
ao objeto de aprendizagem.
Se compartilhamos da premissa de que nosso locus de enunciação é o da ciências
humanas discutida por Bakhtin, chegamos ao final deste relato de experiência não com uma
conclusão pronta, fechada, mas damos nossa contrapalavra em forma de considerações. Isto é,
ao refletirmos sobre a práxis empregada no decorrer das oficinas de “Introdução aos gêneros
[discursivos] acadêmicos”, pudemos nos colocar, enquanto bolsistas do PET Letras e, sobretudo,
professores em formação, em um contato direto entre a relação de ensino e aprendizagem, numa
perspectiva qualitativa em que a reflexão a respeito da aprendizagem recai sobre a interlocução
entre os sujeitos, em outras palavras sobre os enunciados presentes na construção dos gêneros
discursivos que integram a esfera acadêmica e que não se limitam, ao que Geraldi (2010)
denominou, uma repetição de estruturas fixas de gêneros.
BAKHTIN, Mikhail. Estética da criação verbal. Tradução do russo de Paulo Bezerra. São
Paulo: Martins Fontes, 2011.
BAKHTIN, Mikhail. Os gêneros do discurso. Bezerra, Paulo. Notas da edição russa: Serguei
Botcharov. São Paulo: Editora 34, 2016.
FREIRE, Paulo. Pedagogia da autonomia. Rio de Janeiro: Paz e Terra, 2002.
GERALDI, João Wanderley. A aula como acontecimento. São Carlos: Pedro \& João Editores,
2010.

\includepdf{pdfs/Oficinas-De-Introducao-Ao}

\addcontentsline{toc}{section}{Os Desafios E Oportunidades Das Atividades On-Line Do Grupo Petamb Durante A Pandemia}

\section*{Os Desafios E Oportunidades Das Atividades On-Line Do Grupo Petamb Durante A Pandemia}

Renata Mendes Serralheiro, Déborah Bozz,Diana Elena Sosa Gimenez,Luis Felipe Pinheiro,Tairone Cesar da Silva Pereira Junior,Laercio Mantovani Frare

Introdução
A pandemia, nas universidades, foi a responsável pela adoção das medidas de distanciamento 
social. Devido às providências aplicadas para evitar a disseminação do vírus, estudantes de todo o 
país deixaram de frequentar as instituições de forma presencial passando a realizar atividades em 
formato remoto. Nos treze campus da Universidade Tecnológica Federal do Paraná (UTFPR) as 
aulas presenciais foram suspensas em dezesseis de março de 2020. Juntamente com as suspensões 
das atividades acadêmicas, iniciaram as incertezas das formas para realizar as atividades 
programadas durante o planejamento do grupo PETAMB (Programa de Educação Tutorial em 
Ambiental). O PETAMB funciona no campus Medianeira e é constituído por acadêmicos dos 
cursos de Engenharia Ambiental e Tecnologia em Gestão Ambiental e tem, como propósito, 
desenvolver atividades na tríade de ensino, pesquisa e extensão. Portanto, desde o início deste 
período de afastamento social, enfrentou vários desafios para desenvolver o seu objetivo devido à 
suspensão das atividades presenciais. Neste contexto, o objetivo deste trabalho é apresentar os 
principais desafios e oportunidades que surgiram neste período para realização das atividades, a 
partir das percepções do grupo.
Metodologia
O PETAMB, para conseguir cumprir suas atividades de ensino, pesquisa e extensão, adotou as 
seguintes estratégias no período de pandemia: manteve as reuniões semanais por meio de vídeoconferências e a discussão de todas as atividades programadas com vistas a alterá-las para serem 
realizadas on-line. Durante as reuniões foram apresentadas as atividades realizadas pelo grupo e 
estudadas as possibilidades de adaptação das mesmas em formato não presencial a partir do uso 
de ferramentas adequadas. No decorrer dos encontros, surgiram diversas ideias, por parte dos 
integrantes, tais como: webinários, vídeo aulas, podcast, postagens em redes sociais, desafios online entre outras atividades substituíram as programadas de forma presencial. A partir destas ideias, 
as atividades foram adaptadas, divididas entre os integrantes do grupo e executadas. 
Resultados e Discussões 
Os seminários tradicionais do PETAMB, conhecidos como Ciclo de Debates, foram reformulados 
para webinários. Os webinários foram transmitidos pela plataforma StreamYard para o canal do 
PETAMB no YouTube. A organização destes webinários possibilitou, a todos os integrantes do 
grupo, adquirirem e/ou fortalecerem competências que incluíram o aprendizado de novas 
tecnologias, a necessidade da organização de todas as etapas durante a divulgação, a transmissão, 
a certificação dos participantes, o trabalho em equipe de forma remota e o comprometimento com 
cada um dos webinários. As videoaulas foram uma das atividades mais utilizadas durante esse
período. As videoaulas são muito usadas na educação à distância, com o objetivo de ilustrar, 
reforçar e complementar o conteúdo de um curso ou disciplina. Por meio da videoaula o professor 
tem a liberdade para ensinar o conteúdo das mais variadas formas. Com as atuais metodologias 
ativas, gravar aulas é uma forma de tornar o processo de ensino/aprendizagem mais efetivo para o 
aluno. As videoaulas foram disponibilizadas tanto para a turma quanto nas plataformas Google 
Classroom e também no canal do YouTube do PETAMB. Os podcast, chamados de PETCast, 
também proporcionaram ótimos retornos. Os áudios foram disponibilizados na plataforma Anchor 
e gravados por integrantes do PETAMB com diversos convidados especiais (professores, egressos, 
profissionais de outras áreas, etc) com assuntos inovadores. 
Conclusão 
Mediante o exposto, todas as atividades desenvolvidas durante a pandemia foram importantes para 
manter o grupo ativo e produtivo de forma on-line. A percepção e avaliação, pelos membros do 
grupo, demonstrou o desafio que foi manter o grupo funcionando de forma remota e o grande 
número de oportunidades de abranger novos métodos de trabalho e a aquisição de mais 
conhecimento por parte dos integrantes. Além disso, essas atividades desenvolvidas de forma online são possíveis de serem mantidas quando as atividades voltarem a ser presenciais permitindo 
uma maior flexibilidade e atingindo um público alvo maior

\includepdf{pdfs/Os-Desafios-E-Oportunidad}

\addcontentsline{toc}{section}{Perfil De Egressos Do Curso De Tecnologia Em Alimentos Da Universidade Tecnologica Federal Do Paraná – Campus Francisco Beltrão}

\section*{Perfil De Egressos Do Curso De Tecnologia Em Alimentos Da Universidade Tecnologica Federal Do Paraná – Campus Francisco Beltrão}

Lucas Dalaqua Ribeiro, Amanda Georg Gebim,Carlos Alberto Câmara Leal de Oliveira,Larissa Barbosa de Oliveira,Alexandre da Trindade Alfaro

O Curso Superior de Tecnologia em Alimentos da Universidade Tecnológica Federal do Paraná, 
campus Francisco Beltrão (UTFPR – FB), foi ofertado no período de 2008-2014. A partir do ano 
de 2015, o curso migrou para a Engenharia de Alimentos, não havendo mais a entrada de novos 
estudantes. O Curso Superior de Tecnologia em Alimentos, conferia aos discentes habilidades e 
competências para atuarem em diferentes funções, ligadas a área da ciência e tecnologia de 
alimentos. A Tecnologia em Alimentos, formou dezenas de profissionais habilitados para atuar na 
industrialização de alimentos, desse modo, suprindo a necessidade da região por mão de obra 
qualificada e contribuindo com a expansão do setor. O presente trabalho teve como objetivo traçar 
o perfil dos egressos do Curso Superior de Tecnologia em Alimentos da UTFPR – FB, 
identificando se estão atuando na área de formação. Para realização do trabalho, foi fornecida pela 
coordenação do curso, uma lista com os dados de contato (e-mail e telefone) dos egressos. O 
contato com os egressos foi realizado através do aplicativo de mensagens Whatsapp e pelas redes 
sociais Facebook e Instagram. Após o contato com o egresso, era realizada uma breve 
contextualização do objetivo da informação e três preguntas eram feitas: (1) Se o egresso atuava 
ou não na área de formação? (2) Se “sim”, onde trabalhava e a quanto tempo atuava na área? (3) 
Se “não”, qual o motivo? A pesquisa foi realizada com 164 egressos, sendo obtida a resposta de 
135. Dos egressos que responderam o questionário, 91 afirmaram “sim” e 44 “não” para a primeira 
pergunta. Para a pergunta 2, 18 egressos relataram que atuam na área de controle de qualidade, 7 
na gestão de produção e 4 na gerência industrial, além de 20 que informaram que atuam como 
bolsistas em programas de pós-graduação. Os demais responderam que trabalham em diversas 
áreas/setores da tecnologia em alimentos. Na pergunta 3, parte considerável dos egressos relataram 
a falta de incentivo para profissionais que estão ingressando no mercado de trabalho, enquanto
outros, informaram que buscaram melhores oportunidades em outros setores. Uma parcela menor 
de egressos, relatou que buscaram formação complementar em áreas correlatas a tecnologia em 
alimentos. Desse modo, observou-se que 67,4% dos egressos atuam na área de formação, onde 71
(52%) desses egressos estão trabalhando em diferentes empresas processadoras de alimentos e 20 
(15%) estão se capacitando em programas de pós-graduação da área.

\includepdf{pdfs/Perfil-De-Egressos-Do-Cur}

\addcontentsline{toc}{section}{Perfil, Condições E Desafios Da Formação Do(A)S Estudantes De Graduação Noturna Da Saúde/Ufrgs:  Serviço Social, Odontologia, Psicologia E Saúde Coletiva}

\section*{Perfil, Condições E Desafios Da Formação Do(A)S Estudantes De Graduação Noturna Da Saúde/Ufrgs:  Serviço Social, Odontologia, Psicologia E Saúde Coletiva}

Rafaella Tomasi, Ariel Bertoni Lopes (PET Conexões de Saberes, Cursos da Saúde Noturno - UFRGS),Cibele Pitthan da Silva (PET Conexões de Saberes, Cursos da Saúde Noturno - UFRGS),Layla Nicoly Mattos Medeiros (PET Conexões de Saberes, Cursos da Saúde Noturno - UFRGS),Loan Tonial Tomiello (PET Conexões de Saberes, Cursos da Saúde Noturno - UFRGS),Wellington Luis Xavier Mancilha (PET Conexões de Saberes, Cursos da Saúde Noturno - UFRGS),Orientador(a): Loiva Mara de Oliveira Machado (PET Conexões de Saberes, Cursos da Saúde Noturno - UFRGS)

Introdução:
Ao longo de quase uma década, com frequência os/as estudantes participantes deste PET, relatam as
dificuldades vivenciadas por serem alunos/as dos cursos noturnos. Essas dificuldades são encontradas tanto na
dimensão de estrutura da universidade, quando a consideram não adequada para atender de forma igualitária os/as
estudantes que realizam graduação diurna e noturna, bem como, destacam o aspecto pedagógico que não reconhece
suficientemente as particularidades dos/as estudantes noturnos. Nesse sentido, os/as estudantes noturnos sentem-se
prejudicados em não acessar a formação ofertada pela UFRGS com a mesma excelência dos cursos diurnos.
Justificativa:
Na última década, o governo federal propôs a expansão e a reestruturação das universidades por meio do
Programa de Apoio a Planos de Reestruturação e Expansão das Universidades Federais (REUNI). Uma das
dimensões do REUNI apontou a ampliação da oferta de educação pública superior, com o aumento de vagas para
ingresso na universidade, especialmente no período noturno. Esse cenário de mudanças afetou também a
Universidade Federal do Rio Grande do Sul (UFRGS) que, em outubro de 2007, aprovou o encaminhamento ao
Ministério da Educação, de Proposta Institucional UFRGS, em atendimento ao REUNI.
Os/as estudantes do ensino diurno, que não trabalham, vivem a universidade de uma maneira muito
diferente dos/das estudantes que trabalham o dia todo e somente à noite conseguem participar da vida acadêmica.
Tais alunos/as, geralmente participam das aulas, mas não conseguem, por exemplo, participar de atividades do tripé
(ensino, pesquisa e extensão). O/a trabalhador/a que também estuda é duplamente um trabalhador/a e o seu trabalho
de estudar está incluído na categoria de sobre trabalho (POCHMANN, 2004), pois ultrapassa muitas vezes a jornada
de 44 horas de trabalho semanal (MESQUITA, 2010, p. 82).
De acordo com Vargas e Paula (2013), é fundamental que a instituição de ensino se prepare para o desafio
de oferecer um curso noturno, com relação às condições oferecidas ao estudante do curso noturno, ao significado de
estudar à noite e trabalhar, à comparação entre cursos diurno e noturno e à equalização das oportunidades de estudo
que o curso oferece ao estudante trabalhador/a e não trabalhador/a. Aspectos relacionados ao acesso à biblioteca, aos
laboratórios, às salas de aula e aos equipamentos, além de espaços de convivência, como cantina, centro acadêmico
e áreas de vivência, também podem se constituir como motivadores ou não para que o/a estudante deseje frequentar
a instituição de ensino (TERRIBILI FILHO, 2009).
Objetivo geral:
Analisar como se conforma o perfil dos estudantes e as condições em que se realiza a formação nos cursos
da saúde noturnos da UFRGS com vistas a contribuir para o fortalecimento da qualidade do ensino superior noturno.
Objetivos Específicos:
a) Investigar as principais demandas e dificuldades dos estudantes dos cursos noturnos da saúde da
UFRGS com a finalidade de entender os elementos que comprometem a permanência e garantia da
conclusão de uma formação de qualidade.
b) Caracterizar o perfil dos estudantes dos cursos da saúde noturnos com vistas a analisar os desafios e
possibilidades para permanência e conclusão do curso nesta universidade.
c) Analisar as particularidades do trabalhador estudante com vistas a refletir sobre medidas necessária no
âmbito do ensino superior para que se mantenha a qualidade do ensino, permanência e conclusão do
curso.
d) Verificar as possibilidades proporcionadas pelo ensino superior noturno, a fim de contribuir para o
fortalecimento e ampliação das mesmas.
e) Possibilitar a vivência da pesquisa aos petianos, com vistas a desvendar a realidade vivenciada pelos
estudantes dos cursos da saúde noturnos e propor ações a partir do desvendamento desta realidade.
Metodologia:
O caminho metodológico desta pesquisa desenvolveu-se, primeiramente, a partir da criação de um Grupo
de Estudo e Trabalho (GET) composto pela tutora do grupo e, pelo menos, um estudante de cada um dos cursos que
formam o PET. As reuniões do GET ocorriam uma vez a cada semana. Nos primeiros encontros foram realizadas
formações sobre como elaborar um projeto de pesquisa; posteriormente, seguiu-se para etapa de busca por
bibliografias sobre o tema e a devida apropriação teórica. Então iniciou-se a escrita coletiva do projeto de pesquisa.
Concluídos todos os elementos do projeto de pesquisa, este foi ao Comitê de Pesquisa e Comitê de Ética da
universidade, incluindo o instrumento de coleta de dados e o termo de consentimento. Após avaliação e aprovação,
deu-se início à fase de coleta da pesquisa. Foi utilizado formulário eletrônico, encaminhado às Comissões de
Graduação dos cursos e mídias sociais, objetivando atingir o maior número possível de estudantes.
Resultados preliminares:
Atualmente a pesquisa se encontra na fase de análise e tratamento dos dados, processo que vem se
realizando através do software SPSS. Ainda assim, já é possível realizar alguns apontamentos mais gerais sobre o
perfil do público-alvo da pesquisa com base nas respostas obtidas através de formulário eletrônico. No total, 204
estudantes participaram da pesquisa. Destes, em relação ao gênero, as mulheres representam 79,9%, os homens
18,1%, e pessoas não-binárias e não declaradas totalizam 1% cada. No quesito raça/cor, estudantes brancos são
maioria, constituindo 62,3% dos participantes, já os/as estudantes negros representam 37,2% dos participantes,
sendo este grupo constituído por pretos (18,1%) e pardos (19,1%). Nenhum estudante indígena respondeu o
formulário, entretanto, destaca-se que há alunos/as indígenas matriculados nos cursos analisados. Os dados em
análise vêm apontando para o necessário planejamento pedagógico e garantia de infraestrutura institucional, de
modo a atender as demandas dos/as estudantes de cursos noturnos que também produzem conhecimentos em
diferentes áreas da universidade.
Referências:
MESQUITA, M. C. G. D. O Trabalhador estudante do ensino superior noturno: Possibilidades de acesso,
permanência com sucesso e formação. 2010. 193 f. Tese (Doutorado em Ciências Humanas) - Pontifícia
Universidade Católica de Goiás, GOIÂNIA, 2010.
POCHMANN, Marcio. Educação e trabalho: como desenvolver uma relação virtuosa. Educação e Sociedade,
Campinas, v. 25, n. 87, p. 383-399, maio/ago. 2004. Disponível em:. Acesso em: 20
out 2019
VARGAS, H. M.; PAULA, M. F. C. A inclusão do estudante-trabalhador e do trabalhador estudante na educação
superior: desafio público a ser enfrentado. Avaliação, Campinas, v.18, n. 2, p. 459-485, jul. 2013.
TERRIBILI FILHO, A.; NERY, A. C. B. Ensino superior noturno no Brasil: história, atores e políticas. RBPAE,
Goiânia, v. 25, n. 1, p. 61-81, 2009

\includepdf{pdfs/Perfil--Condicoes-E-Desaf}

\addcontentsline{toc}{section}{Pet Convida (Igtv)}

\section*{Pet Convida (Igtv)}

Roberto Lorenzo Carminatti, Juliana Jobim Jardim (UFRGS),Roberta Machado (UFRGS),Gustavo Almansa (UFRGS),Milena Piccinini (UFRGS),Débora Scheck (UFRGS),Ludmila Duarte Dias (UFRGS)

Problemática: A atividade de extensão PET-Convida surgiu da oportunidade de trazer 
conhecimentos de diferentes áreas para o dia-a-dia da graduação da Odontologia. Com essa 
atividade, os estudantes adquirem conhecimentos que não se limitam apenas à área técnica e 
prática da Odontologia, ampliando seu horizonte de conhecimento, auxiliando na sua convivência 
consigo mesmo e outrem. 
Metodologia: Durante a rotina usual da universidade, eram convidados professores doutores, para 
que ministrassem uma palestra, e ao fim da mesma, participar e orientar a discussão do assunto 
com os alunos da graduação. No atual contexto de pandemia, a atividade foi adaptada para o 
formato audiovisual, com vídeos sendo gravados pelos convidados e postados no IGTV do Pet 
Odontologia UFRGS (@petodontoufrgs), respondendo a perguntas de um roteiro elaborado pelo 
grupo, sobre assuntos de saúde do interesse da comunidade acadêmica da Faculdade de 
Odontologia. 
Resultados: A atividade conseguiu grande adesão do público, e gerou alto grau de engajamento 
dos participantes no modelo presencial. No entanto, no modelo de atividade por meio das mídias 
audiovisuais, obteve-se uma alta escala de difusão, já que cada vídeo chegou na casa de centenas 
e até milhares de visualizações, com repercussões positivas vindas de dentro e fora da comunidade, 
com grande felicidade também dos doutores que se dispuseram a se colocar frente às câmeras. 
Discussão: Diante das positivas repercussões, da facilidade de execução da atividade e da difusão 
de conhecimento de maneira quase que perpétua – por estar hospedado na internet – a continuação 
da atividade não é dúvida para o grupo, ainda em situação excepcional de distanciamento social. 
Porém, quando houver o retorno às atividades presenciais na faculdade, deverá se ponderar sobre 
qual modelo e futuro a atividade terá: o de retorno ao presencial ou a continuação da atividade por 
meio das mídias, e sobre qual o projeto se beneficiará mais. 
Conclusão: A atividade cumpre seu objetivo primário de difundir conhecimentos que urgem a 
comunidade acadêmica, bem como dicas de estilo de vida e estudos. A alta difusão dos vídeos 
representa um bom engajamento, assim como a adesão da comunidade acadêmica ao 
acompanhamento da atividade. Portanto o PET Convida, hoje hospedado no IGTV, está e 
continuará como uma das principais atividades de extensão do grupo.

\includepdf{pdfs/Pet-Convida-(Igtv)}

\addcontentsline{toc}{section}{Pet Discute A Engenharia Civil}

\section*{Pet Discute A Engenharia Civil}

Leticia Barcellos De Moraes, Amanda Bitencourt (Lucila Bitencourt e Evandro Bitencourt),Laisa Cancian (Nelci Maria Uliana Cancian e Adilson Angelo Cancian),Lara Rosa Ceolin (Rubiara da Silva Rosa e Adriano Ceolin),Luan Somavilla da Rosa (Andrea Prevedello Somavilla e Mychael Carvalho da Rosa),Mauricio Machado Mendes Peres (Deise Rejane Oliveira Machado e Gerson Mendes Peres)

O curso de Engenharia Civil da Universidade Federal de Santa Maria (UFSM) já passou por cinco 
estruturas curriculares diferentes desde sua implementação, sendo a última datada do ano de 2005. 
O atual currículo é tradicional; e as aulas são, em sua maioria, inteiramente expositivas, com pouca 
interação dos acadêmicos. Todavia, é de conhecimento geral que, desde 2005, muitas tecnologias 
foram criadas e difundidas. Hoje é comum os estudantes disporem de smartphones, notebooks e 
tablets. As salas e corredores da universidade dispõem de redes wifi; e o Moodle e os portais da 
universidade passaram a ser elementos indispensáveis para a interação dos discentes com o curso. 
Essa evolução deveria chegar, também, às salas de aula. No entanto, infelizmente muitas dessas 
inovações não são contempladas no currículo vigente nem nas práticas de ensino. A matriz 
curricular atual, com a maioria das disciplinas obrigatórias e poucas optativas, sem disciplinas 
integradoras e de síntese, e que não valoriza as atividades extracurriculares, reflete a desconexão 
do atual Projeto Pedagógico do Curso (PPC) com as demandas da sociedade e dos acadêmicos. 
Para mais, a publicação das novas Diretrizes Curriculares Nacionais (DCNs) prevê que os 
currículos precisam ser dinâmicos, alicerçados em competências e não em conteúdos, além de 
atribuir maior expressão para a extensão. Nesse sentido, surge a necessidade de um projeto que 
busque ampliar a discussão do novo currículo do curso e das dinâmicas em sala de aula. O PET 
Discute propõe essa discussão ampla, uma vez que conduz a comunidade do curso para participar 
dessa construção junto a todos os envolvidos. À vista disso, o projeto tem como objetivo discutir 
os estudos e o curso de Engenharia Civil da UFSM como um todo, a fim de otimizar a 
aprendizagem dos acadêmicos, bem como promover as reformas necessárias para uma graduação 
mais íntegra e eficiente. Busca-se uma matriz curricular e um PPC que sejam capazes de dialogar 
com uma sociedade e um mercado de trabalho dinâmicos, em constante evolução. Além disso, o 
projeto visa possibilitar uma maior interação entre discentes e docentes, por meio de um espaço 
de troca de experiências e questionamentos. Isto posto, inicialmente foram feitas reuniões entre os 
organizadores para a estruturação prévia do projeto e a conformidade de ideias, através da análise 
de estruturas e práticas que precisavam ser lapidadas e/ou renovadas. A partir disso, entre 25 de 
janeiro e 28 de fevereiro do ano de 2021, foi aplicado um questionário online direcionado aos 
acadêmicos e egressos do curso de Engenharia Civil da UFSM, de modo que foi possível elaborar 
um levantamento das principais considerações dos alunos em relação à matriz curricular vigente, 
às experiências vividas nas salas de aula e demais temáticas intrínsecas ao curso. O questionário 
contou com 28 perguntas, objetivas e dissertativas, e, ao final, totalizou 120 respostas. Por 
conseguinte, os dados recolhidos – e as inferências provenientes destes – foram repassados à 
Coordenação do Curso e ao Núcleo Docente Estruturante (NDE), responsáveis pela reforma 
curricular do curso, de forma a incorporá-los na discussão que está em vigor. No que diz respeito 
especificamente às reuniões semanais que o grupo PET Engenharia Civil passou a ter com a 
coordenação, foi possível debater não somente questões curriculares, mas também todos os demais 
relatos coletados ao longo do questionário referentes às adversidades experienciadas no ambiente 
do curso, como, por exemplo, as relacionadas aos docentes. Como consequência do trabalho 
desenvolvido pelo grupo desde o início do ano de 2021 – questionários, levantamentos e reuniões 
–, o NDE deu seguimento à discussão da reforma curricular, com progressos no que se refere à 
construção das disciplinas que constituirão a futura matriz, e passou a trabalhar em conjunto à 
Coordenação do Curso na elaboração de novas Disciplinas Complementares de Graduação 
(DCGs). A coordenação, ainda, está construindo, junto ao grupo PET, um Fórum Estudantil. Para 
mais, o grupo segue ampliando o projeto com um estudo para a otimização da grade de horários 
do curso. Diante do exposto, conclui-se que o projeto tem proporcionado um espaço de vínculo 
dos estudantes com os docentes do curso, com maior conforto, acolhimento e bem-estar no que se 
refere ao diálogo e à interação bilateral. Pode-se dizer que o PET Discute tirou o curso de 
Engenharia Civil da UFSM da zona de conforto, de modo que estimulou o pensamento crítico dos 
estudantes quanto às falhas no curso, bem como impulsionou os docentes a promoverem as 
reestruturações necessárias. Além disso, o grupo PET se mostrou um canal de comunicação 
eficiente para as demandas estudantis. A médio e longo prazo, ainda, a atividade deverá promover 
uma formação de melhor qualidade aos discentes do curso de Engenharia Civil da UFSM, em 
razão da futura reforma curricular, que fará com que os acadêmicos se sintam mais motivados no 
decorrer da graduação e confiantes quando chegar o momento de ingresso no mercado de trabalho

\includepdf{pdfs/Pet-Discute-A-Engenharia-}

\addcontentsline{toc}{section}{Pet Explica: Conhecimento Para Além Da Universidade}

\section*{Pet Explica: Conhecimento Para Além Da Universidade}

Eduarda Kleemann De Ponte, Juliana Jobim Jardim (Universidade Federal do Rio Grande do Sul),Isadora Mello Carvalho (Universidade Federal do Rio Grande do Sul),Guilherme Vidal (Universidade Federal do Rio Grande do Sul),Giulia de Oliveira Bisotto (Universidade Federal do Rio Grande do Sul),Júlia Vanni (Universidade Federal do Rio Grande do Sul),Michelle Justen (Universidade Federal do Rio Grande do Sul)

Atualmente sabemos que as redes sociais ocupam um grande espaço na vida dos usuários. Nesses canais de comunicação a população procura por informação, entretenimento e relacionamento. No entanto, nos últimos anos, o surgimento das Fake News, que são notícias falsas compartilhadas como se fossem verdadeiras, tem se tornado uma preocupação. O compartilhamento em massa dessas informações faz um desserviço à sociedade, pois causam confusão aqueles que não estão aptos a interpretar tal informação e saber se ela é verdadeira ou não.

Aproveitando o espaço de troca com a comunidade da Faculdade e o bom engajamento que existia no perfil do PET Odontologia UFRGS no Instagram, somado a necessidade de informar com veracidade e descomplicar alguns assuntos, a atividade PET Explica: Conhecimento para Além da Universidade, teve seu início no dia 15 de março de 2020.

A atividade é organizada pelo grupo PET Odontologia UFRGS, formado por 12 discentes bolsistas e um docente tutor. Os alunos são sorteados para estabelecer um cronograma e cada bolsista se torna responsável pela confecção de uma postagem. Em uma reunião o grupo decide as datas que irão acontecer cada uma das postagens e se haverá uma temática específica. No período de confecção, cada aluno busca na literatura vigente informações sobre o assunto do seu post. Após a produção da postagem, a professora tutora revisa o conteúdo criado e as referências utilizadas, de modo que sejam reduzidas as chances de haver uma informação incorreta, finalizada essa etapa, a atividade é publicada. O conteúdo abordado nas postagens possui temática diversa, sendo de livre escolha do aluno que irá confeccionar, dentro do tema geral decidido pelo grupo.

Até o momento já fizemos três séries de doze postagens cada e uma série de 6 postagens, totalizando 42 publicações. Além disso, o número de seguidores do perfil do grupo passou de 991 para 2448, tendo quase triplicado desde o início da atividade. Alguns dos temas abordados nas diferentes séries de posts foram: atividades realizadas pelo grupo, saúde bucal, funções do Sistema Único de Saúde (SUS) e mitos e verdades relacionados a odontologia.

Espera-se que essa atividade contribua para a melhora da qualidade dos conteúdos informados nas redes sociais, de modo que seja compartilhado com o maior número de pessoas possível, para que a população em geral seja orientada corretamente com informações atualizadas, bem como qualificar os petianos a utilizar as ferramentas de confecção das postagens e formar senso crítico a respeito dos assuntos selecionados.

\includepdf{pdfs/Pet-Explica--Conhecimento}

\addcontentsline{toc}{section}{Pet News, Integrando Profissionais E Suas Áreas De Atuação Específicas, Compiladas E Distribuídas Para O Entendimento E Desenvolvimento Social}

\section*{Pet News, Integrando Profissionais E Suas Áreas De Atuação Específicas, Compiladas E Distribuídas Para O Entendimento E Desenvolvimento Social}

Gabriel Michalichen, Andres Angel Leonardo Lindao, Mateus Rosante Grisang, Vitoria Sena Braga, Fernanda Gama Cerqueira, Dinéia Tessaro e Maria Madalena Santos da Silva

O trabalho refere-se a um projeto  com eixo temático envolvendo o Ensino, qual foi criado e  teve  inicio juntamente com o período da Pandemia da Covid 19, onde o  Grupo Pet Engenharia Florestal, da Universidade Tecnológica Federal do Paraná encontrou  uma forma de aproximar e integrar  o publico geral  , mesmo no formato remoto, expondo assuntos pertinentes a área florestal e interesses gerais, isso através de postagens em forma de conteúdos educativos e auto explicativos que despertam o interesse do publico fazendo  com que sempre que ocorra o lançamento de novas edições, ocorra a procura do publico geral por nossos conteúdos.

\includepdf{pdfs/Pet-News--Integrando-Prof}

\addcontentsline{toc}{section}{Pet Zootecnia No Ensino Médio}

\section*{Pet Zootecnia No Ensino Médio}

Caroline Baratela Alves, Camilla Tieko Ogasawara,Clara Lopes Siqueira Massi,Fernando Aidar Larini,Giovanna Lima Silva,Letícia Neves de Oliveira

Os cursos de graduação em Zootecnia no Brasil têm uma história relativamente recente, quando
comparada com os outros cursos das agrárias. O primeiro curso surgiu no ano de 1966 na PUC
da cidade de Uruguaiana, e as poucos foram abrindo outros cursos no Brasil. A baixa procura
pelo curso de graduação em Zootecnia pode estar relacionada pela pouca informação e
confundimento com os cursos de Medicina Veterinária e Agronomia. Assim, os estudantes do
ensino médio possuem pouco conhecimento sobre o curso de zootecnia e da profissão
zootecnista. Além disso, muitos dos estudantes encontram adversidades relacionadas à situação
econômica, em função da necessidade de trabalhar (o curso é período integral) ou mudar de
cidade. Também, falta conhecimento sobre os programas sociais que apoiam o acesso e a
permanência dos estudantes na UEL, como bolsa de inclusão social, ensino, pesquisa e extensão
que a Universidade oferece, a Moradia Estudantil e o Restaurante Universitário. Todos esses
fatores contribuem pela baixa procura. Essa atividade foi elaborada com o objetivo de disseminar
informações e conhecimentos sobre o Curso de Zootecnia da Universidade Estadual de Londrina,
com a finalidade de aumentar a procura pelo curso pelos estudantes do ensino médio. Para esse
projeto, foram selecionadas escolas de ensino médio, sendo uma escola pública, a Escola Idalina
Vianna Ferro, localizada no município de Bariri, São Paulo, e três particulares, sendo o Colégio
SESI de Apucarana, Colégio SESI de Londrina e Colégio SESI de Arapongas. Foram
ministradas palestras virtuais para alunos do terceiro ano do ensino médio, utilizando
plataformas como Zoom e Microsoft Teams. Os temas abordados foram sobre o curso de
Zootecnia da Universidade Estadual de Londrina, sobre o vestibular e outras formas de ingresso
na Universidade, bolsas de inclusão social, pesquisa e extensão, a Moradia Estudantil, o
Restaurante Universitário e as oportunidades de intercâmbio e estágios durante a graduação. As
palestras foram realizadas pelos petianos que também relatavam as suas experiências e
expectativas em relação ao curso. A realização do PET no Ensino Médio garantiu o maior
conhecimento e entendimento acerca do curso de Zootecnia da Universidade Estadual de
Londrina. Logo após as primeiras apresentações, o professor responsável em uma das escolas
selecionadas se interessou em continuar a orientar essa atividade do PET e encaminhá-la para
outras escolas, possibilitando que mais alunos fossem contemplados. Ao final das apresentações
sempre havia dúvidas sobre as diversas áreas de atuação do zootecnista, sobre a universidade e
sobre curiosidades referentes ao agronegócio. A atividade intitulada “PET no Ensino Médio” foi
importante para ampliar os conhecimentos em relação ao curso de Zootecnia e as áreas de
atuação do zootecnista. Também, serviu para divulgar o curso de Zootecnia da UEL.

\includepdf{pdfs/Pet-Zootecnia-No-Ensino-M}

\addcontentsline{toc}{section}{Pet Faz Arte}

\section*{Pet Faz Arte}

Gabriel Antonio De Matos Diogo, Aline Savam,Cristina Beatriz Buzelli Noronha,Franciele de Almeida Nascimento,Jennifer Ceciliano Terencio,João Marcos Barbosa Piai,Mayara Tayna Messias dos Santos.

Introdução: As mudanças causadas pela pandemia da Covid-19 em 2020, mudaram a rotina dos
petianos. Um dos impactos dessas mudanças foi a maneira na qual atividades programadas pelo 
grupo foram realizadas. Em função das medidas sanitárias, as atividades presenciais foram 
suspensas e houve a adaptação, da maioria das atividades, para o modo remoto. Neste contexto, 
um dos desafios foi a manutenção da integração dos membros do grupo, bem como o estímulo à 
participação individual. O estresse e a exaustão física e psicológica que tais mudanças causaram 
se tornaram problema para toda a cadeia educacional. Na tentativa de buscar práticas que pudessem 
trazer relaxamento e uma visão diferente sobre a área farmacêutica, foi proposta a atividade 
intitulada: “PET faz arte”. Essa atividade foi inspirada em uma prática já realizada por professores 
da disciplina de Fisiologia da UNESP-Araraquara, que utilizam temas da disciplina de Fisiologia 
para desenvolvimento de trabalhos artísticos. Considerando que no grupo PET-Farmácia temos 
participantes de todos os anos do curso, o projeto dos colegas da UNESP foi adaptado à realidade 
do nosso grupo. Foi proposto que cada petiano escolhesse um tema referente a alguma disciplina 
do curso, bem como um movimento artístico, ou obra como referência para desenvolver um 
trabalho. Com essa atividade, buscou-se trazer aos participantes uma visão mais leve, divertida e 
curiosa do seu próprio universo de formação. Sequencialmente, houve o desafio da elaboração de 
uma peça artística para futura apresentação aos outros membros do grupo. Para que essa peça fosse 
realizada, os alunos realizaram pesquisas sobre temas artísticos. Em um dia marcado, cada obra 
foi inicialmente contextualizada por meio de uma breve introdução das referências artísticas de 
cada trabalho. Com isso, os participantes puderam ter a experiência de conduzir uma discussão 
sobre tema relativo a Artes. Após esta introdução, o petiano apresentou seu trabalho por filmagem, 
ou fotografia, dependendo do tipo de obra. Foram apresentadas 15 obras. Objetivo: Considerando 
que somos um grupo de estudantes de Farmácia, o objetivo deste trabalho, foi avaliar a percepção 
da atividade “PET faz Arte” dentre os petianos do grupo PET Farmácia UEM. Metodologia: Para 
a avaliação das percepções sobre a atividade, uma comissão composta por 1 petiano e a tutora 
elaborou um questionário, que foi posteriormente enviado a todos os petianos, incluindo o aluno 
que elaborou o questionário. O questionário foi composto por 5 questões, sendo 4 perguntas 
objetivas e 1 pergunta aberta. As perguntas tiveram o objetivo de capturar a primeira impressão 
sobre a atividade, incluindo possíveis medos e desconfortos em função da necessidade de 
realização de atividade artística. Além disso, buscou capturar também, a sensação do participante 
da atividade, ao longo da feitura de suas obras, e o seu contentamento quanto ao resultado do 
trabalho artístico desenvolvido. Esse questionário foi aplicado após 8 meses da finalização da 
atividade. Resultados e Discussão: Foram coletadas 15 respostas. A síntese das respostas mostrou
que, a primeira impressão/percepção sobre a atividade foi de apreensão para a maioria (66,7%)
dos petianos em função de entenderem ser de difícil realização. Essa percepção de receio 
possivelmente tem origem na insegurança sobre as habilidades artísticas de cada um. Porém, 
apesar de inseguros, também descreveram entusiasmo com essa tarefa. Sobre o aproveitamento no 
aprendizado ao longo do processo de pesquisa, 40% dos petianos indicaram ter aprendido muito 
durante a fase de pesquisa, outros 40% disseram que aprenderam razoavelmente, 13,3% que 
aprenderam pouco e 6,7% que aprenderam muito pouco. Esses dados mostram que mais de 80% 
tiveram a sensação satisfatória de aproveitamento ao longo do processo de pesquisa, expandindo 
seus universos culturais. A informação sobre a percepção do processo criativo individual mostrou 
que 66,7% gostaram muito dos seus processos de desenvolvimento de suas obras, enquanto 33,3% 
somente gostaram. Não houve resposta negativa quanto a esse item. Finalmente, a avaliação sobre 
os objetivos de relaxamento e diversão da atividade terem sido cumpridos, 86,7% dos petianos 
responderam que o objetivo foi atingido e somente 6,7% (1 petiano) respondeu que não sentiu que 
a atividade tenha proporcionado relaxamento e diversão. Questionados se gostaram dos resultados
finais produzidos, 60% dos petianos responderam que gostaram muito do resultado e 40% 
disseram que gostaram. Embora a atividade não possa ser enquadrada como arteterapia 
propriamente dita, sabe-se que a utilização da realização de trabalhos manuais, ou artísticos pode 
auxiliar na redução de fatores negativos relacionados ao emocional social, como é o caso do 
processo de distanciamento social. Estudar e desenvolver a arte, contribui muito na busca do
autoconhecimento e bem-estar consigo mesmo (JARDIM et al., 2020). O prazer da feitura de suas 
obras trouxe satisfação quanto ao resultado final dos trabalhos para a grande maioria dos 
envolvidos. A atividade PET faz arte continuará sendo aplicada dentro do grupo, como forma de 
trabalho interno na busca de desenvolvimento dos participantes. Para as próximas edições, o 
mesmo questionário de avaliação será reaplicado, imediatamente após a sua finalização, visando 
capturar melhor as memórias emocionais dos envolvidos. Considerando que essa atividade foi 
inspirada em uma iniciativa de sucesso de professores de uma disciplina específica da área 
biológica (Fisiologia), como relatado acima, vimos que a conexão da arte com diferentes áreas de 
estudos poderia ser feita com resultados bastante interessantes. O processo da educação formadora 
e integrativa unindo de modo mais intuitivo as áreas das Ciências Humanas com áreas mais 
tecnicistas, como é o caso das Ciências da Saúde, poderia ser melhor explorada dentro do currículo 
de formação geral dos alunos. A arte e suas referências teóricas, trazem em si, questionamento 
individual e coletivo necessários para formação de pessoas em cidadãos mais capazes de 
compreender movimentos transformadores. Essa atividade pode ser um elemento embrionário para 
uma futura proposta de curricularização de conteúdos que alavanquem os conhecimentos das áreas 
de humanas, e, no âmbito da psicologia, de situações que estimulem o auto desafio como prática 
de amadurecimento pessoal e profissional. Conclusão: A atividade “PET faz Arte”, teve um 
impacto positivo para os petianos. Ela possibilitou trazer novos conhecimentos, tanto culturais 
como artísticos, e também auxiliou aos participantes a lidarem melhor com tensões causadas pela 
pandemia durante o momento de sua realização. A atividade também trouxe aos petianos uma nova 
percepção sobre suas próprias habilidades. 
Referências 
JARDIM, V. C. F. DA S. et al. Contribuições da arteterapia para promoção da saúde e qualidade 
de vida da pessoa idosa. Revista Brasileira de Geriatria e Gerontologia, v. 23, n. 4, 2020.

\includepdf{pdfs/Pet-Faz-Arte}

\addcontentsline{toc}{section}{Pet-Eventos E A Pandemia}

\section*{Pet-Eventos E A Pandemia}

Moara Zambonim Rossi Dos Santos, Isabella Flud (PET LETRAS UFSC),Mayumi Esmeraldino (PET LETRAS UFSC),Isabella Flud (PET LETRAS UFSC),Mayumi Esmeraldino (PET LETRAS UFSC)

Para esta proposta, analisou-se o funcionamento de um dos projetos estruturantes do PET-Letras UFSC, o PET-Eventos: planejamento e organização, no que se refere à sua importância e às suas possíveis contribuições à vida acadêmica e pessoal de seus participantes. Abordaram-se os aspectos de gestão do projeto com destaque para as estratégias de proposição e de realização de eventos, no âmbito do PET-Letras, e de integração aos eventos maiores, tais como o SulPET e o InterPET. A partir disso, os eventos on-line realizados pelo projeto, durante a pandemia da COVID-19 — entre março de 2020 e agosto de 2021—, foram listados e avaliados em relação: (i) à sua proposição e organização; (ii) à sua efetivação; e (iii) às suas possíveis contribuições. Por fim, observou-se a relevância de um projeto com o objetivo de viabilizar o envolvimento da equipe do PET-Letras com eventos de diversas naturezas, tanto como participantes quanto como idealizadores e organizadores.

\includepdf{pdfs/Pet-Eventos-E-A-Pandemia}

\addcontentsline{toc}{section}{Projeto Covid-19}

\section*{Projeto Covid-19}

Hektor Gabriel Amaral Vargas, Fernanda Zanini dos Santos Bentancur,Luciana Rodrigues de Medeiros,Lucca Bragança Castagnino Viana,Roberta Delgado Bauer,Ana Júlia Vicari,Paula Carlotto Pacheco

O projeto começou em 2020, com o início da pandemia de COVID-19, surgiu a incerteza da
população e da comunidade médica frente a essa nova realidade. A divulgação de notícias falsas
veiculadas por todas plataformas marcaram o início da pandemia no Brasil, gerando surto
coletivo, consumo indevido de medicamentos e lotação de estabelecimentos por pessoas em
busca de mantimentos, que atingiram preços exorbitantes. Em 2021, com o início da campanha
de vacinação no Brasil, novamente a desinformação era compartilhada, gerando desconfiança
com relação à segurança e efetividade da vacina. Percebemos a necessidade de combater essa
desinformação através da criação e divulgação de um conteúdo elucidativo, simples e didático,
baseado em dados científicos. Para isso elaboramos uma série de postagens em nossas redes
sociais, com informações sobre a transmissão do vírus e sua ação no corpo humano, formas de
prevenção e a importância do isolamento social, práticas para preservar a saúde mental e o
porquê de não acumular itens essenciais. Outros tópicos abordados foram: os tipos da vacina e
sua eficácia, seus efeitos no corpo e cuidados posteriores, o progresso da vacinação mundial,
além da origem e das consequências ambientais da pandemia. Foram feitas 18 publicações nas
redes sociais do PETBio (Instagram, Twitter e Facebook) utilizando textos e sequências de cards
informativos com gráficos e simulações, falas de especialistas e artigos científicos citados como
referência. O projeto foi encerrado no mês de julho com grande engajamento do público com
esses materiais. Já tendo cumprido suas metas e programações para o presente momento, não
tem previsão de retorno ou atividades adicionais.

\includepdf{pdfs/Projeto-Covid-19}

\addcontentsline{toc}{section}{Projeto Feq/Ieq: Aplicação De Ferramentas Computacionais Nas Disciplinas De Engenharia Química.}

\section*{Projeto Feq/Ieq: Aplicação De Ferramentas Computacionais Nas Disciplinas De Engenharia Química.}

Bruno Bertolo Caetano, Felipe Rodrigues Batista

INTRODUÇÃO
Seja nos estudos, na indústria ou em qualquer outro campo de atuação do Engenheiro 
Químico, é imprescindível a utilização de softwares que auxiliam na resolução de problemas que 
envolvem uma extensa rotina de cálculos. O conhecimento da utilização destes sistemas 
operacionais é visto com grande interesse por parte das empresas, considerando a possibilidade de 
otimização e redução de custos e, consequentemente, maior geração de lucros. Desse modo, 
dominar esses softwares torna-se um diferencial para o profissional no mercado de trabalho.
Visando complementar a graduação de seus alunos na Universidade Estadual de Maringá, 
o Programa de Educação Tutorial do curso de Engenharia Química promove a elaboração de aulas 
voltadas às disciplinas de Introdução à Engenharia Química (IEQ) e Fundamentos da Engenharia 
Química (FEQ) por meio da aplicação de softwares didáticos, sendo o Microsoft Excel e o DWSIM.
Tais disciplinas abordam os conteúdos de balanço de massa e energia, que são 
fundamentais para todas as outras disciplinas que os alunos irão cursar ao longo da graduação. Nas 
disciplinas mais avançadas e na futura vida profissional do engenheiro, o número de variáveis 
envolvidas nos problemas estudados é significantemente maior. Nestes casos, a resolução 
utilizando apenas calculadoras fica praticamente inviável em função do tempo que demandaria e 
se faz necessário o uso de ferramentas mais completas, como softwares que agilizem os cálculos.
O objetivo deste projeto é proporcionar o primeiro contato dos alunos da primeira e 
segunda série do curso de Engenharia Química com ferramentas computacionais. Deseja-se que 
ao final do período letivo os graduandos dominem as estratégias básicas de resolução de exercícios 
de balanços de massa e energia utilizando o Microsoft Excel e que sejam capazes de simular 
processos químicos com o DWSIM.
METODOLOGIA
 O Projeto FEQ/IEQ compreende na organização e execução das aulas, para isso é formada 
uma comissão para organizar tal atividade. Inicialmente, a comissão é responsável pela revisão e 
atualização dos materiais que serão utilizados, como as apostilas com os conteúdos sobre os 
softwares que foram elaboradas pelos PETianos e as listas de exercícios. Tais exercícios são 
escolhidos seguindo as instruções das professoras de FEQ/IEQ e, para que não haja problema, são
testados e resolvidos à mão primeiramente e, logo após, digitalizados para o Excel e para o 
DWSIM. Dessa forma, as resoluções são apresentadas aos professores responsáveis pelas 
disciplinas para verificar se estão corretas ou se necessitam de alguma alteração.
Além disso, é do cargo da comissão executar um processo seletivo voltado aos graduandos 
de Engenharia Química que já cursaram o primeiro ano, para que sejam selecionados monitores 
para as aulas de Excel e DWSIM na disciplina de FEQ e IEQ. Esse processo seletivo é avaliado 
através da apresentação da resolução de um exercício pelo candidato para os membros da comissão 
para verificar-se o conhecimento dos softwares e das disciplinas envolvidas no projeto. Então, é 
realizada uma reunião para avaliação e decisão dos selecionados para monitorar as aulas.
As datas para a realização das aulas são antecipadamente definidas com os professores das 
respectivas matérias, sendo duas aulas de Excel aos alunos de IEQ e duas aulas de DWSIM aos de 
FEQ. No Ensino Remoto Emergencial, as aulas foram feitas de maneira assíncrona, sendo postadas 
no Google Classroom das respectivas turmas. 
Ao final das aulas é realizado um feedback dos graduandos sobre sua qualidade, que é 
analisada pela comissão e repassada a todos do grupo em uma reunião geral para debate e 
compartilhamento de ideias.
RESULTADOS E DISCUSSÃO
Para as turmas de IEQ, na primeira aula são passados comandos e funções básicas do Excel. 
Alguns exemplos também são ministrados e depois são deixados exercícios para os graduandos 
praticarem o que aprenderam. Todo esse processo facilita a fixação do conteúdo. Já na segunda 
aula, exercícios envolvendo balanço de massa são resolvidos por meio do software. Esses 
exercícios são parecidos ou até mesmo iguais aos exercícios resolvidos por eles em sala de aula de 
maneira convencional. Deste modo, o graduando é estimulado a utilizar o Excel, pois a resolução 
se dá de forma muito mais simples e direta. Além disso, como essa ferramenta tem grande 
importância no mercado de trabalho da Engenharia Química, o projeto ajuda o aluno a se ambientar 
à ela.
Já nas turmas de FEQ, as duas aulas abordam o simulador DWSIM. A dinâmica das aulas 
contempla trazer funções novas do software à medida que elas se fazem necessárias para resolver 
os exercícios. Sendo assim, os primeiros conteúdos são mais simples, como determinar 
propriedades de substâncias, por exemplo. Em sequência, exercícios com balanços materiais e 
balanços energéticos são ministrados. Semelhantemente à disciplina de IEQ, o objetivo é 
apresentar o simulador aos graduandos. Ele pode ser utilizado futuramente para facilitar a 
compreensão e a lógica dos exercícios de diversas disciplinas e também para a simulação dos mais 
variados processos, podendo ser utilizado até em iniciações científicas e na carreira profissional.
CONCLUSÃO
A aproximação dos alunos do curso de Engenharia Química a ferramentas computacionais 
é de grande importância. Quando essas ferramentas são trabalhadas em sala de aula, seu 
conhecimento por parte dos graduandos pode ajudá-los futuramente quando enfrentarem o 
mercado de trabalho. Esse projeto, então, afeta diretamente os discentes de Engenharia Química 
da Universidade Estadual de Maringá de forma positiva e satisfatória.

\includepdf{pdfs/Projeto-Feq-Ieq--Aplicaca}

\addcontentsline{toc}{section}{Para Além Da Leitura: Cidadania Em Ação}

\section*{Para Além Da Leitura: Cidadania Em Ação}

Eduardo Prates Macedo, Ana Julia Rodrigues (UFSM),Gabriela Viera dos Santos (UFSM),Igor Vianna Bianchin (UFSM),Julia Lopes Marafiga (UFSM),Renata Santos Abitante (UFSM)

O projeto de extensão Para Além da Leitura: Cidadania em Ação está vinculado ao Programa de 
Educação Tutorial Ciências Sociais Aplicadas (PET CiSA), da Universidade Federal de Santa 
Maria (RS), composto pelos cursos de Comunicação Social - Produção Editorial, História 
Bacharelado e Licenciatura e Meteorologia. Após analisar as baixas perspectivas de retorno das 
atividades presenciais no ano de 2021, em função da pandemia de COVID-19, o projeto de 
extensão foi adaptado, com o objetivo de promover atividades voltadas a discentes de escolas 
públicas, do ensino médio. Devido à mudança abrupta do método de ensino, a proposta de atender 
as demandas dos professores pertencentes à escola pública se fez muito importante para aproximar 
universidade e comunidade, bem como auxiliar esses profissionais, de modo a repassar experiência 
e conhecimento aos petianos e descobrir propostas de ensino atrativas aos alunos.
A metodologia norteadora trata-se da Pesquisa-ação, a qual fundamenta-se em Thiollent (2009, p. 
16), como: “um tipo de pesquisa social com base empírica que é concebida e realizada em estreita 
associação com uma ação ou com a resolução de um problema coletivo e no qual os pesquisadores 
e os participantes representativos da situação ou do problema estão envolvidos de modo 
cooperativo, ou participativo”. A primeira fase da pesquisa contou com a elaboração e envio de 
um formulário disponibilizado aos docentes, no qual mapeamos o perfil destes para reconhecer 
como o projeto poderia auxiliar na produção de conteúdo e ações de ensino. A partir desses dados, 
nos reunimos com os professores no dia 21 de julho para apresentar o projeto e conversar sobre os 
rumos que os encontros tomariam. Os docentes apresentaram interesse na proposta e 
complementaram com sugestões que eles acreditaram pertinentes aos seus alunos. Com essas 
informações, avaliamos quais demandas poderíamos suprir e, assim, decidimos atender ao que nos 
foi solicitado pelos professores. Na segunda fase, ocorreram os encontros via Google Meet, no 
primeiro semestre de 2021, no dia 07 de Julho em parceria com a Escola Técnica Estadual Rubens 
da Rosa Guedes (ETERRG) de Caçapava do Sul, e no dia 14 de Julho com a escola E. E. E. M. 
Plácido de Castro de Rosário do Sul.
No primeiro encontro, participaram os docentes e discentes em conjunto com os petianos, que 
propuseram uma troca de saberes com os educandos acerca das oportunidades oferecidas pela 
UFSM quanto ao ingresso, programas de permanência e assistência estudantil. O encontro contou 
ainda com um espaço para diálogo visando sanar dúvidas. À luz de Paulo Freire (1983, p. 66) “o 
sujeito pensante não pode pensar sozinho; não pode pensar sem a co-participação de outros sujeitos 
no ato de pensar sobre o objeto. Não há um ‘penso’, mas um ‘pensamos’. É o ‘pensamos’ que 
estabelece o ‘penso’ e não o contrário”. Em vista disso, o contato entre graduandos e alunos 
promove o “pensamos” através do conjunto de indivíduos em diferentes estágios sociais que se 
complementam pela troca de conhecimento e experiências. Este primeiro encontro proporcionou 
a aproximação entre estudantes da rede pública de ensino com a realidade da universidade, 
incentivando-os a continuarem buscando sua entrada no ensino superior, agora tendo a sua 
disposição maiores conhecimentos.
Em um segundo encontro com os discentes da ETERRG no dia 18 de agosto, foram apresentados 
alguns cursos de interesse apontados pelos próprios discentes após um levantamento prévio, sendo 
eles: Zootecnia, Medicina Veterinária, Arquitetura, Produção Editorial e História Bacharelado e 
Licenciatura, sendo tratadas questões de carga horária, grade curricular, áreas de atuação, estágios 
e esclarecimento de dúvidas. O encontro contribuiu na desmistificação acerca dos cursos de 
graduação, tendo em mente que muitos alunos podem/irão sair do ensino médio direto para o 
ensino superior, a atividade cumpriu seu objetivo de apresentar os cursos e suas características 
principais. Os discentes demonstraram-se motivados no decorrer do encontro. Pode-se afirmar que 
a reunião se pauta na “interação dialógica entre universidade e sociedade, caracterizada pelo 
intercâmbio de experiências e saberes entre Universidade e demais setores da sociedade” 
(POLÍTICA DE EXTENSÃO DA UFSM, 2019, p. 1).
Os resultados prévios apontam a importância deste projeto para a promoção da visibilidade dos 
cursos de graduação da UFSM, bem como para a orientação aos jovens estudantes das escolas 
públicas, quanto ao ingresso em uma universidade. Desse modo, as ações proporcionaram aos 
alunos maior aproximação com o ambiente universitário, visto que para alguns deles este foi seu 
primeiro contato e conhecimento sobre a UFSM, bem como proveu inspiração e motivação, 
possibilitando que estes discentes amadureçam suas ideias e planos, permitindo assim que 
consigam seguir seus sonhos e futuras carreiras. O projeto terá continuidade no planejamento de 
2022, sendo que serão ampliados os espaços de aplicação estendendo-se à outras instituições de 
Ensino Público. Deste modo, acreditamos que o PET CiSA estará contribuindo para a aproximação 
entre universidade e sociedade, na medida que promove um projeto de extensão voltado aos 
interesses desta e na formação de cidadãos conscientes dos seus direitos e dos espaços que podem 
ser conquistados no Ensino Superior.

\includepdf{pdfs/Para-Alem-Da-Leitura--Cid}

\addcontentsline{toc}{section}{Pavimentação Utilizando Concreto Permeável}

\section*{Pavimentação Utilizando Concreto Permeável}

Matheus Silva Campos, GUILHERME CONDE DIAS - UNIVERSIDADE FEDERAL DO PARANÁ

Um dos principais impactos do acelerado desenvolvimento urbano é o crescimento de superfícies 
impermeabilizadas. Isto eleva consideravelmente o escoamento superficial nas grandes cidades, 
sobrecarregando os sistemas de drenagem e facilitando picos de cheias. Cresce assim a 
possibilidade de enchentes e inundações. Em tal cenário sedimentos e impurezas da superfície 
chegam aos corpos hídricos provocando assoreamento e poluição. Tendo em mente a mitigação 
desses processos o PET Engenharia Civil desenvolveu o projeto Pavimentação de Concreto 
Permeável. Os problemas ligados à impermeabilização podem ser observados no próprio campus 
Politécnico da UFPR, onde determinadas regiões tornam-se inacessíveis por períodos prolongados 
devido ao nível de água que tem dificuldade de infiltrar no solo após eventos pluviométricos mais 
extremos como em setembro de 2018 e novembro de 2020. Uma alternativa é o uso de materiais 
como o concreto permeável que reduz o runnoff. Superfícies utilizando concreto permeável 
absorvem parte ou a totalidade do escoamento para dentro de um reservatório de brita, construído 
sobre subleito do pavimento. O volume captado pode então ser conduzido a um reservatório, e 
posteriormente levada ao sistema comum de drenagem, ou simplesmente infiltrada no subsolo. 
Pavimentos permeáveis são uma solução muito importante no cenário do atual biênio de estiagem 
que a região metropolitana de Curitiba experimenta. O projeto consiste, portanto, de um estudo de 
caso no próprio campus, utilizando-se uma área de estacionamento existente. O objetivo é produzir 
um projeto executivo para a área e apresentá-lo como possível solução e sugestão de 
implementação para a divisão responsável pela infraestrutura da UFPR. O grupo conta com a 
supervisão de professor especialista na área e a primeira parte do projeto consiste de estudo 
bibliográfico relacionado ao tema e também das normas técnicas aplicáveis. Em sequência o 
projeto segue duas frentes, sendo a primeira responsável pelo estudo hidrológico do local para 
determinação da vazão de projeto. Enquanto a segundo dedica-se ao dimensionamento da 
infraestrutura necessária para acomodar tal vazão. O objetivo final é, então, produzir um relatório 
de projeto que inclua desde a fenomenologia e o diagnóstico das inundações do local, bem como 
o dimensionamento, o detalhamento e a estimativa de custos das soluções propostas. Desse modo 
o projeto, cumpre um dos papéis mais relevantes do PET proporcionar uma formação de qualidade 
aos estudantes atuando também em uma necessidade importante da comunidade acadêmica e 
gerando conhecimento que pode ser levado aos demais integrantes da sociedade.


\includepdf{pdfs/Pavimentacao-Utilizando-C}

\addcontentsline{toc}{section}{Percepção Dos Alunos Do Curso De Agronomia Sobre O Estímulo Da Família E Da Universidade No Processo De Sucessão Familiar}

\section*{Percepção Dos Alunos Do Curso De Agronomia Sobre O Estímulo Da Família E Da Universidade No Processo De Sucessão Familiar}

Lucas De Mattos, Denise Maria Vicente,Katiane Abling Sartori,Larrisa Lamperti Tonello

A sucessão familiar consiste na transposição dos meios sociais, do trabalho e da produção 
para os próximos sucessores, cuja atividade juntamente com a interação dinâmica e profissional 
implica diretamente nos empreendimentos (SCHINEIDER, 2016). Atualmente, tem se mostrado 
constantes os debates referentes ao envelhecimento da população rural, dos conflitos existentes 
entre as gerações que empreendem na agricultura, como também, das dificuldades que se 
apresentam no momento dos jovens decidirem permanecer ou não na propriedade. Isto, em virtude 
dos poucos incentivos dos pais, governo e sociedade e pela pouca disponibilidade de área 
agricultável. Se não houver comprometimento, administração e percepção de negócio, a 
possibilidade de ganhos financeiros pode ficar comprometida (FLORES, 2006). A saída dos jovens 
do campo, principalmente nas pequenas propriedades, pode ser vista como uma perda, tanto para 
as famílias quanto para a sociedade, já que se bem administradas, essas atividades podem gerar 
considerável ganho econômico. 
O objetivo deste trabalho foi analisar a percepção e opinião dos alunos do curso de 
Agronomia da UFSM Campus Frederico Westphalen sobre o assunto propriedade rural e a 
sucessão familiar, além dos possíveis fatores que levam esses acadêmicos à decisão de permanecer 
ou sair da propriedade. Para obtenção das variáveis, nos meses de janeiro a março de 2021, aplicou-se aos acadêmicos do curso um questionário com perguntas abertas e fechadas, através da 
ferramenta Google Formulários, o qual os participantes tiveram acesso pelas redes sociais. A 
pesquisa constituiu-se de 21 perguntas direcionadas aos entrevistados. Entretanto serão discutidos 
os resultados de apenas 3 destas. As perguntas e respostas realizadas foram: “Qual o incentivo da 
sua família para o seu retorno à propriedade?\"Sempre me incentivaram, me ensinando a 
desenvolver as atividades; Recebi pouco incentivo; Não recebi incentivo, pois minha família 
acredita que atividades fora do campo me trariam mais oportunidades; Reside em área urbana. “Se 
mesmo com pouco incentivo familiar, você gostaria de retornar à propriedade, possui ideias de 
atividades que gostaria de implantar?” Sim; Não.“Acredita que a graduação oferece estímulo para 
o retorno à propriedade? Qual a sua percepção sobre o processo de sucessão?”. Desta relação, 
foram coletadas 56 respostas dos alunos dos diferentes semestres para análise e verificação dos 
resultados.
Quando questionado sobre o incentivo da família para o retorno a propriedade, 37,5% 
responderam que sempre receberam incentivo, 26,8% responderam que receberam pouco 
incentivo, 10,7% responderam que não receberam nada de incentivo e a família acredita que as 
atividades fora do campo trazem mais oportunidades, o restante, cerca de 25% são residentes de 
área urbana. De acordo com esses resultados, percebe-se que ainda há pouco incentivo da família 
para o processo de sucessão, muitas ainda preferem que seus filhos busquem outras oportunidades, 
lembrando que, para as pessoas que responderam que recebem pouco incentivo, foi questionado 
sobre a vontade de realizar o processo de sucessão mesmo assim e cerca de 77% responderam que 
sim, possuem essa vontade e tem ideias para implantar na propriedade, o restante, cerca de 23% 
responderam que não tem vontade de realizar a sucessão. Isso evidencia o desejo dos estudantes 
em retornar à propriedade, mesmo que isso não seja tão trabalhado no ambiente familiar. 
Quando questionado sobre o estímulo que a graduação oferece para o retorno à propriedade 
e a percepção sobre o processo de sucessão, foram obtidas 55 respostas, onde 67,3% delas 
responderam positivamente quanto ao estímulo da graduação, 12,7% responderam que a 
graduação oferece pouco estímulo devido á poucas disciplinas voltadas para pequenas 
propriedades, 12,7% relataram que a graduação não oferece estímulo e 7,3% acreditam que o 
estímulo é dependente do interesse do indivíduo e do tamanho da propriedade. 
Sobre a percepção do processo de sucessão, 38% das respostas obtidas relatam a 
importância da aceitação e incentivo dos familiares, bem como do interesse da pessoa em realizar 
a sucessão, como é o caso dessa resposta obtida “A sucessão é um ponto muito importante e 
necessário a ser debatido, muitos pais querem a permanência dos filhos, mas não dão aberturas 
para novas ideias e formas de administrar a propriedade! Precisa haver um consenso entre as 
partes, para que estimule os sucessores a ficarem na propriedade. Maior estímulo vem dos pais e 
não dá faculdade!”, 32,7% descrevem a sucessão como um processo natural de quem possui 
propriedade rural, como é o exemplo dessa resposta obtida “A sucessão é um processo natural, 
hora ou outra será necessário, o diferencial será a preparação e responsabilidade de quem irá 
assumir a propriedade\", 12,7% atribuem o sucesso da sucessão ao tamanho da propriedade, onde 
propriedades maiores têm mais chance de dar certo, o restante, cerca de 18% não responderam a 
esse questionamento.

\includepdf{pdfs/Percepcao-Dos-Alunos-Do-C}

\addcontentsline{toc}{section}{Podcast Como Meio De Transformação Social}

\section*{Podcast Como Meio De Transformação Social}

Beatriz Leal Thompson Claro, Emily Guimarães,Marcone de Freitas Marques,Pablo Viana

Durante o ano de 2020 e com um contexto de atividades que só poderiam ser executadas
remotamente devido a pandemia, o projeto PetC3 Cast1
foi criado. O projeto tem como principal
objetivo promover discussões acerca de temas que contribuam para o desenvolvimento não
somente do público tecnológico, mas para os demais públicos. O podcast, já conta com uma
temporada concluída, onde abordamos assuntos mais voltados para o cotidiano, compartilhando
experiências em áreas distintas, com o objetivo de gerar uma unificação de grupos diversos
dentro da nossa sociedade.
A produção dos episódios é feita em quatro etapas, sendo a primeira de planejamento,
que consiste no levantamento de informações sobre o tema; a segunda etapa é a de roteirização,
onde o episódio é escrito e revisado, para posteriormente entrar em contato um possível
convidado que tenha propriedade no assunto, com o viés de enriquecer as discussões; terceira
parte trata exclusivamente da gravação do episódio; e por fim, na quarta etapa os episódios são
editados e publicados.
Durante a primeira temporada buscou-se levar ao público assuntos mais voltados à
experiências diversas, conforme citado anteriormente, e com isso atingir um público
diversificado. Fato que ficou evidente a partir de uma pesquisa realizada pelos membros do
podcast, após o primeiro episódio, para os ouvintes, no intuito de conhecermos melhor sobre o
público que estávamos atingindo.
Tendo em vista os resultados da primeira temporada, onde conseguimos atingir um
público considerável e através dos levantamentos, conseguimos instigar os ouvintes a buscar
fatores acerca de suas dificuldades e curiosidades, compreendemos que obtivemos sucesso
dentro de um primeiro momento do nosso projeto, pois além de atingirmos de certa forma o
primeiro objetivo traçado, percebemos o quanto poderíamos ampliar a nossa meta buscando uma
nova forma de agregar conhecimento para a sociedade.

\includepdf{pdfs/Podcast-Como-Meio-De-Tran}

\addcontentsline{toc}{section}{Preparando Para O Mundo Profissional (Pmp)}

\section*{Preparando Para O Mundo Profissional (Pmp)}

Milena Zago, Jonas Cardoso de Oliveira,Clara Cristina Ansolin,Thays Cassiano,Alexandre da Trindade Alfaro

O Engenheiro de Alimentos é responsável por todo o processamento e conservação de alimentos 
de origem animal e vegetal, abrangendo funções, como: controle de qualidade, pesquisa e 
desenvolvimento, planejamento de projetos, controle de processos, entre outras. Deve se 
considerar ainda, que cada alimento e/ou bebida possui particularidades especificas no seu 
processamento, e dessa forma, o Engenheiro de Alimentos possui uma gama considerável de áreas 
que pode atuar. Essa diversidade de possíveis áreas de atuação, gera nos discentes muitas dúvidas 
sobre qual seria o melhor “caminho” a seguir. Além disso, durante um curso de graduação e, 
principalmente, na proximidade da sua conclusão, os estudantes têm muitas dúvidas sobre o 
mercado de trabalho, entrevistas, remuneração, dentre outros. Dentro desse contexto, a atividade 
“Preparando para o Mundo Profissional (PMP)”, teve como objetivo formar uma mesa redonda 
com diferentes profissionais que atuam em empresas processadoras de alimentos, para 
compartilhar experiências e sanar dúvidas dos estudantes. Para a realização da atividade o grupo 
PET-Alimentos, discutiu quais temas poderiam agregar mais conhecimento aos discentes do curso 
de Engenharia de Alimentos e entrou em contato com profissionais das áreas levantadas. 
Participaram da atividade os seguintes profissionais: 1 coordenador de recursos humanos, 1 
supervisor de industrializados, 1 analista de pesquisa e desenvolvimento, 1 gerente regional do 
Conselho Regional de Engenharia e Arquitetura. A atividade foi mediada pelo coordenador da 
Engenharia de Alimentos da UTFPR-FB. Cada profissional teve 20 minutos para uma explanação 
inicial, e após, foi aberto para as perguntas do público. A atividade foi realizada no auditório da 
Universidade Tecnológica Federal do Paraná, campus Francisco Beltrão. Na mesa redonda, foram 
abordados assuntos referentes ao mercado de trabalho, oportunidades de atuação, remuneração, 
entrevistas de emprego, idiomas, dentre outros. O público presente participou com perguntas e 
questionamentos, e dentre as dúvidas observou-se um número maior de perguntas a assuntos que 
relacionam o campo de trabalho e a remuneração. Também foram recorrentes as perguntas sobre 
as exigências das empresas quanto à qualificação do profissional, e o que as empresas esperam dos
profissionais a serem contratados. O evento contou com a participação de aproximadamente 120 
discentes de todos os semestres do curso de Engenharia de Alimentos. Na mesa redonda, foram 
relatadas importantes informações sobre o perfil profissional desejado pelas empresas. Desse 
modo, a atividade “Preparando para o Mundo Profissional (PMP)” auxiliou a sanar dúvidas e 
preparar os acadêmicos da Engenharia de Alimentos para o mercado de trabalho.

\includepdf{pdfs/Preparando-Para-O-Mundo-P}

\addcontentsline{toc}{section}{Projeto Arboreto}

\section*{Projeto Arboreto}

Ana Beatriz Barbosa, Ana Carolina Coelho Schimaleski,Isabelle Mesadri Gewehr,Lucas Vieira da Freiria,Alessandro Camargo Angelo

O projeto “Arboreto” é uma iniciativa de caráter prático que envolve o planejamento,
implantação e manutenção de áreas com espécies florestais de interesse econômico e ambiental.
Desta forma, o projeto almeja atender ao trinômio “ensino, pesquisa e extensão”, preconizados
pelo programa PET. O projeto tem como objetivo gerar e disseminar informações sobre
silvicultura de espécies de interesse comercial e/ou ambiental para o estado do Paraná. A partir
de reuniões com a comunidade interessada foram definidas as principais demandas ligadas ao
tema para balizar as escolhas técnicas, o elenco de espécies testadas e os procedimentos em
campo. Os plantios foram realizados na Fazenda Experimental Canguiri da Universidade Federal
do Paraná (UFPR) no município de Pinhais - PR, região metropolitana de Curitiba, em área com
relevo suave ondulado e clima Cfb, segundo a classificação de Koppën. O preparo do solo se deu
com subsolagem (40 cm), seguida de gradagem devido a compactação oriunda do tráfego de
tratores. As áreas foram implantadas em dezembro de 2012, sob o espaçamento de 3x2m
realizado com motocoveador. Os experimentos foram montados conforme um delineamento
inteiramente casualizado, as parcelas do plantio foram constituídas por 80 plantas de cada
espécie e os experimentos foram replicados em propriedades parceiras do projeto na região
Centro-Sul do estado. A partir desse momento, seguiu-se um cronograma de manutenção da
área, envolvendo a limpeza e intervenções como podas e desbastes. Dentre as espécies estudadas
estão a Araucaria angustifolia (Bertol.) Kuntze, Ilex paraguariensis (St. Hil.), Eucalyptus
benthamii (Maiden \& Cambage), E. dunnii (Maiden), E. saligna (Smith), um híbrido de E.
urophylla (S.T. Blake) x E. globulus (Labill), Pinus maximinoi (H.E. Moore), P. taeda (L.),
Cryptomeria japonica (L.F.) e Liquidambar styraciflua (L.) e Acca sellowiana (O. Berg Burret),
ao todo o projeto possui atualmente mais de 60 espécies arbóreas e arbustivas diferentes
espalhadas nas áreas de experimento. O projeto demanda um conjunto de atividades rotineiras,
como a realização de coroamento, roçada, fertilização, poda baixa e poda alta, demarcação de
tratamentos e coleta de dados. Estas atividades são realizadas pelos membros do PET em
conjunto com atividades didáticas de disciplinas de graduação e pós-graduação. Entretanto, em
virtude da pandemia causada pelo COVID-19, essas atividades foram restringidas para preservar
a saúde dos integrantes, principalmente a dos produtores rurais no interior do estado, que
forneciam estadia aos participantes em períodos de visita, manutenção e coleta de dados nos
experimentos. O acompanhamento e as avaliações periódicas dos experimentos buscam
solucionar as dúvidas da comunidade de interesse sobre o uso das espécies em questão, bem
como sobre o uso de insumos testados, contribuindo com a silvicultura destas espécies. As
atividades com um público maior e mais acessível prosseguiram virtualmente, através de
reuniões e palestras conduzidas pelos petianos a respeito do que é desenvolvido no projeto. Desta
forma é contemplado o componente “ensino” do trinômio mencionado acima. No elemento
“pesquisa” os dados destas áreas têm sido usados para a elaboração de documentos científicos,
desde resumos de iniciação científica até artigos em periódicos indexados, bem como livros
relacionados ao tema. As variáveis frequentemente analisadas nestas pesquisas são a
sobrevivência, diâmetro de colo (DAP), circunferência a altura do peito (CAP), altura total,
projeção de copa, produção de biomassa e teor de óleos essenciais. O local já foi utilizado para o
processo formativo de iniciações científicas de ensino médio e graduação até o doutoramento de
profissionais. O elemento “extensão” é contemplado através do fluxo de produtores rurais,
empresas e instituições de ensino interessados no tema. Com o apoio do IDR- PR e da Embrapa,
parceiros do projeto, são realizados os “Dias de Campo”, nos quais são realizadas reuniões,
palestras e visitas nas áreas implantadas, com intuito de mostrar aos alunos, pesquisadores e
produtores rurais as informações geradas durante todo o período de existência do projeto, sejam
elas em sistemas agroflorestais, plantios homogêneos ou florestas nativas, durante esses dias de
campo os participantes têm a possibilitante também de cooperar na manutenção e instalação dos
plantios. Os discentes do PET-Floresta participam e organizam o evento, a contribuição ocorre
pela elaboração de materiais de divulgação, confecção de textos referentes aos temas que serão
abordados e a apresentação do material. Com a elaboração dessas atividades no projeto, é
possível levar aos produtores rurais informações práticas sobre silvicultura, e, por outro lado,
oportunizar aos estudantes uma vivência técnica, aprimorando o contato com os produtores
rurais, extensionistas e propriedades rurais, além de estimular soft skills. Fazendo com que este
projeto conte com atividades de ensino, pesquisa e extensão. Por fim, as unidades demonstrativas
estão se tornando uma referência para a região, atraindo o interesse de produtores rurais. Há um
número significativo de pesquisas a nível de graduação e pós-graduação sendo desenvolvidas
nessas áreas, o que faz com o projeto ganhe uma maior visibilidade em um contexto geral. Outro
aspecto a ser considerado é a oportunidade gerada aos graduandos, pós-graduandos e aos
membros do PET-Floresta a vivência e capacitação através destas atividades práticas realizadas
em conjunto com outros cursos de graduação, produtores rurais, extensionistas, instituições
privadas e de ensino.

\includepdf{pdfs/Projeto-Arboreto}

\addcontentsline{toc}{section}{Projeto Brotar Em Classe: Uma Intervenção Escolar Sobre Saneamento Hídrico Na Educação Remota   }

\section*{Projeto Brotar Em Classe: Uma Intervenção Escolar Sobre Saneamento Hídrico Na Educação Remota   }

Gabriel Henrique Monteiro Silvestre, Rafael de Lima,Bruna Lins,Geovana Izabela Mota,Mariana Silva Corrêa,Renato  Hajenius Aché de Freitas

O município de São José (SC) é uma região metropolitana de Florianópolis e possui
cerca de 250.000 habitantes. Apesar do saneamento ser garantido pela lei federal 11.445/07,
de acordo a Companhia Catarinense de Águas e Saneamento (CASAN), apenas 38,31% da
população tem o seu esgoto tratado, o que significa que a maioria do esgoto residencial acaba
chegando aos rios extremamente poluído e contaminante, oferecendo riscos para populações e
meio ambiente. Tendo isso em vista, ações que visam reduzir e conscientizar sobre os
impactos sócio-ambientais são extremamente necessárias.
Com o objetivo de desenvolver estudos e atividades em Educação Ambiental Crítica
(EA), o projeto de extensão Brotar foi criado em 2013 pelo PET Biologia UFSC. Desde
então, atua nos espaços formais de ensino, sobretudo em escolas públicas, abordando os
princípios da EA através de práticas e confecções de materiais pedagógicos. Em decorrência
da pandemia do Covid-19, as atividades do Brotar precisaram ser adaptadas ao Ensino
Remoto Emergencial adotado pela universidade associada e pelas escolas do Ensino Básico.
Assim, o projeto realizou suas práticas pedagógicas remotamente na escola pública
Centro Educacional Municipal Interativo (CEM Interativo), no município de São José (SC).
Nós, integrantes do Brotar, atuamos em duas turmas do 6º ano do Ensino Fundamental, uma
com 16 e a outra com 20 estudantes, através da mediação da profª Ana Carolina Bosio, com
o tema norteador “abastecimento, consumo e descarte de água”, na disciplina de Ciências.
No momento em que a atividade foi realizada, por medidas de saúde sobre a ocupação
da sala, as turmas eram divididas em A e B. Além das turmas híbridas, a escola também tinha
estudantes que não possuíam acesso à internet, e que compareciam à escola somente para
pegar atividades impressas deixadas pelos professores. Portanto, foi necessário elaborar um
plano de atividade que fosse auto explicativo e propositivo para ser realizado tanto em sala
quanto por aqueles que não tinham acesso à internet, ou não podiam ir para a escola.
Primeiramente, propusemos uma atividade intitulada “Onde está a água?”. Nela,
redigimos um texto introdutório sobre aspectos gerais dos recursos hídricos, sua utilização na
sociedade e sistemas de abastecimento de água e tratamento de esgoto no município de São
José, junto de quatro questões sobre este assunto. Estas questões tinham como principais
objetivos o reconhecimento de mananciais e recursos hídricos do município, além de sua
valoração pelos estudantes. Para isso, no final do primeiro encontro solicitamos a eles que
desenhassem um mapa do percurso da água, ilustrando desde a fonte natural em que é
captada, o seu local de tratamento, o momento de chegada em suas casas, até o destino final
da água residual em um corpo d’água.
No período da prática pedagógica, realizamos dois encontros com o grupo \"A\'\' e dois
encontros com o grupo \"B\'\' de cada turma, todos realizados pela plataforma de
videoconferência Google Meet, em momento síncrono com os educandos em sala de aula.
Para que eles pudessem nos ver, a professora projetou nossa chamada na parede da lousa,
enquanto nossa comunicação com eles se dava através de um microfone.
Para o primeiro encontro com cada grupo das duas turmas, organizamos três
momentos: apresentação do projeto Brotar e de seus integrantes; transmissão da ‘nuvem de
palavras’ pela plataforma Menti Meter; e apresentação da atividade pedagógica com
exemplos de mapas feitos pelos integrantes do projeto. Utilizando a nuvem de palavras,
levantamos os seguintes questionamentos aos educandos: “Onde você encontra água no seu
dia a dia?” e depois “De onde vem a água que você bebe?”, para tomarmos conhecimento de
sua compreensão sobre o assunto. Para o segundo encontro, também utilizamos o recurso da
nuvem de palavras para comparar suas respostas anteriores com o que foi aprendido depois de
realizada a confecção do mapa, e, a partir disso, discutimos com os educandos a respeito de
sua percepção sobre a atividade realizada.
Avaliando a atividade como um todo, o uso da ‘nuvem de palavras’ através de uma
plataforma de interação digital foi de extrema importância nesse contexto remoto, pois
permitiu que os alunos assumissem uma postura ativa na construção das respostas aos nossos
questionamentos. A atividade de confecção do mapa de percurso da água também
proporcionou engajamento dos educandos, já que a maioria conseguiu realizar a atividade
proposta. Dentre os 32 mapas elaborados pelos estudantes, observamos três tipos de
ilustração: mapas que indicavam as especificidades locais, aqueles que apresentavam
elementos genéricos e os que se assemelhavam ao desenho esquemático do livro didático
sobre tratamento e abastecimento de água.
Neste modelo de aula, o principal desafio encontrado pelo grupo foi relacionado à
comunicação com os alunos, que acontecia em sua maioria através da mediação da professora
Ana Carolina. Não era possível ouvir com clareza o que estava sendo dito pelos educandos,
pois os mesmos estavam muito distantes do microfone associado ao computador, e não foi
possível visualizá-los em sala, pois a webcam não tinha motilidade para ser voltada para eles
e, portanto, permaneceu voltada de frente para a parede da lousa.
Apesar dos desafios enfrentados no ensino híbrido, acreditamos ter atingido nossos
objetivos com a prática pedagógica, mesmo de forma online. Foi uma experiência
enriquecedora para os integrantes do Brotar, segundo seus próprios relatos documentados
dentro do projeto. A partir de ferramentas que estimulam a participação ativa dos estudantes,
promovemos o engajamento dos alunos na discussão sobre problemáticas ambientais
regionais sem utilizar uma abordagem conteudista. Além do sucesso com os educandos, as
aulas também foram elogiadas pela própria professora responsável pelas turmas, e teve boa
recepção pelos gestores e outros profissionais da escola.
Referências
MENTIMETER. Mentimeter, 2014. Página inicial. Disponível em:
. Acesso em: 16 de set. de de 2021.
Prefeitura de São José. Revisão do Plano Municipal de Saneamento Básico de São
José/SC. São José: Prefeitura de São José, 2020.

\includepdf{pdfs/Projeto-Brotar-Em-Classe-}

\addcontentsline{toc}{section}{Projeto Vamos Entender}

\section*{Projeto Vamos Entender}

Milena Mayumi Costa Makimori, Laysa Samara da Silva,Tutor: Jorge Luís Nunes de Góes

A aplicação de conhecimentos científicos na Engenharia é indispensável, e é evidente que 
essa ciência tem evoluído ao longo da história. Em consequência, a responsabilidade das 
instituições de ensino, de capacitar indivíduos, tem se tornado um desafio cada vez mais complexo.
O Ministério da Educação (MEC) estabeleceu que as instituições de ensino de Engenharia no 
Brasil devem atender as exigências das Leis de Diretrizes e Base da Educação Nacional (1996) e 
a Resolução CNE/CES 11, de 11 de março de 2002, a fim de promover o desenvolvimento de 
competências e habilidades esperadas pela sociedade. Entretanto, Bazzo (2000) afirma que os 
professores detêm conteúdos e procedimentos didático-pedagógicos insuficientes e inadequados 
para viabilizar a desejável formação do engenheiro. Em virtude disso, os alunos apresentam 
dificuldades de correlacionar conteúdos e aplicar a teoria em situações práticas. À vista disso, 
membros do Programa de Educação Tutorial do curso de Engenharia Civil da Universidade 
Tecnológica Federal do Paraná (PET CIVIL UTFPR-CM) elaboraram o projeto de extensão 
“Vamos Entender”, com o intuito de proporcionar a aplicação prática de conceitos teóricos através 
de estudos de caso. Tal ação estimula o desenvolvimento do senso crítico e produção de 
conhecimentos profissionais. A relevância deste projeto se encontra na tentativa de aperfeiçoar a 
formação dos engenheiros por intermédio de um repertório científico, técnico e cultural.
Baseando-se na atualidade e no impacto causado na sociedade, o tema escolhido para a 
primeira edição do “Vamos Entender” foi o rompimento das barragens de Brumadinho e Mariana. 
De início, houve o contato com o Departamento de Construção Civil da Universidade de Campo 
Mourão sobre a ideia, a qual foi aprovada pelo mesmo. Em sequência, foi feito o convite ao 
Professor Dr. Ewerton Clayton da Fonseca, especialista em Geotecnia, que elaborou o material de 
apresentação. Finalizado esta etapa, foi liberado um formulário de inscrição pela internet e feito a 
divulgação presencial no campus Campo Mourão, além da divulgação nas redes sociais do PET. 
Esta atividade ocorreu no dia 13 de junho de 2019, às 18h15min no anfiteatro da UTFPR – Campus 
Campo Mourão, com duração total de 1h30min. Ao final do evento, os participantes receberam 
um questionário de satisfação para que deixassem um feedback da atividade, com finalidade de 
melhorias posteriores.
Em decorrência das medidas emergenciais para prevenção do COVID-19, o projeto não foi 
realizado no ano de 2020, porém, teve sua segunda edição realizada em 11 de agosto de 2021, às 
19h30min através de uma live na plataforma Youtube. Para essa segunda edição, o tema definido 
foi o atentado de 11 de setembro, por ser um marco na história e estar completando 20 anos em 
2021. Inicialmente, foi feito o contato com o Professor Dr. Ronaldo Rigobello, especialista em 
estruturas, e após sua confirmação no evento, começou a divulgação por meio das redes sociais do 
PET. O evento teve duração de 1h30min, e ao final, foi disponibilizado um formulário de feedback 
da atividade, com intuito de saber a opinião dos espectadores.
Seguindo a metodologia descrita acima, serão elaboradas edições anuais destas palestras. 
Além disso, existe a pretensão de formar parcerias com demais coordenações do campus e 
profissionais da comunidade externa, a fim de promover a interdisciplinaridade.
Os participantes da edição de 2019, ao responderem o formulário de satisfação, 
demonstraram, através dos comentários expostos pelos mesmos, que projetos como esse são 
importantes para formar alunos com perfil, humanista, crítico, reflexivo, ético e apto a pesquisa. 
Da mesma forma, a edição de 2021 obteve ótimos níveis de aprovação, além de sugestões para 
outros temas, o que demonstra interesse da parte do público em relação ao projeto. Ficou
perceptível em ambas as edições, a forma como a comunidade acadêmica anseia evoluir e 
aprender. Daí a necessidade de se preocupar com a maneira em que um profissional pode atender 
a sua comunidade, pois, isso pode trazer diferenciais na vida dos estudantes quando entrarem para 
o mercado de trabalho.
É necessário avaliar e elucidar temáticas que abordam os estudantes num geral, uma vez 
que isso se encontra na adequação de que todas as áreas caminham juntas e são dependentes de 
fatores que correlacionam todos os campos de conhecimento, mesmo não estando inseridos uns 
aos outros diretamente. Visto que a engenharia é uma formação dedicada a transformar
conhecimento em solução e pode exigir dos profissionais o saber prático em diversos e diferentes 
campos de atuação, a repercussão dessa abordagem pode ser ainda mais relevante, por se tratar de 
uma ciência com caráter predominantemente tecnológico, que demanda estratégias didáticas que 
favoreçam uma postura ativa, crítica e inovadora dentro e fora da universidade.
Os estudos de casos apresentados no projeto discorrido são apropriados para investigação 
de fenômenos quando há uma grande variedade de fatores e relacionamentos que podem ser 
diretamente observados, não havendo regras e diretrizes que ditam o nível de importância desses 
fatores. Tendo consciência disso, abrem-se as portas para criação de novas tecnologias, dando 
ênfase a diversidade de dimensões, características e diferentes métodos de resolução de um 
problema e permitindo uma análise em profundidade dos processos e das relações entre eles. 
Projetos com propostas como essa são capazes de mudar o que antes poderia resultar no chamado 
“ciclo básico” dos currículos tradicionais das universidades. Tal feito é decorrente da motivação e 
elucidação de obras de Engenharia, onde é possível comprovar a interdisciplinaridade de forma 
dinâmica.
REFERÊNCIAS:
BAZZO, Walter Antônio; PEREIRA, Luiz Teixeira do Vale; LINSINGEN, Irlan von. Educação 
tecnológica: enfoques para o ensino da engenharia. Florianópolis: Editora da UFSC, 2000.
BRASIL. Lei Federal Nº 9.394, de 20 de dezembro de 1996. Estabelece as diretrizes e bases da 
educação nacional. Disponível em: .
Acesso em: 25 ago. 2021.
BRASIL. Secretaria de Ensino Superior. Resolução CNE/CES 11, de 11 de março de 2002. 
Diretrizes Curriculares para os Cursos de Engenharia. Disponível em:
. Acesso em: 25 ago. 2021.

\includepdf{pdfs/Projeto-Vamos-Entender}

\addcontentsline{toc}{section}{Projeto “Capacitação E Qualificação Petiana”: Promoção De Práticas De Ensino E Formação  Profissional Ao Futuro Turismólogo. }

\section*{Projeto “Capacitação E Qualificação Petiana”: Promoção De Práticas De Ensino E Formação  Profissional Ao Futuro Turismólogo. }

Angelice Raquel Motter Manzino Motter Manzino, Ligia Dalchiavon

O projeto de “Capacitação e qualificação PETiana” consistem na organização, participação e 
promoção de oficinas, cursos ou palestras aos membros do Grupo PET Turismo. Surgiu com o 
intuito de profissionalizar, capacitar e qualifica os discentes enquanto petianos, alunos e futuros 
profissionais, com abordagem nos temas relacionados a área do turismo e sua interdisciplinaridade 
com outras áreas. A escolha dos temas para cada encontro ou curso aplicado busca suprir as 
necessidades do Grupo, capacitar em temas e ofícios que perpassam os demais projetos ou 
especialização em conhecimento técnico e científico para a realização de ações e tarefas 
desenvolvidas internamente. Busca-se dar ênfase em temas pouco ou não abordados em sala de 
aula, mas necessários no dia a dia do futuro profissional em turismo. O Projeto visa a qualificação 
e a integração dos petianos e das equipes de trabalho organizadas pelo Grupo para a execução dos 
projetos de seu planejamento anual. Também, o desenvolvimento de habilidades e a descoberta de 
potenciais dos alunos. Em determinados momentos, o projeto desafia os petianos a promoverem 
as palestras e oficinas, visando a qualificação pessoal, acadêmica e profissional. O projeto almeja 
o fortalecimento das fragilidades do Grupo PET Turismo, visando a plena realização das 
atividades propostas, a integração dos membros e o desenvolvimento individual de afinidades. 
Tem como objetivos a realização de capacitações contínuas no Grupo; a troca de ideias e 
experiências técnico-cientificas entre aluno-aluno e aluno-profissional, visando a aproximação do 
PET com a comunidade acadêmica e externa; o aperfeiçoamento e a descoberta de habilidades; o 
desenvolvimento interpessoal dos petianos; o entendimento amplo da área de turismo e temas 
inerentes à profissão; estimular o estudo em outras línguas, como espanhol, inglês e francês. De 
forma ampla, o projeto tem o intuito de fortalecer as fragilidades do grupo com o desenvolvimento 
pessoal, profissional e acadêmico. A “Capacitação e qualificação PETiana” têm como 
metodologia, a escolha dos temas das palestras, oficinas, cursos e workshops definidos em 
coletivo, a partir da sinalização de temáticas indicadas pelo líder do projeto, o qual é escolhido um 
petiano para organizar o projeto durante o ano. Posteriormente, é estabelecido um cronograma 
para a realização das atividades; a forma de execução e o tipo da atividade; a organização do 
evento pelo responsável do projeto; a realização da atividade e, por último, a avaliação individual 
dos participantes, analisando pontos fortes e fracos para aperfeiçoar futuras capacitações e 
qualificações. No ano de 2021, o Grupo PET Turismo participou de 2 oficinas, 4 cursos, 1 palestra 
e do curso de línguas, respectivamente: Funções básicas da página institucional; Organização de 
eventos online; Psicologia do Trabalho; Gestão de pessoas - conceitos e processos; O uso de 
Aplicativos Web na Construção de Materiais Educacionais; Trilha para o Empreendedorismo -
organizando a sua ideia de criar; Conversa com o Ministério do Turismo; Conversação em 
Espanhol; Curso Regular de Francês e Espanhol. Majoritariamente os petianos participaram de 
todas as atividades propostas e consideraram a aprendizagem satisfatória e essencial para o
fortalecimento do Grupo e para o desenvolvimento pessoal como petiano, atendendo suas 
expectativas pessoais e coletivas. Após a realização da oficina de funções básicas do site 
institucional, foi sinalizada a necessidade de aprofundar o aprendizado no tema pelo Grupo, sendo 
insuficiente o aprendizado com apenas uma oficina. As capacitações e qualificações se iniciaram 
em abril e finalizaram no mês de agosto. Acontecendo de forma contínua nas datas estipuladas
pelo cronograma do projeto. O projeto desenvolvido cumpriu com o objetivo de promover o
aprendizado de forma contínua, possibilitar o aprofundamento de temas e o desenvolvimento de 
novas atividades, adequadas ao perfil e características do Grupo PET Turismo. Assim, visando o 
fortalecimento de suas habilidades e mitigando as suas fragilidades. Através dos bons resultados 
das atividades realizadas em 2021 e das avaliações positivas dos petianos, concluiu-se ser essencial
para a formação ampla dos acadêmicos a continuação do projeto “Capacitação e qualificação 
PETiana” com sua inclusão no plano de trabalho anual do Grupo para os próximos anos. Para 
edições futuras do projeto, faz-se necessário a adequação das temáticas a serem abordadas de 
acordo com as necessidades do Grupo. Bem como a ampliação da qualificação de temas já 
desenvolvidos e que pela avaliação da atividade mostrou-se importante o aprofundamento dos 
estudos. Por fim, conclui-se que as formas de inserção e aplicação das “Capacitações e 
qualificações PETiana” corroboraram não somente para o desenvolvimento acadêmico do petiano, 
mas oportunizaram momentos de aprendizado e integração entre os petianos com profissionais, 
com o mercado de trabalho e com temas emergentes e relevantes à formação do futuro 
turismólogo

\includepdf{pdfs/Projeto-Capacitacao-E-Qu}

\addcontentsline{toc}{section}{Projeto “Programa De Atendimento Ao Calouro - Pac”}

\section*{Projeto “Programa De Atendimento Ao Calouro - Pac”}

Rafaela Ramos Fofonka, Ana Clara Petry,Felipe Pereira Vergara,Isadora Santin Fochi,Júlia Farias,Katlin Modesto Dorneles,Cesar Alberto Ruver (Tutor)

O Programa de Atendimento ao Calouro (PAC) é um projeto de ensino realizado pelo
Grupo PET Engenharia Civil da Universidade Federal do Rio Grande do Sul (UFRGS), que foi
desenvolvido com o objetivo de levar até os calouros do curso Engenharia Civil da UFRGS,
diversas informações essenciais para uma melhor adaptação à Universidade. Além disso, traz
uma proximidade maior com os calouros para que participem dos projetos, processos seletivos e
cursos desenvolvidos pelo Grupo PET. O projeto inicialmente foi concebido no regime
presencial, onde uma das atividades desenvolvidas era a apresentação, com visitas guiadas, aos
diversos setores da Universidade, incluindo laboratórios, restaurantes universitários, bibliotecas e
outros espaços. Após o início da pandemia, devido ao Covid-19, com a restrição presencial, o
projeto foi executado de forma online e síncrona. Além das informações trazidas, são propostas
no início do encontro dinâmicas com o objetivo de deixar os calouros mais confortáveis para
interagir e tirar suas dúvidas. Neste início de semestre (2021/1), foi utilizado um site para
realizar um jogo de “Stop”, onde temas foram pré-definidos e uma letra do alfabeto era sorteada
para que fosse escrito algo sobre cada tema com a primeira letra da palavra sorteada. Os calouros
tinham um tempo pré estabelecido para terminar de escrever em todas as palavras sobre os
temas, no entanto havia a possibilidade de alguém terminar antes do tempo e chamar o “Stop”.
Sendo assim, todos deveriam parar de escrever. Havia também um somatório de pontos levando
em conta se a palavra fosse repetida ou não e conferência se elas estavam corretamente
empregadas em seus respectivos temas, podendo estabelecer um vencedor. Esses encontros têm
sido realizados de forma remota, onde, após a realização da dinâmica, foram apresentadas as
plataformas que são utilizadas durante o período acadêmico, dentre elas o Moodle Acadêmico,
Portal do Aluno, G - Suíte, oportunidades dentro da universidade, como Empresa Júnior de
Arquitetura e Engenharia Civil(EJECiv), o Centro dos Estudantes de Engenharia Civil da
UFRGS(CECIV), o Centro de Estudantes Universitários de Engenharia da UFRGS (CEUE),
Associação Atlética Acadêmica da Escola de Engenharia da UFRGS (AAEE) e o próprio PET
Engenharia Civil. Foi falado também sobre os auxílios para alunos que fazem parte das cotas
socioeconômicas, tendo o auxílio alimentação e auxílio transporte como exemplo. Foi utilizado o
espaço também para dar dicas de ferramentas e softwares que auxiliam no desempenho do aluno,
como Excel, Photomath, Symbolab, e dicas sobre o currículo do curso e como melhor
administrar as disciplinas mais difíceis. O projeto é muito bem recebido pelos calouros que
compareceram e participaram, apesar de no modo online não haver tanta adesão ao projeto.

\includepdf{pdfs/Projeto-Programa-De-Aten}

\addcontentsline{toc}{section}{Redes Pedagógicas: As Tecnologias Digitais Como Ferramentas De Ações Do Pet Pedagogia Em Tempos De Pandemia}

\section*{Redes Pedagógicas: As Tecnologias Digitais Como Ferramentas De Ações Do Pet Pedagogia Em Tempos De Pandemia}

Lucas Da Costa Lage, Misaeli Botelho Lima - Bolsista PET Pedagogia da Universidade Federal do Pampa,Profa. Dra. Juliana Brandão Machado - Tutora do PET Pedagogia da Universidade Federal do Pampa

A pandemia de COVID-19 apresentou novos desafios para pensarmos o cenário
educacional. Nesse sentido, com a paralisação das atividades presenciais de ensino, os espaços
educacionais tiveram que repensar suas ações, de modo a se remodelar para a nova configuração
instaurada. A partir desse contexto, os grupos do Programa de Educação Tutorial (PET)
reorganizaram as atividades propostas em seu planejamento, viabilizando sua execução.
Em meio ao isolamento social, as tecnologias digitais surgem como ampliação de novos
horizontes, conforme aponta Sibilia (2012, p. 186): “a conexão às redes dissolve o espaço -
sobretudo aquele que é pautado pelo confinamento -, mas também dilui o tempo, ambos como
fontes capazes de organizar a experiência”. Nessa mudança de cenário, tivemos os projetos
extensionistas reformulados e com novos alcances possibilitados pelas tecnologias digitais,
tornando-se um eficaz meio de comunicação entre a universidade e a comunidade externa.
Este trabalho visa apresentar algumas reflexões realizadas pelo PET Pedagogia no
decorrer dos anos de 2020 e 2021, reafirmando o compromisso do PET frente ao novo contexto
social imposto, fomentando as redes pedagógicas para a construção do saber de modo que haja
uma interação dialógica entre a tríade ensino, pesquisa e extensão.
Conforme aponta Lévy (1999, p. 81) “a comunicação por mundos virtuais é, portanto,
em certo sentido, mais interativa que a comunicação telefônica, uma vez que implica, na
mensagem, tanto a imagem da pessoa como a da situação, que são quase sempre aquilo que está
em jogo na comunicação”. Nessa perspectiva, o contato por meio de plataformas como o Google
Meet se tornou essencial para o andamento das proposições do grupo PET Pedagogia.
O meio digital se apresenta como uma metodologia interativa e de aproximação entre os
sujeitos, que precisaram, devido a pandemia, manterem-se em isolamento. As ações promovidas
nas diferentes esferas possibilitaram um maior alcance em relação às atividades presenciais,
como o “Grupo de Estudos Epistemologias da Docência para o Século XXI” que, durante seus
encontros em 2019, limitava-se apenas aos bolsistas e no decorrer de 2020 outros participantes
integraram os debates promovidos. Nesse mesmo sentido, destacamos o projeto “Formação para
a Pesquisa Científica em Ciências Humanas”, com o intuito de aprofundar e qualificar o
desenvolvimento das produções científicas dos bolsistas e acadêmicos envolvidos.
As reuniões de planejamento trouxeram grandes contribuições reflexivas acerca das
realizações dos projetos, pois ao mesmo tempo que o grupo executava as atividades, se colocava
como participantes, de modo que avaliavam os métodos e sua percepção frente à aprendizagem.
Assim como as avaliações apresentadas pelos demais participantes através de formulários
reflexivos, que nos permitem construir uma prática voltada para a interação entre os meios e os
sujeitos.
Sem a interação dialógica, permitida pelas atividades extensionistas, a Universidade
corre o risco de ficar isolada, ensimesmada, descolada dos problemas sociais mais
prementes e incapaz de oferecer à sociedade e aos governos o conhecimento, as
inovações tecnológicas e os profissionais que o desenvolvimento requer (FORPROEXT,
2012, p. 23).
Outros projetos foram viabilizados com o uso das tecnologias, o qual não limitou-se
apenas às ações extensionistas, mas interligaram a pesquisa e o ensino, desenvolvendo a relação
indissociável defendida pelo PET. Conforme os objetivos apresentados na portaria 976/2010, no
que diz respeito a “contribuir para a elevação da qualidade da formação acadêmica dos alunos da
graduação” (BRASIL, 2010), o PET Pedagogia tem se destacado como um dos primeiros
contatos de novos graduandos na universidade, realizando, para além das semanas de
acolhimento aos ingressantes, ações que pontuem didaticamente o uso de mecanismos
fundamentais para a vida acadêmica em meio ao período de ensino remoto emergencial.
Portanto, compreendemos que a formação é contínua e necessita ser ressignificada frente
aos novos desafios que são impostos em tempos de crises, a fim de potencializar uma gama de
projetos que considerem as epistemologias docentes. Por fim, as reflexões do grupo PET
Pedagogia frente às atividades desenvolvidas em meio a pandemia, corroboram a ideia de
“formular novas estratégias de desenvolvimento e modernização do ensino superior no país”
(BRASIL, 2010), fortalecendo os laços de uma educação crítica-reflexiva e que se conecta às
necessidades de diálogos que contribuam para a prática pedagógica.
Conclui-se que as tecnologias digitais, por vezes estigmatizadas pelos sistemas
educacionais, são ferramentas potencializadoras da efetividade das ações nos ambientes
acadêmicos, sendo a linguagem que reduz as distâncias, ultrapassando os muros e nos permitindo
se fazer presente em tempos tão adversos, contribuindo para evolução da educação em caminhos
digital, tecendo redes coletivas pautadas na interação, diálogo e construção de conhecimentos.
Referências:
BRASIL. Ministério da Educação. Coordenação de Aperfeiçoamento de Pessoas de Nível
Superior (CAPES). Portaria no 976 de 27 de Julho de 2010. Brasília, DF. MEC/CAPES,
2010.
FÓRUM DE PRÓ-REITORES DE EXTENSÃO DAS UNIVERSIDADES PÚBLICAS
BRASILEIRAS (FORPROEXT). Política Nacional de Extensão Universitária. “Coleção
Extensão Universitária”. Porto Alegre: UFRGS, 2012.
LÉVY, Pierre. Cibercultura. Tradução Carlos Irineu da Costa. São Paulo: Ed. 34, 1999. 264 p.
(Coleção TRANS)
SIBILIA, Paula. Redes ou paredes: a escola em tempos de dispersão. Rio de Janeiro:
Contraponto, 2012.

\includepdf{pdfs/Redes-Pedagogicas--As-Tec}

\addcontentsline{toc}{section}{Relato De Experiência: O Uso Da Rede Social Instagram Como Ferramenta Para Disseminação De Conteúdo Sobre Alimentação E Nutrição.}

\section*{Relato De Experiência: O Uso Da Rede Social Instagram Como Ferramenta Para Disseminação De Conteúdo Sobre Alimentação E Nutrição.}

Milena Santa Anna Fuhrmann, Greta Gabriela Rychescki,Aline Chinenye Anyanwu,Beatrice Orthmann,Clara Nogueira Pacheco,Ana Beatriz Moraes da Silva,Francilene Kunradi Vieira

Problemática: Recentemente, as redes sociais tornaram-se um excelente meio de divulgação
científica e acesso a conhecimentos. Com a pandemia de Covid-19 e a substituição de diversas
atividades presenciais para a modalidade remota, o Instagram se fortaleceu como meio de
divulgação de ações, conteúdos e espaço para produção educacional em nutrição. Visto as
limitações para a realização de ações de ensino, pesquisa e extensão de forma presencial, o
Programa de Educação Tutorial do curso de Nutrição da Universidade Federal de Santa Catarina
(PET Nutrição UFSC), adequou-se a esta nova realidade, utilizando o Instagram como
ferramenta para manter o contato com a comunidade externa e acadêmica e disseminar
informações confiáveis e relevantes.
Justificativa: Em alternativa ao cenário existente, foi utilizada a ferramenta tecnológica
Instagram, para educação em saúde objetivando o fácil acesso ao conhecimento de forma
simples e compreensível, suportada em respaldo científico. Dessa forma, objetivou-se
disponibilizar informações relacionadas à alimentação e nutrição, no intuito de contribuir para a
promoção da saúde, prevenção de doenças, mudanças de comportamento, formação de opiniões
e disseminação de informações científicas, além de divulgar outros eventos realizados pelo grupo
PET.
Metodologia: As ações realizadas pelo grupo PET Nutrição UFSC, tais como, as atividades de
ensino e extensão, os eventos e projetos vinculados ao programa, realizadas anteriormente de
forma presencial, foram adaptadas iniciando-se a utilização do perfil público já criado na rede
social Instagram (@petnutriufsc). A rede social é gerenciada pelos petianos integrantes do cargo
de marketing, tendo a responsabilidade de organizar o cronograma de postagens, assim como
confeccioná-las e postá-las, compartilhar nos stories e responder aos comentários. As temáticas
foram definidas pela demanda do grupo, afinidade dos petianos, assuntos relevantes e/ou vistos
em sala de aula e também conforme o alcance das postagens. No decorrer do tempo, planejou-se
publicar duas postagens semanais.
Resultados e Discussão: As postagens informativas e sobre alimentação e nutrição tiveram
início em abril de 2020 e até o momento totalizam 148 posts (de um total de 383 no perfil).
Foram realizadas várias séries de postagens com diferentes temas: orientações do Guia
Alimentar para a População Brasileira, rotulagem nutricional, divulgação de ebook com receitas
da estação, “Parece mas não é”, fitoterapia, alimentos regionais e receitas, diferença de produtos
e vitaminas. Além disso, alguns posts únicos, sobre hipertensão, alergias alimentares, microbiota
e transgênicos, também foram realizados. No último mês, tivemos o lançamento de uma série
quinzenal de postagens, “Notícias da Nutrição’’, onde trazemos as principais notícias
relacionadas à nutrição e alimentação do momento, a primeira dessas postagens é uma das que
apresenta maior número de comentários (n = 14). Em relação ao alcance das postagens, obtemos
uma média de 477 visualizações, sendo a postagem sobre temperos naturais a com maior alcance
(1608 visualizações) e maior número de salvamentos (80). Os posts da série “Parece mas não é”
(parece suco, parece tempero e parece carne) também tiveram números significativos, com uma
média de 1105 visualizações (1286, 1041, 990, respectivamente). Ainda, tivemos um
crescimento médio de 15% no número de seguidores em um ano (1582 seguidores em 20/08/20;
1822 seguidores em 13/09/21), dos quais 39% residem na Grande Florianópolis, 41,5% tem de
25 a 34 anos e 79,7% são mulheres.
Conclusão: Considerando os resultados apresentados, o impacto e alcance das publicações
possibilitam maior acessibilidade à informação pela comunidade externa e, ainda, viabilizam
ações de educação alimentar e nutricional de forma clara e compreensível. Além disso,
capacitam os integrantes responsáveis pelas postagens que, também, obtêm conhecimento na
elaboração dos materiais. Assim, o Instagram demonstra ser uma importante ferramenta no
processo de promoção de saúde, prevenção de doenças, mudanças de comportamento, formação
de opiniões e disseminação de informações científicas relacionados à nutrição.

\includepdf{pdfs/Relato-De-Experiencia--O-}

\addcontentsline{toc}{section}{Relato De Experiência: Pet Talks Contribuindo Na Escolha Da Área De Atuação Do Estudante De Nutrição.}

\section*{Relato De Experiência: Pet Talks Contribuindo Na Escolha Da Área De Atuação Do Estudante De Nutrição.}

Vitoria Tondo Santini, Mary de Paulo Irmão,Ana Luísa Baurich Vidor,Brisa Rocha,Francilene Kunradi Vieira

Problemática: Em 2018, o Conselho Federal de Nutricionistas apresentou as seis áreas de
responsabilidade técnica do nutricionista, quais sejam: nutrição em alimentação coletiva,
nutrição clínica, nutrição em esportes e exercício físico, nutrição na cadeia de produção, na
indústria e no comércio de alimentos e nutrição no ensino, na pesquisa e extensão. Tais
diversidades temáticas devem ser contempladas na matriz curricular deste curso de graduação da
área da saúde. Além disso, embora haja uma variedade de cursos de especialização na área da
nutrição, o grande descontentamento de mais da metade dos brasileiros com seus empregos, gera
anseios nos graduandos que precisam seguir decidindo sua carreira profissional após a formatura.
Justificativa: Dessa forma, o projeto de ensino PET Talks objetiva a troca de experiências entre
profissionais e estudantes de nutrição. Este oferece aos graduandos a possibilidade de adquirir
conhecimentos sobre a prática e o caminho trilhado pelos nutricionistas das diferentes áreas e,
assim, prestar suporte à tamanha decisão, oferecendo a comunidade profissionais mais
realizados. Metodologia: O projeto, adaptado a modalidade remota, realizado através do
Instagram do PET Nutrição UFSC (@petnutriufsc), foi desenvolvido a partir do
compartilhamento de vídeos de nutricionistas de renome nas mais diversas áreas de atuação,
explanando suas trajetórias, desde o início da graduação até o presente momento de suas
carreiras. Os profissionais foram convidados a participar através das suas redes sociais
(Instagram) e/ou e-mail. Após o consentimento, os petianos envolvidos encaminharam um
roteiro com as instruções para elaboração dos vídeos e os publicaram à medida que foram
recebidos. Resultados e discussão: A primeira publicação ocorreu em julho e a última em
dezembro de 2020. Ao todo foram publicados onze vídeos no perfil do Instagram do PET
Nutrição. Cada vídeo, com duração média de 10 minutos, contou com a participação de um
nutricionista das distintas áreas de atuação, tais como nutrição em alimentação coletiva, nutrição
clínica, nutrição esportiva, nutrição hospitalar, nutrição materno infantil, entre outros. Com
relação aos dados de compartilhamento, alcance, curtidas, comentários, envios e salvamentos dos
vídeos, percebeu-se uma variação, sendo o alcance médio deles de 430 pessoas. O de maior
alcance chegou a 764 pessoas, teve 112 curtidas, 12 envios, 4 salvamentos e 20 comentários. Os
vídeos com maior alcance foram das áreas de nutrição clínica, esportiva e comportamental,
demonstrando um maior interesse dos estudantes por tais áreas. Conclusão: Acredita-se que os
números de alcance, curtidas, comentários, envios e salvamentos atingidos demonstram o grande
alcance do projeto e contemplam o objetivo de compartilhar a trajetória de nutricionistas de
renome em diferentes áreas. A modalidade remota do projeto permitiu o alcance do conteúdo a
alunos de nutrição da UFSC como também à alunos de outras instituições que acompanham as
redes sociais do PET Nutrição da UFSC, possibilitando assim uma maior expansão do
conhecimento.

\includepdf{pdfs/Relato-De-Experiencia--Pe}

\addcontentsline{toc}{section}{Resíduo Eletrônico: Descarte, Reciclagem E Conscientização.}

\section*{Resíduo Eletrônico: Descarte, Reciclagem E Conscientização.}

Thalia Delgado Bezerra, Dienifer Viana e Silva,Milena Mayara dos Santos,Paula Rosa Pujol de Lima

1 INTRODUÇÃO
Resíduo eletrônico ou e-lixo são termos utilizados para referir-se a todos os equipamentos
e dispositivos elétricos e eletrônicos, que são descartados por não possuírem mais utilidade. O
Brasil éconsiderado o sétimo país na posição de países produtores de lixo eletrônico, mas fica 
em primeiro lugar em relação aos países da américa latina, produzindo em média 1,5 toneladas, 
por ano1
. Entretanto somente 3% desse é lixo reciclado ou descartado de maneira adequada1
.
A Lei n°12.305/10 – Política Nacional de Resíduos Sólidos (PNRS) – determina a
necessidade de desenvolver um sistema de logística reversa dos produtos elétricos/eletrônicos
para que evitem os danos causados à sociedade quando descartados de forma incorreta5
. Na
maioria dos casos, a falta de conhecimento das pessoas é o motivo principal do descarte incorreto,
já que pouco se é divulgado sobre os perigos do aterro ou incineração desses resíduos e sobre a
importância da reciclagem dos mesmos4
.
No Brasil, a reciclagem do lixo eletrônico ainda não ocorre3
. O material geralmente é
enviado para outros países com a justificativa de que se necessita de alto investimento financeiro
e uma grande quantidade de material reciclável para se tornar economicamente viável3
. Dessa
forma, este trabalho tem como objetivo mostrar os benefícios quanto ao descarte correto,
reciclagem e conscientização da população.
2 METODOLOGIA
A metodologia utilizada na realização deste trabalho foi um levantamento bibliográfico a
respeito do lixo eletrônico no Brasil e sobre formas de conscientização quanto à reciclagem, e o
reaproveitamento do lixo eletrônico. Este estudo inicial proporcionou a base teórica para a
elaboração de um projeto de conscientização, descarte e reutilização do lixo eletrônico.
3 RESULTADOS E DISCUSSÃO
A pesquisa bibliográfica permitiu verificar que o descarte incorreto do lixo eletrônico 
pode causar os seguintes impactos no meio ambiente: contaminação e poluição, redução do
tempo de vida útil dos aterros sanitários - pois o lixo eletrônico tem, em sua composição, 
materiais que demandam muito tempo para se decompor, o que acaba aumentando ovolume de
lixo no aterro, contaminação por metais pesados e danos à saúde pública - principalmente da 
população que vive entorno do aterro2
. A reutilização desse material é um processo que exige
mão-de-obra qualificada, fator gerador de empregos, já que quanto maior a demanda de 
reciclagem mais vagas de emprego serão criadas.
No Brasil, existem locais corretos para se realizar o descarte de produtos eletroeletrônicos
e a PNRS5
obriga as empresas a aceitarem o retorno de seus produtos como eletroeletrônicos,
pilhas e baterias descartados e de se responsabilizar pelo destino desses, a chamada logística
reversa, que é fundamental para a geração de lucros a partir dos resíduos e para prevenir o meio
ambiente da contaminação dos solos e outras matrizes ambientais. Entretanto, muitas pessoas não
possuem conhecimento a respeito disso.
Nessa perspectiva, propomos realizar uma campanha, através de posts no Instagram,
trazendo mais informações para a população em relação ao descarte e reutilização dos resíduos 
elétricos/eletrônicos. Os posts no Instagram tem uma duração de três meses, sendo publicados 
dois posts por mês, totalizando seis posts ao todo. Essa campanha teve início no mês de setembro
com térmio precvisto para e terminará no mês novembro do corrente ano.
4 CONSIDERAÇÕES FINAIS
Com o início da campanha esperamos trazer mais informações sobre o lixo eletrônico para
os internautas, visando a conscientização quanto ao descarte e a reutilização dos resíduos
elétricos/eletrônicos. A pesquisa realizada revela que a falta de ações no país para promover o
descarte responsável e a conscientização da população leva a sérios problemas, afetando a natureza
e a sociedade como um todo, revelando que o acúmulo de lixo eletrônico é muito importante e
deve ser tratado com muita seriedade e responsabilidade.

\includepdf{pdfs/Residuo-Eletronico--Desca}

\addcontentsline{toc}{section}{Revista Informe Letras Como Veículo De Divulgação Do Conhecimento Sobre Discursos De Resistência}

\section*{Revista Informe Letras Como Veículo De Divulgação Do Conhecimento Sobre Discursos De Resistência}

Guilherme Henrique Paro, Arthur Teixeira Ernesto (UNIPAMPA),Carolina Fernandes (UNIPAMPA),Larissa do Prado Martins (UNIPAMPA)

A Revista Informe Letras é uma revista de estilo magazine publicada em formato digital
produzida anualmente pelo Grupo PET-Letras, da Universidade Federal do Pampa, campus
Bagé, e que, no ano de 2021, chega em sua décima segunda edição, tendo como objetivo a
promoção e popularização do conhecimento científico produzido na área das Letras. Dessa
forma, a revista proporciona o acesso do público em geral aos textos produzidos pelos bolsistas,
pensando na linguagem em seu caráter social e transformador. Com isso, essa atividade tende a
levar o petiano a desenvolver as habilidades de pesquisa, leitura e produção textual, exercitando
métodos de coleta de dados com entrevistas, assim como uma reflexão sobre os resultados
encontrados, a fim de manter a indissociabilidade entre pesquisa, ensino e extensão. Para a
realização das análises, recorremos à Análise do Discurso (AD) de vertente materialista
concebida por Michel Pêcheux, e a partir de conceitos-base está sendo possível compreender de
que forma a ideologia se materializa na linguagem, bem como, as contradições que possibilitam
a produção de novos sentidos por meio das “condições ideológicas da reprodução/transformação
das relações de produção” (PÊCHEUX, 1995, p. 133) de cada discurso. A partir disso, o grupo
definiu a temática geral que abarca todos os textos publicados, para a produção da décima
segunda edição, levando em conta “a arte como resistência”. Assim, para compor essa temática,
foram ponderados os subtemas que giram em torno de produções artísticas como: livros, filmes,
músicas e artes plásticas, e uma entrevista com uma artista convidada. Com isso, procuramos
promover discussões sobre temas como a violência racial, a desigualdade social, o lugar social
das mulheres diante da ideologia patriarcal, e a violência que ocorre com frequência contra os
indígenas e as pessoas LGBTQIA+. Dessa forma, para a elaboração da Revista Informe Letras, o
grupo conta com uma equipe de bolsistas responsável pela edição geral. Nesse processo, ocorrem
etapas de desenvolvimento corresponde à: confecção dos textos que são elaborados por cada
petiano dentro de um prazo inicial de escrita, essa etapa conta com a orientação da tutora para
um melhor direcionamento do trabalho, depois os textos são trocados entre duplas de acordo com
os temas em que melhor se aproximam, e, após feita as correções, os textos são compartilhados
com a tutora novamente para uma correção final e, em seguida, ocorre a revisão final feita por
bolsistas responsáveis pela revisão geral da revista. Em uma outra etapa, outros petianos são
encarregados de editarem a revista dentro da plataforma escolhida, no caso, o site Canva. As
decisões de enquadramento, layout, fonte e cores, são discutidas entre a equipe editora e depois
levado para os demais integrantes do grupo para uma aprovação final do produto. Dentro desse
processo de elaboração, ocorre o desenvolvimento e a apropriação de habilidades quanto ao uso
de tecnologias para a composição estética da revista. Após a revisão e edição do material, é feita
a publicação da revista por meio da exportação do PDF para a plataforma Issuu, onde é possível
ter acesso a ela. Através da divulgação da revista, conseguimos desencadear algumas reflexões
sobre os “efeitos de sentidos” produzidos a partir de cada materialidade, considerando os
discursos de resistência que se manifestam através da arte. Portanto, através desses textos, é
possível entender como ocorre o processo de produção de sentidos de cada discurso, visto que,
será através do funcionamento discursivo da linguagem que iremos compreender os processos
discursivos que produzem os sentidos na sociedade.
Referências:
PÊCHEUX, Michel. Semântica e Discurso: Uma crítica à afirmação do óbvio. 2 edição.
Campinas - SP: Editora da UNICAMP, 1995.

\includepdf{pdfs/Revista-Informe-Letras-Co}

\addcontentsline{toc}{section}{Saberes Pedagógicos: Diálogos Com Jovens Pesquisadores}

\section*{Saberes Pedagógicos: Diálogos Com Jovens Pesquisadores}

Crislaine Lopes De Oliveira, Karolyn Elizabeth Fernandes Dacri - Bolsista PET Pedagogia,Jéssica Reis de Melo - Bolsista PET Pedagogia,Isadora Cabreira da Silva - Egressa PET Pedagogia,Profa. Dra. Juliana Brandão Machado - Tutora PET Pedagogia

O presente trabalho visa apresentar um relato de experiência sobre o projeto de extensão
intitulado “Segundas do PET Pedagogia: Diálogos Interdisciplinares em Educação”, que teve
como objetivo promover diálogos sobre Ensino Superior, Gênero e Sexualidades, Direitos
Humanos, Educação para as relações Étnico-raciais e Cibercultura, temáticas que são
consideradas transversais no campo da formação docente e que integram o núcleo de estudo e
pesquisa do PET Pedagogia da UNIPAMPA.
Tal proposta se justifica pela necessidade de dialogar com jovens pesquisadores e
professores universitários, das respectivas áreas supracitadas, com a finalidade de promover
trocas de experiências entre os participantes em cada temática. Cabe ressaltar que o grupo faz
parte do projeto de pesquisa coordenado pela tutora, intitulado “Docência no século XXI:
políticas, narrativas, práticas e proposições para a construção de uma epistemologia do trabalho
docente”, que visa trabalhar temáticas diversas de interesse dos bolsistas citadas acima. Portanto,
essa proposta foi construída pelos alunos bolsistas, a partir do interesse de promover uma rede de
diálogos com pessoas que construíram suas experiências de pesquisa durante o período de
graduação.
Como base teórica para a construção do projeto, utilizamos a perspectiva da
autoformação, embasada em Josso (2004); bem como o conceito de conhecimento
pluriversitário, concebido como um conhecimento transdisciplinar que, em sua propriedade,
obriga o exercício dialógico ou confrontativo com outros tipos de conhecimento (SANTOS,
2011); e os relacionamos com os saberes docentes, constituído “de vários saberes provenientes
de diferentes fontes” (TARDIF 2014, p. 33).
O projeto foi desenvolvido por meio de webnários realizados através da plataforma
Google Meet, com duração de duas horas cada um. Os encontros e os convidados foram
definidos pelo grupo em reunião de planejamento, assim como a organização dos encontros e sua
metodologia. Estabeleceu-se as segundas-feiras para a execução do projeto, tendo em vista que,
no planejamento do primeiro semestre de atividades remotas emergenciais (2020/1), o PET
Pedagogia reivindicou junto à comissão de curso que este dia ficasse disponível para nossas
atividades formativas, o que possibilitaria a adesão de discentes ao projeto. Os temas ficaram
definidos conforme segue: o primeiro encontro ocorreu em 09/11/20 e tematizou o Ensino
Superior; o segundo encontro, realizado em 18/11/20, debateu a temática dos Direitos Humanos;
no terceiro encontro, em 30/11/20, abordamos o tema Gênero e Sexualidades; o quarto encontro,
em 07/12/20 debateu a Educação para as Relações Étnico-Raciais; e, o último encontro,
ocorrendo integrado à VII Semana Acadêmica do curso de Pedagogia, aconteceu no dia
14/12/20, abordando a cibercultura.
Os encontros tiveram a seguinte estruturação: inicialmente, a bolsista mediadora fazia
uma fala de abertura e apresentação do palestrante convidado. Em seguida, o mesmo realizava
sua explanação, finalizando-se com a participação dos inscritos através de perguntas, dúvidas e
relatos. Ao final do encontro foi disponibilizado um formulário de presença e eram registrados
prints com todos os participantes para o registro do encontro. Durante a realização do projeto
atingimos em média 30 participantes. No último encontro, foi enviado aos participantes um
formulário de avaliação do projeto, o qual possibilitou a análise apresentada neste trabalho. O
questionário foi composto por 18 questões que versavam sobre a dinâmica, temática, abordagens
dos convidados nos webnários e também a possibilidade de aprendizagem com eles.
Nesse sentido, os webinários se assumem como espaço de formação e troca de
experiências e perspectiva dos jovens pesquisadores sobre suas pesquisas, projetos e estudos que
participam nas universidades públicas desde a sua graduação. Dessa forma, as relações
construídas durante a realização do projeto apresentam novas possibilidades de integrar à tríade
universitária em nosso cotidiano, assegurando uma auto reflexão de nossa trajetória acadêmica.
A partir dos dados coletados, percebemos que o grupo conseguiu atingir seu propósito, pois seus
resultados em maioria foram positivos, tendo a aprovação pela maioria em relação aos temas e
suas abordagens.
Ressaltamos a relevância dos diálogos realizados durante o projeto, concretizando os
objetivos propostos pelo Programa de Educação Tutorial, que prima por uma educação
interdisciplinar capaz de possibilitar uma relação prática com sua própria formação. As
experiências relatadas, tanto dos palestrantes quanto dos participantes, possibilitaram a
indagação da necessidade de uma formação continuada dos temas propostos pois, conforme
Freire (1996, p. 16), “faz parte da natureza da prática docente a indagação, a busca, a pesquisa”,
a qual ocasiona uma reflexão intrínseca sobre a formação docente. Isso representa que, neste
projeto, o diálogo e o conhecimento sobre os temas foram abordados para aqueles que
participaram e compartilharam seus saberes, possibilitando a realização de conexões
pedagógicas.
Sendo assim, a prática extensionista trouxe novas oportunidades de aproximação com a
comunidade externa e universitária, de modo que vislumbrou o compromisso assumido pelo
Programa de Educação Tutorial no apoio à formação acadêmica, contribuindo significativamente
para uma educação de qualidade.
Referências:
FREIRE, Paulo. Pedagogia da Autonomia: saberes necessários à prática educativa. São Paulo:
Paz e Terra, 1996. - (Coleção Leitura).
JOSSO, Marie-Christine. Experiências de vida e formação. São Paulo: Cortez, 2004.
TARDIF, Maurice. Saberes docentes e formação profissional. 16. ed. Petropólis: Vozes, 2014.

\includepdf{pdfs/Saberes-Pedagogicos--Dial}

\addcontentsline{toc}{section}{Saúde E Resistência Da População Negra E Indígena: Um Relato De Experiência}

\section*{Saúde E Resistência Da População Negra E Indígena: Um Relato De Experiência}

Pedro Henrique Paiva Bernardo, Giovana Munhoz Dias - Petiana,Ana Luísa Serrano Lima - Petiana,Lucas Vinícius de Lima - Petiano,Bianca Monti Gratão - Petiana,Vitória Maytana Alves dos Santos - Petiana,Vanessa Denardi Antoniassi Baldissera - Tutora

Os efeitos da pandemia de covid-19 atingiram de forma diferente as classes sociais, agravando o quadro de exclusão social e vulnerabilidade que atingem a população negra, periférica e indígena, as quais representam os grupos mais atingidos, segundo dados epidemiológicos. Fica evidente que as desigualdades sociais, produzidas em função da classe social, raça e gênero, colocam as populações vulneráveis em situações mais precárias de adoecimento e morte. Além disso, nesse processo o racismo tem um papel potencializador de iniquidades. Há inúmeros conceitos e tipologias de racismo, no entanto, para Almeida (2019), o racismo é estrutural em toda suas relações sociais, visto que o mesmo possui uma estruturação que se dá por meio da formalização de um conjunto de práticas institucionais, históricas, culturais e interpessoais dentro de uma sociedade capitalista que frequentemente coloca um grupo social ou étnico em uma posição melhor em detrimento de outro. Ademais, a Organização Mundial de Saúde (OMS) considera o racismo como um dos determinantes sociais do processo de adoecimento e morte. Sendo assim, o grupo PET Enfermagem da Universidade Estadual de Maringá organizou, em parceria com a Associação dos Universitários Indígenas (AUIND), o Coletivo Negro de Psicologia (CONEPSI) e o Programa de Pós-Graduação em Enfermagem (PSE) da mesma Universidade, o “I Seminário de Saúde e Resistência Negra e Indígena: Debates sobre a Permanência e a Promoção da Saúde Mental na Universidade”, visto que, diante do momento vivenciado, torna-se extremamente importante um evento que vise discutir como a pandemia de covid-19 tem afetado as diferentes populações e de que modo a Universidade pode contribuir para a atenuação dessas desigualdades. Com isso, por meio de um relato de experiência, será relatada a experiência de concepção, planejamento e realização do evento supracitado. O evento de extensão objetivou discutir a importância da implementação de políticas para a permanência estudantil, reconhecer a essencialidade da articulação estudantil no processo de enfrentamento do racismo na Universidade e sociedade e debater técnicas e métodos das abordagens participativas no ensino, pesquisa e extensão, além de apresentar como a covid-19 tem afetado a população negra e indígena. A concepção e planejamento do evento partiram da necessidade em gerar um espaço de reflexão crítica sobre as diversas formas que o racismo permeia a educação superior, especialmente em razão de 2021 ser o primeiro ano em que as cotas raciais foram introduzidas na Universidade Estadual de Maringá, bem como as diferentes formas que a violência contra a população negra e indígena impacta na saúde em todo o ciclo vital, mas especialmente na saúde mental, e as diferentes experiências para o enfrentamento do racismo nas instituições de ensino superior e seus desafios. De tal modo, propiciou-se um espaço de reflexão crítica e articulações estratégicas individuais e coletivas de estudantes, docentes, profissionais da área da saúde e comunidade externa em geral para o processo de defesa da educação pública, integral, gratuita e
de qualidade, incluindo o racismo estrutural e a saúde da população negra e indígena na formação acadêmica, e nos processos de educação permanente e cuidado dos profissionais da saúde. O tema é considerado de relevante importância, levando em conta a intensa vulnerabilidade desta população no território brasileiro, o que contribui fortemente para a desigualdade social e a falta de oportunidades para acesso ao ensino superior. Portanto, aqueles que conseguem ingressar enfrentam um ambiente hostil, mudanças culturais e preconceito de colegas e professores, além de passarem a duvidar da própria capacidade de vivenciar a Universidade, lidam com sentimentos de inferioridade, necessidade de constante autoprovação e tentativa de evitar o fracasso ao máximo. Fatores esses que influenciam diretamente na carga mental e na maioria das vezes esses estudantes não procuram auxílio. Durante as reuniões, foram designadas as tarefas para cada integrante da comissão organizadora, como as artes para divulgação do evento, contato e suporte com os palestrantes, divulgação do formulário para inscrição, mediação de cada dia de palestra, sanar dúvidas no chat, cuidar da lista de frequência e transmitir o evento. O evento foi realizado durante o período noturno, de forma online, através da transmissão ao vivo pela plataforma YouTube, tendo dois palestrantes por dia, durante 5 dias. O evento contou com a participação média de 366 pessoas da comunidade interna e externa. Todos os objetivos propostos para o evento foram alcançados com êxito, conseguindo proporcionar uma reflexão crítica sobre as diversas formas de racismo presente no ensino superior, contada por pessoas que vivenciaram e não tiveram o apoio que se objetiva gradativamente ofertar nos dias atuais. Ademais, contribuiu para a construção de conhecimento da saúde negra e indígena, ressaltando a importância da integralidade no Sistema Único de Saúde.

REFERÊNCIAS

ALMEIDA, S. Racismo estrutural. São Paulo: Editora Pólen Livros, 2019.

\includepdf{pdfs/Saude-E-Resistencia-Da-Po}

\addcontentsline{toc}{section}{Seminários Odontológicos: Desenvolvendo O Conhecimento, A Pesquisa E O Senso Crítico}

\section*{Seminários Odontológicos: Desenvolvendo O Conhecimento, A Pesquisa E O Senso Crítico}

Romulo Ruan Da Silva, Gabriela Camarotto de Almeida,José Alexandre Felix de Camargo,Luísa Gonçalves Cardoso,Leonardo Galvão da Silva Garcia,Mariana Podadeiro de Andrade,Carlos Alberto Herrero de Morais

No âmbito acadêmico, diversas situações exigem do acadêmico a capacidade de falar em
público. Desse modo, demonstrar boas habilidades interpessoais e domínio de oratória são de
suma importância para que o acadêmico denote boa performance no ensino superior e em
relações sociais. Ademais, atualmente, as práticas educativas em Universidades se preocupam
em formar profissionais aptos para o mercado de trabalho e para a vida, com senso crítico e
capacidade de gerar novos conhecimentos. Os seminários odontológicos proporcionam ao
acadêmico adquirir novos conhecimentos de diferentes assuntos, bem como oferecer formas
pertinentes de interdisciplinaridade, por meio de um sistema de educação integral. Diante disso,
o objetivo é permitir que o acadêmico aprimore a desenvoltura em apresentações e oratória de
seminários, propiciando a capacidade de pesquisa, sistematização dos fatos, raciocínio, como
também reflexões acerca dos temas dentro da área odontológica. É sabido que, a realização dos
seminários permite não somente a habilidade em falar em público, como também atua na
redução da ansiedade, consequentemente resulta em um aperfeiçoamento no desempenho do
discente. Nesse viés, o presente trabalho tem como finalidade, relatar a experiência da atividade
promovida pelo Programa de Educação Tutorial - PET do curso de Odontologia da Universidade
Estadual de Maringá durante os anos de 2019 e 2020, que visa à complementação da formação
do petiano, denominada \"Seminários Odontológicos\". Trata-se de uma atividade aberta à toda
comunidade acadêmica interna e externa, divulgada em meio digital por meio de nossas redes
sociais. Os integrantes foram organizados em grupos para apresentação dos seminários e
composição da banca avaliadora, inseridos em um sistema de rodízio. Através da apresentação
do seminário, os petianos buscam interagir com docentes e convidados, expandindo
conhecimentos e interesse nos mais variados assuntos do contexto odontológico assegurando
uma formação interdisciplinar completa. A experiência em compor a banca examinadora auxilia
no desenvolvimento de visão e senso crítico. Fica evidente, portanto, que as ações
interdisciplinares oportunizam experiências diferenciadas em um cenário inovador de ensino e
aprendizagem. Por conseguinte, o contexto contribui para a formação dos futuros
cirurgiões-dentistas, preparados para proceder atividades de forma humanizada e interligada com
a capacidade de maior habilidade em apresentação ao público.


\includepdf{pdfs/Seminarios-Odontologicos-}

\addcontentsline{toc}{section}{Simpósio Online: Do Desafio À Oportunidade}

\section*{Simpósio Online: Do Desafio À Oportunidade}

Gabrielle Da Silva Flores De Campos, Carmem Eduarda Rohr Flores,Gabriela Cabral Tondolo,Marina Michels Dotto,Renata Rodrigues Soilo,Vitória Luiza Beier,Luísa Helena do Nascimento Tôrres

O Grupo PET Odontologia da Universidade Federal de Santa Maria sempre buscou oferecer aos
discentes do curso, palestras de diversas temáticas associadas à Odontologia a fim de contribuir
com a formação acadêmica com atividades além das da grade curricular vigente do curso. Essas
palestras, até o ano de 2020, eram feitas de forma presencial, todavia após a chegada da
pandemia e devido a restrições aos encontros presenciais, o grupo deparou-se com o grande
desafio de manter o vínculo com os acadêmicos, pois pausar o projeto até ser permitido voltar à
“normalidade” não era uma opção. Dessa forma, os simpósios mostraram-se como uma
oportunidade de levar conteúdo aos alunos e manter o vínculo com os mesmos, em tempos de
ensino remoto de caráter emergencial. O I Simpósio Online foi organizado inteiramente de forma
remota, foram escolhidos 3 dias consecutivos com uma palestra à tarde e outra à noite, sendo
totalmente gratuito a fim de permitir maior participação de ouvintes. Em um primeiro momento,
foi realizada uma enquete com os estudantes do curso no perfil do Instagram do Grupo PET
solicitando sugestões quanto a temas de interesse. Assim que os temas foram definidos, o ciclo
de palestras foi organizado por integrantes do grupo que entraram em contato com palestrantes
selecionados segundo o tema da palestra e disponibilidade de data e horário. O grupo realizou a
compra do pacote Zoom para transmitir as palestras pela plataforma YouTube, a divulgação do
evento foi realizada pelo perfil do grupo no Instagram e no Facebook, as inscrições foram
realizadas via formulário do google e todas as informações foram enviadas aos e-mails de
coordenação de cursos de Odontologia do país. A comissão organizadora do simpósio entrava na
sala criada para cada palestrante no Zoom e transmitia para o Youtube. Os participantes tiveram
acesso a sala de transmissão através do canal do grupo no YouTube e a presença foi
contabilizada através de dois formulários disponibilizados no início e final de cada palestra.
Além disso, os vídeos gerados de todas as palestras ficaram salvos no canal do YouTube e
podem ser assistidos novamente. Depois do sucesso do I Simpósio Online do grupo PET
Odontologia da UFSM, o grupo planejou a execução do II Simpósio em 2021, sendo esse
planejado totalmente de maneira remota, e seguindo os padrões semelhantes ao primeiro, como a
organização em 3 dias consecutivos, duas palestras por dia, com palestrantes de diversas áreas e
de diferentes localidades, já que foi uma oportunidade de trazer convidados que não
conseguiriam vir até Santa Maria no modo presencial. Nesse segundo evento, o grupo optou por
realizar a inscrição dos ouvintes pela plataforma Doity, prezando por uma melhor organização,
além de usar a plataforma StreamYard para a realização da transmissão das palestras. Os vídeos
também ficaram salvos no canal do Youtube a fim de permitir a visualização a qualquer
momento. Em relação aos resultados do I Simpósio obtivemos 2,769 inscrições. Destes, 92,6%
dos inscritos eram estudantes de Odontologia. De 2,564 estudantes, 292 inscritos eram discentes
da UFSM e o restante de outras instituições. O número de visualizações de cada palestra foi de
2,129 . Através dos comentários no chat do Youtube, feedback enviados pelos acadêmicos ao
perfil do grupo e pelos debates internos, percebemos um aproveitamento do evento e satisfação
com as escolhas dos palestrantes e com a forma de realização do I Simpósio. Já o II Simpósio
teve um alcance menor com 1306 inscrições, assim sendo 13% referentes a alunos da própria
instituição e o restante público externo, logo um grande alcance além da UFSM . Portanto,
mesmo com os desafios impostos pelo ensino remoto, o grupo PET Odontologia UFSM
conseguiu manter seu planejamento das atividades de maneira virtual através dos Simpósios.
Assim, atingiram-se os objetivos desejados pelos integrantes do grupo através de eventos que
agregaram temas relevantes para a formação, de interesse dos participantes e com a presença de
convidados renomados.

\includepdf{pdfs/Simposio-Online--Do-Desaf}

\addcontentsline{toc}{section}{Série Documental: Controle Social Nas Comunidades Periféricas  (Episódio 1 - Controle E Participação Social)}

\section*{Série Documental: Controle Social Nas Comunidades Periféricas  (Episódio 1 - Controle E Participação Social)}

Brenda Barros Dias, CAROLINE BASTOS DA SILVA - UFRGS,FREDERICO VIANA MACHADO - UFRGS,GUILHERME DE ALMEIDA NICHES - UFRGS,LEOCIR MULLER RIBEIRO - UFRGS,SAMUEL SANTOS DA ROSA - UFRGS

Série Documental: Controle Social nas comunidades periféricas  (Episódio 1 - Controle e Participação Social)

Controle Social diz respeito às práticas de participação popular que visam incluir a população nos processos decisórios que determinam as prioridades e as práticas de saúde em seus territórios. Assim, Conselhos e Organizações Sociais podem contribuir para o estabelecimento da agenda política que vai determinar as melhorias nas condições de serviços e atendimento em saúde. As pessoas que realizam o Controle Social em saúde compreendem o funcionamento do Sistema Único de Saúde e o itinerário do cuidado, construindo as condições para pressionar o poder público em busca de uma agenda política que atenda efetivamente aos interesses dos usuários. No entanto, os mecanismos institucionais de participação não são devidamente divulgados e, por vezes, a população não conhece seus direitos e os caminhos que podem ser percorridos em busca de melhores serviços e condições de saúde. 

O Programa de Educação Tutorial Conexões Participação e Controle Social em Saúde (PET PCSS) da Universidade Federal do Rio Grande do Sul (UFRGS), no ano de 2021, iniciou o desenvolvimento do projeto “Série Documental: Controle Social nas comunidades periféricas”. A proposta do projeto é expor a realidade da Participação e do Controle Social nos bairros vulnerabilizados e comunidades periféricas em Porto Alegre no Rio Grande do Sul e relatar como os Conselhos de Saúde se relacionam com as lutas sociais e políticas autônomas. Fazem parte do escopo do projeto o registro e a discussão documental de fatos históricos relevantes para a compreensão das políticas de saúde e o registro de relatos de pessoas com atuação política e social, observando como impactam nos serviços de saúde dos seus territórios e da cidade.

O projeto busca apresentar o Conselho Municipal de Saúde e as práticas políticas relacionadas à participação social em saúde divulgando seus desafios e conquistas para a sociedade. Estão sendo trabalhados os aspectos históricos dos conselhos (municipal, distritais e locais) de saúde da cidade, como seu surgimento e desenvolvimento, as funções e ações dos conselhos, e, essencialmente, sua relevância na construção de políticas públicas na área da saúde. Tendo em vista a fundamental importância dos atores sociais na elaboração de ações que promovem saúde, tratamos de dar voz a esses indivíduos, ao mesmo tempo que contribuímos para a democratização do conhecimento para além do âmbito universitário. Nos propomos a verificar o entendimento da população sobre o termo Controle Social, avaliar qual a possibilidade de ação em agendas políticas e a composição de políticas públicas, avaliar o conhecimento da população sobre os Conselhos de Saúde, e principalmente criar uma ferramenta acessível para a qualificação e o  fomento à Participação Social.

A série documental foi elaborada com recursos audiovisuais, com entrevistas realizadas por meio de chamadas de vídeo na plataforma digital “Google Meet”. Expondo as informações de forma acessível em uma dinâmica dialógica, a produção audiovisual combina relatos orais de pessoas locais com elementos visuais históricos, contextualizando as falas em seu contexto histórico. A partir de um roteiro pré-definido, foram realizadas entrevistas virtuais com ativistas e usuárias do SUS. Foram feitas perguntas sobre os serviços de saúde existentes no território, a atuação de Conselhos Locais de Saúde e sobre o entendimento dos termos Controle Social e Participação Social. Além das entrevistas, foi realizada a busca de materiais históricos como notícias e vídeos já produzidos pela mídia local e pelo Laboratório de Políticas Públicas, Ações Coletivas e Saúde (LAPPACS/UFRGS), que disponibilizou sua produção audiovisual para o projeto através de uma parceria entre os grupos. Os vídeos foram editados por membros do PET PCSS, em parceria com uma bolsista de Iniciação Científica do LAPPACS, e posteriormente publicado no canal do grupo no Youtube (Disponível em: https://youtu.be/xJdTYP-m2U0). 

Este é o primeiro episódio e estão previstos, ao todo, cinco vídeos que farão parte da Série Documental. O lançamento deste episódio foi realizado em uma Live, na qual as PETianas puderam apresentar para o público o conteúdo produzido e os desafios para a realização do trabalho. Atualmente o vídeo conta com quase 400 visualizações no Youtube (Controle Social nas Comunidades Periféricas -   Episódio 1). Os resultados do projeto foram encaminhados para os Conselhos de Saúde de Porto Alegre e para as Comissões de Graduação da UFRGS, assim como para as pessoas entrevistadas e contatos de referência nas comunidades da cidade. Tornar esse conteúdo acessível foi um importante resultado do projeto, considerando que, além de utilizarmos uma plataforma gratuita e popular para a veiculação da Série Documental, ainda pudemos custear a interpretação em LIBRAS ampliando o acesso de pessoas surdas.


\includepdf{pdfs/Serie-Documental--Control}

\addcontentsline{toc}{section}{Ted-Pet: Método De Aperfeiçoamento Da Oratória No Pet Odonto}

\section*{Ted-Pet: Método De Aperfeiçoamento Da Oratória No Pet Odonto}

Natalia Brito Soares, Rosiane Pereira de Oliveira,Antônio Marcos Gonçalves Duarte,Luiza Souza  Schmidt,Douglas Bender Stopassola,Letícia Da Silva Pires,Josué Martos

1. INTRODUÇÃO
A apresentação de trabalhos é inerente à vida acadêmica tanto em disciplinas quanto em
seminários, jornadas ou congressos. Sendo assim, para que as apresentações sejam claras, objetivas
e de forma natural, o Grupo PET Odontologia da Universidade Federal de Pelotas implementou
em 2016 um projeto adaptado do método TED (acrônimo de Technology, Entertainment, Design).
O TED consiste originalmente em uma série de conferências, sem fins lucrativos, destinadas à
disseminação de conhecimento, com duração máxima de apresentação de 18 minutos,
incentivando os palestrantes a serem objetivos em suas explicações e argumentos, de acordo com
GALLO (2014) e ANDERSON (2016).
A partir da primeira conferência de 1990, começaram a ocorrer mundialmente diversas
apresentações que foram disponibilizadas posteriormente no site do TED, de forma gratuita,
visando disseminar os diversos conteúdos apresentados. Este tipo de apresentação tornou-se
cotidiana na vida profissional e acadêmica tanto em disciplinas quanto em seminários, jornadas ou
congressos.
Levando isso em consideração, esta atividade teve como objetivo promover entre os
petianos o hábito da oratória, buscando aprimorar suas habilidades e alcançar uma apresentação
clara, objetiva e de forma natural e de uma maneira mais abrangente.
2. METODOLOGIA
Os encontros do TED PET ocorrem uma vez por mês, na sala do Programa de Educação
Tutorial no prédio da Faculdade de Odontologia da UFPel (Universidade Federal de Pelotas).
Porém, devido a pandemia da COVID-19, estes encontros estão ocorrendo de forma remota desde
o ano de 2020, através da plataforma da UFPel, o WebConf. Cada petiano aborda um tema
relevante em que ele possa ter total domínio sobre o assunto. Para esta atividade o aluno poderá
usufruir da Biblioteca PET para escolha dos diferentes assuntos ou livros a serem abordados e uma
planilha com o cronograma de todo o TED/PET permite a organização interna das apresentações. A ordem das apresentações é definida no início do ano, por sorteio ou por comum acordo.
Cada integrante do grupo estuda um assunto/livro de seu interesse ou de interesse do grupo e
apresenta entre 18/30 minutos, explicando de maneira mais clara possível para que seus ouvintes
compreendam. No fim de cada apresentação há um debate sobre o tema apresentado, que envolve
o apresentador, o tutor e o grupo. Além disso, ao final ocorre a avaliação do grupo sobre pontos
importantes e sugestões ao apresentador. Esta avaliação não tem o objetivo de competitividade e
sim, de conhecimento dos pontos fortes e fragilidades.
3. RESULTADOS E DISCUSSÃO
O método TED para o PET Odontologia tem sido relevante para os integrantes do grupo para
o desenvolvimento da capacidade de oratória e dinâmica de apresentação dos petianos. Os
momentos de apresentações espelham a criatividade das mesmas de forma mais clara possível e
também geram um ambiente rico para discussões e para a autocrítica ao final do evento. Além de
possibilitar a avaliação do grupo ao apresentador, possibilita também que os alunos avaliadores
exercitem sua visão julgadora sobre as apresentações orais. Esta atividade vem proporcionando
mais segurança nas apresentações dos bolsistas em Palestras e Congressos.
Os momentos de apresentações geram um ambiente rico para discussões e para a autocrítica,
além de possibilitar que o grupo avalie o apresentador, possibilitando também que os alunos
avaliadores exercitem sua visão crítica sobre as apresentações orais.
Apresentadores acadêmicos do TED são muitas vezes escolhidos como destaque em
Congressos e Jornadas, demonstrando a importância do método TED, na evolução do processo de
comunicação, desenvolvimento da capacidade de oratória (GALLO, 2014; ANDERSON, 2016).
4. CONCLUSÕES
O TED PET exercita a capacidade dos integrantes do grupo PET Odontologia a melhorarem
o seu processo de comunicação e de disseminação de ideias, inspirando-os a falarem em público.
5. REFERÊNCIAS BIBLIOGRÁFICAS
ANDERSON, C. TED Talks: The official TED guide to public speaking. Nicholas Brealey
Publishing: Boston, 2016. 288p.
GALLO, C. Talk Like TED. St. Martin\'s Press: New York, 2014. 287p.

\includepdf{pdfs/Ted-Pet--Metodo-De-Aperfe}

\addcontentsline{toc}{section}{Transformação - Convertendo Gestos Em Objetos}

\section*{Transformação - Convertendo Gestos Em Objetos}

Jonnifer De Freitas Feltrin, Laina Miho Takaqui,Giovana Manchini Mendonça,Guilherme Henrique Oliveira Silva,Milena Lopes dos Santos,Maria Eduarda Brun,Thais Lumy Hatanaka

Problemática: Sabe-se que no Brasil o consumo irracional de medicamento vem crescendo 
demasiadamente e o lixo medicamentoso vem se tornando um problema social cada vez maior 
(FERREIRA et al, 2018; PFIZER, 2020). O alumínio e o plástico que estão presentes nesses 
materiais, são recursos que estão presentes na vida da população mundial diariamente e a milhares 
de anos. Nas últimas décadas muitos têm se questionado sobre o impacto que estes têm causado 
no meio ambiente e toda a questão de sustentabilidade e reciclagem vem vindo à tona (GORNI, 
2003). O alumínio vem sendo muito discutido nesse âmbito e em muitos casos é o escolhido para 
substituir o plástico na fabricação de inúmeros produtos, uma vez que seu custo é mais baixo e seu 
índice de reciclabilidade é bem mais alto, devido às suas características físico-químicas (DAVIES 
et al, 2018; GORNI, 2003). Na cidade de Maringá-Paraná, existe uma ONG chamada “Assistência 
a Reabilitação e Bem-estar de Convalescentes” (ARBEC) que busca maneiras de ajudar a 
população e o meio ambiente. Uma das maneiras propostas é por meio da coleta de blisters e 
cartelas de medicamentos feitas de alumínio e/ou plástico. A ONG repassa esse material a 
empresas de reciclagem, que, por sua vez, compra e entrega à ARBEC, aparelhos de auxílio a 
locomoção como cadeiras de banho, cadeiras de rodas, camas hospitalares e muletas. A ARBEC 
auxilia a população emprestando esses aparelhos. Para o empréstimo é necessário que o usuário se 
cadastre e se responsabilize equipamento. Justificativa: O objetivo da atividade “TransformAção” 
foi apoiar a ONG ARBEC por meio da promoção de coleta de blisters na comunidade, e difundir 
a importância do descarte racional de medicamentos e resíduos. Metodologia: Pontos de coleta de 
cartelas e blisters de medicamentos foram colocados na cidade de Maringá e região. Na cidade de 
Maringá foram selecionados 3 pontos de coleta, sendo um dentro da Universidade Estadual de 
Maringá, outros dois em comércios nos arredores da Universidade. Na cidade vizinha de Marialva, 
foi realizada uma parceria com o Interact Club local, e na cidade de São Jorge do Patrocínio, o 
ponto de coleta foi colocado em uma farmácia local. O recipiente destinado à coleta de cartelas, 
foi adaptado a partir de galões de água com capacidade de 20L onde continha placa explicativa do 
tipo de produto a ser coletado (cartelas e blisters de comprimidos). O coletor foi confeccionado 
pelos PETianos. A cada 20 dias estes coletores eram esvaziados e o material coletado ficava 
armazenado na sede do PET-Farmácia da UEM. Este material arrecadado foi posteriormente 
separado de forma manual pelos PETianos, e os medicamentos residuais encontrados foram 
retirados das embalagens e destinados ao descarte correto a uma Unidade Básica de Saúde local. 
As cartelas coletadas foram encaminhadas para a ARBEC. Resultados e Discussão: Após 
destinação, o material recolhido pesado perfez um total de 86,4 kg de blisters. Conforme prática 
da ARBEC, esse material fica armazenado na instituição até um total de uma tonelada, quando 
será retirado pela empresa que recicla e converte os blisters, transformando-os em insumos para 
produção de distintos materiais. Esse insumo é uma mistura de plástico e alumínio. O Projeto 
TransformAção atendeu as expectativas do grupo, tanto em volume de material arrecadado, quanto 
no engajamento dos PETianos e da comunidade. Além do produto arrecadado per se, o Projeto 
TransformAção trouxe a oportunidade de divulgar informações sobre o descarte correto de 
medicamentos em redes sociais, e também pela comunidade afeta aos pontos de coleta. Essa 
percepção veio de comentários recebidos pelos PETianos, dos responsáveis dos estabelecimentos 
onde os pontos de coleta foram instalados. Atribui-se como beneficiários direto do projeto a 
população que poderá usufruir dos equipamentos disponíveis, e também o meio ambiente, já que 
todo o plástico, alumínio e medicamentos foram dispensados da maneira correta. Indiretamente, 
os PETianos do grupo, que puderam sentir a experiência do trabalho social voluntário, fortificando 
assim os laços de companheirismo. Conclusão: O projeto TransformAção obteve êxito nos seus 
objetivos, alcançando um bom volume de materiais arrecadados, despertando o interesse da 
comunidade pelo tema e sensibilizando os PETianos pela causa do voluntariado.
Referências
DAVIES, F. R.; DA SILVA, A. P. L. OS TRÊS PILARES DA SUSTENTABILIDADE NA KNX 
PLÁSTICO E ALUMÍNIO. Revista Tecnológica, v. 27, n. 1, p. 59-69, 2018.
Descarte correto de medicamentos também salva vidas. Disponível em:> 
https://www.pfizer.com.br/noticias/ultimas-noticias/descarte-correto-de-medicamentos-tambemsalva-vidas.> Acesso em: 29 de ago de 2021.
FERREIRA, R. L.; JÚNIOR, A. T. T.. ESTUDO SOBRE A AUTOMEDICAÇÃO, O USO 
IRRACIONAL DE MEDICAMENTOS E O PAPEL DO FARMACÊUTICO NA SUA 
PREVENÇÃO: Imagem: Vida e Saúde. Revista Científica da Faculdade de Educação e Meio 
Ambiente, v. 9, n. edesp, p. 570-576, 2018.
GORNI, A. A. INTRODUÇÃO AOS PLÁSTICOS. Revista plástico industrial, v. 10, n. 9, 2003.

\includepdf{pdfs/Transformacao---Converten}

\addcontentsline{toc}{section}{Utilização De Redes Sociais Como Ferramenta Para A Disseminação Das Geociências Pelo Pet Geologia Ufpr}

\section*{Utilização De Redes Sociais Como Ferramenta Para A Disseminação Das Geociências Pelo Pet Geologia Ufpr}

Thaisa Stoco Dos Santos, Amanda Rompava Lourenço - UFPR,Bianca Leticia Marghoti - UFPR,Nicolas dos Santos Rosa - UFPR,Paulo Henrique Ferreira da Silva - UFPR,Tarso Feraboli Curcino - UFPR,Prof. Dr. Fábio Braz Machado - UFPR

Com o início e rápida disseminação da pandemia de COVID-19 a nível global, o isolamento
social, lockdowns e outras medidas restritivas e sanitárias, acarretaram na intensificação do uso
de mídias sociais como forma de comunicação e disseminação de informações. Com isso, cresce
em conjunto o número de propagação de fake news, e consequentemente a necessidade de
ampliar a divulgação científica. Marandino et al. (2003), discorre sobre a quem deve a ocupação
com divulgação científica, por um lado, o pesquisador pela sua “natural” competência, e seu
compromisso social com aqueles que lhe financiam. E por outro, profissionais formados em
comunicação científica, com estudos voltados a tal área. Porém um ponto é convergente: o
processo de divulgar ciência implica em uma transformação da linguagem científica com vistas a
sua compreensão pelo público. Tendo isso em vista, objetiva-se relatar experiências e
metodologias do grupo PET Geologia UFPR na divulgação geocientífica, visando aproximar-se
da comunidade com linguagens acessíveis, aspectos do cotidiano e humor. Durante o segundo
semestre de 2020, o projeto interno intitulado “Um olhar sobre a Terra” atuou em produções de
vídeos que buscaram compreender temas de amplo contorno geocientífico e, ao mesmo tempo,
de fácil entendimento para o público externo. A produção seguiu por etapas como pesquisa
bibliográfica, roteirização, gravação, edição e postagem dos vídeos nas principais redes sociais,
Instagram e Facebook. O modelo utilizado assemelhou-se ao modelo de Trailer, com duração
média dos vídeos de 4 minutos, sendo apresentado os temas geologia geral, minerais e rochas,
paleontologia, estrutura interna da Terra, vulcanismo e desastres naturais. Após isso, durante o
primeiro semestre de 2021, levando em conta o modelo de organização interna do grupo PET em
questão, a demanda de comunicação com o público externo, no que refere-se a estruturação e
administração das redes sociais, encontra-se a cargo da Secretaria de Relações Externas. Assim
sendo, a organização da secretaria, utilizando linguagem adequada ao público alvo, pesquisou e
reuniu temas, estruturou postagens e compartilhou os produtos nas redes Instagram e Facebook.
Priorizou-se temas julgados “em alta” na comunidade geocientífica e que despertassem a
curiosidade, tanto dos discentes do curso de Geologia como do público externo à universidade. O
modelo de postagens propôs a utilização de textos curtos e diretos, e de imagens ilustrativas de
fácil entendimento, sugestionando uma pesquisa pessoal mais aprofundada posteriormente. Para
mais, os temas empregados foram “Machine learning nas geociências”, “Mineração espacial”,
paleontologia e “Red Dead Redemption 2 e seus aspectos relacionados a geologia”, sendo este
último evidenciando os aspectos geológicos das paisagens gráficas que o jogo “Red Dead
Redemption 2” expõe. Os vídeos do “Um olhar sobre a Terra” obtiveram 1.851 visualizações no
Instagram e alcance orgânico de 10.525 no Facebook. Além disso, as postagens do primeiro
semestre de 2021 foram compartilhadas e curtidas por centenas de pessoas, utilizando as
ferramentas de funcionamento próprio das redes sociais já mencionadas. Portanto, os integrantes
do PET Geologia UFPR puderam, além de instruir-se quanto aos temas pesquisados, entrar em
contato com a comunidade externa e, consequentemente, com a linguagem necessária e
adequada para disseminar a ciência desenvolvida dentro e fora do Programa de Educação
Tutorial. Parafraseando Aracri et al. (2015), o saber científico não pode ficar restrito a um único
grupo, visto que suas implicações promovem a melhoria na qualidade de vida no planeta. A
produção audiovisual empregada acarretou, também, no contato e experiências em programas de
produção e edição de imagens e vídeos, o que aumentou o repertório de pré-requisitos
individuais dos integrantes. Por fim, difundiu-se as geociências durante o período atípico da
pandemia de COVID-19, aproximando os discentes das atividades suspensas pelos protocolos
sanitários.
Referências
ARACRI, Eveline Milani Romeiro Pereira et al. A Olimpíada Brasileira de Geociências:
contribuição para a popularização das Ciências da Terra. Terræ Didatica, v. 11, n. 2, p. 108-116,
2015.
MARANDINO, Martha et al. A educação não formal e a divulgação científica: o que pensa
quem faz. Atas do IV Encontro Nacional de Pesquisa em Ensino de Ciências, 2004.
SILVEIRA, Renata Vasconcelos Alves. Saúde sem Fronteiras: ações de divulgação científica
em tempos de pandemia. 2020.

\includepdf{pdfs/Utilizacao-De-Redes-Socia}

\addcontentsline{toc}{section}{Versos Do Índico: Grupo Cênico-Literário Contarolando (Pet  Pedagogia Ufsc) Na Pandemia}

\section*{Versos Do Índico: Grupo Cênico-Literário Contarolando (Pet  Pedagogia Ufsc) Na Pandemia}

Jayziela Jessica Fuck, Juliana Breuer Pires,Rafael da Silva,Elizabeth de  Souza Neckel,Carlos Henrique De Moraes Barbosa,Lucas Rodrigues Menezes

O PET de Pedagogia (UFSC) em 2011 iniciou ações de contação de histórias por meio do 
Grupo Cênico-Literário Contarolando, ação articulada com a tríade universitária (pesquisa, ensino 
e extensão). O projeto busca aproximar as/os bolsistas e estudantes da Pedagogia, futuras/os 
professoras/es, a um repertório literário que valorize o texto estético (palavra/imagem e 
materialidade). O Contarolando tem sido contínuo no Grupo, promovendo a fruição literária e a 
criação artística, promovendo momentos onde os integrantes do PET Pedagogia experienciam a 
literatura junto ao público, em particular, crianças em espaços educativos, mas não só. Desde 2017, 
o Grupo tem centrado o seu fazer a partir da temática da cultura africana e afro-brasileira, trazendo
para o cenário da formação/ação a reflexão sobre ela, contribuindo para pensar a educação das 
relações étnico-raciais, pois cremos que o cumprimento da Lei nº 10.639/, de 2003, é um dos 
deveres da Universidade (BRASIL, 2003). Em 2020, isolados socialmente pela pandemia da Covid 
19, o grupo teve que se reorganizar e realizar as suas ações de forma remota, o que ocorre até o 
momento (setembro de 2021). Neste trabalho apresentamos “Versos do índico” um dos projetos 
realizados neste período que buscou contemplar as literaturas africanas de língua portuguesa, em 
particular aquela produzida em Moçambique.
Acreditamos que ações que promovam o acesso a leitura de títulos de escritores oriundos 
do continente africano, dialoga com a Lei de Diretrizes e Bases (LDB) (BRASIL, 1996), a Lei 
10.639/2003 que cria a obrigatoriedade do ensino da História e Culturas Afro-brasileiras e 
Africanas, as Diretrizes Curriculares Nacionais para a Educação das Relações Étnico-Raciais 
(BRASIL, 2004), e por certo impacta na sensibilidade estética oriunda da palavra literária, bem 
como o alargamento de um repertório, que ultrapasse o acervo eurocêntrico. 
Assim, o projeto Versos do Índico busca socializar, por meio de vídeos nas redes sociais, 
poemas para infância de escritores Moçambicanos. Para isso, detemo-nos nos poemas dos livros 
Passos de Magia ao Sol, de Mauro Brito, O Gil e a Bola Gira e outros poemas para brincar, de 
Celso C. Cossa, e Viagem pelo Mundo num Grão de Pólen e Outros Poemas, de Pedro Pereira
Lopes. Os dois primeiros títulos têm publicação somente em Moçambique, o último tem 
publicação no Brasil. A escolha desses escritores e títulos se deve a proximidade do grupo com 
essa literatura, em razão de uma visita de Mauro Brito em 2017 ao grupo, e da participação, em 
2019, dos três escritores no 8 Seminário de Literatura Infantil e Juvenil, evento que teve como 
coorganizador o PET e foi realizado na UFSC.
Metodologicamente o projeto, “Versos do Índico”, foi pensado para levar a literatura de 
Moçambique para diferentes infâncias e foi elaborado por meio de ensaios semanais de forma 
remota pelas plataformas Microsoft Teams ou Google Meet. Os bolsistas foram orientados pela 
bolsista da Secretaria de Cultura (SECARTE/UFSC) Lílian Zoldan, estudante do curso de Artes 
cênicas e pela doutoranda do Programa de Pós-Graduação em Educação Waleska Regina Becker 
Coelho De Franceschi, formada em artes cênicas. Os ensaios permitiram a escolha dos poemas e
o exercício da leitura em voz alta, posteriormente, devido à necessidade de isolamento social, os 
vídeos foram gravados individualmente pelos integrantes do grupo. Ao final do processo criativo, 
os vídeos dos poemas foram publicados no canal do YouTube (https://youtu.be/K3lxXqYDPQY), 
no Instagram (@petpedagogia07) e compartilhados no Facebook
(https://www.facebook.com/PETPedagogiaUFSC) do PET de Pedagogia da UFSC, com o 
objetivo de divulgar os poemas, o grupo, o projeto, fortalecendo as possibilidades cênico-literárias 
na formação docente.
O compromisso coletivo do grupo de promover a palavra literária de forma criativa 
reverberou em diferentes espaços, “a pandemia do Coronavírus nos separou fisicamente, mas não 
nos desmobilizou para repensar nossas ações, fazendo com que elas nos alimentassem a nós e aos 
outros” (DEBUS; ZOLDAN; FRANCESCHI, 2021, p. 71). A realização das ações também foi um 
ato de resistências.
Referências
BRASIL. Lei no
9.394, de 20 de dezembro de 1996. Estabelece as diretrizes e bases da educação 
nacional. Diário Oficial da União, Brasília, DF, 23 dez. 1996.
BRASIL. Lei no
10.639, de 9 de janeiro de 2003. Altera a Lei no 9.394, de 20 de dezembro de 
1996, que estabelece as diretrizes e bases da educação nacional, para incluir no currículo oficial 
da Rede de Ensino a obrigatoriedade da temática \"História e Cultura Afro-Brasileira\", e á outras 
providências. Diário Oficial da União, Brasília, DF, 10 jan. 2003.
BRASIL. Diretrizes Curriculares Nacionais para a Educação das Relações Étnico-Raciais e 
para o Ensino de História e Cultura Afro-Brasileira e Africana. Brasília: Conselho Nacional 
de Educação, 2004.
BRITO, Mauro. Passos de Magia ao Sol. Ilustração de Bárbara Marques. Maputo: Editorial 
Escola Portuguesa de Moçambique, 2016. 
COSSA, Celso C. O Gil e a Bola Gira e outros poemas para brincar. Ilustração de Luis Cardoso. 
Maputo: Editorial Escola Portuguesa de Moçambique, 2016. 28p. 
DEBUS, Eliane; ZOLDAN, Lilian M.; FRANCESCHI, Waleska Regina Becker Coelho de. O 
grupo contarolando e a pandemia do coronavirus: os desafios do ano de 2020. In: GOULART, Ilsa 
do Carmo V.; CABRAL, Giovanna R.; NEVES, Ludmila M. Reinvenção da arte de contação de 
histórias. Rio de Janeiro: e-Publicar, 2021. 
LOPES, Pedro Pereira. Viagem pelo Mundo num Grão de Pólen e Outros Poemas. Ilustração 
de Filipa Pontes. Maputo: Editorial Escola Portuguesa de Moçambique, 2015.

\includepdf{pdfs/Versos-Do-Indico--Grupo-C}

\addcontentsline{toc}{section}{Vi Semana Da Agricultura Familiar: Mulheres Rurais, Mulheres De Direitos, Mulheres De Respeito}

\section*{Vi Semana Da Agricultura Familiar: Mulheres Rurais, Mulheres De Direitos, Mulheres De Respeito}

Fernanda Spagnol, ANDREZA BITENCOURT,GABRIELA QUIEZI,JEAN POSSENTI,KAROLINE SASKOSKI,LEONARDO ZIMMER,MATHEUS SANTOS

A questão da discriminação de gênero se estende por décadas. Existe um grande 
contingente feminino no meio rural desenvolvendo atividades agrícolas e não agrícolas, 
colaborando para a produção dentro da agricultura familiar, mas que infelizmente, pouca 
visibilidade alcança (CARVALHO, 2012).
Segundo Marion e Bona (2016) são nítidas as poucas atribuições e o número de mulheres 
que se engajam nessa diversificação, pois entendem que ainda existe um preconceito quanto à 
capacidade de realizar atividades desenvolvidas pelo sexo oposto, como operar máquinas 
agrícolas, atuar como financiadoras de investimentos e serem vistas como proprietárias.
No entanto, o número de mulheres que participam como gestoras na agricultura empresarial 
e familiar, tem aumentado ao longo dos anos e elas desempenham muitas funções básicas para 
essas atividades. De acordo com o último censo do IBGE (2017), 18,6% das mulheres que se 
declararam chefes de empreendimentos rurais desempenhavam funções como agricultoras e 
gerentes da propriedade.
Com o cenário atual da pandemia mundial devido à Covid-19, o grupo PET-AF manteve 
seu foco nas atividades planejadas, buscando-se adptar à nova realidade de distânciamento social 
e, de forma proativa, adotou a modalidade da tecnologia digital para a realização dos seus eventos. 
Desta maneira focou seus objetivos para que os pilares de ensino, pesquisa e extensão 
desenvolvidos pelo grupo, fossem mantidos.
Logo, uma das atividades desenvolvidas nesse período foi a Semana da Agricultura 
Familiar. Na sua sexta edição, o evento se caracteriza por ser de extensão rural e ocorre 
anualmente baseado no tema da Organização das Nações Unidas para a Alimentação e Agricultura 
(FAO). Para 2020, a campanha da FAO teve o tema “Mulheres Rurais, Mulheres com Direitos”.
Com algumas adaptações necessárias realizadas pelo grupo, definiu-se a sexta edição como 
“Mulheres Rurais, Mulheres de Direitos, Mulheres de Respeito”.
Neste sentido, a VI Semana da Agricultura Familiar promovida pelo Grupo PET- Conexões 
de Saberes Agricultura Familiar, buscou homenagear e incentivar mulheres agricultoras, 
conscientizando o núcleo familiar quanto à valorização e a melhor distribuição das atividades 
realizadas. Além disto, incentivar mulheres jovens e adultas a serem protagonistas, como gestoras 
e produtoras rurais, reconhecendo o seu lugar no âmbito da agricultura.
O evento é destinado ao público interno e externo à Universidade Tecnológica Federal do 
Paraná (UTFPR). Nesta edição, buscou-se principalmente, a participação de mulheres ligadas à 
agricultura, pois toda a programação do evento, foi concebida neste sentido. Todas as palestras, 
foram proferidas por mulheres, de forma a valorizar o sexo feminino no meio rural e divulgar 
informações técnicas e de produção para as pequenas propriedades rurais. Desta maneira, o evento, 
deu-se de forma totalmente on-line através do Canal da Plataforma Youtube do PET Agricultura 
Familiar na semana de 23 a 27 de novembro de 2020. As transmissões das palestras tiveram sempre 
início às 20 horas de Brasília, com duração de aproximadamente uma hora e meia cada uma. Após, 
foram realizados questionamentos, mesas redondas e debates com a participação do público. 
Os temas abordados foram “O poder da Mulher na Propriedade Rural”; “Produção de Ovos 
Caipira: Como funciona na prática”; “Cultivo do Porongo à Produção da Cuia: Alternativa na 
Agricultura Familiar”; “Projeto Flores para Todos: Incentivo à Produção Local de Flores e 
Incremento de Renda para a Agricultura Familiar” e; “A trajetória de uma Mulher Frente a 
Produção Orgânica de Alimentos”.
Ao decorrer das transmissões, as convidadas discorreram sobre suas práticas cotidianas e 
demonstraram as atividades que podem ser realizadas como alternativa de renda na agricultura 
familiar. Os resultados foram bem significativos, visto que a participação do público feminino foi 
muito boa, além das dúvidas que puderam ser sanadas. Através das palestras, foi possível a 
percepção de um grande engajamento por parte das internautas, por conseguinte aumentando a 
visibilidade do canal do PET Agricultura Familiar juntamente com a interação com a comunidade.
Assim, o evento proporcionou informações técnicas para o público em geral e incentivo 
às mulheres. Neste sentido, a atuação dos grupos PET’s nas universidades é fundamental, uma 
vez que frente ao momento de pandemia atual, as atividades de ensino, pesquisa e extensão de 
forma remota devem ser mantidas.
Referências 
BRASIL. Instituto Brasileiro de Geografia e Estatística - IBGE. Censo 2017. Disponível em: <
https://censos.ibge.gov.br/agro/2017/>. Acesso: setembro de 2021.
CARVALHO, D. J. D. O empoderamento da mulher na agricultura familiar de Carvalhópolis, 
UFF.Rio de Janeiro, p. 137, 2012.
MARION, A.A.; BONA, A.N. A importância da mulher na agricultura familiar. Curso de 
Cooperativismo Solidário e Crédito Rural. Pública Cresol. Francisco Beltrão, p. 1-11, 2016.

\includepdf{pdfs/Vi-Semana-Da-Agricultura-}

\addcontentsline{toc}{section}{Vivências Do Grupo Pet Comunidades Do Campo: A Adaptação Para O Modelo Remoto }

\section*{Vivências Do Grupo Pet Comunidades Do Campo: A Adaptação Para O Modelo Remoto }

Emanoela Gabriela Da Silva Conceicao, Thainara Rocha do Nascimento; Georgia Rossi de Aguiar; Luisa Fernandes de Almeida; Udson Rodrigues da Silva; Gabriela Matos Pereira de Carvalho; Valdir F. Denardin  pet.comunidadesdocampo@gmail.com,PET Comunidades do Campo (PET CC),Universidade Federal do Paraná (UFPR)

No final do ano de 2019 uma pandemia causada pelo vírus SARS-CoV-2 (COVID-19)
trouxe graves consequências para a saúde humana e sociedade em escala mundial. O período
coincidiu com a volta do ano letivo da UFPR no início de 2020, logo, o grupo PET Comunidades
do Campo teve de adaptar-se para um formato remoto. O desafio era executar de forma virtual o
planejamento anual, concebido para o ambiente presencial. Com a passagem do tempo, foram
surgindo contratempos constantes envolvendo: acesso à internet, funcionamento das plataformas
de videoconferências para a realização das reuniões e outras questões relacionadas ao mundo
virtual. Ao buscar soluções para esses desafios, foram surgindo alternativas, que, após debates
coletivos, foram sendo adaptadas, reformuladas e outras criadas para além do planejamento
estabelecido antes da pandemia.
Uma das ideias de adaptação foi o ciclo de estudos do campo, iniciado no primeiro
semestre de 2020, contando com leituras e fichamentos sobre educação ambiental e questão
agrária no Brasil a fim de dar suporte aos debates entre estudantes. O diálogo socioambiental foi
constante promovendo à comunidade petiana avanços neste campo de conhecimento.
Devido ao isolamento social as atividades relacionadas à extensão foram fortemente
impactadas, não podendo ser realizadas. Como exemplo temos o Seminário de Desenvolvimento
Sustentável, organizado anualmente pelo PET Litoral Indigena, Litoral Social e Comunidades do
Campo, ambos da UFPR Litoral. Os grupos, então, aproveitaram a data comemorativa de 10
anos dos PET do Setor Litoral e organizaram um evento com a presença de egressas (os) e ex
tutores, possibilitando compreender os benefícios do programa dentro e fora da academia.
Além disso, o grupo percebeu que a pandemia escancarou as desigualdades territoriais e
raciais estruturadas no Brasil e no mundo, a desigualdade no acesso aos direitos básicos como
saúde e trabalho, deixando as populações negras, indígenas, tradicionais e periféricas ainda mais
vulneráveis. Foi consenso que debater sobre o racismo estrutural seria fundamental, surgindo
atividades com esse viés: a “Mesa Online sobre Saúde Mental para População Negra”, “Mesa
Redonda Negritude e Representatividade: Perspectivas para 2021” e “Cabelos que Crescem para
Cima” poema de autoria da PETiana Juliana Modesto. Esse debate é considerado urgente dentro
de todos os programas PET, e também em toda sociedade para o fortalecimento da luta
antirracista. Também participamos de um evento promovido pelo PET Psicologia da UNB, que
trouxe a abordagem do racismo estrutural no país, contribuindo para a ampliação do
conhecimento sobre o assunto no grupo e promovendo o intercâmbio de ideias e abordagens.
Considerando que o PET CC desde seu princípio, objetiva conectar conhecimentos e se
envolver com outros PET, criando uma conexão de saberes, o formato online permitiu uma maior
aproximação com esses grupos. Com o PET Políticas Públicas da UTFPR, através de debates
sobre os textos por eles criados, surgiu a inspiração para o artigo “Ensino remoto na rede pública
de ensino do Paraná em tempos de pandemia” que foi submetido e aceito para publicação na
Revista Extensão em Foco, ainda em 2021. Com o PET Química da UFPR, ocorreu o evento
online “Agrotóxicos: é agro ou tóxico”, que resultou na publicação de uma cartilha. Em parceria
com o PET Litoral Social criamos uma lista colaborativa de audiovisuais que foram produzidos
sobre e no Litoral do Paraná, retratando as paisagens e a biodiversidade da Mata Atlântica, sua
história e a cultura dos seus povos ancestrais M’byá Guarani, caiçaras, pescadores artesanais,
agricultores e quilombolas. Compondo um acervo no nosso canal no Youtube, de fácil acesso,
online para todos que desejam conhecer estas temáticas.
No período remoto trabalhamos na construção da cartilha “Manual basico e dicas para
utilizar a ferramenta Drive”, que se constitui por um material pedagógico simples e acessível
para informatização da comunidade de forma não presencial e também servirá como material de
apoio para as próximas oficinas ministradas presencialmente pós pandemia. Já o ENAPET,
organizado pela UFPR, no modelo presencial, nos dificultava se integrar na organização devido
às distâncias entre a capital e a nossa sede em Matinhos, 135 km, porém com a mudança para o
formato online o grupo se inseriu não apenas no apoio, mas também na organização. O evento
teve grande proporção e espaços de debate.
Para a atividade Cine Saberes, houve uma adaptação para Cine Conexões, uma versão
online de sessão de transmissão coletiva e debate acerca de materiais audiovisuais. Essa
adaptação foi desafiadora e ao mesmo tempo interessante, pois o fato de ser online possibilitou a
participação de ouvintes e convidadas(os) de outras regiões que não conseguiriam tão facilmente
participar presencialmente.
Por fim, foi realizada a criação de uma base de dados sobre unidades de conservação e
comunidades tradicionais para divulgação e popularização da ciência que foi divulgada em
nossas redes sociais.
Mesmo com tantos desafios e adaptações, o grupo se surpreendeu e renovou. As
ferramentas onlines proporcionaram ampliar o conhecimento, além de aumentar nossa interação
com alguns eventos e convidadas(os) distantes do nosso território. Apesar de ser um momento
difícil para todos, esse formato remoto proporcionou oportunidades e novas informações que
antes não eram consideradas. No entanto, após praticamente dois anos de atividades remotas
intensas, o grupo está mentalmente exausto e é perceptível o quanto o contato humano possui um
valor incomensurável. As tecnologias irão permanecer nos nossos planejamentos futuros, pois já
mostraram seu inequívoco valor, mas consideramos que nada substitui o contato presencial na
construção do conhecimento.

\includepdf{pdfs/Vivencias-Do-Grupo-Pet-Co}

\addcontentsline{toc}{section}{Yellow Cow: Reformulando O Aprendizado De Idiomas}

\section*{Yellow Cow: Reformulando O Aprendizado De Idiomas}

Roberta Xavier Giovanetti, Ana Carolina Cardoso Gomes Marcelino (UFPR),Celeste Miyuki Nagase Ikeda (UFPR),Giovanna Beatriz Sari Hey (UFPR),Nathália Carolina Barreiro Marques (UFPR)

Visando desempenhar um dos objetivos do programa, o de “estimular a melhoria do
ensino de graduação por meio da atuação dos bolsistas como agentes multiplicadores,
disseminando novas ideias e práticas entre o conjunto dos alunos do curso” (BRASIL, 2006),
o Grupo PET Farmácia/UFPR idealizou o projeto intitulado Yellow Cow, o qual subdivide-se
atualmente em “Yellow Cow Básico” e “Yellow Cow Plus”, ambos realizados com
periodicidade mensal. Tal iniciativa objetiva ampliar o conhecimento de idiomas estrangeiros
do grupo e do corpo discente do curso de Farmácia da Universidade Federal do Paraná, por
meio de apresentações, dinâmicas e rodas de conversas.
Para esta finalidade, durante os anos de 2020 e 2021, tendo em vista o contexto de
distanciamento social, o subprojeto Yellow Cow Básico utilizou-se de apresentações de slides,
vídeos e jogos interativos no idioma escolhido pelos próprios integrantes, incentivando o
aprendizado de diversas línguas, como o inglês, espanhol, francês e alemão. Já o subprojeto
“Yellow Cow Plus” é aberto para a graduação, e visa exercitar e promover a conversação em
inglês de forma dinâmica, por meio de exibição de vídeos, jogos e rodas de conversa entre
discentes internos ou externos ao Programa. Ambos foram promovidos de forma remota pela
plataforma Microsoft Teams.
O Yellow Cow Básico é realizado apenas entre os próprios integrantes do PET; para
tal, o grupo de trabalho responsável pelo projeto se divide em grupos de 2 a 3 pessoas e
seleciona o idioma a ser apresentado. Posteriormente, o material é trabalhado com o grupo em
reunião ordinária, promovendo um aprendizado sobre a gramática e o vocabulário dos
diferentes idiomas citados, de forma estimulante, a partir da interação entre os integrantes.
No caso dos encontros do Yellow Cow Plus, 2 a 3 integrantes do grupo de trabalho
elaboram materiais gráficos como apresentação de slides e vídeos no idioma inglês, e jogos
em plataformas como o Kahoot, proporcionando o contato entre os participantes e a
possibilidade de conversação neste idioma, a fim de aprimorar as habilidades de pronúncia da
língua inglesa.
Dessa forma, o projeto Yellow Cow auxilia na ampliação do repertório de idiomas
estrangeiros por meio de apresentações dinâmicas, tanto internamente, para o grupo PET
Farmácia UFPR, quanto para a graduação, auxiliando no aprimoramento profissional de tais
estudantes, visto que, no mundo cada vez mais globalizado, a fluência em línguas estrangeiras
tem sido cada vez mais valorizada no mercado de trabalho de diferentes áreas. Além disso, o
projeto também incentiva a tutoria e a didática dos(as) PETianos(as) que coordenam as
atividades, aprimorando, além dos conhecimentos básicos da língua e a capacidade de
conversação, habilidades de oratória, síntese, comunicação, criatividade, entre outras.
Tendo em vista os benefícios proporcionados por este projeto, o PET Farmácia UFPR
dará continuidade ao Yellow Cow no ano de 2022, de forma presencial ou remota, em virtude
da unânime opinião positiva do grupo e dos feedbacks recebidos por participantes externos.
Referência:
BRASIL, Ministério da Educação. Manual de Operações Básicas do Programa de Educação
Tutorial, versão 2006. Brasília, DF.

\includepdf{pdfs/Yellow-Cow--Reformulando-}

\addcontentsline{toc}{section}{Zoopet - O Futuro Da Zootecnia Chegou: Uma Possibilidade De Aprimoramento Por Meio  Remoto}

\section*{Zoopet - O Futuro Da Zootecnia Chegou: Uma Possibilidade De Aprimoramento Por Meio  Remoto}

Vanessa Bolonhesi Da Silva, Ana Flavia Nascimento e Silva,Claudia Inez  Domenes Danner,Laura Maria Borri de Souza

O Encontro Nacional dos Grupos PET de Zootecnia - ZOOPET - é um evento bienal, sendo sediado pela primeira vez na Universidade Estadual de Maringá (UEM), no ano de 2005, pelo grupo PET Zootecnia da Instituição. Em 2021, na sua 9° edição, o evento voltou a ser sediado novamente pela UEM, mas adaptado para o cenário atual, o que culminou em seu formato totalmente remoto sob o tema: Zootecnia 4.0: O Futuro Chegou. O evento ocorreu de 27 a 29 de abril de 2021, contando com 248 inscritos que puderam apreciar palestras e workshops, além de rodas de debates, apresentação de trabalhos, mostras culturais e atividades de integração.


\includepdf{pdfs/Zoopet---O-Futuro-Da-Zoot}

